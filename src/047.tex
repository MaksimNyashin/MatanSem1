(134)

\T \q О существовании и непрерывности обратной функции. Gecnm $f: C(\ang ab \to \R),\s f$ строго монотонна, \F{$m = \dsl\inf{x \in \ang ab}{} f(x), \quad M = \dsl\sup{x\in\ang ab}{} f(x)$.} Тогда справедливы следующие утверждения.\\
1. $f$ обратима, $f^{-1}: \ang mM \to \ang ab$~--- биекция.\\
2. $f^{-1}$ строго монотонна одноимённо с $f$.\\
3. $f^{-1}$ непрерывна.

\D Пусть для определения $f$ строго возрастает.\\
Если $x_1, x_2 \in \ang ab,\s x_1 < x_2$, то $f(x_1) < f(x_2)$; следовательно $f$ обратима. По теореме о сохранении промежутка $f(\ang ab) = \ang mM$. По общим свойтсвам обратимого отображения $f^{-1}$~--- биекция $\ang mM$ и $\ang ab$.\\
Докажем, что $f^-1$ строго возрастает. Если $y_1, y_2 \in \ang m, M,\s y_1 < y_2$, то $y_ 1 = f(x_1),\s y_2 = f(x_2)$, где $x_1, x_2 \in \ang ab,\s x_1 = f^{-1}(y_1),\s x_2 = f^{-1}(y_2)$. При этом $x_1 < x_2$, так как возможность $x_1 \ge x_2$ исключена в силу строгого возрастания $f$.\\
Возрастающая функция $f^{-1}$ задана на промежутке $\ang mM$, а её множество значений~--- промежуток $\ang ab$. По теореме о разрывах и непрерывности монотонной функции, она непрерывна.

\Zam1. Для обратимости строго омнотонной функции и строгой монотонности обратимой функции непрерывность не нужна.

\Zam2. 1. Множество точек разрыва монотонной функции не более чем счётно.\\
2. Если функция задана на промежутке, непрерывна и обратима, то она строго монотонно и, следовательно, обратная функция непрерывна.\\
3. Отображение, обратное к непрерывному, может окад=заться разрывным. Сопоставим каждой точке $x$ подуинтервала $[0, 2\pi)$ точку $f(x)$ единичной окружности $S$, такую что длина дуги, отсчитываемой от точки $f(0) = (1, 0)$ до точки $f(x)$, равна $x$ (или, что тоже самое $\arg f(x) = x$). Отображение $f$ биективно и непрерывно, но $f^{-1}$ терпит разрыв в точке $(1, 0)$. Близким к ней точкам окружности с отрицательной ординатой соответствуют точки полуинтервала, близкие к $2\pi$, а не к 0.\\
Но если отображение задано на компакте, непрерывно и обратимо, то обратное отображение непрерывно.\\
4. Существует обратимая функция $f: \R \to \R$, непрерывная в точке 0, но такая, что $f^{-1}$ разрывна в точке $f(0)$.