(163)

\def\ti{\tilde}

\T \q Замена на эквивалентную при вычислении пределов. Пусть $X$~--- метрическон пространство, $f, \ti{f}, g, \ti{g}: D\subset X\to \R(\C),\s x_0$~--- предельная точка $D$, \FF{f(x) \sim \ti{f}(x),\s g(x) \sim \ti{g}(x),\quad x \to x_0.} Тогда справедливы следующие утверждения.\\
1. $\dsl\lim{x\to x_0}{} f(x)g(x) = \dsl\lim{x\to x_0}{} \ti{f}(x)\ti{g}(x)$.\\
2. Если $x_0$~--- предельная точка области определения $\frac fg$, то $\dsl\lim{x\to x_0}{} \frac {f(x)}{g(x)} = \dsl\lim{x\to x_0}{} \frac{\ti{f}(x)}{\ti{g(x)}}$\\
В обоих утверждениях пределы одновременно существуют и равны или не существуют.

\Zam1. Если $g(x) \not\equiv 0$ в $\dot{V_a}\cap D$, то и $\ti{g}(x) \not\equiv 0$ в $\dot{\ti{V}}_{x_0}\cap D$ и обратно. Поэтому точка $x_0$ одновременно является или не является предельной для областей определения $\frac fg$ и $\frac{\ti{f}}{\ti{g}}$.

\D По определению эквивалентной функции, существуют окрестности $U_{x_0}, V_{x_0}$ и функции $\varphi, \psi$, стремящиеся к 1 при $x \to x_0$, такие, что \FF{f = \varphi \ti{f}$ на $\dot{U}_{x_0}\cap D, \quad g = \psi \ti{g} \textit{на} \dot{V}_{x_0}\cap D.} Тогда на множестве $\dot{W}_{x_0} \cap D$, где $W_{x_0} = U_{x_0} \cap V_{x_0}$, верны оба равенства. Значит, на $\dot{W}_{x_0}\cap D$ \F{$fg = (\varphi\psi)(\ti{f}\ti{g})$}. Следовательно если $\dsl\lim{x\to x_0}{} \ti{f}\ti{g}$ существует и равно $A$, то по теореме о пределе произведения $\dsl\lim{x\to x_0}{} f(x)g(x)$ существует и равен $A$. Верно и обратное. Аналогично доказыается для предела частного (может понадобиться сузить  окрестность, чтобы $\varphi, \psi$ не обращались в ней в нуль).

Пусть $f \sim g, f\sim h$. Если $f - h = o(f - g)$, то говорят, что асимптотически равенство $f\sim h$ точнее чем $f\sim g$.

(167)

\Op Пусть $x_0 \in \R$, функция $f$ задана по крайненй мере на $\ang a{x_0}$ или $\ang{x_0}b$ и действует в $\R$ прямая $x = x_0$ называется вертикальной асимптотой функции $f$, если $f(x_0+)$ или $f(x_0-)$ равны $+\infty$ или $-\infty$.

\Op Пусть $\langle a, +\infty) \subset D \subset \R,\s f: D\to \R,\s \alpha, \beta \in \R$. Прямая $y = \alpha x + \beta$ называется наклонной асимптотой функции $f$ при $x \to + \infty$, если \F{$f(x) = \alpha x + \beta + o(1),\quad x\to +\infty$}
Аналогично определяется наклонная асимптота при $x \to -\infty$ функции заданной по крайней мере на $(-\infty, b)$.