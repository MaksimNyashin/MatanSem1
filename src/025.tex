(86)

\T \q О стягивающихся отрезках. Пусть $\se {[a_n, b_n]}{n = 1}\infty$~--- последовательность стягивающихся отрезков. Тогда пересечение всех этих отрезков состоит из одной точки, то есть
\F{$\exists c \in \R: \dsl \bigcap{n = 1}\infty [a_n, b_n] = \{c\}$}
при этом $a_n \to c,\ b_n \to c$

\D То, что пересечение не пусто, следует из аксиомы о вложенных отрезках. Пусть $c, d \in \dsl \bigcap{n = 1}\infty [a_n, b_n]$. Докажем, что $c = d$. Поскольку $a_n \le c \le b_n$ и $a_n \le d \le b_n$, имеем \F{$a_n - b_n \le c - d \le b_n - a_n$}. По теореме о предельном переходе в неравенстве, $0 \le c - d \le 0$, то есть $c = d$. Так как \F{$0 \le c - a_n \le b_n - a_n \quad 0 \le b_n - c \le b_n - a_n$} по теореме о сжатой последовательности $a_n \to c$ и $b_n \to c$.

\Op Пусть $E \subset \R,\ E \neq \varnothing,\ E$ ограничено сверху. Наименьшая из верхних границ множества $E$ называется точной верхней границей, или верхней гранью, или супремумом множества $E$ и обозначается $\sup E$.

\T \qСуществование верхней грани. Всякое непустое ограниченное сверху множетво имеет верхнюю грань.

\D По условию, существует точка $x_0 \in E$ и $M$~--- верхняя граница $E$, $x_0 \le M$. Обозначим $[a_1, b_1] = [x_0, M]$. Отрезок $[a_1, b_1]$ Удовлетворяет двум условиям:\\
1. $[a_1, b_1] \cap E \neq \varnothing$\\
2. $(b_!, +\infty) \cap E = \varnothing$\\
Рассмотрим середину отрезка $[a_1, b_1]$~--- точку $\frac{a_1 + b_1}2$. Положим $[a_2, b_2] = [a_1, \frac{a_1 + b_1}2]$, если $(\frac{a_1 + b_1}2, b_1] \cap E = \varnothing$, и $[a_2, b_2] = [\frac{a_1 + b_1}2, b_1]$, если $(\frac{a_1 + b_1}2, b_1]\cap E \neq \varnothing$. В обоих случаях условия сохраняются.\\
Будем продолжать этот процесс неограниченно, при этом оставляя выполняться эти условия. При этом, $b_n - a_n = \frac{b_1 - a_1}{2^{n - 1}} \to 0$. По теореме о стягивающихся отрезках существует единственная точка $c$, принадлежащая одновременно всем отрезкам, причём $a_n \to c,\ b_n\to c$\\
Проверим, что $c = \sup E$. Если $x \in E, n \in N$, то $x\le b$ по свойству 2. По теореме о предельном переход в неравенстве $x \le c$. То есть $c$~-- верхняя граница $E$. Возьмём $\eps > 0$ и дркажем, что $c - \eps$ не является верхней гранцей $E$. Так как $a_n \to c$, найдётся номмер $N$, для которого $a_N > c - \eps$ (по определению предела все элементы с некоторого номера удовлетворяют этому неравенству). По свойству 1, найдётся точка $x \in [a_N, b_N] \cap E$, а тогда $x > x - \eps$.

\Zam2. Если множество $E$ не ограничено сверху считают, что $\sup E = +\infty$, при этом определение супремума в $\ol{\R}$ существует у любого непустого множества. Ограниченночть $Y$ сверху равносильна неравенству $\sup E < +\infty$

\Zam3. Если $D \subset E\subset \R,\ D \neq \varnothing$, то $\sup D \le \sup E$

\D Если $sup E = +\infty$, то неравенство тривиально. Пусть $\sup E < +\infty$. Если $x \in D$, то $x \in E$ и, слеедовательно, $x \le \sup E$, то есть $\sup E$ какая-то верхняя граница $D$. Но $\sup D$~--- наименьшая верхняя граница $D$, поэтому $\sup D \le \sup E$.

\Zam4. Если $E, F \subset \R,\ E, F \neq \varnothing, t > 0$, то\\
$\sup(E + F) = \sup E +\sup F$\\
$\sup(tE) = t\sup E$\\
$\sup(-E) = -\inf E$