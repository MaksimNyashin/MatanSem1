(129)

\Op Пусть $X, Y$~--- метрические пространства $f: X\to Y$. Отображение $f$ называется равномерно непрерывным на $X$, если \F{$\forall \eps > 0\s\exists \delta > 0\s\forall\ol{x}, \ool{x} \in X: p_X(\ol{x}, \ool{x}) < \delta\s p_Y(f(\ol{x}), f(\ool{x})) < \eps$.} Ясно, что всякое раномерно непрерывное отображение непрерывно.

\T \q Кантор. Непрерывное на компакте отображение равномерно непрерывно.

\D Пусть $X$~--- компактно, $f \in C(X \to Y)$. Предположим, что $f$ не являетс яравномерно непрерывным. Тогда существует такое $\eps^* > 0$, что при каждом $n \in \N$ для числа $\delta = \frac 1n$ найдутся точки $\ol x_n, \ool x_n \in X$: \F{$\rho_X(\ol x_n, \ool x_n) < \frac 1n,\quad \rho_Y(\ol y_n, \ool y_n) \ge \eps^*$,} где $\ol y_n = f(\ol x_n),\s \ool y_n = f(\ool x_n)$.\\
Пользуясь секвенциальной компактностью $X$, выделим из последовательности $\{\ol x_n\}$ точек $X$ подпоследовательность $\{\ol x_{n_k}\}$, имеющую предел в $X$: $\ol x_{n_k} \to c \in X$. Тогда и $\ool x_{n_k} \to c$, так как \F{$\rho_X(\ool x_{n_k}, c) \le \rho_X(\ool x_{n_k}, \ol x_{n_k}) + \rho_X(\ol x_{n_k}, c) < \frac 1{n_k} + \rho_X(\ol x_{n_k}, c) \to 0$}. По непрерывности $f$ в точке $c$ \F{$\ol y_{n_k} \to f(c),\quad \ool y_{n_k} \to f(c)$.} Следовательно, $\rho_Y(\ol y_{n_k}, \ool y_{n, k}) \to 0$ и, начиная с некоторого номера $\rho_Y(\ol y_{n_k}, \ool y_{n, k}) < \eps^*$, что противоречит построению.