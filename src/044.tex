Билет 44: Сохранение промежутка (с леммой о характере промежутков). Сохранение отрезка
(131)

\L \q Характеристика промежутков. Пусть $E \subset \R$. Тогда следующие утверждения равносильны.\\
1. $E$~--- промежуток (возможно вырожденный).\\
2. Для любых $x, y$, принадлежащих $E\s (x < y),\s [x, y] \subset E$

\D Второе утверждение следует из первого тривиально. Докажем обратный переход. Пусть $e \neq \varnothing$. Обозначим $m = \inf R,\s M = \sup E$. Ясно, что $E \subset [m, M]$. Докажем, что $(m, M) \subset E$. Пусть $m < z < M$. Тогда по определению граней существуют точки $x, y \in E: x < z < y$. По условию $z \in E$

\T \q О сохранении промежутка. Непрерывный образ промежутка~--- промежуток.

\D Пусть $f \in C\ang ab$, \F{$m = \dsl\inf{x \in \ang ab}{} f(x),\quad M = \dsl\sup{x \in \ang ab}{} f(x)\quad (m, M \in \R)$.} По теореме Больцано-Коши о промеуточных значениях непрерывной функции, множество $E = f(\ang ab)$ выпукло, а, по лемме, $E$~--- промежуток, то есть $f(\ang a,b) = \ang mM$.

\Zam2. Промежуток $\ang mM$ может быть другого типа, нежели $\ang ab$

\S1. \q О сохранении отрезка. Непрерывный образ отрезка~--- отрезок. 

\D Действительно, множество $f([a, b])$~--- промежуток, а, по теореме Вейерштрасса, имеет наибольший и наименьший элемент.

\Zam3. Наибольшее и наименьшее значения не обязательно достигаются на концах отрезка.