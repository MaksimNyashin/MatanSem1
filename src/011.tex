(51)

\T \q Предельныйпереход в неравенстве. Пусть ${x_n}$, ${y_n}$~--- вещественные последовательности $x_n \le y_n$ при всех натуральных $n$, $a, b \in \R$ $x_n \to a,\ y_n \to b$. Тогда $a \le b$.

\D Предположим противное: пусть $a > b$. Тогда $\varepsilon = \frac{a - b}{2} > 0$. По определению предела найдутся такие номера $N_1, N_2$, что $a = \varepsilon < x_n$ для всех $n > N_1$, а $y_n < b + \varepsilon$ для всех $n > N_2$. Значит если $n > \max\{N_1, N_2\}$, то\\
$y_n < b + \varepsilon = \frac{a + b}{2} = a - \varepsilon < x_n$,\\
что противоречит условию.

\Zam1. $x_n = -\frac{1}{n}$, $\frac{1}{n}$ показывает, что из $x_n < y_n$ не следует $\lim x_n < \lim y_n$.

\S1. 1. Если $x_n \le b$ при всех $n \in \N$ и существует $\lim x_N$, то $\lim x_n \le b$.\\
2. Если $x_n \ge a$ при всех $n \in \N$ и существует $\lim x_N$, то $\lim x_n \ge a$.\\
3. Если $x_n \in [a, b]$ при всех $n \in \N$ и существует $\lim x_N$, то $\lim x_n \in [a, b]$.

\Zam2. Свойство отрезка из утверждения 3 (неверное для других типов отрезков) называется замкнутостью.

\T \q О сжатой последовательности (О двух милиционерах). Пусть $\{x_n\},\ \{y_n\}, \{z_n\}$~--- вещественные последовательности, $x_n \le y_n \le z_n$, при всех $n \in \N,\ a \in \R$, $\lim x_n = \lim z_n = a$. Тогда предел $\{y_n\}$ существует и равен $a$.

\D Возьмём $\varepsilon > 0$. По определению, предела найдутся такие номера $N_1, N_2$, что $a - \varepsilon < x_n$ для всех $n > N_1$, а $z_n < a + \varepsilon$ для всех $n > N_2$. Пусть $N = \max\{N_1, N_2\}$. Тогда при $n > N$\\
$a - \varepsilon < x_n \le y_n \le z_n < a + \varepsilon$.\\
В силу произвольности $\varepsilon$ предел $\{y_n\}$ существует и равен $a$.

\Zam2. Отметим, что если $|y_n| \le z_n$ при всех $n \in \N$ и $z_n \to 0$, то $y_n\to 0$.

\Zam3. В обеих теорема достаточно  выполнения неравенств для всех номеров, начиная с некоторого.
