(79)

\L Пусть $\se{[a^{(n)}, b^{(n)}]}{n = 1}\infty$~--- последовательность вложенных параллелепипедов в $\R^m$, то есть\\
$a^{(n)}_k\le a^{(n)}_{k + 1}\le b^{(n)}_{k + 1} \le b^{(n)}_k$ при всех $n \in N$ и $j \in [1: m]$\\
Тогда $\dsl\bigcap{n = 1}\infty [a^{(n)}, b{(n)}] \neq \varnothing$.

\D При каждом $k \in [1: m]$ имеем последовательность вложенных отрезков $\se{[a^{(n)}_k, b^{(n)}_k]}{n = 1}\infty$. По аксиоме Кантора, найдётся точка $x^*_k$, принадлежащая всем отрезкам. Тогда $m$-мерная точка $x^* = (x^*_1, \cdots\, x^*_m)$ принадлежит всем параллелепипедам $[a^{(n)}, b^{(n)}]$.

\L Замкнутый куб в $\R^m$ компактен.

\D Пусть $I = [a, b]$~--- куб в $\R^m,\ \delta$~--- его диагональ. Допустим, тчо $I$ не компатен. Обозначим через $\{G_\alpha\}$ такое открытое покрытие $I$, из которого нельзя изввлечь конечное подпокрытие. Разделив каждый отрезок $[a_k, b_k]$ пополам, разобъём куб $I$ на $2^m$ кубов. Среди них найдётся тот, который не покрывается никаким конечным наборо множеств из семества $\{G_\alpha\}$ (так как иначе куб $I$ покрывается конечным набором множеств из этого семейства). Обозначим этот куб $I_1$. Продолжая процесс деления и далее получим последовательность вложенных замкнутых кубов $\se{I_n}{n = 1}\infty$ со следующими свойствами:\\
1) $I\supset I_1\supset I_2 \supset\cdots$.\\
2) $I_n$ не покрывается никаким конечным набором множеств из семейства $G_\alpha$\\
3) Если $x, y \in I_n$, то $|x - y| < \frac{\delta}{2^n}$\\
По лемме существует точка $x^*$, принадлежащая одновременно всем кубам. Следовательно, $x^* \subset I$. Тогда $x^*$ принадлежит некотороиу элементу покрытия $G_{\alpha^*}$. Поскольку $G_{\alpha^*}$ открыто, найдётся такое $r > 0$, что $B(x^*, r) \subset G_\alpha$. Так как $\frac{\delta}{2^n}\xra[n\to\infty]{} 0$, найдётся такое $n$, что $\frac{\delta}{2^n} < r$, то есть $I_n \subset B(x*, r)$. Значит куб $I_n$ покрывается одним множеством, что противоречит свойству 2.