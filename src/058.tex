(188)

\T \q Ферма. Пусть $f: \ang ab \to \R,\s x_0 \in (a, b),\s f(x_0) = \dsl\min{x \in (a, b)}{} f(x)$ или $f(x_0) = \dsl\max{x \in (a, b)}{} f(x),\s f$ дифференцируема в точке $x_0$. Тогда $f'(x_0) = 0$.

\D Пусть для определённости значение в точке $x_0$ наибольшее, то есть $f(x) \le f(x_0)$ при всех $x \in \ang ab$. Тогда $\frac{f(x) - f(x_0)}{x - x_0} \le 0$ при всех $x \in (x_0, b\rangle$. По теореме о предельном переходе в неравенстве \F{$f'(x_0) = f'_+(x_0) = \li{x}{x_0+} \frac{f(x) - f(x_0)}{x - x_0} \le 0$.}  Аналогично $\frac{f(x) - f(x_0)}{x - x_0} \ge 0$ при всех $x \in \langle a, x_0)$, и поэтому \F{$f'(x_0) = f'_-(x_0) = \li{x}{x_0-}\frac{f(x) - f(x_0)}{x - x_0} \ge 0$.} Следовательно $f'(x) = 0$.

\Zam1. Геометрический смысл теоремы Ферма: если во внутренней точке максимума (минимума) существует касательная, то эта касательная горизонтальна.

\Zam2. $f(x) = |x|$~--- пример функции не имеющей касательной в тоске минимума.

\Zam3. Условие, что $x_0$~--- внутренняя точка существенно: $f(x) = x^2$ на отрезке $[0, 1]$ принимает наибольшее значение в точке 1, при этом $f'(1) = 2$.