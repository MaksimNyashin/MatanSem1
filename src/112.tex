(54)\\
Пусть $X$~--- векторное пространство над $\R$ или $\C$. Нормой в $X$ называют функцию $p: X\to\R_+$, удовлетворяющаяя следующим условиям.\\
\q1.\q Положительная определённость.\\
$p(x) = 0 \lra x = \theta$\\
\q2.\q Положительная однородность.\\
$p(\lambda x) = |\lambda|p(x)$\\
\q3.\q неравенство треугольника (полуаддитивность).\\
$p(x + y) \le p(x) + p(y)$\\
Принято обозначать норму двойными палочками $p(x) = ||x||$.\\
Пара $(X, ||\cdot||)$ называется нормированным пространством.
Если функция $p: X\to\R_+$ удовлетворяет аксиомам 2 и 3, то она называется \textit{полунормой}.