(114)\\
Пусть $(X, \rho_X)$ и $(Y, \rho_Y)$~--- метрические пространства, $f: D\subset X \to Y$, $x_0 \in D$. Отображение $f$ называется непрерывным в точке $x_0$ если выполняется одно из следующих утвержений:\\
\q1. Предел отображения $f$ в точке $x_0$ существует и равен $f(x_0)$. (Применимо, если $x_0$~--- предельная точка D).\\
\q2. \q На $\varepsilon$-языке или по Коши.\\
$\forall \varepsilon > 0\ \exists \delta > 0\ \forall x \in D:\ \rho_X(x, x_0) < \delta \Ra \rho_Y(f(x), f(x_0)) < \varepsilon$\\
\q3. \q На языке окрестностей.\\
$\forall V_{f(x_0)}\ \exists V_{x_0}\ f(V_{x_0} \cap D) \subset V_{f(x_0)}$\\
\q4. \q На языке последовательностей или по Гейне.\\
$\forall \{x_n\}:\ x_n \in D, x_n \to x_0\ f(x_n) \to f(x_0)$\\
\q5. Бесконечно малому приращению аргумента соответствует бесконечно малое приращение отображения. $\Delta y \xra[\Delta x \to \theta_X]{} \theta_Y$

Отображение называется непрерывным на множестве $D$, если оно непрерывно в каждой точке множества $D$.\\
Множество отображений $f: D\subset X \to Y$ непрерывных на множестве $D$, обозначают $C(D\subset X\to Y)$ или $C(D\to Y)$