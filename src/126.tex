(130, 133)\\
\qТеорема Больцано -- Коши о промежуточном значении.
Пусть функция $f$ непрерывна на $[a, b]$. Тогда для любого числа $C$, лежащего между $f(a)$ bи$f(b)$, найдётся такое $c \in (a, b)$, что $f(c) = C$\\
\qТеорема Больцано -- Коши о непрерывных отображениях.\\
Пусть $X, Y$~--- метрические пространства, $X$ линейно связно, $f \in C(X\to Y)$. Тогда $f(x)$ линейно связно.\\
Другими словами непрерывный образ линейно связного образа линейно связен.

\q Линейная связность.\\
$Y$~--- метрическое пространство, $E \subset Y$. Множество $E$ называется линейно связным, если любые две его точки можно соединить путём в $E$: $\forall A, B \in E\ \exists \gamma \in C([a, b] \subset \R \to E):\ \gamma(a) = A, \gamma = A, \gamma(b) = b$
