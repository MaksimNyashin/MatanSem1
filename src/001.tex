(6-12)\\
Множество состоит из элементов\\
Если каждый элемент множества $X$ принадлежит множеству $Y$, то говорят $X$ подмножество $Y$  $X \subset Y$.\\
$X = Y \lra X \subset Y \wedge Y \subset X$\\
Пустое множество $\varnothing$~--- множество в котором нет элементов

Пусть $\{X_\alpha\}_{\alpha \in A}$~--- семейство множеств. Объединением семейства $\{X_\alpha\}_{\alpha \in A}$ называется множество всех элементов, которые принаджежат хотя бы одному из множеств$X_n$:\\
$\ds\bigcup\limits_{\alpha \in A} X_\alpha = \{x:\ \exists\alpha \in A\ x \in X_\alpha\}$\\
$(X\cup Y) \cup Z = X \cup (Y\cup Z),\quad X\cup Y = Y \cup X,\quad X \cup X = X\cup \varnothing = X$

Пусть $\{X_\alpha\}_{\alpha \in A}$~--- семейство множеств. Пересечением семейства $\{X_\alpha\}_{\alpha \in A}$ называется множество всеъ элементов, которые принадлежат каждому из множеств $X_\alpha$\\
$\ds\bigcap\limits{\alpha \in A} X_\alpha = \{x:\ \forall \alpha \in A\ x \in X_\alpha\}$\\
$(X\cap Y) \cap Z = X \cap (Y\cap Z),\quad X\cap Y = Y \cap X,\quad X\cup \varnothing = \varnothing$

Разностью множеств $X$ и $Y$ называется множество всех элементов, которые принадлежат $X$, но не принадлежат $Y$.\\
$X \bsl Y = \{x: x\in X, x \notin Y\}$\\
Определение не предполагает, что $Y \subset X$. Если же $Y \subset X$, то разность $X\bsl Y$ называют дополнением множества $X$ до множества $Y$ и обозначают $CX, \ol{X}, X^c$\\
$(X^c)^c = X,\quad X \cup X^c = U,\quad X \cap X^c = \varnothing$

Декартовым или пряиыи произведением множеств $X$ и $Y$ называется множество всех упорядоченных пар, таких что первый элемент принадлежит $X$, а второй~--- $Y$\\
$X \times Y = \{(x, y):\ x\in X\ y \in Y$