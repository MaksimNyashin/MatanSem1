(95)

\Op Точка $a \in \ol{\R}$ называется частичным пределом последовательности $\{x_n\}$, если существует подпоследовательность $\{x_{n_k}\}$, стремящаяся к $a$.

\Op Пусть последовательность $\{x_n\}$ ограничена сверху. Величина
\F{$\ol{\dsl\lim{n\to\infty}{}}x_n = \dsl\lim{n\to\infty}{} \dsl\sup{k \ge n}{} x_k$} называется верхним пределом последовательности $\{x_n\}$

Пусть последовательность $\{x_n\}$ ограничена снизу. Величина
\F{$\ul{\dsl\lim{n\to\infty}{}}x_n = \dsl\lim{n\to\infty}{} \dsl\inf{k \ge n}{} x_k$} называется нижним пределом последовательности $\{x_n\}$

\Zam1. Последовательности $z_n - \dsl\sup{k \ge n}{} x_k$ и $y_n = \dsl\inf{k \ge n}{} x_k$ иногда называют верхней и нижней огибающими последовательности $\{x_n\}$.\\
Если $\{x_n\}$не ограничена сверху, то $z_n = +\infty$ при всех $n$ и поэтому полагают $\ol{\lim} x_n = +\infty$. Аналогично, если $\{x_n\}$ не ограничена снизу полагают $\ul{\lim} x_n = -\infty$.

\Zam2. Верхний и нижний пределы вещественной последовательности $\{x_n\}$ существуют в $\ol{\R}$, причем $\ul{\lim} x_n \le \ol{\lim}x_n$

\D Пусть $\{x_n\}$ ограничена и сверху и снизу. Так как при переходе к подмножеству супремум не увеличивается, а инфинум не уменьшается $\{y_n\}$ возрастает, а $\{z_n\}$ убывает. При всех $n$ $y_q \le y_n \le z_n \le z_1$. По теореме о преле монотонной последовательности $\{y_n\},\ \{z_n\}$ сходятся, то есть существуют конечные пределы $\lim b$. Если хотя бы одна последовательность неограничена, то очевидно.

\T \q О верхнем и нижнем пределе последовательности. Пусть $\{x_n\}$~--- вещественная последовательность, тогда справедливы следующий утверждения\\
1. Верхний предел~--- наибольший, а нижний~--- наименьший из частичных пределов $\{x_n\}$\\
2. Предел $\{x_n\}$ в $\ol{\R}$ существует тогда и только тогда, когда $\ul{\lim} x_n = \ol{\lim} x_n$, при этом $\lim x_n$ равен этому значению.

\D \RNumb{1}. Пусть $\{x_n\}$ ограничена и сверху и снизу\\
1. Обозначим $b = \ol{\lim}x_n$. Докажем, что $b$~--- частичный предел $\{x_n\}$, для чего построим полпоследовательность последовательности $\{x_n\}$, стремящуюся к $b$. При всех $n$ будет $z_n \ge b$. Поскольку \F{$z_1 = \sup\{x_1, x_2,\cdots\} > b -1$} найдётся номер $n_1$, для которого $x_{n_1} > b - 1$. Поскольку \F{$z_{n_1 + 1} = \sup\{x_{n_1 + 1}, x_{n_1 + 2}, \cdots\} > b - \frac 12$} найдется номер $n_2 > n_1$, для которого $x_{n_2} > b - \frac 12$. Этот процесс может продолжаться неограничено. Таким образом, построена подпоследовательность $\{x_n\}$, члены которой удовлетворяют неравенству \F{$b - \frac 1k < x_{n_k} \le z_{n_k}$} Подпоследоватеьность $\{z_{n_k}\}$ последовательности $\{x_n\}$, стремящейся к $b$ тоже стремится к $b$. По теореме о сжатой последоавтельности и $x_{n_k} \to b$.\\
Если $\{x_{m_l}\}$~--- подпоследовательность $\{x_n\}$, $\{x_{m_l}\} \to \beta$, то, сделав предельный переход в неравенстве $x_{m_l} \le z_{m_l}$, получим, что $\beta \le b$, то есть $b$~--- наибольший частичный предел $\{x_n\}$.\\
Аналогично для $\ul{\lim} x_n$~--- наименьшего частичного предела.
2. По определению $y_n$ и $z_n$, при всех $n$ будет \F{$y_n \le x_n \le z_n$} Если $\ul{\lim} x_n = \ol{\lim} x_n$, то по теореме о сжатой последовательности существует $\lim x_n$ и $\lim x_n = \ul{\lim} x_n = \ol{\lim} x_n$

\RNumb{2}. Пусть $\{x_n\}$ ограничена сверху, но не снизу. Тогда, по определению, $\ul{\lim} x_n = -\infty$. По замечанию 2 к принципу выбора, из $\{x_n\}$ можно выбрать подпоследовательность, стремящуюся к $-\infty$, то есть $-\infty$~--- частный предел $\{x_n\}$. Разумеется $-\infty$ меньше любого другого частичного предела, если они есть. То что $\ol{\lim} x_n$~--- наибольший частичный предел, доказано в пункте \RNumb{1}. Если $\ul{\lim} x_n = \ol{\lim} x_n$, то есть $z_n \to -\infty$, то и $x_n \to -\infty$, так как $x_n \le z_n$. Обратно, если $\lim x_n = -\infty$, то и $\ol{\lim} x_n = -\infty$, как частичный период.

\RNumb{3}. Если $x_n$ не ограничена ни сверху, ни снизу, то превое утверждение очевидно, а второе не реализуется