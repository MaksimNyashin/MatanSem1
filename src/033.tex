(110)

\T \q Критерий Больцано-Коши для отображений. Пусть $X, Y$~--- метрические пространства, $Y$ полно, $f: D\subset X \to Y,\ a$~--- предельная точка $D$. Тогда существование в точке $a$ предела $f$, принадлежащего $Y$, равносильно следующему утверждению.
\F{$\forall \eps > 0\s\exists V_a\s\forall \ol{x}, \ool{x} \in \dot{V_a}\cap D\s\rho_Y(f(\ol{x}), f(\ool{x})) < \eps$.}

\D 1. Пусть $\dsl\lim{x\to a}{}f(x) = A \in Y$. Возьмём $\eps > 0$. По определению предела, найдётся такая окрестность $V_a$ точки $a$, что $p_Y(f(X), A) < \frac \eps2$ Для всех $x \in \dot{V_a} \cap D$. Тогда если $\ol{x}, \ool{x} \in \dot{V_a}\cap D$, то
\F{$\rho_Y(f(\ol{x}), f(\ool{x})) \le \rho_Y(f(\ol{x}), A) + \rho_Y(A, f(\ool{x})) < \eps$} В силу произвольности $\eps$ формула выполнена.

2. Пусть выполнена формула. Докажем существование предела $f$ в точке $a$ на языке последовательностей. \Geine, и докажем, что существует $\lim f(x_n) \in Y$. Возьмём $\eps > 0$ и подберём окрестность $V_a$ из формулы. По орпеделению предела $\{x_n\}$ найдётся такой номер $N$, что $x_n \in V_a$ для всех $n > N$; тогда $x_n \in \dot{V_a}\cap D$ для тех же $n$. По выбору $V_a$, для всех $n, l > N$ будет $\rho_y(f(x_n), f(x_1)) < \eps$. таким образом последовательность сходится в себе, а, значит, в силу полноты $Y$, сходится к некоторому пределу, принадлежащему $Y$. Тогда, в силу замечания к определению предела существует $\dsl\lim{x\to a}{} f(X) \in Y$.

\Zam1. Полнота $Y$ использовалась только во второй части доказательства.