(103-106)

\T \q Единственность предела отображения. Отображение в данной точке не может иметь более одного предела: если $X, Y$~--- метрические пространства, $f: D \subset X \to Y, a$~--- предельная точка $D,\ A, B \in Y,\ f(x)\t A,\ f(x)\t B$, то $A = B$.

\D Возьмём последовательность $\{x_n\}: x_n \in D,\ x_n \neq a,\ x_n \to a$. По определению Гейне $f(x_n)\to A$ и $f(x_n) \to B$. По единственности предела последовательности, $A = B$.

\Zam1. Если $Y = \R$, то, как и для последовательностей, а теореме можно считать, что $A, B \in \ol{\R}$.

\T. \q Локальная ограниченность отображения имеющего предел. Пусть $X, Y$~--- метрические пространства, $f: D\subset X\to Y$, $a$~--- предельная точка $D,\ a\in Y,\ f(x)\t A$. Тогда существует такая окрестность $V_a$ точки $a$, что $f$ ограниченв в $V_a\cap D$ (то есть $f(V_a \cap D)$ содержится в некотором шаре в $Y$).

\D Возьмём окрестность $V_A = B(A, 1)$. По определению предела на языке окрестностей, найдётся такая окрестность $V_a$ точки $a$, что $f(V_a \cap D) \subset B(A, 1)$. Если $a \notin D$, то на этом доказательство заканчивается. Иначе
\F{$f(V_a \cap D) \subset B(A, R)$, где $R = \max\{1, \rho_Y(f(a), A)\}$}

\Zam2. Отображение, имеющее предел в точке не обязано быть ограниченным. Например функция $f(x) = x$. Поэтому в названии теоремы присутствуетс лово "локальная".

\Zam3. Если $X$~--- метрическое пространство, $Y$~--- нормированное пространство с нулём $\theta,\ D \subset X,\ a$~--- предельная точка $D,\ g: D\to Y,\ \dsl\lim{x\to a}{}g(x) = B,\ B \neq \theta$, то существует такая окрестность $V_a$, что $g(x) \neq \theta$ для всех $x \in \dot{V_a}\cap D$.

\D Пусть не так: тогда для любого $n \in \N$ существует точка $x_n \in V_a(\frac1n)\cap D$, для которогй $g(x_n) = \theta$. Построенная последовательность $\{x_n\}$ стремиться к $a$. По определению предела $g(x_n) \to B$, откуда $B = \theta$, что протмворечит условию.

\T \q Арифмметические действия над отображениями, имеющими предел. Пусть $X$~--- метрическое пространство, $Y$~--- нормированное пространство, $D \subset X,\ f, g: D \to Y,\ \lambda: D\to \R(\C),\ a$~--- предельная точка $D,\ A, B \in Y,\ \lambda_0 \in \R(\C),\ f(x) \t A,\ g(x)\t B,\ \lambda(x)\t \lambda_0$. Тогда \\
1. $f(x) + g(x) \t A + B$;\\
2. $\lambda(x)f(x) \t \lambda_0 A$;\\
3. $f(x) - g(X) \to A - B$;\\
4. $\|f(x)\| \t \|A\|$

\T \q Арифметические действия над функциями, имеющими предел. Пусть $X$~--- метрическое пространство, $f, g: D \subset X \to \R(\C),\ a$~--- предельная точка $D,\ a, b \in\R,\ f(x) \t A,\ g(x)\t B$. Тогда\\
1. $f(x) + g(x) \t A + B$;\\
2. $f(x)g(x) \t A B$;\\
3. $f(x) - g(x) \t A - B$;\\
4. $|f(x)| \t |A|$;\\
5. Если $B \neq 0$, то $\frac{f(x)}{g(x)} \t \frac AB$.

\D С помощью определения на языке последовательностей эти теоремы сводятся к аналогичным про последовательности. Докажем, например, первое утверждение. Возьмём последовательность $\{x_n\}$ со свойствами $x_n \in D,\ x_n \neq a, x_n \to a$. Тогда по определению Гейне $f(x_n) \to A, g(x_n) \to B$. По теореме о пределе суммы для последовательностей $f(x_n) + g(x_n) \to A + B$. В силу произвольности последовательности $\{x_n\}$ это и означает, что $f(X) + g(x) \t A + B$. При доказательстве утверждения о пределе частного следует учесть, что, по замечанию 3, существует такая окрестность $V_a$, что частное $\frac fg$ определено по крайней мере на множестве $V_a \cap D$.

\Zam 4. Теорема про функции верна и для бесконечных пределов, за исключением случаев неопределённости вида $\infty - \infty, 0\cdot\infty, \frac 00, \frac\infty\infty$.

\Zam5. Определение бесконечно малой и бесконечно большой переносятся на функции (и отображения со значениями в нормированно пространстве). Так, функция, стремящаяся к нулю в точке $a$ называется бесконечно малой в точке $a$. Утверждение о том, что  произведение бесконечно малой функции на ограниченнуб есть бесконечно малая, и о связи иежду бесконечно большими и бесконечно малыми сохраняют свою силу.