(219)

\T \q Доказательство неравенств с помощью производной. Пусть функции $a, g$ непрерывны на $[a, b\rangle$ и дифференцируема на $(a, b),\s f(a) \le g(x)$ и $f'(x) \le g'(x)$ для всех $x \in (a, b)$. Тогда $f(x) < g(x)$ для всех $x \in [a, b\rangle$.

\D Положим $h = g - f$. Тогда $h' = g' - f' \ge 0$ на $(a, b)$. По теореме функция $h$ возрастает на $[a, b\rangle$. следовательно для всех $x \in [a, b\rangle$ \FF{h(x) \ge h(a) = g(a) - f(a) \ge 0,} то есть $g(x) \ge f(x)$.

\Zam1. Аналогичное утвердение справедливо вместе сдоказательством и в случае, когда исходно значение функций сравниваются на правом конце.\\
Пусть функции $f, g$ непрерывны на $\langle a, b]$  и дифференцируема на $(a, b),\s f(b) \ge g(b)$ и $f'(x) \ge g'(x)$ для всех $x \in  (a, b)$. Тогда $f(x) \ge g(x)$ для всех $x \in \langle a, b]$.

\Zam2. Если в условиях теоремы будет $f'(x) < g'(x)$ для всех $x \in (a, b)$, то $f(x) < g(x)$ для всех $x \in (a, b)$.

\Pr1. Докажем, что $\cos x > 1 - \frac{x^2}2$ при всех $x\neq 0$.\\
В силу чётности обеих сторон, достаточно локазать при $x > 0$. Положим $f(x) = 1 - \frac{x^2}2,\s g(x) = \cos x$, тогда $f(0) = g(0) = 1$ и \FF{g'(x) = -\sin(x) > -x = f'(x) \textit{ при всех } x > 0.}

\Pr2. Докажем, что $\sin x > x = \frac{x^3}6$ при всех $x > 0$.\\
Положим $f(x) = x - \frac{x^3}6,\s g(x) = \sin x$. Тогда $f(0) = g(0) = 0$. По предыдущему примеру, \FF{g'(x) = \cos x > 1- \frac{x^2}2 = f'(x)\textit{ при всех }x > 0.}

\Pr 3. Докажем, что $\sin x > \frac 2\pi x$ при всех $x \in (0, \frac\pi2)$.\\
Положим $f(x) = \frac{\sin x}{x}$ при $x \neq 0,\s f(0) = 1$ \FF{f'(x) = \frac{x\cos x - \sin x}{x^2} = \frac{\cos x(x - \tg x)}{x^2} < 0.} Так как $x < \tg x$, а $\cos x$ и $x^2$ неотрицательны на заданном промежутке. Тогда $f$ строго убывает на промежутке, то есть $f(x) > f(\frac\pi2)$.