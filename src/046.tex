(134)

\T \q О разрывах и непрерывностях монотонной функции. Пусть $f: \ang ab \to \R,\s f$ монотонна. Тогда справедливы следующие утверждения.\\
1. $f$ не может иметь разрывов второго рода.\\
2. Непрерывность $f$ равносильна тому, что её множество значений~--- промежуток.

\D Пусть для определения $f$ возрастает.

1. Пусть $x_0 \in (a, b),\s x_1 \in \langle a, x_0).$ Тогда $f(x_1) \le f(x) \le f(x_0)$ для всех $x \in (x_1, x_0)$, пожтому $f$ возрастает и ограничена сверху на $\langle a, x_0)$. По теореме о пределе монотонной функции, существует конечный предел $f(x_0-)$, причем, по теореме о предельно переходе в неравенстве, $f(x_1) \le f(x_0-) \le f(x_0)$. Аналогично доказывается, что для любой точки $x_0 \in \langle a, b)$ существует конечный предел $f(x_0+)$, причём $f(x_0) \le f(x_0+) \le f(x_2)$ для всех $x_2 \in (x_0, b\rangle$.

2. Ввиду следствия о сохранении промежутка остается доказать достаточность. Пусть $f(\ang ab)$~--- промежуток. Докажем непрерывность $f$ слева в любой точке $x_0 \in \ang ab$ от противного. Пусть $f(x_0-) < f(x_0)$ (существование конечного левостороннего предела уже доказано). Возьмём $y \in (f(x_0-), f(x_0))$. Тогда если $a < x_1 < x_0$, то $y \in [f(x_1), f(x_0)]$. Следовательно, $y \in f(\ang ab)$, то есть $y$~--- значение функции. С другой стороны, для всех $x \in \langle a, x_0)$ будет $f(X) \le f(x_0-)< y$, а для всех $x \in [x_0, b\rangle$ будет $f(x) \ge f(x_0) > y$, то есть функция не принимает значение $y$. Полкченное противоречме доказывает, что $f(x_0-) = f(x_0)$. Аналогично $f$ непрерывна справа в любой точке $x_0\ in \langle a, b)$.