(40)\\
\T Отрезок $[0, 1]$ несчётен.

\D Допустим противное пусть отрезок $[0, 1]$ счётен, то есть все числа отрезка $[0, 1]$ можно разположить в виде последовательности [0, 1] = $\{x_1, x_2, x_3, \cdots\}$.\\
Разобъём отрезок $[0, 1]$ на три равных отрезка $[0, \frac{1}{3}]$, $[\frac{1}{3}, \frac{2}{3}]$, $[\frac{2}{3}, 1]$ и обозначим через $[a_1, b_1]$ тот из них, который не содержит точки $x_1$. Далее разобъём отрезок $[a_1, b_1]$ на три отрезка и обозначим через $[a_2, b_2]$ тот из них, который не содержит точки $x_2$. Этот процесс продолжим неограниченно. В результате мы построим последовательность вложенных отрезков $\{[a_n, b_n\}^\infty_{n = 1}$, причём $x_n\notin [a_n, b_n]$ при любом $n$. По аксиоме о вложенных отрезках существует точка $x^*$, принадлежащая всем отрезкам $[a_n, b_n]$. Но тогда $x^* = x_m$ при некотором $m$. По построению $x^* \notin [a_m, b_m]$, что противоречит принадлежности $x^*$ всем отрезкам.

\S2. Множества вещественныз чисел $\R$ и иррациональных чисел $\R \bsl \Q$ несчётны.

\Op Если множество эквивалентно отрезку $[0, 1]$, то говорят, что оно имеет мощность континуума.

\Zam1. Любой невырожденный отрезок имеет мощность континнума. Также как и любой промежуток, вся прямая $\R$ и всё пространство $\R^m\ m \in \N$.

\Zam2. Множество всех функций $f: [0, 1] \to \R$, более богато элементами, чем отрезок $[0, 1]$, то есть оно не равномощно отрезку, но имеет часть, равномощную отрезку.

\Zam3. Мощность~--- класс эквивалентности. Два множества попадают в один класс, если они эквивалентны.

\Zam4. Если из бесконечного множества удалить конесное число элемнтов, множество окажется бесконечным, то получится множество равномощное исходному. Поэтому любое бесконечное множество имеет равномощное подмножество, не совпадающее с исходным множеством.