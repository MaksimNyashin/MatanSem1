(220)\\
Пусть $f: D \subset \R \to \R_+$, $x_0 \in D$. Если существует такое $\delta > 0$, что:
\begin{itemize}
	\item для любого $x \in (x_0 - \delta, x_0 + \delta) \cap D$ выполняется неравенство $f(x) \le f(x_0)$, то $x_0$ называется точкой максимума функции $f$;
	\item для любого $x \in (x_0 - \delta, x_0 + \delta) \cap (D\bsl \{x_0\})$ выполняется неравенство $f(x) < f(x_0)$, то $x_0$ называется точкой строгого максимума функции $f$.
\end{itemize} 

Если противоположные неравенства, то $x_0$ соответсвенно точка минимума и точка строгого минимума.

Необходимое условие экстремума\\
Пусть $f: \langle a, b\rangle \to \R$, $x_0\in (a, b)$~--- точка экстремума $f$. $f$ дифференцируема в точке $x_0$. Тогда $f'(x_0) = 0$ (Лемма Ферма).