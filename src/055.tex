(180)

\T \q Производная композиции. Если функция $f: \ang ab \to \ang cd$ дифференцируема в точке $x \in \ang ab$, а функция $g: \ang cd \to \R$ дифференцируема в точке $f(x)$, то функция $g \circ f$ дифференцируема в точке $x$ и \F{$(g \circ f)'(x) = g'(f(x))\cdot f'(x)$.}

\D Обозначим $y = f(x)$. Воспользуемся определением 1 дифференцируемости и запишем \F{$f(x + h) = f(x) + f'(x)h + \alpha(h)h$,} \F{$g(y + k) = g(y) + g'(y)k + \beta(k)k$,} где функции $\alpha, \beta$ в нуле непрерывны и равны нулю. Подставляя во второе равенство $k = f'(x)h + \alpha(h)h = \varkappa(h)$, получаем
\F{$g(f(h + x)) = g(f(x)) + g'(f(x))(f'(x)h + \alpha(h)h) + \beta(\varkappa(h))\varkappa(h) = g(f(x)) + g'(f(x))f'(x)h + \gamma(h)h$,} где \F{$\gamma(h) = g'(y)\alpha(h) + \beta(\varkappa(h))(f'(x) + \alpha(h))$.} Ясно, что $\gamma(0) = 0$ b $\gamma$ непрерывна в нуле по теореме о непрерывности композиыции и результатов арифметических операций. Пожтому выполнено определение дифференцируемости композиции $g\circ f$ в точке $x$ и верно равенство.