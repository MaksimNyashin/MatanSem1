(75)

\T \q Открытость и замкнутость в прострвнстве и подпространстве. Пусть $(x, p)$~--- метрическое пространствоб $D \subset Y\subset X$.\\
1. $D$ открыто в $Y$ тогда и только тогда, когда существует множество $G$, открытое в $X$, что $D = G \cap Y$\\
2. $D$ замкнуто в $Y$ тогда и только тогда, когда существует такое множество $F$, замкнутое в $X$, что $D = F\cap Y$

\D 1. Пусть $D = Y \subset G$, где $G$ открыто в $X$. Возьмём точку $a \in D$. В силу открытости $G$ в $X$ существует окрестность $V^X_a$ точки $a$ в $X: V^X_a \subset G$. Тогда $V^Y_a = V^X_a \cap Y$~--- окрестность содержащаяся в $D$. В силу произвольности $a$ множество $D$ открыто в $Y$.\\
Обратно, пусть $D$ открыто в $Y$. Тогда для каждой точки $a \in D$ найдётся её окрестность в $Y$, содержащаяся в $D: V^Y_a = B^Y(a, r_a) \subset D$. Обозначим $G = \ds\bigcup_{a \in D} B^X(a, r_a)$. Тогда $G$ открыто в $X$ как объединение открытых множеств и $G \cap Y = \ds\bigcup\limits_{a \in D}(B^x(a, r_a)\cap Y) = \bigcup\limits_{a \in D}B^Y(a, r_a) = D$\\
2. По теореме замкнутость $D$ в $Y$ равносильна открытости $Y\bsl D$ в $Y$. По доказанному, последнее равносильно существованию такого открытого в $X$ множества $G$, что $Y\bsl D = G \cap Y$. Осталось обозначить $F = G^c$ и учесть, что соотношения $S = F \cap Y$ и $Y\bsl D = G\bsl Y$ равносильны.