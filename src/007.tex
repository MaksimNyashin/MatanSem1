(39)

\L Пусть элементы множества $A$ предстаимы в виде бесконечной в обоих направлениях матрицы (занумерованы с помощью упорядоченной пары натуральных чисел по одному разу с использованием всех возможных пар). Тогда $A$ счётно.

\D Занумеруем элементы множества $A$ по диагоналям $A = \{a_{11}, a_{12}, a_{21}, a_{13}, a_{22}, \cdots\}$.

\Op Не более чем счётное множество~--- счётное, конечное или пустое множество

\T Не больее чем счётное объединение не более чем счётных множеств.

\D Пусть $B = \ds\bigcup\limits^{n}_{k=1} A_k$ или $B = \ds\bigcup\limits^{\infty}_{k=1} A_k$, множества $A_k$ не более чем счётны. Запишем элементы $A_1$ в первую строку матрицы, элементы $A_2\bsl A_1$~--- во вторую строку и так далее, то есть если задано множество $A_k$, то элементы $A_k \bsl \ds\bigcup\limits^{k - 1}_{j = 1} A_j$ запишем в $k$-тую строку матрицы. Таким образом все элементы множества $B$ окажутся записанными в клетки матрицы (но при этом некоторые клетки могут остаться пустыми). Значит $B$ Равномощно некоторому подмножеству счётного множества $\N \times \N$. А подмножество счётного либо счётно, либо конечно, либо пусто, то есть не более чем счётно.