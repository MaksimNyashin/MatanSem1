(68-70)

 \Op Точка $a$ называется внутренней точкой множества $D$, если существует окрестность точки $a$, содержащаяся в $D$.

\Op Множество $D$ называется открытым, если все его точки внутренние.

\Op Пусть $(X, \rho)$~--- метрическое пространство, $a \in X, r > 0$. Множество\\
$B(a, r = \{x \in X: \rho()x, a < r\})$\\
называются открытым шаром радиуса $r$ с центром в точке $a$ или окрестностью ($r$-окрестностью) точки $a$ и обозначается $V_a(r)$ или $V_a$, если  значение $r$ несущественно.

Покажем, что открытый шар $B(a, r)$~--- открытое множество. Пусть $p \in B(a, r)$, то есть $\rho(p, a) < r$\\
Положим $h = r - \rho(p, a)$ $(h > 0)$ и проверим что $B(p, h) \subset B(a, r)$. Пусть $x \in B(p, h)$, то есть $\rho(x, p) < h$, тогда\\
$\rho(x, a) \le \rho(x, p) + \rho(p, a) < h + r - h = r$\\
То есть $x \in B(a, r)$.

\T \q Свойства открытых множеств.\\
1. Объежинение любого семейства открытых множеств открыто.\\
2. Перечечение конечного семейства открытых множеств открыто

\D 1. Пусть задано семейство открытых множеств $\{G_\alpha\}_{\alpha \in A},\ G = \ds\bigcup\limits_{\alpha \in A} G_\alpha,\ x \in G$. Докажем, что $x$~--- внутренняя точка $G$. Так как $x \in G$, найдётся $\alpha: x \in G_\alpha$, тогда существует шар $B(x, r) \subset G_\alpha$, тогда $B(x, r) \subset G$\\
2. Пусть задано конечное семейство открытых множество $\{G\}^n_{k = 1},\ G = \bigcup\limits^n_{k = 1} G_k,\ x \in G$. Тогда x принадлежит каждому из множеств и в силу открытости найдутся такие $r_1, r_2, \cdots, r_n$, что $B(x, r_k) \subset G_k$. При $r = \min\{r_1, \cdots, r_n\}$ $B(x, r) \subset G_k$ для любого $k \in [1 : n]$. Тогда, по определению, $B(x, r) \subset G$.

\Zam 1. Пересечени бесконечного семейства открытых множеств не обязано быть открытым. Например\\
$\ds\bigcap\limits^\infty_{n = 1} (-\frac{1}{n}, \frac{1}{n}) = \{0\}$\\
а одноточечное мнодество не является открытым в $\R$.

\Op Множество всех внутренних точне множества $D$ называется внутренностью $D$ и обозначается $\mathring{D}$ или Int $ D$

\Zam 2. Внутренность $D$ есть:\\
а) объединение всех открытых подмножеств $D$\\
б) максимальное по включению открытое подмножество $D$.

\D Пусть $G$~--- объединение всех открытых подмножеств $D$, тогда $G \subset D$, $G$ содержит любое открытое подмножество $G$ и открыто по теореме, то есть $G$~--- максимальное по включению открытое подмножество $D$. Если $x$~--- внутренняя точка $D$, то $D$ содержит окрестность $V_x$ точки $x$, а тогда $x \in V_x \subset G$. С другой стороны, если $x \in G$, то $x$ принадлежит некоторому открытому подмножеству $D$ и значит является его внутренней точкой и, тем более, внутренней точкой $D$.

\Zam3. Множество открыто тогда и только тогда, когда оно совпадает со своей внутренностью.