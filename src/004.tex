(25-26)\\
Число $M$ называется максимумом или наибольшим элементом множества $E \subset \R$, если $M \in E$ и $\forall x \in E\ x \le M$, обозначается $\max E$.

\T Во всяком конечном подмножестве $\R$ есть наибольший и наименьший элемент.

\D Проведем индукцию по числу $n$ элементов сножества. База индукции $n = 1$: если в множестве всего один элемент, то он и наибольший, и наименьший. Для определённости индукционный переход проведём в случае максимума. Пусть всяке $n$-элементное подмножество $\R$ имеет максимум, $E$~---$(n + 1)$-элементное подмножество $\R$:\\
$E = \{x_1, \cdots, x_n, x_{n + 1}\}$.\\
Обозначим\\
$c = \max\{x_1, \cdots, x_n\}$.\\
Если $c \le x_{n + 1}$, то очевидно $x_{n + 1} = \max E$, иначе $c = max E$.

\S1. Во всяком непустом ограниченнои сверху (снизу) подмножестве $\Z$ есть наибольший (наименьший) элемент.

\D Пусть $E \subset \Z$, $E \neq \varnothing$, $E$ ограничено сверху. Выберем какой-нибудь элемент $n_0 \in E$ положим\\
$E_1 = \{n \in E: n >\ge n_0\}$.\\
Поскольку $E$ ограничено сверху, то множество $E_1$ конечно (в нем не более $M - n_0 + 1$ элементов, где $M$~--- верхняя граница $E$). По теореме в множестве $E_1$есть наибольший элемент, он же будет наибольшим элементом в $E$.

\S2. Во всяком непустом подмножестве $\N$ есть наименьший элемент.

