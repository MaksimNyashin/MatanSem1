(169)

\Op Пусть $f: \ang ab \to\R,\s x_0 \in \ang ab$. Существует такое число $A \in \R$, что \F{$f(x) = f(x_0) + A(x - x_0) + o(x - x_0),\quad x \to x_0$,} то функция называется диффернцируемой в точке $x_0$. При этом чисдо $A$ называется производной фугкции $f$ точке $x_0$.

\Op Пусть $f: \ang ab \to \R,\s x_0\in \ang ab$. Если существует предел \F{$\li{x}{x_0} \frac{f(x) - f(x_0)}{x - x_0}$,} равный числу $A \in \R$, то функция $f$ называется дифференцируемой в точке $x_0$, а число $A$~--- её производной в точке $x_0$.

\T Определения дифференцирцуемости и производной равносильны.

\D 1. Пусть $f$ дифференцируема, а $A$~--- ее производная в точке $x_0$, в смысле определения 1, которое говрит, что \F{$f(x) = f(x_0) + A(x - x_0) + \varphi(x)(x - x_0), \quad \varphi(x) \xra[x\to x_0]{} 0$.} Перенося в $f(x_0)$ в левую часть и деля на $x - x_0$ находим, что \F{$\frac{f(x) - f(x_0)}{x - x_0} = A + \varphi(x)\xra[x \to x_0]{} A$.} то есть $f$ дифференцирцема, а $A$~--- ее производная.\\
1. Обратно, пусть функция $f$ дифференцируем, а $A$~--- ее производная в смысле определения 2. Обозначим \F{$\varphi(x) = \frac{f(x) - f(x_0)}{x - x_0} - A$.} Тогда $\varphi(x) \xra[x\to x_0]{} 0$ и выполнено равенство из превой части доказательства, то есть $f$ дифференцируема, а $A$~--- призводная в смысле определения 1.

\Pr1. $f(x) = |x|$ $f'_\pm(0) = \li{x}{0\pm}\frac{|x| - 0}{x - 0} = \pm 1$. Поэтому она не дифференцируема в нуле.

\Pr2. $f(x) = x \sin \frac 1x$ при $x \neq 0,\s f(0) = 0$ $f'(0) = \frac{f(x) - f(0)}{x - 0} = \sin \frac 1x$ не имеет предела.

\Pr3. $f(x) = \sqrt[3]{x}$ $f'(0) = \frac{\sqrt[3]{x} - 0}{x - 0} \to x^{-2/3} \to +\infty$.

\Pr4. $f(x) =$ sign$x$ $f' = \frac{\textit{sign}x - \textit{sign}0}{x - 0} = \frac 1{|x|} \to +\infty$.