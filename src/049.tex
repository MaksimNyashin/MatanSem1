(140)

\q1. Функция $\exp_a$ строго возрастает на $\R$ при $a > 1$ и строго убывает на $\R$ при $0 < a < 1$.

\D Пусть $a > 1,\s x < y$. Докажем, что $a^x < a^y$. Возьмём рациональные числа $\ol r, \ool r$, такие что \F{$x < \ol r < \ool r < y$} и две последовательности рациональных чисел $\{\ol r_n\}$ и $\{\ool r_n\}$, такие что \F{$\ol r_n < x < y < \ool r_n\quad \ol r_n \to x,\s \ool r_n \to y$.} Тогда в силу строгой монотонности показательной функции рационального аргумента \F{$a^{\ol r_n} < a^{\ol r} < a^{\ool r} < a^{\ool r_n}$.} По теореме о предельном переходе в неравенстве,
\F{$a^x \le a^{\ol r} < a^{\ool r} \le a^ y$.} Случай $0 < a < 1$ разбирается аналогично.

\q2. $a^{x + y} = a^x a^y$. В частности $a ^{-x} = \frac 1{a^x}$

\D Возьмём две последовательность рациональных чисел $\{\ol r_n\}$ и $\{\ool r_n\}$, стремящиеся к $x, y$ и перейдём к пределу в равенстве
\F{$a^{\ol r_n + \ool r_n} = a^{\ol r_n}a^{\ool r_n}$,} которое верно для рациональных чисел.

\q3. Показательная функция непрерывна на $\R$.

\D Непрерывность показательной функции в нуле доказывается на языке последовательностей. $\{x_n\}$~--- последовательность вещественных чисед, $x_n \to 0$. Возьмём $\eps > 0$ и зафиксируем номер $N_0$ для которого выполняется неравенство $1 - \eps < a^{-1/N_0} < a^{1/N_0} < 1 + \eps$. Тогда найдётся такой номер $N$,что для всех $n > N$ будет $-\frac 1{N_0} < x_n < \frac 1{N_0}$. В силу строгой монотонности показательной функции \F{$1 - \eps < a^{-1/N_0} < a^{x_n} < a^{a/N_0} < 1 + \eps$} для таких $n$. Это и означает, что $a^{x^n} \to 1$. Случай $0 < a < 1$ разбирается аналогично.\\
Непрерывность в произвольной точке $x_0$ следует из доказанной непрерывности в нуле \F{$a^{r_0 + \Delta x} - a^{x_0} = a^{x_0} (a^{\Delta x} - 1) \to 0$}.

\q4. $(a^x)^y = a^{xy}$.

\D Возьмём две последовательности рациональных чисел $\{x_n\},\s \{y_m\}: x_n \xra[n \to \infty]{} x,\s y_m \xra[m \to \infty]{} y$. По известному свойству степени с рациональным показателем $(a^{x_n})^{y_m} = a^{x_n y_m}$. Зафиксируем $m$ и устремими $n$ к $\infty$. Тогда, по определению показательной функции $a^{x_n y_m} \xra[n \to \infty] a^{xy_m}$ и $a^{x_n} \xra[n \to\infty] a^x$, а по непрерывности степенной функции с рациональным показателем $(a^{x_m})^{y_m} \xra[n\to\infty]{} (a^x)^{y_m}$. Поэтому $(a^x)^{y_m} = a^{xy_m}$. Осталось устремить $m$к $\infty$и воспользоваться непрерывностью показательной функции.

\q5. $(ab)^x = a^x b^x$

\D Сделаем предельный переход в равенстве для степеней с рациональным показателем.

\q6. $\exp_a: \R\xra[]{\textit{на}}(0, +\infty)$.

\D Пусть $a > 1$. Функция $exp_n$ строго возрастает, поэтому существуб пределы $\dsl\lim{x \to \pm \infty}{} a^x/$. по неравенству Бернулли
\F{$a^n = (1 + \alpha)^n \ge 1 + na \to +\infty, \quad a^{-n} = \frac 1{a^n} \to 0$.} Значит по свойствам промежутака $\exp_a(\R) = (0, + \infty)$. Кроме того, значение 0 не принимается в силу строгой монотонности: если $a^{x_0} = 0$, то $a^x < 0$ при $x < x_0$, чего быть не может. Доказательство при $0 < a < 1$ аналогично.

\q1. $\log_a(xy) = \log_a x + \log_a y$ при всех $x, y > 0$.

\D По свойству 2, \F{$a^{\log_a x + \log_a y} = a^{\log_a x}a^{\log_a y} = xy$.}

\q2. $\log_a x^\alpha = \alpha \log_a x$ при всех $x > 0,\s \alpha \in \R$ В частности $\log_a \frac 1x = -\log_a x$.

\D По свойству 4, \F{$a^{\alpha \log_a x} = (a^{\log_a x})^\alpha = x^\alpha$.}

\q3. $\log_a x = \frac{\log_b x}{\log_b {a}}$ при всех $x > 0$. В частности, $\log_a b = \frac 1{\log_b a}$.

\D По свойству 4 \F{$b^{\log_b a\log_a x} = (b^{\log_b a})^{\log_a x} = a^{\log_a x} = x$}.