(50)

\T \q Единственность предела последовательности. Последовательность в метрическом пространстве не может иметь более одного предела: если $x_n \to a$, $x_n \to b$, то $a = b$.

\D Предположим противное: пусть $a \neq b$. Тогда по аксиоме $\rho(a, b) > 0$. Возьмём $\varepsilon = \frac{\rho(a, b)}{2}$. По определению предела, $\exists N_1, N_2$, что $\forall n > N_1\ \rho(x_n, a) < \varepsilon$ и $\forall n > n_2\ \rho(x_n, b) < \varepsilon$. Тогда если $n > \max(N_1, N_2)$, то по аксилмам расстояния\\
$\rho(a, b) \le \rho(a, x_n) + \rho(x_n, b) < \varepsilon + \varepsilon = \rho(a, b)$\\
Что невозможно.

\T Сходящаяся последовательность ограничена.

\D Пусть $x_n \to a$. Взяв $\varepsilon = 1$ найдем $N$, что для всех номеров $n > N$ будет $\rho(x_n, a) < 1$. Пусть\\
$R = \max\{\rho(x1, a), \cdots, \rho(x_n, a), 1\}$,\\
тогда $\rho(x_n, a) \le R$ при всех $n \in \N$.