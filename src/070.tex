(217)

\T \q Критерий монотонности функции. Пусть функция $f$ непрерывна на $\ang ab$ и дифференцируема на $(a, b)$. Тогда $f$ возрастает(убывает) на $\ang ab$ в том и только том случае, когда $f'(x) \ge 0 (f'(x) \le 0)$ для всех $x \in \ang ab$.

\D 1. Необходимость. Пусть $f$ возрастает. Возьмем $x\in (a, b)$, тогда $f(y)\ge f(x)$ для всех $y \in (a, b)$, поэтому \FF{f'(x) = f'_+(x) = \li y{x+} \frac{f(y) - f(x)}{y - x} \ge 0 .}

2. Достаточность. Пусть $f'(X) \ge 0$ для всех $x \in \ang ab$ Возьмем $x_1, x_2 \in \ang ab: x_1 < x_2$ и докажем, что $f(x_1) \le f(x_2)$. По теореме Лагранжа существует такое $c \in (x_1, x_2)$, что \FF{f(x_2) - f(x_1) = f'(c)(x_2 - x_1) \ge 0}.
Случай убывающей функции рассматривается переходом к $-f$.

\S1. \q Критерий постоянства функции. Пусть $f: \ang ab\to \R$. Тогда $f$ постоянна тогда и только тогда, когда $f'(x) = 0$ при всех $x \in \ang ab$.

\D Известно, что производная постоянной функции равна 0. Обратно если $f \in C\ang ab$ и $f'(x) = 0$ для всех $x \in (a, b)$, то по теореме функция $f$ одновременно и возрастает и убывает, то есть постоянна.

\S2. \q Критерия строгой монотонности функции. Пусть $f$ непрерывна на $\ang ab$. Тогда $f$ строго возрастает на $\ang ab$ в том и только том слкучае, когда:\\
1) $f'(x) \ge 0$ для всех $x\in (a, b)$;\\
2) $f'$ не обращается в ноль тождественно ни на каком интервале.

\D По следствию 1, $f$ не постоянна ни на каком интервале. Поэтому из строго строгого возрастания $f$ вытекает утверждение 2. а утверждение 1 верно по теореме 1.\\
Пусть теперь выполнены утверждения 1 и 2. Из нетрицательности производной следует возрастание $f$. Если возрастание нестроое, то найдутся точки $x_1, x_2 \in \ang ab$, что $x_ 1 < x_2,\s f(x_1) = f(x_2)$. Тогда $f$ постоянна на $[x_1, x_2]$, что противоречит условию 2.

\Zam. Теорема и оба следствия обобщаются наситуацию, когда $f$ непрерывна на $\ang ab$, а дифференцируема на $\ang ab$ за исклбчением конечного множества точек.

\D Пусть $a_1,\cdots, a_n$~--- все те точки интервала $\ang ab$, в которых $f$ не дифференцируемма; $a_1 < \cdots < a_n$. Если $f$ возрастает на $\ang a, b$, то $f$ возрастает на каждом промежутке $\langle a, a_1],\s [a_1, a_2],\cdots,[a_n, b\rangle$. Тогда $f'\ge 0$ на каждом рпромежутке по теореме.\\
Обратно, если $f \in C\ang ab$ и $\s f' \ge 0$ на каждом промежутке, то $f$ возрастает на кждом из них и, следовательно, на $\ang ab$.