(133)

\Op Пусть $Y$~--- метрическое пространство, $E \subset Y$. Непрерывное отображение отрезка в множество $E$: \F{$\gamma \in C([a, b]) \subset \R \to E$} называется путём в $E$. Точка $\gamma(a)$ называется началом, $\gamma(b)$~--- концом пути.

\Op Пусть $Y$~--- метрическое пространство, $E \subset Y$. Множество $E$ называется линейн связным, если любые две его точки соединены путём.
\F{$\forall A, b \in E\s \exists \gamma \in C([a, b] \subset \R \to E) :\s \gamma (a) = A,\s \gamma(b) = B$}

\T \q Больцано-Коши О непрерывных отображениях. Пусть $X, Y$~--- метрические пространства, $X$ линейно связно, $f \in C(X \to Y)$. Тогда $f(X)$ линейно связно. Другими словами: непрерывный образ линейно свзного множества линейно связен.

\D Пусть $A, B \in f(X)$. Тогда, по определению образа, существуют точки $\alpha, \beta \in X: a = f(\alpha),\s B = f(\beta)$. Так как $X$ линейно связно, точки $\alpha, \beta$ можно соединить путём в $X$, то есть существует путь $\gamma \in C([a, b] \to X): \gamma(a) = \alpha,\s \gamma(b) = \beta$. Но тогда, по теореме о непрерывности композиции $f \circ \gamma$~--- путь в $f(X)$; при этом $(f \circ \gamma)(a) = A,\s (f \circ \gamma)(b) = B$.

\Zam4. Согласно лемме, на прямой линейно связными могут быть толко промежутки.

\Zam4. Теорема о сохранении промежутка, вообще говоря, не допускает обращения. Так, множество значений функции \F{$f(x) = \begin{cases}x, &x \in [0, 1],\\ 0, & x \in (1, 2].\end{cases}$} есть отрезок $[0, 1]$. Однако для монотонной функции обратное утверждение верно.