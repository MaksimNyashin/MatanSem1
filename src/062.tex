(195)

\T \q Правило Лопиталя для неопределённостей вида $\frac \infty\infty$. Пусть $\infty \le a < b \le +\infty$, функции $f$ и $g$ диференцируемы на $(a, b),\s g'(t) \neq 0$ для любого $t \in (a, b,\s \li x{a+}g(x) = \infty)$ и существует предел \F{$\li x{a+}\frac{f'(x)}{g'(x)} = A \in \ol{\R}$.} Тогда предел $\li x{a+} \frac{f(x)}{g(x)}$ тоже существует и равен $A$.

\D 1. Пусть $A = 0$. Возьмём последовательность $\{x_n\}$ со свойствами: $x_n \in (a, b),\s x_n \to a$, и докажем, что $\frac{f(x_n)}{g(x_n)}\to 0$. Зафиксируем число $\sigma > 0$. по условию, найдется такое $y \in (a, b)$, что для любого $c \in (a, y)$ будет $g(c)\neq 0$ и $|\frac{f'(с)}{g'(c)}| < \sigma$. Начиная с некоторого номера $x_n \in (a, y)$, поэтому можно считать, что $x_n \in (a, y)$ для всех $n$. По теореме Коши, для любого $n$ найдётся такое $c_n \in (x_n, y)$, что
\F{$\frac{f(x_n)}{g(x_n)} = \frac{f(x_n) - f(y)}{g(x_n) - g(y)}\frac{g(x_n) - g(y)}{g(x_n)} + \frac{f(y)}{g(x_n)} = \frac{f'(c_n)}{g'(c_n)}(1 - \frac{g(y)}{g(x_n)}) + \frac{f(y)}{g(x_n)}$.} Учитывая, что $g(x_n) \to \infty$, находим \F{$|\frac{f(x_n)}{g(x_n)}| \le \sigma(1 + |\frac{g(y)}{g(x_n)}|) + |\frac{f(y)}{g(x_n)}| \xra[n \to \infty]{} \sigma$.} Поэтому $\ol{\lim} |\frac{f(x_n)}{g(x_n)}| \le \sigma$. Но так как $\sigma$ произвольно $\lim\frac{f(x_n)}{g(x_n)} = 0$.

2. Пусть $A \in \R$ произвольно. Положим $h = f - Ag$. Тогда \F{$\li x{a+} \frac{h'(x)}{g'(x)} = \li x{a+} (\frac{f'(x)}{g'(x)} - A) = 0$.} По доказанному, $\frac{h(x)}{g(x)} \xra[x\to a+]{} 0$, то есть $\frac{f(x)}{g(x)} \xra[x\to a+]{} A$.

3. Пусть $A = +\infty$ расматривается аналогично случаю $A = 0$. При этом всместо $|frac{f'(c)}{g'(c)}| < \sigma$ используется неравенство $\frac{f'(c)}{g'(c)} > M$ и доказывается, что $\ul{\lim} \frac{g(x_n)}{g(x_n)} \ge M$. Случай $A = -\infty$ разбирается переходом к функции $-f$.

\Zam1. Утверждение, аналогичное теореме справедливы и для левостороннего предела, а следовательно, и для двухстороннего предела

\Zam2. В теореме функци $f$ не предполагается бесконечно большой, хотя на практике правило Лопиталя обычно применяют при наличии неопределеностей.

\Zam3. В цчловиях правила Лопиталя существование предела отношения функций выводится из существования предела отношений их производных. Обратное неверно. Если $g(x) = x,\s f(x) = x + \sin x$, то предел на бесконечности 1, а отношение производных предела не имеет.

\Pr1. $\li x{+\infty} \frac{\ln x}{x^\alpha} = 0,\quad \alpha > 0$.\\
$\li x{+\infty} \frac{\ln x}{x^\alpha} = \li x{+\infty} \frac{1/x}{\alpha x^{\alpha - 1}} = \li x{+\infty} \frac 1{\alpha x^\alpha} = 0$.

\Pr2. При $a > 1$. $\li x{+\infty} \frac x{a^x} = \li x{+\infty} \frac 1{a^x \ln a} = 0$\\
Тогда $\li x{+\infty} \frac{x^k}{a^x} = \li x{+\infty} (\frac{x}{(a^{1/k})^x})^k = 0$.