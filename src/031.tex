(106)

\T \q Предельный переход в неравенстве для функций. Пусть $X$~--- метрическое пространство, $f, g: D\subset X\to \R,\ a$~--- предельная точка $D,\ f(x)\le g(x)$ для всех $x\in D\bsl \{a\},\ A, B\in \ol{\R}, f(x)\t A,\ g(x)\t B$. Тогда $A \le B$

\D Возьмём последовательность $\{x_n\}$ со свойствами $x_n \in D,\ x_n \neq a,\ x_n\to a$. Тогда, по определению Гейне, $f(x_n) \to A,\ g(x_n)\to B$. По теореме о предельном переходе в неравенстве для последовательностей $A \le B$.

\T \q О сжатой функции. Пусть $X$~--- метрическое пространство, $f, g, h: D\subset X \to \R, a$~--- предельная точка $D,\ f(x)\le g(x)\le h(x)$ для всех $x \in D\bsl\{a\},\ A \in \R,\ f(x)\t A,\ h(x)\t A$. Тогда и $g(x)\t A$.

\D Возьмём последовательность $\{x_n\}$ со свойствами $x_n \in D,\ x_n \neq a,\ x_n\to a$. Тогда, по определению Гейне $f(x_n) \to A,\ h(x_n) \to A$. Кроме того, по условию для всех $n \in N$ \F{$f(x_n) \le g(x_n) \le h(x_n)$.} По теореме о сжатой последовательности $g(x_n) \to A$. В силу произвольности последовательности $\{x_n\}$ это и значит, что $g(x)\t A$.

\Zam1. Аналогично доказывается, что если $f(x) \le g(z)$ для всех $x \in D\{a\}$ и $f(x) \t +\infty (g(x)\t -\infty)$,то и $g(x) \t +infty (f(x) \t -\infty)$.

\Zam2. В теоремах и замечании достаточно выполнения неравенств на множестве $\dot{V_a}\cap D$, где $V_a$~--- какая-нибудь окрестность точки $a$.

Пусть $f: D \subset X\to Y,\ D_1 \subset S,\ a$~--- предельная точка $D_1$ (а следовательно и $D$). Тогда если предел $f$ в точке $a$ существует и равен $A$, то предел сужения $f$ на $D_1$ в точке $a$ также существует и равен $A$. В самом деле, если соотношение $f(X) \in V_A$ выполняется для всех $x \in \dot{V_a}\cap D$, то оно тем более выполняется для всех $x$ из $\dot{V_a}\cap D_1$. Однако возможна ситуация, когда предел сужения существует, а предел отображения нет.