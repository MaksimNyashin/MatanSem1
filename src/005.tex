(26-27)

\Op Пусть $x \in \R$. Наибольшее целое число, не превосходящее $x$, называется целой частью $x$ и обозначается $[x]$. Она существует так как подмножество целых чисел, имеющее верхнюю границу, имеет максимум.

\Zam1. Из определения следует, что\\
$[x] \le x < [x] + 1, \qquad x - 1 < [x] \le x$

\T \q Плотность множества рациональных чисел. Во всяком непустом иитервале есть рациональное число.

\D Пусть $a, b\in \R$, $a < b$, тогда $\frac{1}{b - a} > 0$ и по аксиме Архимеда найдётся такое $n\in N$, что $n > \frac{1}{b - a}$, то есть $\frac{1}{n} < b - a$. Положим $c = \frac{[na] + 1}{n}$, тогда $c \in \Q$ и\\
$c \le \frac{na + 1}{n} = a + \frac{1}{n} < a + b - a = b$\\
$c > \frac{na - 1 + 1}{n} = a$\\
то есть $c \in (a, b)$.

\S3. Во всяком интервале бесконечно много рациональных чисел.

\D Пксть в некотором интервале $(a, b)$ количество рациональных чисел конечно. Обозначим $x_1$ наименьшее из них. Тогда в интервале $(a, x_1)$ нет ни одного рационпльного число, что противоречит теореме.