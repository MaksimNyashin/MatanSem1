Билет 63: Теорема Дарбу, следствия
(198)

\T \q Дарбу. Если функция $f$ дифференцируема на $[a, b]$, то для любого числа $C$, лежащего между $f'(a)$ и $f'(b)$, найдется такое $c \in (a, b)$, что $f'(x) = C$.

\D 1. Пусть сначала $f'(a)$ и $f'(b)$ разных знаков; докажем, что существует такое $c \in (a, b)$, что $f(c) = 0$. Для определенности будем считать, что $f'(a) < 0 < f'(b)$. Поскольку $f$ непрерывна на $[a, b]$, по теореме Вейерштрасса найдётся точка $c \in [a, b]$, для которой $f(c) =\dsl\min{x\in[a, b]} f(x)$. Если $c\in (a, b)$, то, по теореме Ферма, $f'(c) = 0$. Поэтому достаточно доказать, что $c \neq a$ и $c \neq b$. Если $c = a$, то есть функция принимает наименьшее значение на левом конце отрезка, то $\frac{f(x) - f(a)}{x - a} \ge 0$ при всех $x \in (a, b]$, а поэтому и $f'(a) \ge 0$, что противоречит условию. Аналогично доказывается, что $c\neq b$.

2. Рассмотрим теперь общий случай. Пусть для определености $f'(a) < C < f'(b)$. Положим $\varphi(x) = f(x) - Cx$. Тогда \F{$\varphi'(a) = f'(a) - C < 0  < f'(b) - c = \varphi'(b)$.} По доказанному, найдется такое $c \in (a, b)$, что $\varphi'(c) = 0$, то есть $f'(c) = C$.

\S1. Если функция $f$ дифференцируема на $\ang ab$, то $f'(\ang ab)$~--- промежуток.\\
При доказательнстве сослаться на лемму 1 параграфа 2 главы 3 о характерисике промежутков.

\S2. Производная дифференцируемой на промежутке функции не может иметь на нем разрывов первого рода.