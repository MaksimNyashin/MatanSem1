(190)

\T \q Лагранжа. Пусть функция непрерывна на $[a, b]$ и дифференцируема на $(a, b)$. Тогда найдётся такая точка $c \in (a, b)$, что \F{$\frac{f(b) - f(a)}{b - a} = f'(c)$}.

\T \q Коши. Пусть функции $f, g$ непрерывны на $[a, b]$ и дифференцируемы на $(a, b)$. Тогда найдётся такая точка $c \in (a, b)$, что \F{$\frac{f(b) - f(a)}{g(b) - g(a)} = \frac{f'(c)}{g'(c)}$.}

\Zam1. Теорема Лагранжа~--- частный случай теоремы Коши, поэтому ее доказывать небудем, но она применяется часто, поэтому её выделили в отдельную теорему.

\D Заметим, что $g(a) \neq g(b)$, так как иначе по теореме Ролля нашлась бы точка $t \in (a, b)$, в которой $g'(t) = 0$. Положим $\varphi = f - Kg$, где $K = \frac{f(b) - f(a)}{g(b) - g(a)}$, чтобы $\varphi(a) = \varphi(b)$. Тогда $\varphi$ удовлетворяет условиям теоремы Ролля. Поэтому найдётся такая точка $c \in (a, b)$, что $\varphi'(c) = 0$, то есть $f'(c) = Kg'(c)$, что равносильно требуемому.

\Zam7. Пусть функция $f$непрерывна на $\ang ab$ и дифференцируема на $(a, b)$. Тогда для любых различных точек $x, x+ \Delta x$ из $\ang a, b$ найдётся такое $\theta \in (0, 1)$, что \F{$f(x +\Delta x) - f(x) = f'(x + \theta \Delta x)\Delta x$} Доказательство по теореме Лагранжа с концами $x, x + \Delta x$. Надо учесть, что $c$ между $x$ и $x + \Delta x$, то есть $\theta = \frac{c - x}{\Delta x} \in (0, 1)$.

\S1. \q оценка приращения функции. Пусть функция $f$непрерывна на $\ang ab$, дифференцируема на $(a, b)$, а число $M > 0$ такого, что $|f'(t)| \le M$ для всех $t \in (a, b)$. ТОгда для любых точек $x$ и $x + \Delta x$ из $\ang ab$ \F{$|f(x + \Delta x) - f(x)| \le M|\Delta x|$.} Другими словами, если производная функции ограничена, то приращение функции не более чем в $M$ раз превзойдет приращение аргумента.\\
Очевидно вытекает из замечания.

\S2. Функция,имеющая на $\ang ab$ ограниченную производную, равномерно непрерывна на $\ang ab$.

\D Пусть $M > 0$ таково, что $|f'(t)| \le M$ для всех $t \in \ang ab$. Возьмем $\eps > 0$ и положим $\delta = \frac \eps M$. Тогда, если $x, y \in \ang ab,\s |x-y| < \delta$,то последствию 1 \F{$|f(x) - f(y)| \le M|x-y| \le M\delta = \eps$,} что и доказывает равномерную непрерывность $f$.