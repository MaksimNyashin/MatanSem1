(146-154)

\L Если $0 < x < \frac \pi2$, то \F{$\sin x < x < \tg x$.}

\D Изобразим единичную окружность и угол в $x$ радиан.\\
\ig{sintan}\\
На рисунке \F{$\triangle OAB \subset \textit{сект.} OAB \subset \triangle OAD$.} Поэтому фигуры связаны неравенством \F{$S_{\triangle OAB} < S_{\textit{сект.} OAB} < S_{\triangle OAD}$} Учитывая, что \F{$S_{\triangle OAB} = \frac12 |OA||BC|$,} \F{$S_{\textit{сект.} OAB} = \frac 12|OA|^2 x,\quad S_{\triangle OAD} = \frac 12 |OA||AD|$}, \F{$|OA = 1|,\s |BC| = \sin x,\s |AD| = \tg x$.}

\S1. При всеx $x \in \R$ $|\sin x| \le |x|$

\D При $|x| \in (0, \frac \pi2)$ доказано по лемме, иначе $|sin x| < 1 < \frac \pi2 \le x$

\S2. Функции синус и косинус непрерывны на $\R$.

\D Для любой точки $x_0 \in \R$ имеем: \F{$|\sin x - \sin x_0| = |2\sin \frac{x - x_0}2 \cos\frac{x + x_0}2 \le 2\cdot\frac{|x - x_0|}2\cdot 1 = |x - x_0| \xra[x \to x_0]{} 0$.} Непрерывность косинуса доказывается с помощью формулы приведения $\cos x = \sin (\frac \pi2 - x)$ и теоремы о непрерывности композиций.

Тангенс и котангенс непрерывны на своих областях определения, по теореме о непрерывности частного.

$\arcsin = (\sin|_{[-\frac \pi2, \frac \pi2]})^{-1}$. По теореме о существовании и непрерывности обратной функции, функция арксинус строго возрастает и непрерывна

$\arccos = (\cos|_{[0, \pi]})^{-1}$. Аналогично функция арккосинус строго убывает и непрерывна.

$\arctg = (\tg|_{(-\frac \pi2, \frac \pi2)})^-1$. Аналогично функция арктангенс строго возрастает  и непрерывна.

$\arcctg = (\ctg|_{(0, \pi)})^{-1}$. Аналогично функция аркоктангенс строго убывает и непрерывна.