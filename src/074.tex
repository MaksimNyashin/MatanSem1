(228)

\L \q О трех хордах. Пусть функция $f$ выпукла вниз на $\ang ab,\s x_1, x_2, x_3 \in \ang ab,\s x_1 < x_2 < x_3$. Тогда \FF{\frac{f(x_2) - f(x_1)}{x_2 - x_1} \le \frac{f(x_3) - f(x_1)}{x_3 - x_1} \le \frac{f(x_3) - f(x_2)}{x_3 - x_2} }
\ig{threechords}

\D По определению выпуклости $f(x_2) \le tf(x_1) + (1 - t) f(x_3)$, где $t = \frac{x_3 - x_2}{x _3 - x_1},\s 1-t = \frac{x_2 - x_1}{x_3 - x_1}$. Преобразуем неравенство двумя способами. С одной стороны, \FF{f(x_2) \le f(x_1) + (1 - t)(f(x_3) - f(x_1)) = f(x_1) _ (x_2 - x_1)\frac{f(x_3) - f(x_2)}{x_3 - x_1} ,} что равносильно левому неравенству. С другой стороны, \FF{f(x_2) \le f(x_3) - t(f(x_3) - f(x_1)) = f(x_3) - (x_3 - x_2)\frac{f(x_3) - f(x_1)}{x_3 - x_1},} что равносильно правому неравенству.

\T \q Односторонняя дифференцируемость выпуклой функции. Пусть $f$ выпукла вниз на $\ang ab$. Тогда для любой точки $x \in \ang ab$ существуют конечные $f'_-(x), f'_+(x)$, причем $f'_-(x) \le f'_+(x)$.

\D Возьмем $x \in (a, b)$ и положим \FF{g(\xi) = \frac{f(\xi) - f(x)}{\xi - x},\quad \xi \in \ang ab \bsl\{x\}.} По лемме о трех хордах $g$ возрастает на $\ang ab\bsl\{x\}$. Поэтому, если $a < \xi < x < \eta < b$, то $g(\xi) \le g(\eta)$, то есть \FF{\frac{f(\xi) - f(x)}{\xi - x} \le \frac{f(\eta) - f(x)}{\eta - x}. } Следовательно, $g$ ограничена на $\langle a, x)$ сверху, а на $(x, b\rangle$ снизу. По теореме о пределе монотонной функции, существуют конечные пределы $g(x-)$ и $g(x+)$, которые, по определения, являются односторонними производными $f'_-(x)$ и $f'_+(x)$. Устремляя $\xi$ к $x$ слева, а $\eta$~--- справа, получаем, что $f'_-(x) \le f'_+(x)$.