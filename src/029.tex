(99)

\Op на $\eps$-языке (по Коши)\\
$\forall \eps > 0\ \exists \delta > 0\ \forall x \in D \bsl \{a\}: \rho_X(x, a) < \delta\ \rho_Y(f(x), A) < \eps$.

\Op на языке последовательностей (по Гейне)\\
$\forall \{x_n\}: x_n\in D\bsl\{a\}, x_n \to a f(x_n) \to A$.

(102)
\T Определения предела отображения по Коши и по Гейне эквивалентны.

\D Для определённости докажем теорему при $a \in X, A \in Y$.

1. Пусть $A$~--- предел отображения $f$ в точке $a$ по Коши; докажем, что тогда $A$~--- предел и по Гейне. Возьмём последовательность $\{x_n\}$ со свойствами из определения Гейне: $x_n \in D,\ x_n \neq a,\ x_n\to a$. Требуется доказать, что $f(x_n) \to A$. Возьмём $\eps > 0$. По определению Коши подберём такое $\delta> 0$, что для всех $x\in D$, для которых $x\neq a$ и $p_X(x, a) <\delta$  будет $p_Y(f(X), A) < \eps$. По определению предела последовательности $\{x_n\}$ для числа $\delta$ найдётся такой  номер $N$, что при всех $n > N$ верно неравенство $p_X(x_n, a) < \delta$. Но тогда $p_Y(d(x_n), A) < \eps$ Для всех $n > N$ В силу произвольности $\eps$ это значит, что $f(x_n) \to A$.

2. Пусть $A$~--- предел отображения $f$ в точке $a$ по Гейне; докажем, что тогда $A$~--- предел $f$ по Коши. Предположим противное: пусть $A$ не есть предел по Коши. Записывая отрицание определения Коши, получаем
\F{$\exists \eps^* > 0\ \forall \delta > 0 \exists x \in D \bsl\{a\}: p_X(x, a) < \delta,\ p_Y(f(X), A) \ge \eps^*$} Следовательно, для каждого $n \in \N$ по число $\delta = \frac 1n$ найдётся такая точка $x_n$, что \F{$x_n \in D \bsl\{a\}\quad p_X(x_n, a) < \frac 1n,\quad p_Y(f(x_n), A) \ge \eps^*$} По теореме о сжатой последовательности, построенная последоввательность $x_n$ стремится к $a$, так как \F{$0 < p_X(x_n, a) < \frac 1n$.} Тогда, по определению Геёне, $f(x_n) \to A$. По определению предела последовательности $\{f(x_n)\}$ для числа $\eps^*$ найдётся такой номер $N$, что для всех номеров $n > N$ будет $p_Y(f(x_n), A) < \eps^*$, что противоречит условию.