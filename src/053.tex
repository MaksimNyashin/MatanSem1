(174)

\q Геометрический (задача Лейбница о касательной).\\
Пусть $f: \ang ab \to \R,\s x_0 \in \ang ab,\s y_0 = f(x_0),\s f$ непрерывнав точке $x_0$. Точка $M_0 = (x_0, y_0)$. Возьмём на графике функции $f$ ещё одну точку $M_1 = (x_1б y_1):\s x_1 \in \ang ab,\s x_1 \neq x_0,\s y_1 = f(x_1)$. Проведём прямую $M_0 M_1$, которуб будем называть секущей. Уравнение секущей $M_0 M_1$ имеет вид\F{$u = y_0 + \frac{y_1 - y_0}{x_1 - x_0}(x_1 - x_0)$.} Касаткльной называетя предельное положение секущей при $M_1 \to M_0 (x_1 \to x_0)$ $k_{\textit{кас.}} = \li{x_1}{x_0} \frac{y_1 - y_0}{x_1 - x_0} = f'(x_0)$. Иными словами, производная в точке~--- угловой коэффициент касательной в этой точке. \F{$y = f(x_0) + f'(x_0)(x - x_0)$.}

\Zam1. Если $l(x) = f(x_0) + f'(x_0)(x - x_0)$,то \F{$f(x) - l(x) = o(x - x_0)\quad x \to x_0$}. Вто же время ни одна другая прямая не обладает этим свойством, поэтому егопринимают за определение касательной (не вертикальной).

\q физический (задача Ньютона о скорости).\\
Пусть материальная точка движется по прямой. Обозначим $s(t)$~--- путсьпройденный точкой за время от начального момента $t_0$ до $t$. Тогда путь.пройденный от момента $t_1$ до момента $t_1 + \Delta t$, равен $\Delta s = s(t_1 + \Delta t) - s(y_1)$. Средняя скорость между этими моментами времени вычисляется формулой $v_{\textit{ср}} = \frac{\Delta x}{\Delta t}$. $v_{\textit{мгн.}}(t_1) = \li{\Delta t}{0} v_{\textit{ср}}$~--- мгновенная скорость. По определению производной она равнв $s'(t_1)$. Подобным образом производная встречается и в ситуациях, когда речь идйт о скорости изменения одной величины относительно другой.