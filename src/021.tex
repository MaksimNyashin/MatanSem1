(81)

\Zam1. Для строго возрастающей последовательности $\{n_k\}$ при всех $k$ будет $n_k \ge k$. Действительно, $n_1 \ge 1$ и из неравенства $n_k \ge k$ следует, что $n_{k + 1} \ge n_k + 1 \ge k + 1$.

\L Всякая подпоследовательность сходящейся последовательности сходится, и притом к тому же пределу: если $\{x_n\}$~--- последовательность в метрическом пространстве $X$, $x_{n_k}$~--- её подпоследовательность, $a \in X, x_a$, То $x_{n_k} \to a$.

\D Возьмём $\eps > 0$. По определению предела существует такой номер $N$, что $p(x_n, a) < \eps$ для всех $n > N$. Но тогда, если $K > N$, то по замечанию и $n_k > N$, а значит $p(x_{n _k}, a) < \eps$.

\L Пусть $\{x_n\}$~--- последовательность в метричеком пространстве $X$, $\{x_{n_k}\}$ и $\{x_{m_t}\}$~--- её подпоследовательностьи, причём объединение множеств их индексов равно $\N, a \in X$. Тогда, если $x_{n_k} \to a,\ x_{m_l}\to a$, то и $x_n \to a$.

\D Возьмём $\eps > 0$. По определнию предела одной и другой подпоследовательности найдутся такие номера $k$ и $L$, что\\
$\rho(x_{n_k}, a) < \eps\quad$ для всех $k > K$\\
$\rho(x_{m_l}, a) < \eps\quad$ для всех $l > L$\\
Положим $N = \max\{n_k, m_l\}$. Если $n > N$, то или $n$ равно некоторому $n_k$, причём $k > K$, а тогда $\rho(x_n, a)< \eps$, или $n$ равно некоторому $m_l$, причём $l > L$, а тогда $\rho(x_n, a) < \eps$.

\Zam2. Леммы сохраняют силу для $a = \infty$ в нормированном пространстве и для $a = \pm \infty$ для вещественных последовательностей. В доказательстве следует заменить неравенство вида $\rho(x_n, a) < \eps$ на $|x_n| > E$ и т. п.