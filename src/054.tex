(178)

Если $f, g: \ang ab \to \R$ дифференцируемы в точке $x \in \ang ab$

\q1. \q Производная суммы и разностию. то функция $f + g$ И $f - g$ дифференцируемы в этой точке и \F{$(f \pm g)'(x) = f'(x) \pm g'(x)$.}

\D По определению производной суммы. \F{$\frac{(f + g)(x + h) - (f + g)(x)}h = \frac{f(x + h) - f(x)}h + \frac{g(h + x) - g(x)}h \xra[h \to 0]{} f'(x) + g'(x).$} Это и означает, что сумма дифференцируема в точке и для производной суммы верно равенство.\\
Для разности доказывается аналогичнл.

\q2. \q Производная произведения. то функция $fg$ дифференцируема в точке $x$ и \F{$(fg)'(x) = f'(x)g(x) + f(x)g'(x)$.}

\D \F{$\frac{(fg)(h + x) - (fg)(x)}h = \frac{f(x + h) - f(x)}h g(x + h) + f(x) \frac{g(x + h) - g(x)}h \xra[h \to 0]{} f'(x)g(x) + f(x)g'(x)$.}

\S1. Если $\alpha \in \R$, то функция $\alpha f$ диффиренцируема в точке $x$ и \F{$(\alpha f)'(x) = \alpha f'(x)$.}

\S2. \q Линейность дифференцирования. Если $\alpha, \beta \in \R$, то функция $\alpha f + \beta g$ дифференцируема в точке $x$ и \F{$(\alpha g + \beta g)'(x) = \alpha f'(x) + \beta g'(x)$.}

\q3. \q Производная частоного. и $g(x) \neq 0$, то функция $\frac fg$ дифференцируема в точке $x$ и \F{$(\frac fg)'(x) = \frac{f'(x)g(x) - f(x) g'(x)}{g^2(x)}$.}

\D Прежде всего заметим, что в силу условия $g(x) \neq 0$ и непрерывности функции $g$ в точке $x$ существует такое $\delta > 0$, что $g$ не обращается в ноль на промежутке $(x - \delta, x + \delta)\cap\ang ab$. Поэтому частно $\frac fg$ определено на током промежутке, и модно ставить вопрос о дифференцируемости частного в очке $x$.
\F{$\frac{\frac fg(h + x) - \frac fg (x)}h = \frac 1{g(x + h)g(x)}(\frac{f(x + h) - f(x)}h g(x) - f(x)\frac{g(x + h) - g(x)}h) \xra[h \to 0]{} \frac{f'(x)g(x) - f(x)g'(x)}{g^2(x)}$}