(91)

\T \q Предел монотонной последовательности.\\
1. Всякая возрастающая ограниченная сверху последовательность сходится.\\
2. Всякая убывающая ограниченная снизу последовательность сходится.\\
3. Всякая монотонная ограниченная последовательность сходится.

\D Докажем первое утверждение. Пусть последовательность $\{x_n\}$ ограничена сверху. По теореме существует $\sup x_n = c \in \R$. Докажем, что $c = \lim x_n$. Возьмём $\eps > 0$. По определению супремума (так как $c - \eps$ не является верхней границей последовательности), найдётся такой номер $N$, что $x_N > c - \eps$. В силу возрастания поледовательности, при любом $n > N$ будет $x_n \ge x_N$. Снова по определению супремума $x_n \le c$ при всех $n$. Итак для любого $n > N$\F{$c - \eps < x_N \le x_n \le c < c + \eps$} В силу произвольности $\eps$ это значит $c = \lim x_n$.\\
Второе утверждение доказываетя аналогично. Третье следует из первых двух.

\Zam1. Если последовательность возрастает и не ограничена сверху, то она стремится к плюс бесконечности.

\D Возьмём $E > 0$. Так как $\{x_i\}$ не ограничена сверху, найдется такой номер $N$, что $x_N > E$. Тогда для любого номера $n > N$, в силу возрастания последовательности, тем более $x_n > E$.

\Zam2. Доказано, что любая монотонная последовательность имеет предел в $\ol{\R}$, при этом для всех возрастающих последовательностей \F{$\lim x_n = \sup x_n$} а для убывающих \F{$\lim x_n = \inf x_n$}