(234)

\T \q Дифференциальные критерии выпуклости. 1. Пусть функция $f$ непрерывна на $\ang ab$ и дифференцируема на $(a, b)$. Тогда $f$ (строго) выпукла вниз на $\ang ab$ в том и только том случае, когда $f'$ (строго) возрастает на $(a, b)$.\\
2. Пусть функция $f$ непрерывна на $\ang ab$ и дважды дифференцируемана $\ang ab$. Тогда $f$ выпукла вниз на $\ang ab$ в том и только том случае, когда $f''(x) \ge 0$ для всех $x \in (a, b)$.

\D 1. Необходимость. Возьмем $x_1, x_2 \in (a, b): x_1 < x_2$. По теореме о выпуклости и касательных \FF{f'(x_1) \le \frac{f(x_2) - f(x_1)}{x_2 - x_1} \le f'(x_2), } что и означает возрастание $f$.

Возьмем $x_1, x_2 \in \ang ab: x_1 < x_2$ и $x \in (x_1, x_2)$. По теореме Лагранжа, существуют такие $c_1 \in (x_1, x)$ и $c_2 \in (x, x_2)$, что \FF{\frac{f(x) - f(x_ 1)}{x -x_1} = f'(c_1),\quad \frac{f(x_2) - f(x)}{x_2 - x} = f'(c_2) .} Тогда $x_1 < c_1 < x < c_2 < x_2$, а $f$, по условию, возрастает, поэтому $f'(c_1) < f'(c_2)$, то есть \FF{\frac{f(x) - f(x_1)}{x - x_1} \le \frac{f(x_ 2) - f(x)}{x_2 - x} ,} что равносильно неравенству из определения выпуклости.

2. По пункту 1, выпуклость $f$ равносильна возрастанию $f'$, которое, по критерию монотонности, равносильно неотрицательности $f''$.