(231)

\T \q Выпуклость и касательные. Пусть функция $f$ дифференцируема на $\ang ab$. Тогда $f$ выпукла вниз на $\ang ab$ в том и только в том случае, когда график $f$ лежит не ниже любой своей касательной, то есть для любых $x, x_0 \in \ang ab$ \FF{f(x) \ge f(x_0) + f'(x_0)(x - x_0).}

\D 1. Необходимость. Пусть $f$ выпукла вниз, $x, x_0 \in \ang ab$. Если $x > x_0$, то, по лемме о трех хордах, для любого $\eta \in (x_0, x)$ \FF{\frac{f(\eta) - f(x_0)}{\eta - x_0} \le \frac{f(x) -f(x_0)}{x - x_0} .} Устремляя $\eta$ к $x_0$ справа, получаем неравенство \FF{f'(x_0) \le \frac{f(x) - f(x_0)}{x - x_0} ,} равносильное требуемому.\\
Если $x < x_0$, то, по лемме о трех хордах, для любого $\xi \in (x, x_0)$ \FF{\frac{f(\xi) - f(x_0)}{\xi - x_0} \ge \frac{f(x) - f(x_0)}{x - x_0}. } Устремляя $\xi$ к $x_0$ слева получаем неравенство \FF{f'(x_0) \ge \frac{f(x) - f(x_0)}{x - x_0},} равносильное требуемому (при домножении на $x - x_0 < 0$ меняется знак неравенства).

2. Достаточость. Пусть для любых $x, x_0 \in \ang ab$ верно неравенство. Возьмем $x_1, x_2 \in \ang ab: x_1 < x_2$ и $x \in (x_1, x_2)$. Применяя неравенство дважды: сначала к точкам $x_1, x$, а затем~--- к $x_2, x$, получаем \FF{f(x_1) \ge f(x) + f'(x)(x_1) - x,\qquad f(x_2) \ge f(x) + f'(x)(x_2 - x),} что равносильно \FF{\frac{f(x) - f(x_0)}{x -x_0} \le f'(x) \le \frac{f(x_2) -f(x)}{x_2 - x} .} Крайние части составляют неравенство из определения выпуклости.

\Op Пусть $f: \ang ab\to \R,\s x_0 \in \ang ab$. Прямая, задаваемая уравнением $y = l(x)$, называется опорной прямой для ыункции $f$ в точке $x_0$, если \FF{f(x_0) = l(x_0)\quad \textit{ и } \quad f(x) \ge l(x) \forall x \in \ang ab .} Если же \FF{f(x_0) = l(x_0) \quad \textit{ и } \quad f(x) > l(x) \forall x \in \ang ab \bsl\{x_0\},} то прфмая называется строго опорной для функции $f$ в точке $x_0$.

\S2. пусть функция $f$ (строго) выпукла вниз на $\ang ab$. Тогда длялюбой точки $x_0 \in (a, b)$ существует (строго) опорная прямая функции $f$ в точке $x_0$.

\D По теореме, в каждой точке $x_0 \in (a, b)$ функция имеет односторонние касательные, а они в свою очередь являются (строго) опорнвмы прямыми.