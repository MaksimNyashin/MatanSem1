(13-17)\\
\q1. \q Аксиомы поля. В множестве $\R$ определены две операции называемые сложением и умножением, действующме из $\R \times \R$ в $\R$ и удовлетворяющие следующим свойствам.

1.1. Сочетательный закон (ассоциативноть) сложения:\\
$(x + y) + z = x + (y + z)$.

1.2. Переместительный закон (коммутативность) сложения:\\
$x + y = y + x$.

1.3. Существует число нуль (0, нейтрвльный элемент по сложению), такое что $x + 0 = x$ для всех $x$.

1.4. Для любого числа $x$ существует такое число $\ol{x}$, что $x + \ol{x} = 0$ (это число называется противоположныи числу $x$ и обозначается $-x$).

1.5. Сочетательный закон (ассоциативность) умножения:\\
$(xy)z = x(yz)$.

1.6. Переместительный закон (коммутативность) умножения:\\
$xy=yx$.

1.7. Существует вещественное число единица (1б нейтральный элемент по умеожению), отличный от нуля, такое что $x\cdot 1 = x$ для всех $x$.

1.8. Для любого числа $x$, отличного от нуля, существует такое $x'$, что $xx' = 1$(это число называется обратным к $x$ и обозначается $x^{-1}$ или $\frac{1}{x}$).

1.9. Распределительный закон (дистрибутивность):\\
$x(y + z) = xy + xz$.

\q2. \q Аксиомы порядка. Между элементами $\R$ определено отношение $\le$ со следующими своствами.

2.1. Для любых $x, y$ верно $x \le y$ или $y \le x$.

2.2 Транзитивность: если $x \le y, y\le z$, то $x\le z$.

2.3. Если $x \le y$ и $y \le x$, то $x = y$.

2.4. Если $x \le y$, то $x + z \le y + z$ для любого $z$.

2.5. Если $0 \le x$ и $0 \le y$, то $0 \le xy$

\q3. \q Аксиома Архимеда. Каковы бы ни были положительные числа $x, y \in \R$, существует такое натуральное число $n$, что $nx > y$.

\q4. \q Аксиома Канторва о вложенных отрезках.\\
Пусть $\{[a_n, b_n]\}^\infty_{n = 1}$~--- последовательность вложенных отрезков, то есть\\
$a_n \le a_{n + 1} \le b_{n+1} \le b_n \forall n \in \N$.\\
Тогда существует точка, принадлежащая одновременно всем отрезкам $[a_n, b_n]$, то ечть\\
$\bigcap\limits^\infty_{n = 1} [a_n, b_n] \neq \varnothing$.\\
Важно, что в определение отрезки, так как в случае $(0, \frac{1}{n}]$ пересечение равно $\varnothing$.