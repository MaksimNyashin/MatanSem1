(70-75)

\Op Точка $a$ называется предельной точкой  или точкой сгущения множества $D$, если в любой окрестности точки $a$ найдётся точка множества $D$, отличная от $a$.\\
$\forall \dot{V_a}\ \exists x \in \dot{V_a} \cap D$

\Op Множество $D$ называется замкнутым (в $X$), если он содержит все свои предельные точки.

\T Множество открыто тогда и только тогда, когда его дополнение замкнуто.

\D Пусть $D^c$ замкнуто. Возьмём точку $x \in D$ и докажем, что $x$~--- внутренняя точка $D$; в силу произвольности $x$ это и будет обозначать открытость $D$. Поскольку $x \notin D^c$, а $D^c$ замкнуто $x$ не является прелельнойточкой $D^c$, то есть существует такая окрестность $\dot{V_x} \cap D^c = \varnothing$, тогда $V_x \cap D^c = \varnothing$, то есть $V_x \in D$, то есть $x$~--- внутрення точка.\\
Пусть $D$ открыто. Возьмём точку $x$, предельную для $D^c$, и докажем, что $x\in D^c$; в силу произвольности $x$ это и будет означать замкнутость $D^c$. Поскольку в любой окрестности точки $X$ найдётся точка $D^c$, $x$ не является внутренней точкой $D$, $x \notin D \Ra x \in D^c$

\T \q Свойства замкнутых множеств.\\
1. Пересечение любого семейства замкнутых множеств замкнуто.\\
2. Объединение конечного семейства замкнутых множеств замкнуто.

\D Следует из аналогичной теоремы для открытых множеств и формул.\\
$\ds\bigcap\limits_{\alpha \in A} F_\alpha = (\bigcup\limits_{\alpha \in A} F^c_\alpha)^c \qquad \bigcup\limits^n_{k = 1} F_k = (\bigcap^n_{k = 1}F^c_k)^c$

\Zam1. $\bigcup\limits_{q \in \Q} \{q\} = \Q$ не замкнуто в $\R$. Так как в любом интервале есть рациональное число, все точки числовой прямой являются предельными.

\Op Точка $a$ называется точкой прикосновения множества $D$, если в любой её окрестности есть точка множества $D$

\Op Множество всех точек соприкосновения называется замыканием $D$ и обозначается $\ol{D}$ или Cl$D$.

\Zam2.Точки соприкосновения~--- объединени множества предельных и изолированых точек $D$.

\Zam4. Точка $a$~--- точка прикосновения множества $D$ тогда и толко тогда, когда существует последовательность $\{x_n\}$ точек множества $D$ стремящаяся к $a$.

\D  Пусть $a$~--- точка прикосновения $D$/Если $a \in D$, то можно взять станционарную последовательность, все члены которой равны $a$. Иначе $a$~--- предельная точка $D$ и искомая последовательность сущесивует по замечанию к определению предельной точки.\\
Обратно, если существует последовательность $\{x_n\}$ с перечисленными свойствами, то по определению предела в любой окрестности точки $a$ найдётся член этой последовательности (туда даже попадают все члены, начиная с некоторого номера), то есть точка множества $D$.

\Zam4. Замыкание $D$ есть:\\
а) Пересечение всех замкнцтых множеств, содержащих $D$\\
б) Минимальное по включению замкнутое множество, содержащее $D$.

\D Пусть $F$~--- пересечение всех замкнутых множеств, содержащих $D$. Тогда $D \subset F$, $F$ содержится в любом замкнутом множестве, содержащем $D$, и $F$ замкнуто по теореме, то есть $F$~--- минимальное по включению замкнутое множество содержащее $D$. Если $x \in \ol{D}$ , то есть $x$ точка прикосновения $F$, а тогда $x \in F$ в силу замкнутости $F$. С другой стороны , если $x \notin \ol{D}$, то у точки $x$ существует окрестность $V_x$, содержащаяся в $D^c$. Тогда её дополнение $V^c_x$ замкнуто и содержит $D$, поэтому $F \subset V^c_x$, то есть $V_x \subset F^c$, тогда $x\notin F$;

\Zam5 Множество замкнуто тогда и толко тогда, когда оно совпадает со своим замыканием.

\Op Внутренность дополнения множества $D$ называется внешностью $D$ и обозначается Ext$D$

\Op Точка $a$ называется граничной точкой множества $D$, если\\
$\forall V_a \exists x_1\in D,\ x_2\in D^c: \{x_1, x_2\} \subset V_a$.

\Op множество всезх граничных точек называется границей и обозначается Fr$D$ или $\delta D$

\Op Множество всех предельных точек множества $D$ называется производным множеством множества $D$ и обозначается $D'$

\Zam6. 1. Ext$D = (\ol{D})^c$\\
2. $\delta D = \ol{D}\bsl \mathring{D}$\\
3. Граница замкнута\\
4. Множество $D'$ замкнуто