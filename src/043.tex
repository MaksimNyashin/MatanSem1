(130)

\T \q Больцано-Коши О промежуточном значении непрерывной функции. Пусть $f$ непрерывна на $[a, b]$. Тогда для любого числа $C$, лежащего между $f(a)$ и $f(b)$ найдётся такое $c \in (a, b)$, что $f(c) = C$.

\D 1. Пусть числа $f(X), f(b)$ разных знаков: $f(a)f(b) < 0$; докажем, что существует такая точка $c \in (a, b)$, что $f(c) = 0$. Не умаляя общности, будем считать, сто $f(a) < 0 < f(b)$; второй случай рассматривается аналогично. Рассмотрим середину отрезка $[a, b]$~--- точку $\frac{a + b}2$. Если $f(\frac{a + b}2) = 0$, то теорема доказана~--- можно положить $c = \frac{a +b}2$. Иначе
\F{$[a_1, b_1 = \begin{cases}[\frac{a + b}2, b], & f(\frac{a + b}2) < 0,\\ [a, \frac{a + b}2], & f(\frac{a + b}2) < 0. \end{cases}$} В обоих случаях $f(a_1) < 0 < f(b_1)$. Продолжим этот процесс построения помежутков. Если процес не завершится (не будет найдена точка $c$), то будкт построена последовательность вложенных отрезков, таких что $f(a_n) < 0 < f(b_n)$. При этом отрезки стягивающиеся, так как $b_n - a_n = \frac{b - a}{2^n} \to 0$. По теореме о стягивающихся отрезках существует единственная точка $c$ принадлежащая одновременно всем отрезкам, при этом $a_b \to c,\s b_n \to c$. По теореме о предельном переходе в неравенстве $f(c) \le 0 \le f(c)$, то есть $f(c) = 0$.

2. Докажем теорему в общем случае. Пусть $\varphi = f - C$. Тогда $\varphi \in C[a, b]$ как разность непрерывных функций, $\varphi(a)\varphi(b) < 0$. По доказанному существует такая точка $c \in (a, b)$, что $\varphi(c) = 0$, то есть $f(c) = C$.

\Zam1. Теорему можно переформулировать так: если непрерывная на промежутке функция принимает два какие-то два значения, то она принимает все значения, лежащие между ними.\\
Здесь существенно и то, что функция непрерывна, и то, что она задана на промежутке. Функция sign, заданная на $\R$, разрывна в 0. Она принимает значения --1 и 1, но из чисел между ними только 0. Сужение функции на $\R\bsl\{0\}$ непрерывно, но не принимает значений, лежвщих между -1 и 1.

\Zam4. Другой способ доказательства теоремы Больцано-Коши~--- поверить, что если $f \in C[a, b],\s f(a) < 0 < f(b)$, то точка \F{$c = \sup\{x \in [a, b]: f(X) < 0\}$} есть корень функции $f$.