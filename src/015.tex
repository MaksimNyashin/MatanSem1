Билет 15: Бесконечно большие и бесконечно малые. Арифметические действия над бесконечно большими
(65-67)

\Op Говорят, что вещественная последовательность $\{x_n\}$ стремится к\\
1) плюс бесконечености, и пишут\\
$\lim\limits_{n\to\infty} x_n = +\infty$ или $x_n \xra[n\to\infty]{} +\infty$, если\\
$\forall E > 0\ \exists N \in \N\ \forall n \in \N: n > N\ x_n > E$;\\
2) минус бесконечности, и пишут\\
$\lim\limits_{n\to \infty} x_n = -\infty$ или $x_n \xra[n\to \infty]{} -\infty$, если\\
$\forall E > 0\ \exists N\in\N\ \forall n\in\N: n > N\ x_n < -E$;\\
3) бесконечности, и пишут\\
$\lim\limits_{n\to \infty} x_n = \infty$ или $x_n\xra[]{n\to\infty} \infty$, если\\
$\forall E > 0 \exists N \in \N\ \forall b \in \N: n > N\ |x_n| > E$

\Zam1. Из определений 1 и 2 следует третье, обратно неверно $x_n = (-1)^nn$

\Op Последовательность называется бесконечно большой, если стремится к бесконечности.

\Zam2. Из определения следует, что если $x_n\to\infty$, то $x_n$ неограничена. Обратное неверно $x_n = (1 + (-1) ^n)n$ не ограничена и не стремится к бесконечности.

\Zam3. Понятно, что последовательность не может стремится одновремнно к бескончености и к конечному пределу, а также к бесконечности разных знаков. Тогда предел единственный в $\ol{\R}$.

\Zam4. Определение достаточно проверять лишь для достаточно больших $E$, можно также опустить требование $E > 0$.

\Zam5. Определение сходящейся последовательности не меняется:: последовательность называется сходящейся, если она имеет конечный предел. Бесконечно большие последовательности считаются расходящимися.

\L \q Связь между бесконечно большими и бесконечно малыми. Пусть $\{x_n\}$~--- числовая последовательность $x_n\neq 0$ ни прикаком $n$. Тогда последовательность $\{x_n\}$~--- бесконечно большая в том и только том случае когда $\{\frac{1}{x_n}\}$~--- бесконечно малая.

\D Пусть $x_n\to\infty$. Возьмём $\eps > 0$ и для числа $E = \frac{1}{\eps}$ подбереём $N$, что для всех $n > N$ $|x_n| > E$. Последнее равносильно $|\frac{1}{x_n} < \eps|$, что и означает стремление $\frac{1}{x_n}$ к нулю. в обратную сторону аналогично.

\T \q Арифметические действия над бесконечно большими. Пксть $\{x_n\}, \{y_n\}$~--- числовые последовательности.\\
1. Если $x_n \to +\infty, \{y_n\}$ ограничена снизу, то $x_n + y_n\to +\infty$\\
2. Если $x_n\to -\infty, \{y_n\}$ ограничена сверху, то $x_n + y_n \to -\infty$\\
3. Если $x_n \to \infty, \{y_n\}$ ограничена, то $x_n + y_n \to \infty$\\
4. Если $x_n \to \pm\infty, y_n \ge b > 0$ для всех $n$ (или $y_n \to b_1 > 0$), то $x_n \to \pm\infty$\\
5. Если $x_n \to \pm\infty, y_n \le b < 0$ для всех $n$ (или $y_n \to b_1 < 0$), то $x_n \to \mp\infty$\\
6. Если $x_n \to \infty, |y_n| \ge b > 0$ для всех $n$ (или $y_n \to b_1 \neq 0$), то $x_n \to \infty$\\
7. Если $x_n \to a\neq 0, y_n \to 0, y_n\neq 0$; при всех $n$, то $\frac{x_n}{y_n} \to \infty$\\
8. Если $x_n\to  a \in \C, y_n\to\infty$, то $\frac{x_n}{y_n} \to 0$\\
9. Если $x_N \to \infty, y_n \to b \in \C, y_n\neq 0$ при всех $n$, то $\frac{x_n}{y_n} \to \infty$.

\D Докажем отверждения 1, 6, 8\\
1. Возьмём $E > 0$. По определению ограниченности снизу ,найдётся такое число $m \in \R$, что $y_n \ge$ 0 при всех $n$. По определени. бесконечного предела существует такой номер $N$, что $x_n > E - m$ для всех $n > N$. Тогда для всех $n > N$ $x_n + y_n > E - m + m > E$.\\
6. Пусть $|y_n| \ge b > 0$ для всех $n$. Возьмём $E > 0$. По определению бесконечного предела существует такой номер $N$, что $|x_n| > \frac{E}{b}$ для всех $n > N$. Тогда для всех $n > N$ $|x _n y_n| > \frac{E}{b} \cdot b = E$\\
Gecnm $y_n \to b_1 \neq 0$. Положим $b = \frac{|b_1|}{2}$, если $b_1$~--- число и $b = 1$, если $b_1$~--- бесконечность. Тогда начиная с некоторого номера $|y_n| \ge b$ и применимо толко что доказанное.\\
8. По теореме о пределе произведения и лемме $\frac{x_n}{y_n} = x_n \cdot\frac{1}{y_n}\to a\cdot 0 = 0$

\Zam1. Часть утверждений теорем об арифметических свойствах можно объединить следующей формулировкой. Есле $\{x_n\}, \{y_n\}$`--- вещественные последоватедьности, $x_n \to x_0 \in \ol{\R}, y_n\ to y_0, \in \ol{\R}$, знак $*$ означает одно из четырёх арифметических действий и $x_0*y_0$ определено в $\ol{\R}$, то $x_n*y_n \to x_0*y_0$ \\
Теорема не позволет сделать заключение о значении в следующих четырёх случаях\\
1. $x_n \to +\infty, y_n \to -\infty\qquad x_n + y_n \to ?$\\
2. $x_n \to 0, y_n \to \infty\qquad x_n y_n \to ?$\\
3. $x_n \to 0, y_n \to 0 \qquad \frac{x_n}{y_n} \to ?$\\
4. $x_nto \infty, y_n \to \infty\qquad \frac{x_n}{y_n}\to ?$