(78)

\T \q Простейшие свойства компактов. Пусть $(X, p)$~--- метрическое пространство, $K \subset X$\\
1. Если $K$ компатно, то $K$ замкнуто и ограничено.\\
2. Если $X$ компактно, а $K$ замкнуто, то $K$ компакт.

\D 1. Докажем, что $K^c$ открыто. Возьмём точку $a\in K^c$ и докажем, что $a$~--- внутренняя точка $K^c$; в силу произволности $a$ это и будет означать, что $K^c$ открыто. Для каждой точки $q \in K$ положим\\
$r_q = \frac{p(q, a)}{2},\quad V_q = B(a, r_q)\quad W_q = B(q, r_q)$\\
Тогда $V_q \cap W_q = \varnothing$. Семейство $\iseq WqK$~--- открытое покрытие компакта $K$. Извлечем из него конечное подпокрытие $\{W_{q_i}\}^N_{i = 1}$: $K \subset \dsl \bigcup{i=1}N W_{q_i} = W$. Тогда $V = \dsl \bigcap{i = 1}N V_{q_i}$~--- окрестность точки $a$, причём $V\cap W = \varnothing$. Тем более $V \cap K = \varnothing$, то есть $V \in K^c$.\\
Докажем, что $K$ ограничено. Зафиксируем точку $a \in X$ и рассмотрим покрытие множества $K$ открытыми шарами $\{B(a, n)\}^\infty_{n = 1}$. В силу компактности $K$ покрывается конечным набором шаров $\{B(a, n_i)\}^N_{i = 1}$ и, следовательно, содержится в шаре $B(a, \dsl\max{1 \le \le N}{} n_i)$.\\
2. Пусть $\iseq G\alpha A$~--- открытое покрытие $K$. Тогда, поскольку $K$ замкнуто, $\iseq G\alpha A \cap \{K^c\}$~--- открытое покрытие $X$. Пользуясь компактностью $X$ извлечём из него конечное подпокрытие $X$: $X = \dsl\bigcap{i=1}N G_{\alpha_i}\cup K^c$. Но тогда $\se {G_{\alpha_i}}{i = 1}N$~--- покрытие $K$.