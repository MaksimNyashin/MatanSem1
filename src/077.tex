(238)

\T \q Неравенство Йенсена. Пусть функция $f$ выпукла вниз на $\ang ab,\s n \in \N$. Тогда для любых $x_1,\cdots, x_n \in \ang ab$ и $p_1, \cdots, p_n > 0$ \FF{f\lr{(\frac{\dsl\sum{k = 1}n p_kx_k}{\dsl\sum{k = 1}n p_k}}) \le \frac{\dsl\sum{k = 1}n p_k f(x_k)}{\dsl\sum{k = 1}n p_k}. }

\Zam1. Числа $p_k$ называются весами, а отношение $\frac{\dsl\sum{k = 1}n p_k x_k}{\dsl\sum{k = 1}n p_k} $~--- взвешенным средним (арифметическим) чисел $x_1, \cdots, x_n$. Неравенство Йенсена можно сформулировать так: значение выпуклой вниз функции от взвешенного среднего не превосходит взвешенного среднего значений функций.

\Zam2. Не уменьшая общности можно считать что $\dsl\sum{k = 1}n p_k =1$. При этом условие неравенства Йенсена принимают вид \FF{f\lr{(\dsl\sum{k = 1}n p_k x_k}) \le \dsl\sum{k = 1}n p_k f(x_k) .}

\D Пусть сумма $p_k$ равна 1. Положим \FF{x^* = \dsl\sum{k = 1}n p_k x_k.} Сразу отметим, что если $x_1 = \cdots = x_n$, то $x^*$ с ними совпадает, а неравенство Йенсена обращается в равенство.\\
Пусть среди $x_1,\cdots, x_n$ есть различные. Проверим, что $x^* \in (a, b)$. Действительно, хоть одно из чисел $x_k$ меньше $b$, поэтому \FF{x^* < \dsl\sum{k = 1}n p_k b = b.} Аналогично, $a < x^*$.\\

В точке $x^*$ у функции $f$ существует опорная прямая; пусть она задается уравнением $l(x) = \alpha(x) + \beta$. По определению опорной прямой $l(x^*) = f(x^*)$ и $l(x_k) \le f(x_k)$ при всех $k$. Поэтому, \FF{f(x^*) = l(x^*) = \alpha\dsl\sum{k = 1}n p_k x_k + \beta = \dsl\sum{k = 1}n p_k(\alpha x_k + \beta) = \dsl\sum{k = 1}n p_k l(x_k) \le \dsl\sum{k = 1}n p_k f(x_k) }

\Zam3. Если $f$ строго выпукла, а среди $x_k$ есть различные, то неравенство Йенсена строгое.

\Zam. При $n = 2$ неравенство Йенсена совпадает с неравенством из определения выпуклости.