(206)

\Op Пусть $n \in \N$ функция $f$ дифференцируема $n$ раз в точке $x_0$, или $n = 0$, а функция непрерывна точке $x_0$. Многочлен \F{$T_{n, x_0}f(x) = \dsl\sum{k = 0}n \frac{\pb fk (x_0)}{k!} (x - x_0)^k$} называется многочленом Тейлора порядка $n$ функции $f$ с центром в точке $x_0$. Разность \F{$R_{n, x_0}f(x) = f(X) - T_{n, x_0}f(x)$} называют остаточным членом или остатком формулы Тейлора, а равенство \F{$f(x) = T_{n, x_0}f(x) + R_{n, x_0}f(x)$} --- формулой Тейлора.


\T \q Формула Тейлора -- Пеано. Пусть $n \in \N$, функция $f: \ang ab \to \R$ дифференцируема $n$ раз в точке $x_0 \in \ang ab$. Тогда \F{$f(x) = \dsl\sum{k = 0}n \frac{\pb fk (x_0)}{k!} (x - x_0)^k + o((x - x_0)^n),\quad x\to x_0$.}

\D Для краткости будем писать $T = T_{n, x_0} f,\s R = R_{n, x_0}f$. Требуется доказать, что $R(x) = o((x - x_0)^n)$. Поскольку $R = f - T$, а $\pb Tm (x_0) = \pb fm(x_0)$ при всех $m \in [0: n]$, имеем $\pb Rm(x_0) = 0$ при всех $m \in [0: n]$.\\
Поэтому достаточно доказать, что если $n \in \N$, функция $R$ дифференцируема $n$ раз в точке $x_0$ и $\pb Rm (x_0 = 0)$ при всех $m \in [0: n]$, то $R(x) = o((x - x_0)^n)$ при всех $x \to x_0$. Докажем по индукции по $n$.\\
База индукции $n = 1$. Так как $R(x_0) = R'(x_0) = 0$, по определению дифференцируемости получаем \F{$R(x) = R(x_0) + R'(x_0)(x- x_0) + o(x - x_0) = o(x - x_0) = o(x - x_0).\quad x\to x_0$.}
Индукционный переход: предположим, что для номера $n$ утверждение верно; докажем для номера $n + 1$. Пусть $\pb Rm(x_0) = 0$ при всех $m \in [0: n + 1]$ докажем, что \F{$\frac{R(x)}{(x - x_0)^{n + 1}} \xra[x \to x_0]{} 0.$} Доказательство будем вести на язвке последовательностей. Возбмём последовательность $\{x_\nu\}$ со свойствами $x_0 \in \ang ab,\s x_\nu \neq x_0,\s x_\nu \to x_0$. Тогда для каждого $\nu$ по формуле Лагранжа найдётся такая точка $c_\nu$, лежащая между $x_\nu$ и $x_0$, что \F{$\frac{R(x_\nu)}{(x_\nu - x_0) ^{n + 1}} = \frac{R(x_\nu) - R(x_0)}{(x_\nu - x_0)^{n + 1}} = \frac{R'(c_\nu)}{(x_\nu - x_0)^n}$.} Из неравенства $|x_\nu - x_0| < |x_\nu - x_0|$ следует,что $c_\nu \to x_0$. По индукционному предположению, применненному к функции $R'$ у которой все производные до $n$-ной включительно в точке $x_0$ равны 0, \F{$|\frac{R(x_\nu)}{(x_\nu - x_0)^{n + 1}}| \le |\frac{R'(x_\nu)}{(c_\nu - x_0)^n}| \to 0$,} что и требовалось доказать.