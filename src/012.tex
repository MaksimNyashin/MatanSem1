(53)

\Op Последовательность вещественных или комплексных чисел называется бесконечно малой, если она стремится к нулю.

\L Произведение бесконечно малой последовательности на ограниченную есть бесконечно малая: $\{x_n\}$~--- бесконечно малая числовая последовательность, $\{y_n\}$~--- ограниченная числовая последовательность, тогда $\{x_n y_n\}$~--- бесконечно малая.

\D В силу ограниченности $\{y_n\}$ Найдётся $K > 0$, что $|y_n| \le K$ при всех $n$. Возьмём $\eps > 0$. По определению предела последовательности $\{x_n\}$ существует такой номер $N$, что $|x_n| < \frac{\eps}{K}$ для всех $n > N$. Но тогда для всех $n > N$;\\
$|x_n y_n| < \frac{\eps}{K}	\cdot K = \eps$.\\
В силу произвольности $\eps$ это и означает $x_n y_n \to 0$.

(57)

\T \q Арифметические действия над сходящимися последовательностями в нормированном пространстве. Пусть $(X, ||\cdot||)$~--- нормиованное пространство, $\{x_n\},\ \{y_n\}$ в $X$, $\lambda_n$~--- числовая последовательность $x_0,\ y_0 \in X,\ \lambda_0 \in \R (\textit{или} \C)$, $x_n \to x_0,\ y_n \to y_0,\ \lambda_n \to \lambda_0$. Тогда\\
1. $x_n + y_n \to x_0 + y_0$\\
2. $\lambda_n x_n \to \lambda_0 x_0$\\
3. $x_n - y_n \to x_0 = y_0$
4. $||x_n|| \to ||x_0||$

\T \q Арифметические действия над сходящимися числовыми последовательностями. Gecnm ${x_n},\ {y_n}$~--- числовые последовательности, $x_0, y_0 \in \R$ (или$\C$), $x_n \to x_0,\ y_n\to y_0$. Тогда\\
1. $x_n + y_n \to x_0 + y_0$
2. $x_n y_n \to x_0 y_0$
3. $x_n - y_n \to x_0 - y_0$
4. $|x_n| \to |x_0|$
5. Если $y_n \neq 0$ при всех $n$ и $y_0 \neq 0$, то $\frac{x_n}{y_n} \to \frac{x_0}{y_0}$.

\D

1. Возьмём $\eps > 0$. По определению предела найдутся такие номера $N_1$ и $N_2$, что $||x_n - x_0|| < \frac{\eps}{2}$для всех $n > N_1$, а $||y_n - y_0|| < \frac{\eps}{2}$ для всех $n > \N_2$. Положим $N = \max\{N_1, N_2\}$. Тогда при всех $n > N$ юудет\\
$||(x_n + y_n) - (x_0 + y_0)|| \le ||x_n - x_0|| + ||y_n - y_0|| \le \frac{\eps}{2} + \frac{\eps}{2} = \eps$.

2. По неравенству треугольника\\
$||\lambda_n x_n - \lambda_0 x_0|| = ||(\lambda_n - \lambda_0) x_n + \lambda_0(x_n - x_0)|| \le |\lambda_n - \lambda_0||||x_n|| + ||\lambda_0||||x_n - x_0||$\\
$\{|\lambda_n - \lambda_0|\}$ и $\{||x_n - x_0||\}$~--- бесконечно малые. $\{||x_n||\}$~--- ограничена по теореме. А $\{|\lambda_0|\}$~--- постоянная. Тогда оба слагаемых бесконечно малые, по лемме, тогда их сумма тоже бесконечно малая.

3. $x_n - y_n = x_n + (-1)y_n \to x_0 + (-1)(y_0) = x_0 - y_0$.

4. Следует из $|||x_n|| - ||| \le ||x_n - x_0||$ и теоремы о двух милиционерах.

5. Достаточно доказать, что $\frac{1}{y_n} \to \frac{1}{y_0}$. Поскольку\\
$\frac{1}{y_n} - \frac{1}{y_0} = (y_0 - y_n)\cdot\frac{1}{y_0}\cdot\frac{1}{y_n}$\\
$\{y_0 - y_n\}$~-- бесконечно малая, $\{\frac{1}{y_0}\}$~--- ограниченная. Осталось доказать ограниченность $\{\frac{1}{y_n}\}$.\\
По определению предела для $\eps = \frac{|y_0|}{2} > 0$ существует номер $N$, что $||y_n - y_0|| < \eps$ для всех $n > N$. Тогда при всех $n > N$ по свойствам модуля\\
$|y_n| = |y_0 + y_n - y_0| \ge |y_0| - |y_n - y_0| > |y_0| - \eps = \frac{|y_0|}{2}$\\
Обозначим $k = \min\{|y_1|, \cdot, |y_n|, \frac{|y_0|}{2}\}$. $k > 0$ и $|y_n| \ge k$ при всех $n$. Следовательно $|\frac{1}{y_n}|\le \frac{1}{k}$ при всех $n$, что и зощначает ограниченность $\{\frac{1}{y_n}\}$