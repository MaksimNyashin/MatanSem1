(114)\\
Пусть $(X, \rho_X)$ и $(Y, \rho_Y)$~--- метрические пространства, $f: D\subset X \to Y$, $x_0 \in D$. Отображение $f$ назывется непрерывным в точке $x_0$ если выполняется одно из следующих утвержений.\\
\q1. Предел отображения $f$ в точке $x_0$ существует и равен $f(x_0)$. (Применимо, если $x_0$~--- предельная точка D).\\
\q2. \q На $\varepsilon$-языке или по Коши.\\
$\forall \varepsilon > 0\ \exists \delta > 0\ \forall x \in D:\ \rho_X(x, x_0) < \delta \Ra \rho_Y(f(x), f(x_0)) < \varepsilon$\\
\q3. \q На языке окрестностей.\\
$\forall V_{f(x_0)}\ \exists V_{x_0}\ f(V_{x_0} \cap D) \subset V_{f(x_0)}$\\
\q4. \q На языке последовательностей или по Гейне.\\
$\forall \{x_n\}:\ x_n \in D, x_n \to x_0\ f(x_n) \to f(x_0)$\\
\q5. Бесконечно малому приращению аргумента соответствует бесконечно малое приращение отображения. $\Delta y \xra[\Delta x \to \theta_X]{} \theta_Y$

Отображение называется непрерывным на множестве $D$, если оно непрерывно в каждой точке множества $D$.\\
Множество отображений $f: D\subset X \to Y$ непрерывных на множестве $D$, обозначают $C(D\subset X\to Y)$ или $C(D\to Y)$

\Zam 1. Равносильность определений, когда $x_0$ предельная точка $D$, следует из равносильности различных определений предела. Под номерами 2, 3, 4 записан тот факт, что точка $a = f(x_0)$ является пределом отображения $f$ в точке $x_0$ с одним отличием в каждом случае. 2: опущено условие $x \neq x_0$; 3: окрестность не проколота; 4: опущено условие $x\neq x_0$. Так как это ничего не портит. Определение 5 на любом из языков записывается также, как и определение 1.

\Op пусть $f: D \subset X \to Y, x_0 \to D$. Если отображение $f$ не является непрерывным в точке $x_0$, то говорят, что $f$ разрывно (терпит разрыв, испытывает разрыв)  вточке $x_0$, а точку $x_0$ называют точкой разрыва.

\Pr1. Функция сигнум
\F{sign$x = \begin{cases}1, & x> 0\\
0, & x = 0,\\
-1, & x < 0\end{cases}$} Тогда $f(0+) = 1,\s f(0-) = 1,\s$ 0~--- точка неустранимого разрыва превого рода.

\Pr2. $f(x) = |$sign$(x)|$ Тогда $f(0+) = f(0-) = 1$ и 0~--- точка рустранимого разрыва первого рода.

\Pr3. $f(x) = \frac 1x$ ТОгда $f(0+) = +\infty,\s f(0-) = -\infty$ и 0~--- точка разрыва второго рода.

\Pr4. $f(x) = \frac 1{x^2}$. Тогда $f(0+) = f(0-) = +\infty$ и 0~--- точка разрыва второго рода

\Pr5. $f(x) = \frac{x - 1}{x^2 - 1}$, тогда $f$ определена на $\R \bsl \{-1, 1\}$ и на области определения $f(x) = \frac 1{x + 1}$. Точка --1~--- точка разрыва второго рода, 1~--- точка устранимого разрыва. положим $f(1) = \frac 12$. получим непрерывную  точке 1 функцию.