Билет 59: Теорема Ролля
(189)

\T \q Ролля. Пусть функция $f$непрерывна на $[a, b]$, дифференцируема на $(a, b)$ и $f(a) = f(b)$. Тонда найдётся такая точка $x \in (a, b)$, что $f'(c) = 0$.

\D По теореме Вейерштрасса, существуют точки $x_1, x_2 \in [a, b]$, что $f(x_1) = \dsl\max{x \in [a, b]}{} f(x),\s f(x_2) = \dsl\min{x \in [a, b]}{} f(x)$. Если $x_1, x_2$~--- концевые точки $[a, b]$, то по условию $f(x_1) = f(x_2)$,то есть наибольшее и наименьшее значения $f$ совпадают, поэтому $f$ постоянна на $[a, n]$ и в качестве $c$ можно взять любую точку $(a, b)$. Если же $x_1$ или $x_2$ лежит в $(a, b)$, то, по теореме Ферма, $f'(x_1) = 0$ или $f'(x_2) = 0$; поэтому можно положить $c = x_1$ или $c = x_2$.

\Zam1. Геометрический смысл теормы Ролля: в условиях теоремы найдётся точка с горизонтальной касательной.

\Zam2. Все условия теоремы Ролля существенны. $f(X) = x (x \in [0, 1)) f(1) = 0$ разрывна в точке 1. Функция $f(x) =\sqrt{|x|} (x \in [-1, 1])$ не имеет производной в точке 0. $f(x) = x$ принимает разные значения на концах, но остальным условиям удовлетворяет.

\Zam3. Из дифференцируемости $f$ следует ее непрерывность, поэтому заключение теоремы выролняется для дифференцируемых на $[a, b]$ функций. В теореме Ролля функции разрешается не иметь производной на концах. Так отрезок $f(x) = \sqrt{1 - x^2}$ не дифференцируема на концах отрезка $[-1, 1 ]$, но условиям теоремы удовлетворяет.

\Zam 4. Из теоремы Ролля следует, что между любыми двумя нулями дифференцируемой функции всегда лежит нольеё производной.