(216)

\T \q Применения формулы Тейлора для раскрытия неопределённостей. Пусть $f, g :\ang ab \to\R,\s x_0 \in \ang ab,\s n\ \in\N$,функции $f, g$ дифференцируемы $n$ раз в точке $x_0$: \FF{f(x_0) = f'(x_0) = \cdot = \pb f{n - 1}(x_0) = 0,} \FF{g(x_0) = g'(x_0) = \cdots = \pb g{n - 1}(x _0) = 0.} $\pb gn(x_0) \neq 0$. Тогда \FF{\li x{x_0} \frac{f(x)}{g(x)} = \frac{\pb fb(x_0)}{\pb gn (x_0)}.}

\D По формуле Тейлора -- Пеано при $x\to x_0$ \FF{f(x) = \frac{\pb fn (x_0)}{n!}(x - x_0)^n + o((x- x_0)^n),} \FF{g(x) = \frac{\pb gn(x_0)}{n!}(x -x_0)^n + o((x - x_0)^n) .} Поскольку $\pb gn(x_0) \neq 0$, существует такая окрестность $V_{x_0}$, что $g(x) \neq 0$ для любого $x \in \dot{V}_{x_0} \cap \ang ab$. Значит частное $\frac{f(x)}{g(x)}$ определено при всех таких $x$. Сокращая дробь на $\frac{(x - x_0)^n}{n!}$ получаем \FF{\frac{f(x)}{g(x)} = \frac{\pb fn(x_0) + o(1)}{\pb gn(x_0) + o(1)} \xra[x \to x_0]{} \frac{\pb fn(x_0)}{\pb gn(x_0)} }