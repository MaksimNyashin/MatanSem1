(84)

\Op Пусть $\seq xn1\infty$~--- последовательность в метрическом пространстве $X$. Говорят, что последователбность $x_n$ сходится в себе, если\\
$\forall \eps > 0\ \exists N\ \forall n, l > N\ \rho(x_n, x_l) < \eps$.

\L 1. Сходящаяся в себе последователность ограничена\\
2. Если у сходящейся в себе последовательности есть сходящаяся подпоследовательность, то сама последовательность сходится.

\D 1. Пользуясь сходимостью $\{x_n\}$ в себе, подберём такой номер, что для всех $n, l > N$ будет $\rho(x_n, x_l) < 1$. В частночти, тогда $\rho(x_n, x_{N + 1}) < 1$ для всех $n > N$. Пусть $b \in X$. Следовательно, для всех $n > N$ по неравенству треугольника $\rho(x_n, b) < 1 + \rho(x_n + 1, b)$. Положим\\
$\R = \max\{\rho(x_1, b), \cdots, \rho(x_N, b), 1 + \rho(x_{N + 1}, b)\}$\\
Тогда $\rho(x_n, b) \le R$ для всех номеров $n$.\\
2. Пусть $\{x_n\}$ сходится в сеюе $x_{n_k} \to a$. Возьмём $\eps > 0$. По определению предела, найдётся такой нмер $K$, что $\rho(x_{n_k}, q) < \frac{\eps}{2}$ для всех $k > K$, а по определнию сходимости в себе, найдётся такой номер $N$, что $\rho(x_n, x_l) < \frac{\eps}{2}$ для всех $n, l > N$. Покажем, что найженное $N$~--- требуемое для $\eps$ из определения предела. Пусть $n > N$ Положим $M = \max\{N +1, K +1\}$; тогда $b_M \ge n_{N + 1} \ge N$ и аналогично $n_M \ge K$. Следовательно,\\
$\rho(x_n, a) \le \rho(x_n, x_{n_M}) + \rho(x_{n_M}, a) < \frac{\eps}{2} + \frac{\eps}{2} = \eps$\\
В силу произвольности $\eps$ это и означает, что $x_n \to a$.

\T 1. Во всяком метрическом пространстве любая сходящаяся помледовательность сходится в себе.\\
2. В $\R^m$ любая сходящаяся в себе последовательность сходится.

\D 1. Обозначим $\lim x_n = a$. Возьмём $\eps > 0$. По определению предела, найдётся такой номер $N$, что $\rho(x_n, a) < \frac{\eps}{2}$ для всех $n > N$. Тогда для всех $n, m > N$\\
$\rho(x_n, x_m) \le \rho(x_n, a) + \rho(a, x_m) < \frac{\eps}{2} + \frac{\eps}{2} =\eps$.\\
В силу произвольности $\eps$ это и значит, что $\{x_n\}$ сходится в себе.\\
2. Пусть $\pb xn$~--- сходящаяся в себе последовательность в $\R^m$. По пункту 1 леммы, она ограничена. По принципу выбора Болцано-Вейерштрасса из неё можно извлечь сходящуюся подпоследовательность, а тогда по пункту 2 леммы она сама сходится.

\Op Если в метрическом протсранстве $X$ любая сходящаяся в себе последователбность сходится, то это пространство называется полным.

\Zam1. Второе утверждение теоремы можно сформулировать так: пространство $\R^m$ полно.\\
Пример неполног пространства $\Q$, как подпротсранство $\R$. Если взять десятичные приближения к $\sqrt{2}$, то она будет сходиться в себе, но небудет иметь предела в $\Q$.

\Zam2. Утвержение о том, что в $\R^m$ сходимость и сходимость в себе равносильны, называют критерием Больцано - Коши сходимости последовательности\\
В пространстве $\R^m$ последовательность $\{\pb xn\}$ сходится тогда и толко тогда, когда\\
$\forall \eps > 0\ \exists N \in \N:\ \forall n, l > N|\pb xn - \pb xl| < \eps$\\
Критерий позволяет доказывать существавание предела, не используя само значения предела.