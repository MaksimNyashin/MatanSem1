(222)

\T \q Необходимое суловие экстремума. Пусть $f: \ang ab \to \R,\s x_0 \in (a, b)$~--- точка экстремума$f,\s f$ дифференцируема в точке $x_0$ Тогда $f'(x_0) = 0$.

\D По определению точки экстремума существует такое $\delta > 0$, что \FF{f(x_0) = \dsl\max{x \in (x_0 - \delta, x_0 + \delta)}{} f(x) \textit{ или } f(x_0) = \dsl\min{x \in (x_0 - \delta, x_0 + \delta)}f(x) .} Остается применить теорему Ферма к функции $f|_{(x_0 - \delta, x_0 + \delta)}.$

\Zam1. Как и в теореме Ферма, существенно, что $x_0$~--- внутренняя точка промежутка.

\Zam2. Условие $f'(x_0)$ не является достаточным $f(x) = x^3$.

\Zam3. Функция может быть не дифференцируемой в точке экстремума $f(x) = |x|$.

\T \q Первое правило исследования критических точек. Пусть $f: \ang ab \to \R,\s x_0 \in (a, b)$, функция $f$ непрерывна в точке $x_0$ и дифференцируема на $(a, b) \bsl \{x_0\}$ и существует такое $\delta > 0$, что $f'$ сохраняет знак на $(x_0 - \delta, x_0)$ и $(x_0, x_0 + \delta)$. Обозначим производные на этих промежутках $f'_1$ и $f'_2$ соответсвенно.\\
1. Если $f'_1 < 0$ и $f'_2 > 0$, то $x_0$~--- точка строгого минимума $f$.\\
2. Если $f'_1 > 0$ и $f'_2 < 0$, то $x_0$~--- точка строгого минимума $f$.\\
3. Если $f'_1 > 0$ и $f'_2 > 0$, то $x_0$~--- точка строгого возрастания $f$.\\
4. Если $f'_1 < 0$ и $f'_2 < 0$, то $x_0$~--- точка строгого убывания $f$.

\D Для определенности докажем 1 и 3.

1. Функция строго убывает на $(x_0 - \delta, x_0]$ и на $[x_0, x_0 + \delta)$. Поэтому $f(x) < f(x_0)$ как при всех $x\in (x_0 - \delta, x_0)$, так и при $x \in (x_0, x_0 + \delta)$. Значит $x_0$~--- nочка строгого минимума $f$.

2. Функция строго возрастает на $(x_0 - \delta, x_0]$ и на $[x_0, x_0 + \delta)$. Поэтому $f(x) < f(x_0)$ для всех $x \in (x_0 \ delta, x_0)$ и $f(x) > f(x_0)$ для всех $x \in (x_0, x_0 + \delta)$. То есть $x_0$~--- точка cтрогого возрастания.