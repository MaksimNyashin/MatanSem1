(224)

\T \q Второе правило исследования критических точек. Пусть $f: \ang ab \to \R,\s x_0\in(a, b),\s n \in \N$, функция $f$ дифференцируема $n$ раз в точке $x_0$. \FF{f'(x_0) = \cdots = \pb f{n - 1}(x_0) = 0,\quad \pb fn(x_0) \neq 0}
1. Если $n$ четно и $\pb fn(x_0) > 0$, то $x_0$~--- точка строгого минимума $f$.\\
2. Если $n$ четно и $\pb fn(x_0) < 0$, то $x_0$~--- точка строгого максимума $f$.\\
3. Если $n$ нечетно и $\pb fn(x_0) > 0$, то $x_0$~--- точка строгого возрастания $f$.\\
4. Если $n$ нечетно и $\pb fn(x_0) < 0$, то $x_0$~--- точка строгого убывания $f$.

\D Запишем формулу Тейлора -- Пеано: \FF{f(x) = \dsl\sum{k = 0}n \frac{\pb fk(x_0)}{k!}(x - x_0)^k + o((x - x_0)^n).} Учитывая определение символа $o$ и обнуление производных $f$, перепишем это равенство в виде \FF{f(x) - f(x_0) = (x - x_0)^n(\frac{\pb fn(x_0)}{k!} + \varphi(x)),} где $\varphi(x) \xra[x \to x_0]{} 0$. Доопределим $\varphi(x_0) = 0$. Существует такое $\delta > 0$, что для всех $x \in (x_0 - \delta, x_0 + \delta)$ \FF{\sig(f(x) - f(x_0)) = \sig((x - x_0)^n\pb fn(x_0)).} Осталось сравнить знаки сомножителей.

\Zam. Может быть, что функция не постоянная, но $\forall n \in \N,\s \pb fn(x_0) = 0$. \FF{f(x) = \begin{cases}e^{-\frac 1{x^2}}, & x \neq 0,\\0, & x = 0.\end{cases}} \FF{f \in \pb C\infty (\R),\s \forall n \in \N \pb fn(0) = 0.}

\D Очевидно, что $f \in C(\R \bsl\{0\})$\\
1.Докажем, что $\pb fn = P_n(\frac 1x)e^{-1/x^2}$, где $P_n$~--- многочлен какой-то степени. База $n = 0,\s P_0 = 1$.\\
Переход: \FF{\pb f{n + 1}(x) = P'_n\lr{(\frac 1x})e^{-1/x^2}\lr{(-\frac 1{x^2}}) + P_n\lr{(\frac 1x})e^{-1/x^2}\lr{(\frac 2{x^3}})= P_{n + 1}\lr{(\frac 1x})e^{-1/x^2} } 

2. $\forall n \in \Z_+ \exists \pb fn(0) = 0?$\
База: $n = 0$, по заданию функции.\\
Переход: \FF{\pb f{n + 1}(0) = \li{x}{0} \frac{\pb fn (x) - \pb fn(0)}{x} = \li x0 \frac 1x P_n\lr{(\frac 1x})e^{-1/x^2} = 0 }