(194)

\T \q Правило Лопиталя для раскрытия неопределенностей вида $\frac 00$. Пусть $-\infty \le a < b \le +\infty$, функции $f, g$ дифференцируемы на $(a, b),\s g'(t) \neq 0$ для любого $t \in (a, b),\s \li{x}{a+} f(x) = \li{x}{a+} g(x) = 0$ и существует предел \F{$\li{x}{a+}\frac{f'(x)}{g'(x)} = A \in \ol{\R}$.} Тогда предел $\li{x}{a+} \frac{f(x)}{g(x)}$ также существует и равен $A$.

\D 1. Пусть $a \in \R$. Доопределим функции в точке $a$ нулем: $f(a) = g(a) = 0$. Тогда доопределенные функции $f, g$ будут непрерывны на $[a, b)$. Возбмем последовательность $\{x_n\}: x_n \in (a, b),\s x_n \to a$, и докажем, что $\frac{f(x)}{g(x)} \to A$. Функции $f$ и $g$ удовлетворяют условиям теоремы Коши на каждом отрезке $[a, x_n]$. Поэтому для любого $n \in \N$ найдётся такая точка $c_n \in (a, x_n)$, что \F{$\frac{f(x_n)}{g(x_n)} = \frac{f(x_n) - f(a)}{g(x_n) - g(a)} = \frac{f'(c_n)}{g'(c_n)}$.} По теореме о сжатой последовательности, $c_n \to a$. По определению правостороннего предела на языке последовательностей, $\frac{f'(c_n)}{g'(c_n)} \to A$, а тогда  в силу произволности $\{x_n\}$ и $\frac{f(x)}{g(x)} \xra[x \to a+]{} A$.

2. Пусть $a = \infty$. В силк локадьности предела можно считать, что $b < 0$. Положим $\varphi(t) - f(-\frac 1t),\s \psi(t) = g(-\frac 1t) (t \in (0, -\frac 1b))$. Тогда \F{$\varphi'(t) = \frac 1{t^2}f'(-\frac 1t),\quad \psi(t)=\frac 1{t^2}g'(-\frac 1t) \neq 0$,}
\F{$\li{t}{0+} \varphi(t) = \li x{-\infty},\quad \li t{0+} \psi(t) = \li x{-\infty} g(x) = 0$,} \F{$\li x{0+} \frac{\varphi'(t)}{\psi'(t)} = \li x{-\infty} \frac {f'(x)}{g'(x)} = A$.} По доказанному, \F{$\li x{-\infty} \frac{f(x)}{g(x)} = \li t{0+}\frac{\varphi(t)}{\psi(t)} = A$.}

\Zam1. Утверждение, аналогичное теореме справедливы и для левостороннего предела, а следовательно, и для двухстороннего предела

\Pr1. $\li x0\frac{\sin x}{x} = \frac{\cos x}1 = 1$