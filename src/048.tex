(136)

Степенную функцию с показателем $\alpha$, которая $x$ сопоставляет $x^\alpha$, будем обозначать $\epsilon_\alpha: \epsilon(x) = x^\alpha$. Заранее отметим, что области определнения степенных функций могут быть различны при разных показателях.

При $\alpha = 1$ функция $\epsilon_1 =$ id$_\R$, как уже отмечалось, непрерывна на $\R$.

При $\alpha = n\in\N$ по определению, $x^n = x\cdot x\cdot ... \cdot x$ $n$ раз $x \in \R$. Следовательно $\epsilon_n$непрерывна на $\R$, как произведение непрерывных.

При $\alpha = -n$, где $n \in \N$, полагаем \F{$x^{-n} = \frac 1{x^n}, \quad x \in \R \bsl \{0\}$}. Следовательно, функция $\epsilon_{-n}$ непрерывна на $\R\bsl \{0\}$ как частное непрерывных.

При $\alpha = 0$ по определению полагаем $x^0 = 0$ при всех $x \neq 0$, тогда можно в соответствии с общим соглашением доопределить  по непрерывности $x^0 = 1$ и при $x = 0$.

Если $n \in \N,\s n$ нечётно, то функция $\epsilon_n$ строго возрастает на $\R$, $\dsl\sup{x \in \R}{} \epsilon_n(x) = +\infty,\s \dsl\inf{x \in \R}{} \epsilon_n(x) = -\infty$; по теореме о сохранении промежутка $\epsilon_n(\R) = \R$. Если $n \in \N,\s n$ чётно, то функция $\epsilon_n$ строго возрастает на $\R_+, \dsl\sup{x\in \R_+}{} \epsilon_n(x) = +\infty,\s \dsl\min{x \in \R_+}{} \epsilon_n(x) = 0$; по теореме о сохранении промежутка $\epsilon_n(\R_+) = \R_+$. По теореме о существовании и непрерывности обратной функции существует и непрерывна функция \F{$\epsilon_{\frac 1n} = \begin{cases}\epsilon^{-1}_n, & n \text{ нечетно}, \\ (\epsilon_n|_{\R_+})^ {-1}, & n \text{ чётно},\end{cases}$} которая называется корнем $n$-ной степени и обозначается ещё $\sqrt[n]{(\cdot)}: \epsilon_{1/n} = x^{1/n} = \sqrt[n]{x}$. Итак, \F{$\epsilon_{1/n}: \R \xra[]{\textit{на}} \R,\qquad n$ чётно,} \F{$\epsilon_{1/n}: \R_+ \xra[]{\textit{на}} \R,\quad n$ чётно;} $\epsilon_{1/n}$ строго возрастает и непрерывна.

При $\alpha \in \Q$ $\alpha = r = \frac pq$~--- несократимая дробь $p \in \Z,\s q \in \N$. Полагаем \F{$x^r = (x^p)^{1/q}$,} для всех $x$, для которых правая часть имеет смысл. Другими словами $\epsilon_r = \epsilon_(1/q) \circ \epsilon_p$. Тогда $x^r$ определено в следующих случаях \F{$x > 0,\quad r$ любое,} \F{$x = 0,\quad r \ge 0$} \F{$x < 0,\quad q$ нечётно.} Функция $\epsilon_r$ непрерывна на своей области определения; она строго возрастает на $[0, +\infty)$ при $r > 0$, строго убывает на $(0, +\infty)$ при $r < 0$.

(145)

При всех $x > o,\s a \in \R$ по свойству $a^{xy} = (a^x)^y$  верна формула $x^ \alpha = e^{\alpha \ln x.}$ Поэтому степенная функция $\epsilon_\alpha$ непрерывна на $(0, +\infty)$ при всех $\alpha \in \R$. Если $\alpha$ иррационально, то \F{$\epsilon_\alpha:[0, +\infty) \xra[]{\textit{на}}[0, +\infty),\quad \alpha > 0$,} \F{$\epsilon_\alpha:(a, +\infty) \xra[]{\textit{на}}(0, +\infty),\quad \alpha < 0$.}  Непрерывность $\epsilon_\alpha$ в нуле при $\alpha > 0$ также имеет место: если $x_n > 0, x_n \to 0$, то $y_n = \ln x_n \to -\infty$ и $\epsilon_\alpha(x_n) = e^{\alpha y_n} \to 0 = e_\alpha(0)$