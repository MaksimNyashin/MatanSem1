(125)

\T \q Характеристика непрерывности отображения с помощью прообразов. Пусть $X, Y$~--- метрические пространства, $f: X \to Y$. Тогда для непрерывности $f$ на $X$ необходимо и достаточно, чтобы при отображении $f$ прообраз любого открытого в $Y$ множества был открыт в $X$.

\D 1. Пусть $f$ непрерывно и множество $U$ открыто в $Y$. Докажем, что множество $f^{-1} (U)$ открыто в $X$. Для этого возьмём точку $a \in f^{-1}(U)$ и докажем, что $a$~--- внутренняя точка $f^{-1} (U)$. Так как $f(a)\in U$, а $U$ открыто, существует окрестность $V_{f(a)}$, содержащаяся в $U$. По определению непрерывности $f$ в точке $a$ найдётся окрестность $V_a$ такая, что $f(V_a) \subset V_{f(a)} \subset U$. Следовательно $V_a \subset f^{-1}(U)$, то есть $a$~--- внутренняя точка $f^{-1}(U)$.

2. Пусть прообраз любого открытого множества открыт, $a \in X$. Докажем, что $f$ непрерывно в точке $a$; в силу произвольности $a$ это и будет означать непрерывность $f$ на всём $X$. Возьмём окрестностть $V_{f(a)}\subset Y$. По условию её прообраз $G = f^{-1}(V_{f(a)})$ открыт в $X$, при этом $a \in G$. Значит, найдётся окрестность $V_a: V_a \subset G$. Осталось проверить, что $f(V_a) \subset V_{f(a)}$. Тогда определение непрерывности $f$ в точке $a$ на языке окрестностей будет выполнено. Действительно, если $y \in f(V_a)$, то, по определению образа, $y = f(x)$ для некоторого $x \in V_a$; тем более $x \in G$. По определению прообраза $f(x) \in V_{f(a)}$, то есть $y \in V_f(a)$

\Zam1. Пусть $f:[0, 2] \to \R,\s f(X) = x$. Тогда \F{$f^{-1}(1, +\infty) = (1, 2]$.} Противоречий с теоремой нет: полуинтервал $(1, 2]$ открыт в $X = [0, 2]$, хотя и не является открытым в $\R$.