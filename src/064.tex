(199)

Пусть $f: D \subset \R \to \R,\s D_1$~--- множество дифференцируемости $f$, $f': D_1 \to \R$. Каждая точка $x_0 \in D_1$ удовлетворяет следующему условию: существует такое $\delta > 0$, что $(x_0 - \delta, x_0 + \delta) \cap D$~--- невырожденный промежуток.\\
Далее следует назвать $f''$~--- второй производной и т.д.\\
Производная порядка $n$ фунуции $f$ обозначается $\pb fn$. $\pb f1 = f'$, производные высших порядков определяются по индукции.

\Op Пусть $n - 1 \in \N$, множество $D_{n - 1}$ и функция $\pb f{n-1}: D_{n - 1} \to \R$ уже определены. Обозначим, через $D_n$ множество всех точек $x_0 \in D_{n - 1}$, для которых существует такое $\delta > 0$, что \F{$(x_0 - \delta, x_0 + \delta) \cap D_{n - 1} = (x_0 - \delta, x_0 + \delta) \cap D$,} и $\pb f{n - 1}$ дифференцируема в точке $x_0$.  Если $x_0 \in D_n$, то $f$ называется дифференцируемой $n$ раз в точке $x_0$. Функция \F{$\pb fn = (\pb f{n - 1})'|_{D_n}: D_n \to \R$} называетя производной порядка $n$, или короче, $n$-ной производной функции $f$. Другими словами, \F{$\pb fn (x_0) = \li x{x_0} \frac{\pb f{n - 1}(x) - \pb f{n - 1}(x_0)}{x - x_0},\quad x_0 \in D_n$}. Под нулевой производной подразумевается сама функция $\pb f0 = f$. Односторонние производные высших порядков определяются равенствами \F{$\pb fn_+(x_0) = \pb {(f|_{D \cap [x_0, +\infty)})}n (x_0), \quad \pb fn_-(x_0) = \pb{(f|_{D \cap (-\infty, x_0]})}n (x_0)$.} Другими словами \F{$\pb fn_\pm(x_0) = \li x{x_0\pm} \frac{\pb f{n - 1}(x) - \pb f{n - 1}_\pm (x_0)}{x - x_0}$.}

\T \q Арифметические действия над старшими производными. Пусть $n \in \N$, функции $f, g: \ang ab \to\R$ дифференцируемы $n$ раз в точке $x \in \ang ab$. Тогда\\
1) при любых $\alpha, \beta \in \R$  функция $\alpha f + \beta g$ дифференцируема $n$ раз в точке $x$ и \F{$\pb{(\alpha f + \beta g)}n(x) = \alpha \pb fn(x) + \beta \pb gn (x)$;}
2) функция $fg$ дифференцируема $n$ раз в точке $x$ и \F{$\pb{(fg)}n(x) = \dsl\sum{k = 0}n C^k_n\pb fk(x) \pb g{n - k}(x)$.}

\D Первое утверждение очевидно по индукции. Докажем второе (правило Лейбница) по индукции. При $n = 1$ равенство известно. Пусть утверждение верно для всех номеров не больших $n$, докажем для $n + 1$. Опуская обозначение аргумента $x$, имеем.
\F{$\pb{(fg)}{n + 1} = (\dsl\sum{k = 0}n C^k_n\pb fk \pb g{n-k})' = \dsl\sum{k = 0}n C^k_n \pb f{k + 1}\pb g{n - k} + \dsl\sum{k = 0}n C^k_n\pb fk\pb g{n + 1 - k} = \pb f{n + 1}\pb g0 + \dsl\sum{k = 1}{n} (C^{k - 1}_n + C^k_n) \pb fk \pb g{n+ 1 - k} + \pb f0 \pb g{n + 1} = \dsl\sum{k = 0}{n + 1} C^k_{n + 1} \pb fk \pb g{n + 1 - k}$}

\Pr1. $\pb {(x^\alpha)}n = \alpha(\alpha - 1)\cdots(\alpha - n + 1)x^{\alpha - n}.$\\
При $n = 1$ равенство известно. Индукционный переход \F{$x^{\alpha - n} = (\alpha - n)x^{\alpha - n - a}$.}
При $\alpha = -1$ \F{$\pb {(\frac 1x)}n = \frac{(-1)^n n!}{x^{n + 1}}$}

\Pr2. $\pb{(\ln x)}n = \frac{(-1)^{n - 1} (n - 1)!}{x^n}$.\\
Так как $\pb{(\ln x)}n = \frac 1x$, этот пример вытекает из предыдущего.

\Pr3. $\pb{(a^x)}n = a^x\ln^n a,\quad a > 0$. В частности $\pb{(e^x)}n = e^x$.

\Pr4. $(\sin x)' = \cos x,\s (\sin x)'' = -\sin x,\s (\sin x)''' = -\cos x,\s (\sin x)'''' = \sin x$, далее последовательность повторяется с периодом четыре. По формулам приведения можно записать \F{$\pb{(\sin x)}n = \sin(x + \frac {n\pi}2)$.}

\Pr 4. Аналогчно предыдущему примеру \F{$\pb{(\cos)}n = \cos(x + \frac{n\pi}2).$}