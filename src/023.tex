(82)

\T \q Характеристика компактов в $\R^m$. пусть $K\subset \R^m$. Тогда следующи утверждения равносильны\\
1. $K$ замкнуто и ограничено\\
2. $K$ компатно\\
3. Из всякой плследовательности точек $K$, можно извлечь подпоследовательность, имеющую предел, принадлежащий $K$.

\D проведём по схемк $1 \Ra 2 \Ra 3 \Ra 1$

1 $\Ra$ 2. Поскольку $K$ ограничено, $K$ содержится в некотором замкнутом кубе I. Тогда $K$ замкнуто и по теореме, так как $K = K \cap I$ и $K$ замкнуто в $\R^m$. Куб $I$ компактен по лемме. По теореме заключаем, что $K$ компатно как замкнутое подмножество компакта.

2 $\Ra$ 3 Пусть $\se{x^{(n)}}{n = 1}\infty$~--- последовательность в $K$; обозначим через $D$ множество её значений. Если $D$ конечно, то из $(\{x^{(n)}\})$ можно выделить станционарную подпоследовательность, причём её предел будет совпадать со значением и, значит, приадлежать $K$.\\
Содержателен случай, когда $D$ бесконечно. Докажем от противного, что в этом случае в $K$ есть предельная точка $D$. Если в $K$ нет предельных точек $D$, то у каждой точки $q \in K$ найдётся окрестность $V_q$, содержащая не более одной точки множества $D$. Тогда получаем противоречие с компатностью $K: \iseq VqK$~--- открытое покрытие $K$, из которого нельзя извлечь конечное подпокрытия не только $K$, но даже $D$.\\
Пусть $a \in K$~--- предельная точка $D$. Тогда найдётся номер $n_1$, для которого $p(x^{(n_1)}, a) < 1$. Так как множество $B(a, \frac{1}{2}) \cap D$ бесконечно, найдётся номер $n_2 > n_1$, для которого $p(x^{(n_2)}, a) < \frac{1}{2}$. Этот процесс можно продолжать неограниченно: на шаге с номером $k$, поскольку множество $B(a, \frac{1}{k})\cap D$ бесконечно, найдётся номер $n_k > n_{k - 1}$, для которого $p(x^{n_k}, a) < \frac{1}{k}$. По построению $\{x^{(n_k)}\}$~--- подпоследовательность и $x^{(n_k)}\to a$.

3 $\Ra$ 1. Если $K$ не ограничено, то для каждого натурального $n$ найдётся такая точка $y^{(n)} \in K$, что $|y^{(n)}| > n$. Последовательность $\{y^{(n)}\}$ стремится к бескончности, а тогда из ней нельзя выделить сходящуюс подпоследовательность, так как по лемме любая её полпоследовательность стрмится к бесконечности.\\
Если же $K$ не замкнуто, то у $K$ есть предельная точка $b$, не принадлежащая $K$. Следовательно, существует последовательность точек  $K$, стремящаяся к $b$. Но тогда любая её подпоследовательность стремится к $b$ и, значит, не имеет предела, принадлежащего $K$.

\Zam2. Утверждение о том, что в $\R^m$ всякое замкнутое ограниченное множество компактно и его частный случай лемму называют \q Теоремой Гейне-Бореля.

\Zam2. Как уже отмечалось, $2 \Ra 1$ верна в любом метрическом пространстве, а $1 \Ra 2$~--- не в любом. На самом деле, утверждения 2 и 3 равносильны в любом метрическом пространстве. Утверждение $3\Ra 2$ оставим без доказательства. Свойство 3 называется секвенциальной компактоностью.

\Zam3. В ходе доказательства теоремы установлено, что всякое бесконечное подмножество компакта $K \subset \R^m$ имеет предельную точку, принадлежащую $K$.

\S1. \q Принцип выбора Больцано-Вейерштрасса. Из всякой ограниченной подпоследовательности в $R^m$ можно извлечь сходящуюся подпоследовательность.

\D В силу ограниченности все члены последвоательности принадлжат некоторому замкнутому кубу $I$. Поскольку $I$ компактен, из этой подпоследовательности можно извлечь подпоследовательность, имеющуь предел, принадлежащий $I$.

\Zam4. Если вещественная последовательность неограничена (сверху, снизу) то из неё можно извлечь подпоследовательность, стремящуюся к бесконечности(плюс-минус бесконечности). Для бесконечности без знака утверждение верно и в нормированном пространстве.

\D Для определённость докажем для $+\infty$. Так как последовательность $\{x_n\}$ не ограничена сверху, найдётся номер $n_1$, для которого $x_{n_1} > 1$. Далее найдётся номер $n_2 > n_1$, для которого $X_{n_2} > 2$ (иначе $\{x_n\}$ была бы ограниченв сверху числом $\max\{x_1, \cdots,x_{n_1}, 2\}$). Жтот процесс продолжаем неограниченно: на $k$-том шаге найдётся номер $n_k > n_{k - 1}$, ля которого $x_{n_k} > k$. Тогда $x_{n_k} \to +\infty$.