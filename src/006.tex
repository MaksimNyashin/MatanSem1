(38)

\T Всяякое бесконечное множество содержит счётное подмножество.

\D Пусть множество $A$ бесконечено. Тогда в нём есть элемент $a_1$. Множество $A\bsl \{a_1\}$ бесконечно, поэтому в нём есть элемент $a_3$. Ввиду бесконечности множества $A$ этот процесс не оборвётся ни на каком шаге; продолжая его и далее, получим множество $B = \{a_1, a_2,\cdots\}$, которое по построению будет счётным одмножеством $A$.

\T Всякое бесконечно подмножество счётного множества счётно: есди $A$ счётно, $B \subset A$ и $B$ бесконечно, то $B$ счётно.

\D Расположим элементы $A$ в виде последовательности: $A =\{a_1, a_2, a_3, \cdots\}$.\\
Будем нусеровать элементы $B$ в порядке их появления в этой последовательности. Тем самым каждый элемент $B$ будет занумерован ровно один раз и, так как $B$ бесконечено, для нумерации будет использован весь натуральный ряд.