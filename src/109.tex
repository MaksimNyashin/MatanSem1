(99)\\
Пусть $(X, \rho_X)$ и $(Y, \rho_y)$~--- метрические пространства, $f: D \subset X \to Y$, $a \in X$~--- предельная точка $D$, $A \in Y$. Точку $A$ называют пределомотображения $f$ в точке $a$ и пишут $\lim\limits_{x\to a}f(x) = A$ или $f(x)\xra[x\to a]{}, A$, если выполняется одно из следующих условий:\\
\q1. \qОпределение на $\varepsilon$-языке (по Коши).\\
$\forall \varepsilon > 0\ \exists\delta > 0\ \forall x \in D \bsl \{a\}:\ \rho_X(x, a) < \delta \Ra \rho_Y(f(x), A) < \varepsilon$.\\
\q2. \qОпределение на языке окрестностей.\\
$\forall V_A\ \exists V_a\ f(\dot{V_a} \cap D) \subset V_A$\\
Для любой окрестности $V_A$ точки $A$ существует такая окрестность $V_a$ точки $a$, что образ пересечения проколотой окрестности $\dot{V_a}$ с множеством $D$ при отображении $f$ содержится в окрестности $V_A$.\\
$\forall V_a\ \exists V_a \forall x \in \dot{V_a} \cap D \Ra f(x) \in V_A$. Очевидно, что это~--- переформулировка исходного утверждения.\\
\q3. \qОпределение нв языке последовательностей (по Гейне).\\
$\forall \{x_n\}: x_n \in D \bsl \{a\},\ x_n \to a\Ra(x_n) \to A$