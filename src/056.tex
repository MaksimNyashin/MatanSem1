(182)

\T \q Производная обратной функции. Пусть $f \in C\ang ab,\s f$ строго монотонна, дифференцируема в точке $x \in \ang ab,\s f'(x) \neq 0$. Тогда обратная функция $f^{-1}$ дифференцируема в точке $f(x)$ и \F{$(f^{-1})'(f(x)) = \frac 1{f'(x)}$.}

\D $f^{-1}$ существует, определена на промежутке $P$ (множество значений $f$), строго монотонна и непрерывна по теореме. Обозначим $y = f(x),\s h = f^{-1} (y + k) - f^{-1}(y) = \tau(k)$. ТОгда $h \neq 0,\s x = f^{-1}(y),\s x +h = f^{-1}(y _ k)$ и $f(x + h) - f(x) = k$.
\F{$\frac{f^{-1}(y + k) - f^{-1}(y)}k = \frac{\tau(k)}{f(x + \tau(k)) - f(x)}$} и найдём его предел при $k \to 0$. По условию, \F{$\frac h{f(x + h) - f(x)} \xra[h \to 0]{} \frac 1{f'(x)}$.} Но $\tau(k) \xra[k \to 0]{} 0$ по непрерывности $f^{-1}$ в точке y. Следовательно, \F{$\frac{f^{-1}(y + k) - f^{-1}(y)}k \xra[k \to 0]{} \frac 1{f'(x)}$} по теореме о непрерывности композиции.

\Zam1. Равенство можно переписать так: \F{$(f^{-1})'(x) = \frac 1{f'(f^{-1}(x))}$}

\Zam2. Дифференциал обратной функции в точке $x$~---  функция обратная диференциалу исходной в точке $x$.

\Zam3. Так как графики $f(x)$ и $f^{-1}(x)$ симметричны относительно $y = x$, касательные в симметричных точках тоже симметричны ($\tg \alpha = \ctg \beta$), то есть $f'(x) = \frac 1{(f^{-1})'(y)}$

\q Производная функции заданной параметрически.
Пусть $T$~--- множество, $\varphi, \psi: T \to \R$. Рассмотрим отображение $\gamma(\varphi, \psi): T \to \R^2$. Cистема \F{$\begin{cases}x = \varphi(t)\\ y = \psi(t)\end{cases}$} не всегда определяет функцию $y(x)$. Но если жто так ($\varphi$ обратима), то находим $t = \varphi^{-1}(x)$ из первого уравнения и подставляем во второе. Тогда \F{$y = \psi(\varphi^{-1}(x)),\quad x\in \varphi(T)$,} то есть $f = \psi \circ \varphi^{-1}$.\\
Обычно для встречающихся на практике систем множество $T$ можно разбить на несколько частей, на каждой из которых функция $\varphi$ обратима.\\
Пусть теперь $T = \ang ab,\s t \in \ang ab,\s \varphi \in C\ang ab,\s \varphi$ строго монотонна, $\varphi, \psi$ дифференцируемы в точке $t,\s \varphi'(t) \neq 0,\s f = \psi\circ\varphi^{-1}$~--- параметрически заданная функция. Тогда $f$ дифференцируема в точке $x = \varphi(t)$ и $f'(x) = \frac{\psi'(t)}{\varphi'(t)}.$ Следует из прафил дифференцирования композиции и обратной функции. Часто равенство записывают в виде $y'_x = \frac{y'_t}{x'_t}$.