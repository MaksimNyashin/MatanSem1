(108)

\T \q О пределе монотонной функции. Пусть $f: D\subset \R\to\R,\ a \in (-\infty, +\infty),\ D_1 = D\cap(-\infty, a), a$~--- предельная точка $D_1$.\\
1. Если $f$ возрастает и ограничена сверху на $D_1$, то существует конечный предел $f(a-)$.\\
2. Если $f$ убывает и ограничена снизу на $D_1$, то существует конечный предел $f(a-)$.

\D Докажем первое утверждение, второе доказыается аналогично. Положим $A = \dsl\sup{x\in D_1}{} f(x)$; тогда $A \in \R$ в силу ограниченности функции свурху. Докажем, что $f(a-) = A$. Возьмём $\eps > 0$. По определению верхней грани существует такая точка $x_0 \in D_1$, что $f(x_0) > A - \eps$. Но тогда для всех таких $x\in D_1$, что $x > x_0$, в силу возрастания $f$ \F{$A - \eps < f(x_0) \le f(x) \le A < A + \eps$.} Теперь положим $\delta = a - x_0$ при $a \in \R$ или $\Delta = \max\{x_0, 1\}$ при $a = +\infty$; Тогда неравенство из определения предела выполнено для всех таких $x \in D$, что $0 < a - x < \delta$ (соответсвенно, $x > \Delta$).

\Zam2. Аналогично утверждениям теоремы доказываются\\
3. Если $f$ возрастает и не ограничена сверху на $D_1$, то предел $f(a-)$ существует и равен $+\infty$.\\
4. Если $f$ убывает и не ограничена снизу на $D_1$, то предел $f(a-)$ существует и равен $-\infty$.

\Zam2. Аналогично формулируется и доказывается теорема для правостороннего предела.\\
Пусть $f: D\subset \R\to\R,\ a \in (-\infty, +\infty),\ D_2 = D\cap(a, +\infty), a$~--- предельная точка $D_2$.
1. Если $f$ возрастает и ограничена снизу на $D_2$, то существует конечный предел $f(a+)$.\\
2. Если $f$ убывает и ограничена сверху на $D_2$, то существует конечный предел $f(a+)$.\\
3. Если $f$ возрастает и не ограничена снизу на $D_2$, то предел $f(a+)$ существует и равен $-\infty$.\\
4. Если $f$ убывает и не ограничена сверху на $D_2$, то предел $f(a+)$ существует и равен $+\infty$.