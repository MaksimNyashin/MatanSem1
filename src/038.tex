(122)

\T \q Арифметические действия над непрерывными отображениями. Пусть $X$~--- метрическое пространство, $Y$~--- нормированное пространство, $D\subset X,\s x_0 \in D$, отобрадения $f, g: D \to Y,\s \lambda: D \to \R(\C)$ нерперывны в  точке $x_0$ю Тогда отображения $f + g,\s f - g,\s \lambda f, |f|$, непрерывны в точке $x_0$.

\D Если $x_0$~--- изолированнач точка $D$, то утверждение тривиально. Если же $x_0$~--- предельная точка $D$, то теоремы о непрерывности следуют из теорем о пределах.

\Zam1. \q О стабилизации знака непрерывной функции. Если функция $g: D\to \R$ непрерывна в точке $x_0$, причём $g(x_0) \neq 0$, то существует такая окрестность $V_{x_a}$, что sign$g(x) = \textit{sign}g(x_0)$ для всех $x \in V_{x_0} \cap D$.

\D Для определённости рассмотрим случай, когда $g(x_0) > 0$. Допустим противное: пусть для любого $n \in \N$ существует точка $x_n \in V_{x_0}(\frac 1n) \cap D$, для которой $g(x_n) \le 0$. Построенная последовательность $\{x_n\}$ стремится к $x_0$. По определению непрерывности $g(x_n) \to g(x_0)$. по теореме о предельном переходе в неравенстве, $g(x_0) \le 0$, что противоречит условию.