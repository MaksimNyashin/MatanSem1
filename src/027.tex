(92)

\L \q Неравенство Я. \qБернулли. Если $n \in \Z_+,\ x > -1$, то \F{$(1 + x)^n \ge 1 + nx$}.

\D При $n = 0, n = 1$ (база индукции) неравенство очевидно, обращается в равенство. Сделаем переход: Пусть неравенство верно для номера $n$. Тогда \F{$(1 + x)^{n + 1} = (1 + x)^n(1 + x) \ge (1 + nx)(1 + x) = 1 + (n + 1)x + nx^2 \ge 1 + (n + 1)x$}

\Pr1. Пусть $x \in \C,\ |z| < 1$. Докажем, что $\dsl \lim{n\to\infty}{} z^n = 0$.

В самом деле, последователность $\{|z|^n\}$ убывает и ограничена снизу нулём. Следовательно, существует конечный $\dsl\lim{n\to\infty}{}|z|^n=a$. Перейдя в равенство $|z|^{n + 1} = |z||z|^n$ к пределу, получим $a = |z|a \lra (1 - |z|)a = 0$, отсюда $a = 0$, поскольку $|z| < 1$, таким образом $z^n \to 0$, что равносильно $x^n \to 0$.

\Pr2. \q Число $e$. Докажем, что последовательность $x_n = (1 + \frac 1n)^n$ сходится.

Положим $y_n = (1 + \frac 1n)^{n + 1}$. Ясно, что последовательность $\{y_n\}$ ограничена снизу единицей, кроме того она убывает \F{$\frac{y_{n - 1}}{y_n} = \frac{(1 + \frac{1}{n - 1})^n}{(1 + \frac 1n)^{n + 1}} = \frac{(\frac n{n - 1})^ n}{(\frac{n + 1}n)^{n + 1}} = (\frac{n^2}{n^2 - 1})^{n + 1}\frac{n - 1}n = (1 + \frac 1{n^2-1})^{n + 1}\frac{n - 1}n \ge (1 + \frac{n + 1}{n^2 - 1}) \frac{n - 1}n \ge 1$} Следовательно, $\{y_n\}$ сходится, а тогда по теореме о пределе частного и $x_n = \frac{y_n}{1 + 1/n}$. сходится к этому же пределу.

\Pr3. \q Формула Герона. Пусть $a > 0, x_0 > 0$, \F{$x_{n + 1} = \frac 12 (x_n + \frac a{x_n})$} Ясно, что $x_n > 0$ при всех $n$ и значит, $\{x_n\}$ ограничена снизу. Воспользоваашись очевидным неравенством \F{$t + \frac 1t \ge 2,\quad t > 0$} получим, что при всех $n \in \Z_+$ \F{$x_{n + 1} = \frac{\sqrt{a}}2(\frac{x_n}{\sqrt{a}} + \frac a{x_n}) \ge \frac{\sqrt{a}}{2}2= \sqrt{a}$} Поэтом у для всех $n \in \N$\F{$x_{n + 1} = \frac{x_n}2 (1 + \frac a{x^2_n}) \le x_n$} То есть последовательность $\{x_n\}$ убывает. Следовательно, она сходится; обозначим $\lim x_n =\beta$. Перейдя к пределу в равенстве получим \F{$\beta = \frac12(\beta + \frac a\beta)$} Откуда $\beta = \sqrt{a}$, так как $\beta \ge 0$\\
Равенство $\lim x_n = \sqrt{a}$ называется формулой герона и используется для приближенного вычисления корня.

\Zam3. Пусть $x_n > 0$ при всех $n$, $\lim \frac{x_{n + 1}}{x_n} < 1$ тогда $x_n \to 0$.\\
Отсюда следует, что
\F{$\dsl\lim{n\to\infty}{} \frac{n^k}{a^n} = 0 \quad a > 1,\ k \in \N$}
\F{$\dsl\lim{n\to\infty}{} \frac{a^n}{n!} = 0 \quad a \in \R$}
\F{$\dsl\lim{n\to\infty}{} \frac{n!}{n^n} = 0$}