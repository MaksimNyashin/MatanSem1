(126)

\T \q Вейерштрасса О непрерывных отображениях. Пусть $X, Y$~--- метрические пространчтва, $X$~--- компактно, $f \in C(X\to Y)$. Тогда $f(x)$ компактно. Другими словами: непрерывный образ компакта~--- компакт.

\D Пусть $\iseq G\alpha A$~--- открытое покрытие множества $f(X): f(X) \subset \dsl\bigcup{\alpha \in A}{} G_\alpha$. По теореме о характеристике непрерывности с помощью прообразов, при всех $\alpha \in A$ множества $f^{-1}(G_\alpha)$ открыты в $X$. Проверим, что \F{$X = \dsl\bigcup{\alpha \in A}{} f^{-1} (G_\alpha)$.} В самом деле, если $a \in X$, то $f(A) \in Y$ и, значит, $f(a) \in G_\alpha$ при некотором $\alpha$, то есть $a \in f^{-1}(G_\alpha)$ при некотором $\alpha$. Следовательно, $X$ содержится в объединении $f^{-1} (G_\alpha)$. Обратное включение тривиально.

Пользуясь компактностью $X$, выделим из его откытого покрытия $\se{f^{-1}(G_\alpha)}{\alpha \in A}{}$ конечное подпокрытие: найдётся такой конечный нобор индексов $\alpha_1, \cdots, \alpha_N \in A$, что
\F{$X = \dsl\bigcup{i = 1}N f^{-1}(G_{\alpha_i})$.} Осталось проверить, что \F{$f(X) \subset \dsl\bigcup{i = 1}N G_{\alpha_i}$;} это и будет означать, что из произвольного открытого покрытия удастся извлечь конечное подпокрытие. Действительно, если $y \in f(X)$, то $y = f(x)$ для некоторого $x \in X$. Тогда найдётся такой номер $i \in [1:N]$, что $x \in f^{-1}(G_{\alpha_i})$. Последнее означает, что $f(x) \in G_{\alpha_i}$, то есть $y \in G_{\alpha_i}$.

\S1. Непрерывный образ компакта замкнут и ораничен.

\S2. \q Первая теорема Вейерштрасса о непрерывных функциях. Функция непрерывная на отрезке ограничена.

\Zam1. Оба условия: и непрерывность, и то, что область определения отрезок~--- существенны. Так, функции $f(x) = x,\s g(x) \frac 1x$ непрерывны, но не ограничены, соответсвенно, на $\R$ и $(0, 1]$. Функция \F{$h(x) = \begin{cases} \frac 1x, & x \in (0,1],\\0, & x = 0.\end{cases}$} Задана на $[0, 1]$, разрывна в точке 0, но не ограничена.

\S3. Пусть $X$ компактно, $f \in C(X\to \R)$. Тогда существует $\dsl\max{x \in X}{} f(X), \dsl\min{x \in X}{}$. Другими словами: непрерывная на компакте функция принимает наибольшее и наименьшее значение.

\D Остаётся доказать, что компактно подмножество $E$ числовой прямой $(E = F(X))$ имеет наибольший и наименьший элемент. Существует  $\sup E = b \in \R$. Докажем, что $b \in E$: это и будет означать, что $b = \max E$. По определению супремум для любого $n \in N$ найдётся такая точка $x_n \in E$ , что $b - \frac 1n < x_n \le b$. Построенная последовательность стремится к $b$. Следовательно, $b \in E$ в силу замкнутости $E$. Доказательство для минимума аналогично.

\S4. \q Вторая теорема Вейерштрасса о непрерывных функциях. Функция, непрерывная на отрезке, принимает наибольшее и наименьшее значение.

\Zam2. И здесь оба условия существенны. Так как функции $f, g, h$ из замечания 1 не имеют наибольшего значения. Наибольшего значения не имеет и ограниченная непрерывная функция $f_1(x) = x$ на $[0, 1)$.