(111)

\Op  Пусть $D_1, D_2 \subset\R,\s a$~--- предельная точка $D_1,\s b$~--- предельная точка $D_2,\s D \supset (D_1 \bsl\{a\}) \times (D_2 \bsl\{b\}),\s f: D \to \R$.

1. Если для каждого $x \in D_1 \bsl\{a\}$ существует конечный предел
\FF{\varphi(X) = \dsl\lim{y\to b}{} f(x, y),} то предел функции $\varphi$ в точке $a$ называется повторным пределом функции $f$ в точке $(a, b)$: \FF{\dsl\lim{x\to a}{} \varphi(x) = \dsl\lim{x\to a}{} \dsl\lim{y\to b}{} f(x, y)}

2. Аналогично для $y \in D_2 \bsl \{b\}$ \f($\dsl\lim{y\to b}{} \psi(x) = \dsl\lim{y\to b}{} \dsl\lim{x\to a}{} f(x, y)$)/

3. Точку $A$ называют двойным пределом функции $f$ в точке $(a, b)$ и пишут \FF{\dsl\lim{x\to a, y\to b}{} f(x, y) = A, \qquad f(x, y)\xra[x\to a, y\to b]{} A,} если для любой окрестности $V_a$ точки $A$ существуют такие окрестности $V_a$ и $V_b$ точек $a, b$, что $f(x, y) \in V_A$ для всех $x \in \dot{V_a}\cap D_1,\s y \in \dot{V_b}\cap D_2$.

\T \q О двойном и повторном пределе. Пусть $D_1, D_2 \subset \R,\s a$~--- предельная точка $D_q,\s b$~--- предельная точка $D_2,\s D \supset (D_1 \bsl\{a\}) \times (D_2 \bsl\{b\}),\s f: D \to \R$ и выполнены условия:\\
1. существует конечный или бесконечный двойной предел $\dsl\lim{x\to a,y\to b}{} f(x,y) = A$;\\
2. для каждого $x \in D_1 \bsl\{a\}$ существует конечный предел \FF{\varphi(x) = \dsl\lim{y\to b}{} f(x, y).} Тогда повторный предел $\dsl\lim{x\to a} \varphi(x)$ существует и равен $A$.

\D Для определённости пусть $A \in \R$. Возьмём $\eps > 0$. По определению, двойного предела, найдутся такие окрестности $V_a, V_b$, что для всех $x \in \dot{V_a}\cap D_1,\s y \in \dot{V_b}\cap D_2$ выполняется неравенство \FF{|f(x, y) - A| < \frac \eps2.} Устремляя в нём $y$ к $b$ и пользуюясь непрерывностью модуля, получаем \FF{|\varphi(x) - A| \le \frac \eps2 < \eps} для всех $x \in \dot{V_A}\cap D_1$, что и означает требуемое. В случае бесконечного предела следует изменить неравенство на соответсвующее.

\S1. При выполнении всех трёх условий оба повторных предела существуют и равны двойному.

\Pr1. Пусть $f(x, y) = \frac{x^2 - y^2}{x^2 + y^2}$, тогда в точке $(0, 0)$ повторные пределы различны и равны 1 и -1 соответсвенно, двойного предела не существует.

\Pr2. $f(x, y) = \frac{xy}{x^2 + y^2}$ в точке (0, 0) повторные пределы равны 0, но двойного предела не существует, так как по прямой $y = x$ предел равен $\frac 12$.

\Pr3. $f(x, y) = x\sin\frac 1y + y\frac 1x$ в (0, 0) повторных пределов нет, а двойной существует и равен 0, так как $|f(x, y)| \le |x| + |y|$