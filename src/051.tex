Билет 51: Замечательные пределы
(154-158)

\q1.\F{$\dsl\lim{x\to 0}{} \frac{\sin x}x = 1$}
По лемме $\sin x < x < \tg x \Ra \cos x < \frac {\sin x}x < 1$, применяя предельный переход в неравенстве при $x \to 0$ $\frac{\sin x}x = 1$.

\S1. \F{$\dsl\lim{x\to 0}{}\frac{1 - \cos x}{x^2} = \frac 12,\qquad \dsl\lim{x\to 0}{}\frac{\tg x}{x} = 1$,}
\F{$\dsl\lim{x\to 0}{}\frac{\arcsin x}{x} = 1,\qquad \dsl\lim{x\to 0}{}\frac{\arctg x}{x} = 1$.}

\q2. \F{$\dsl\lim{x\to\infty}{} (1 + \frac 1x)^x = e$}

\D Напомним, что $e = \dsl\lim{n\to\infty}{} (1 + \frac 1n)^n$. Разница между этим и доказываемым в том, что теперь речь идёт о пределе не последовательности, а функции, заданной на $\R \bsl [-1, 0]$: аргумент $x$ необязан принимать натуральные и даже положительные значения.\\
Для доказательства воспользуемся языком последовательностей. Возьмём последовательность $\{x_n\}: x_n \to \infty$ и докажем, что $f(x_n) \to e$.\\
1. пусть сначала $x_n \in \N$ для всех $n$. Возьмём $\eps > 0$ и по определению числа $e$ подберём такой номер $K$, что для всех номеров (то есть натуральных чисел) $k > K$ будет $|f(k) - e| < \eps$. Но начиная с некоторого номера $x_n > k$, а тогда $|f(x_n) - c| < \eps$, что и означает выполнение требования.\\
2. Пусть $x_n \to +\infty$. Тогда, начиная с некоторого номера $x_n > 1$, поэтому, не уменьшая общности, можно считать, что все $x_n > 1$. Уменьшая или увеличивая основание и показатель степени получаем равенство \F{$(1 + \frac 1{[x_n] + 1})^{[x_n]} \le (1 + \frac 1{x_n})^x_n \le (1 + \frac 1{x_n}) ^ {[x_n] + 1}$,} которые перепишем в виде
\F{$\frac{f([x_n] + 1)}{1 + \frac 1{[x_n] + 1}} \le f(x_n) \le (1 + \frac 1{[x_n]})f([x_n])$.} Так как $\{[x_n]\}$ и $\{[x_n] + 1\}$~--- поседлвательности натуральных чисел, стремящихся к $+\infty$, то, по доказанному, $f([x_n]) \to e,\s f([x_n] + 1) \to e$. Следовательно, по теореме о сжатой последовательности, $f(x_n)$ стремится к $e$.\\
3. Пусть $x_n \to -\infty$, тогда $y_n = -x_n \to +\infty$. По доказанному, \F{$f(x_n) = (1 + \frac 1{-y_n})^{-y_n} = (\frac{y_n}{y_n - 1})^{y_n} = (1 + \frac 1{y_n - 1})f(y_n - 1) \to e$.}
4. Пусть $x_n \notin [-1, 0],\s x_n \to \infty$, а в остальном $\{
x_n\}$ произвольная. Если чисел отрицательных (положительных) конечно,т то $x_n \to +\infty (-\infty)$ и требуемое соотношение уже доказано, иначе разобъём на две полпоследовательность положительных и отрицательных чисел. Они обе стремятся к $e$, тогда и вся поседовательность сходится к $e$ по лемме.

\Zam1. Заменимв $x$ на $\frac 1x$ модно получить \F{$\dsl\lim{x\to 0}{} (1 + x)^{1/x} = e$.}

\q3. \F{$\dsl\lim{x\to 0}{}\frac{\log_a(1 +x)}x = \frac 1{\ln a},\qquad a > 0,\s a \neq 1$.} В частности, \F{$\dsl\lim{x\to 0}{} \frac{\ln(1 + x)}x = 1$.}

\D Так как $\log_a(1 + x) = \frac{\ln(1 + x)}{\ln a}$, достаточно доказать равенство для натурального логарирфма. Имеем \F{$\li{x}{0}\frac{ln(1 + x)}x = \li{x}{0} \ln(1 + x)^{1/x} = \ln \li{x}{0}(1 + x)^{1 / x} = \ln e = 1$.} Во втором равенстве воспользовались непрерывностью логарифма в точке $e$ и теоремой о непрерывности композиции (для её применения мы доопределим $(1 + x)^{1/x} = e$ при $x = 0$).

\q4. \F{$\li x0 \frac{(1 + x)^\alpha - 1}x = \alpha,\qquad \alpha\in\R$.}

\D При $\alpha = 0$ тривиально. Пусть $\alpha \neq 0$. Возьмйм последовательность $\{x_n\}: x_n \to 0,\s x_n \neq 0$; не уменьшая общности, можно считать, что $|x_n| < 1$. Тогда в силу непрерывности и строгой монотонности степенной функции $y_n = (1 + x_n)^\alpha - 1 \to 0,\s y_n \neq 0$. При этом \F{$\alpha\ln(1 + x_n) = \ln(1 + y_n)$.} Пользуясь замечательным пределом для логарифма находим
\F{$\frac{(1 + x_n)^\alpha - 1}{x_n} = \frac {x_n}{y_n} = \frac{y_n}{\ln(1 + y_n)}\alpha\frac{\ln(1 + x_n)}{x_n} \to \alpha$.}

\q5. \F{$\li{x}{0}\frac{a^x - 1}x = \ln a,\qquad a > 0$.} В частности \F{$\li{x}{0} = \frac{e^x - 1}x = 1$}

\D При $a = 1$ Доказывается равенство тривиально; пусть $a \neq 1$. Возьмём последователбность $\{x_n\}:x_n\to 0,\s x_n \neq 0$. Тогда в сид непрерывности и строгой монотонности показательной функции $y_n = a^{x_n} - 1 \to 0,\s y_n \neq 0$. При этом \F{$x_n\ln a = \ln(1 + y_n)$.} Пользуясь замечательным пределом для логарифма, находим \F{$\frac{a^{x_n} - a}{x_n} = \frac{y_n}{x_n} = \frac{y_n}{\ln(1 + y_n)} \ln a \to \ln a$.}