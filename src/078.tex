(240)

\Op Числа $p, q \in (1, +\infty)$, связанные соотношением $\frac 1p + \frac 1q = 1$, называются сопряженными показателями.\\
Ясно, что $q = \frac{p}{p - 1},\s p = \frac q{q - 1}$.

\L \q Неравенство Юнга. Пусть $p, q$~--- сопряженные показатели, $a, b \in [0, +\infty)$, тогда $ab \le \frac{a^p}p + \frac{b^q}{q}$.

\D Если $a = 0$ или $b = 0$ очевидно. По неравенству Йенсена для $f(x) = \ln x$ (функция выпукла вверх) \FF{\ln\lr{(\frac {a^p}p + \frac{b^q}q }) \ge \frac 1p \ln(a^p) + \frac 1q \ln(a^q) = \ln a + \ln b .} При возведеениии $e$ в степень обеих честей неравенства, получается требуемое

\T \q Неравенство Гёльдера. Пусть $a, b \in\R^n$ или $\C^n,\s p > 1,\s \frac 1p + \frac 1q = 1$. Тогда \FF{\lr{|\dsl\sum{k = 1}n a_k b_k}| \le \lr{(\dsl\sum{k = 1}n |a_k|^p})^{1/p} \lr{(\dsl\sum{k = 1}n |b_k|^q})^{1/q} .}

\D Так как \FF{\lr{|\dsl\sum{k = 1}n a_k b_k}| \le \dsl\sum{k = 1}n |a_k b_k|,} достаточно доказать неравенство Гёльдера для чисел $|a_k|, |b_k|$. Поэтому, не уменьшая общности, иожно считать, что $a_k, b_k \in \R_+$. Более того, можно считать, что все $b_k > 0$. Действительно, если неравенство доказано для положительных чисел $b_k$, то доказано и неравенство, так как сумма $a_k b_k$ не изменится, сумма $a^q_k$ уселичиться, сумма $b^q_k$ не изменится (при добавелении пар ($a_k, b_k$), где $b_k = 0$).
Функция $f(x) = x^p$ строго выпукла вниз на $[0, +\infty)$. Положим $p_k = b^q_k,\s x_k = a_k b^{1-q}_k$ и применим неравенство Йенсена: \FF{\lr{( \frac{\dsl\sum{k = 1}n p_k x_k}{\dsl\sum{k = 1}n p_k}})^p \le \frac{\dsl\sum{k = 1}n p_k x^p_k}{\dsl\sum{k = 1}n p_k}. } Учитывая, что \FF{p_k x_k = a_k b_k,\quad p_k x^p_k = b^q_k a^q_k b^{p(1-q)}_k = a^p_k,} Получаем \FF{\lr{(\dsl\sum{k = 1}n a_k b_k})^p \le \lr{(\dsl\sum{k = 1}n a^p_k})\lr{(\dsl\sum{k = 1}n b^q_k})^{p - 1}.} Остается возвести обе части неравенства в степень $\frac 1p$ и воспользоваться тем, что $1 - \frac 1p = \frac 1q$.