(124)

\T \q Непрерывность композиции. Пусть $X, Y, Z$~--- метрические пространства, $f : D \subset X \to Y,\s g: E \subset Y \to Z,\s f(D) \subset E, f$ непрерывно в точке $x_0 \in D,\s g$ непрерывно в точке $f(x_0)$. Тогда $g\circ f$ непрерывно в точке $x_0$.

\D Возьмём последовательность $\{x_n\}$, такую что $x_n \in D,\s x_n \to x_0$. Обозначим $y_n = f(x_n),\s y_0 = f(x_0)$; тогда $y_n, y_0 \in E$. По определению непрерывности $f$ в точке $x_0$ на языке последовательностей $y_n \to y_0$. По определению непрерывности $g$ в точке $y_0$ на языке последователбностей $g(y_n) \to g(y_0)$. то есть  $(g \circ f)(x_n) \to (g \circ f)(x_0)$. Последнее в силу проивольности $\{x_n\}$ и означает непрерывность $g \circ f$ в точке $x_0$.

\Zam2. Пусть $f(x) = x \sin\frac 1x,\s g (y) = |$sign$y|$. Тогда $f(x) \xra[x \to 0]{} 0,\s g(y) \xra[y \to 0]{} 1$, но композиция $g \circ f$ не имеет предела в нуле, так как $(g \circ f) (\frac 1{n\pi}) = 0 \to 0$, а $(g\circ f)(\frac 1{(n + 1/2)\pi}) = 1 \to 1$. Этот пример показывает, что утверждение "если $f(x) \t A, g(x) \xra[x \to A]{} B$, то $g \circ f(x) \t B$" может не выполняться. Если же запертить $f(x)$ принимать значение $A$, то утверждение становится верным.

Пусть $X, Y, Z$~--- метрические пространства, $f: D \subset X \to Y,\s g: E \subset Y \to Z,\s f(D) \subset E$ и выполнены условия:\\
1. $a$~--- предельная точка $D,\s f(x) \t A$;\\
2. $A$~--- предельная точка $E,\s g(x) \xra[x\to A]{} B$;\\
3. Существует такая окрестность $V_a$ точки $a$, что $f(X) \neq A$ для любого $x \in \dot{V}_a \cap D$.\\
Тогда $(g \circ f) (x) \t B$.