(53)\\
Пусть $K$~--- поле, $X$~--- множество, и над элементами $X, K$ определены две операции: сложение $X \times X \xra[]{+} X$ и умножение $K \times X \xra[]{} X$, удовлетворяющие следующим условиям:
\\ $x, y, z \in X$, $\lambda, \mu \in K$\\
\q1. $(x + y) + z = x + (y + z)$\\
\q2. $x + y = y + x$\\
\q3. $\exists \theta \in X:\ 0\cdot x = \theta$\\
\q4. $(\lambda + \mu)x = \lambda x + \mu x$\\
\q5. $\lambda(x + y) = \lambda x + \lambda y$\\
\q6. $(\lambda\mu)\cdot x= \lambda \cdot (\mu x)$\\
\q7. $\exists1 \in K:\, 1 \cdot x = x$\\
Тогда $X$ называюттлинейным пространством или линейным множеством над полем $K$. Элементы $X$ называют векторами, элементы $K$~--- скалярами.