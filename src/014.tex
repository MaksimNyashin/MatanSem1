(61)

\T \q Неравенство Коши-Буняковског и неравенство трейгольника в $\R^m$. Для любых вещественных чисел $x_1, \cdots, x_m$, $y_1, y_m$\\
$(\ds\sum\limits^{m}_{k = 1} x_k y_k)^2 \le (\sum\limits^m_{k - 1} x^2_k)(\sum\limits^m_{k = 1} y^2_k)$.\\
$\sqrt{\ds\sum\limits^m_{k = 1}(x_k + y_k)^2} \le \sqrt{\sum\limits^m_{k = 1}x^2_k} + \sqrt{\sum^m_{k = 1} y^2_k}$.

\D Первое~--- КБШ, второе правило~--- треугольника.

\Op Говорят, что послледовательность $\{x^{(n)}\}$ точек $\R^m$ сходится к пределу $x^{(0)} \in \R^m$ покоординатно, если $x^{(n)}_j \xra[n\to \infty]{} x^{(0)}_j$, для всех $j \in [1: m]$

\L В $\R^m$ покоординатная сходимость и сходимость по евклидовой норме равносильны

\D Утверждение следует из неравенств\\
$|x^{(n)} - x^{(0)}_j| \le |x^{(n)} - x^{(0)}| = \sqrt{\ds\sum\limits^m_{k = 1} (x^{(n)}_k - x^{(0)}_k)^2}\le \sqrt{m} \max\limits_{1\le k\le m} |x^{(n)}_l - x^{(0)}_k|$\\
и теоремы о предельном переходе в неравенствах.

\S. Сходимость последовательности комплексных чисел равносильна одновременной сходимости последовательности из их мнимых частей.