(166)

\T \q Единственность асимптотического разложения. Пусть $X$~--- метрическое пространство, $D \subset X,\s x_0$~--- предельная точка $D,\s n \in \Z_+,\s f, g_k: D\to \R(\C)(k \in [0: n])$, при всех $k \in [0: n - 1]$ \F{$g_{k + 1}(x) = o(g_k(X))\quad x \to x_0$,} и для любой окрестности $V_{x_0}$ существует точка $t \in \dot{V}_{x_0} \cap D$, в которой $g_n(t) \neq 0$. Тогда, если асимптотическое разложение $f$ по системе $\{g_k\}$ существует, то оно едиственное: из равенств \F{$\dsl\sum{k = 0}n c_k g_k(x) + o(g_n(x)), \quad x\to x_0$,}
\F{$\dsl\sum{k = 0}n d_k g_k(x) + o(g_n(x)), \quad x\to x_0$,} следует, что $c_k = d_k$ при всех $k \in [0: n]$.

\D По индукции заключаем, что \F{$g_k(x) = o(g_l(x)), \quad x\to x_0,\s l < k$.} Обозначим \F{$E_k = \{x \in D: g_k(x) \neq 0\}, \quad k \in [0: n]$.} Если бы функция $g_k$ тождественно обращалась в ноль на множестве вида $\dot{U}_{x_0} \cap D$, то и функция $g_n = \varphi_k g_k$, где $\varphi_k$~--- функция из определения символа $o$ , обращалась бы в тождественный ноль а множестве $\dot{V}_{x_0} \cap D$, что противоречит условию. Следовательно, $x_0$~--- предельная точка каждого $E_k$.\\
Допустим противное: пусть $c_k = d_k$ не при всех $k \in [0: n]$. Положим \F{$m = \min\{k \in [0: n]: c_k \neq d_k\}$}.
Из разложений следует, что \F{$f(x) = \dsl\sum{k = 0}m c_k g_k(x) + o(g_n(x)), \quad x\to x_0$,}\F{$f(x) = \dsl\sum{k = 0}m d_k g_k(x) + o(g_n(x)), \quad x\to x_0$.} Вычтя получим \F{$0 = (c_m - d_m)g_m(x) + o(g_m(x)),\quad x\to x_0$.} Поделив на $g_m(x)$ при $x \in E_m$ и перейдя к пределу по множеству $Е_m$, получим $c_m = d_m$, что противоречит определению $m$.