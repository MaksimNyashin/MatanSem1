%C:\Users\Maxim\AppData\Local\MiKTeX\2.9\TeXworks\0.6\templates
\documentclass{article}
%\documentclass[10pt,a5paper]{article}
\usepackage[17pt]{extsizes}
\usepackage[left=2cm, right=2cm, top=2cm, bottom=2cm, bindingoffset=0cm]{geometry}
\usepackage{mathtext}
\usepackage[T2A]{fontenc}
\usepackage[utf8]{inputenc}
\usepackage[russian]{babel}
\usepackage{amsfonts}
\usepackage{listings}
\usepackage{xcolor}
\usepackage{amsmath}
\usepackage{mathtools} %/xra
\usepackage{amssymb} %/varnothing
\usepackage{graphicx}
\usepackage{hyperref}
\graphicspath{{./images/}}
\newcommand{\RNumb}[1]{\uppercase\expandafter{\romannumeral #1\relax}}
\newcommand{\ang}[2]{\langle #1, #2 \rangle}
\newcommand{\dsl}[3]{\displaystyle #1\limits^{#3}_{#2}}%{command}{from}{to}
\newcommand{\iseq}[3]{\{#1_{#2}\}_{#2 \in #3}}%{set}{var}{var set}
\newcommand{\seq}[4]{\{#1_{#2}\}^{#4}_{#2 = #3}}%{set}{var}{from}{to}
\newcommand{\se}[3]{\{#1\}^{#3}_{#2}}%{element}{from}{to}
\newcommand{\pb}[2]{#1^{(#2)}}%power in brackets
\newcommand{\ool}[1]{\ol{\ol{#1}}}
\newcommand{\li}[2]{\dsl\lim{#1 \to #2}{}}
\newcommand{\lr}[1]{\left#1\right}

\newcommand{\F}[1]{
\begin{center}
#1\\
\end{center}
}
\lstset { %
    language=C++,
    backgroundcolor=\color{black!5}, % set backgroundcolor
    basicstyle=\footnotesize,% basic font setting
}
\newcommand{\FF}[1]{\F{$\ds #1$}}


\begin{document}
\def\q#1.{{\bf #1.}}
\def\lra{\Leftrightarrow}
\def\B{\mathbb{B}}
\def\N{\mathbb{N}}
\def\Z{\mathbb{Z}}
\def\Q{\mathbb{Q}}
\def\R{\mathbb{R}}
\def\I{\mathbb{I}}
\def\C{\mathbb{C}}
\def\O{\mathcal{O}}
\def\Ra{\Rightarrow}
\def\skip{\bigskip\bigskip\bigskip\bigskip\bigskip}
\def\s{\ \ }
\def\dd{\mathrm{d}}
\def\ds{\displaystyle}
\def\xra{\xrightarrow}
\def\bsl{\textbackslash}
\def\ol{\overline}
\def\ul{\underline}
\def\T{\q Теорема. }
\def\L{\q Лемма. }
%\def\T#1.{\q Теорема. \q#1.}
\def\D{\q Доказательство. }
\def\S#1.{\q Следствие #1. }
\def\Zam#1.{\q Замечание #1. }
\def\Op{\qОпределение. }
\def\eps{\varepsilon}
\def\Sv#1.{\q Свойства #1. }
\def\Pr#1.{\q Пример #1. }
\def\t{\xra[x\to a]{}}
\def\Geine{Возьмём последовательность $\{x_n\}$ со свойствами $x_n \in D,\s x_n \neq a,\s x_n\to a$}
\def\ig{\includegraphics}
\def\sig{\textit{ sign}}

\tableofcontents
\newpage
\section{Билеты}
\subsection{Множества и операции над ними}
(6-12)\\
Множество состоит из элементов\\
Если каждый элемент множества $X$ принадлежит множеству $Y$, то говорят $X$ подмножество $Y$  $X \subset Y$.\\
$X = Y \lra X \subset Y \wedge Y \subset X$\\
Пустое множество $\varnothing$~--- множество в котором нет элементов

Пусть $\{X_\alpha\}_{\alpha \in A}$~--- семейство множеств. Объединением семейства $\{X_\alpha\}_{\alpha \in A}$ называется множество всех элементов, которые принаджежат хотя бы одному из множеств$X_n$:\\
$\ds\bigcup\limits_{\alpha \in A} X_\alpha = \{x:\ \exists\alpha \in A\ x \in X_\alpha\}$\\
$(X\cup Y) \cup Z = X \cup (Y\cup Z),\quad X\cup Y = Y \cup X,\quad X \cup X = X\cup \varnothing = X$

Пусть $\{X_\alpha\}_{\alpha \in A}$~--- семейство множеств. Пересечением семейства $\{X_\alpha\}_{\alpha \in A}$ называется множество всеъ элементов, которые принадлежат каждому из множеств $X_\alpha$\\
$\ds\bigcap\limits{\alpha \in A} X_\alpha = \{x:\ \forall \alpha \in A\ x \in X_\alpha\}$\\
$(X\cap Y) \cap Z = X \cap (Y\cap Z),\quad X\cap Y = Y \cap X,\quad X\cup \varnothing = \varnothing$

Разностью множеств $X$ и $Y$ называется множество всех элементов, которые принадлежат $X$, но не принадлежат $Y$.\\
$X \bsl Y = \{x: x\in X, x \notin Y\}$\\
Определение не предполагает, что $Y \subset X$. Если же $Y \subset X$, то разность $X\bsl Y$ называют дополнением множества $X$ до множества $Y$ и обозначают $CX, \ol{X}, X^c$\\
$(X^c)^c = X,\quad X \cup X^c = U,\quad X \cap X^c = \varnothing$

Декартовым или пряиыи произведением множеств $X$ и $Y$ называется множество всех упорядоченных пар, таких что первый элемент принадлежит $X$, а второй~--- $Y$\\
$X \times Y = \{(x, y):\ x\in X\ y \in Y$
\skip
\subsection{Аксиомы вещественных чисел}
(13-17)\\
\q1. \q Аксиомы поля. В множестве $\R$ определены две операции называемые сложением и умножением, действующме из $\R \times \R$ в $\R$ и удовлетворяющие следующим свойствам.

1.1. Сочетательный закон (ассоциативноть) сложения:\\
$(x + y) + z = x + (y + z)$.

1.2. Переместительный закон (коммутативность) сложения:\\
$x + y = y + x$.

1.3. Существует число нуль (0, нейтрвльный элемент по сложению), такое что $x + 0 = x$ для всех $x$.

1.4. Для любого числа $x$ существует такое число $\ol{x}$, что $x + \ol{x} = 0$ (это число называется противоположныи числу $x$ и обозначается $-x$).

1.5. Сочетательный закон (ассоциативность) умножения:\\
$(xy)z = x(yz)$.

1.6. Переместительный закон (коммутативность) умножения:\\
$xy=yx$.

1.7. Существует вещественное число единица (1б нейтральный элемент по умеожению), отличный от нуля, такое что $x\cdot 1 = x$ для всех $x$.

1.8. Для любого числа $x$, отличного от нуля, существует такое $x'$, что $xx' = 1$(это число называется обратным к $x$ и обозначается $x^{-1}$ или $\frac{1}{x}$).

1.9. Распределительный закон (дистрибутивность):\\
$x(y + z) = xy + xz$.

\q2. \q Аксиомы порядка. Между элементами $\R$ определено отношение $\le$ со следующими своствами.

2.1. Для любых $x, y$ верно $x \le y$ или $y \le x$.

2.2 Транзитивность: если $x \le y, y\le z$, то $x\le z$.

2.3. Если $x \le y$ и $y \le x$, то $x = y$.

2.4. Если $x \le y$, то $x + z \le y + z$ для любого $z$.

2.5. Если $0 \le x$ и $0 \le y$, то $0 \le xy$

\q3. \q Аксиома Архимеда. Каковы бы ни были положительные числа $x, y \in \R$, существует такое натуральное число $n$, что $nx > y$.

\q4. \q Аксиома Канторва о вложенных отрезках.\\
Пусть $\{[a_n, b_n]\}^\infty_{n = 1}$~--- последовательность вложенных отрезков, то есть\\
$a_n \le a_{n + 1} \le b_{n+1} \le b_n \forall n \in \N$.\\
Тогда существует точка, принадлежащая одновременно всем отрезкам $[a_n, b_n]$, то ечть\\
$\bigcap\limits^\infty_{n = 1} [a_n, b_n] \neq \varnothing$.\\
Важно, что в определение отрезки, так как в случае $(0, \frac{1}{n}]$ пересечение равно $\varnothing$.
\skip
\subsection{Метод математической индукции. Бином Ньютона}
(21-24)\\
Пусть $\{P_n\}^\infty_{n = 1}$~--- последовательность утверждений. Если\\
1) $P_1$ верно,
2) для любого $n \in \N$ из $P_n$ следует $P_{n + 1}$\\
то $P_n$ верно для всех $n \in \N$.

\q Бином Ньютона. Если $n \in \Z_+$, $x, y \in \R$ или $\C$, то\\
$(x + y)^n = \ds\sum\limits^n_{k = 0} C^k_n x^k y^{n-k}$.\\
При $n=0$ и $n=1$ очевидно. $n=1$ служит базой индукции.\\
$(x + y)^{n + 1} = (x + y)(x + y)^n = (x + y)\ds\sum\limits^n_{k = 0} C^k_n x^k y^{n-k} = \sum\limits^n_{k = 0} C^k_n x^{k + 1} y^{n-k} + \sum\limits^n_{k = 0} C^k_n x^k y^{n - k + 1} = \sum\limits^{n + 1}_{k = 1} C^{k - 1}_n x^k y^{n + 1 -k} + \sum\limits^n_{k = 0} C^k_n x^k y^{n + 1-k} = C^n_n x^{n + 1}y^0 + \sum\limits^n_{k = 1} (C^{k - 1}_{n} + C^k_n) x^k y^{n + 1-k} + C^0_nx^0y^{n + 1} = C^{n + 1}_{n + 1}x^{n + 1}y^0 + \sum\limits^{n}_{k = 1} C^k_{n + 1} x^k y^{n + 1-k} + C^0_{n + 1}x^0y^{n + 1} = \sum\limits^{n + 1}_{k = 0} C^k_{n + 1} x^k y^{n + 1 - k}$.
\skip
\subsection{Существование максимума и минимума конечного множества, следствия}
(25-26)\\
Число $M$ называется максимумом или наибольшим элементом множества $E \subset \R$, если $M \in E$ и $\forall x \in E\ x \le M$, обозначается $\max E$.

\T Во всяком конечном подмножестве $\R$ есть наибольший и наименьший элемент.

\D Проведем индукцию по числу $n$ элементов сножества. База индукции $n = 1$: если в множестве всего один элемент, то он и наибольший, и наименьший. Для определённости индукционный переход проведём в случае максимума. Пусть всяке $n$-элементное подмножество $\R$ имеет максимум, $E$~---$(n + 1)$-элементное подмножество $\R$:\\
$E = \{x_1, \cdots, x_n, x_{n + 1}\}$.\\
Обозначим\\
$c = \max\{x_1, \cdots, x_n\}$.\\
Если $c \le x_{n + 1}$, то очевидно $x_{n + 1} = \max E$, иначе $c = max E$.

\S1. Во всяком непустом ограниченнои сверху (снизу) подмножестве $\Z$ есть наибольший (наименьший) элемент.

\D Пусть $E \subset \Z$, $E \neq \varnothing$, $E$ ограничено сверху. Выберем какой-нибудь элемент $n_0 \in E$ положим\\
$E_1 = \{n \in E: n >\ge n_0\}$.\\
Поскольку $E$ ограничено сверху, то множество $E_1$ конечно (в нем не более $M - n_0 + 1$ элементов, где $M$~--- верхняя граница $E$). По теореме в множестве $E_1$есть наибольший элемент, он же будет наибольшим элементом в $E$.

\S2. Во всяком непустом подмножестве $\N$ есть наименьший элемент.


\skip
\subsection{Целая часть числа. Плотность множества рациональных чисел}
(26-27)

\Op Пусть $x \in \R$. Наибольшее целое число, не превосходящее $x$, называется целой частью $x$ и обозначается $[x]$. Она существует так как подмножество целых чисел, имеющее верхнюю границу, имеет максимум.

\Zam1. Из определения следует, что\\
$[x] \le x < [x] + 1, \qquad x - 1 < [x] \le x$

\T \q Плотность множества рациональных чисел. Во всяком непустом иитервале есть рациональное число.

\D Пусть $a, b\in \R$, $a < b$, тогда $\frac{1}{b - a} > 0$ и по аксиме Архимеда найдётся такое $n\in N$, что $n > \frac{1}{b - a}$, то есть $\frac{1}{n} < b - a$. Положим $c = \frac{[na] + 1}{n}$, тогда $c \in \Q$ и\\
$c \le \frac{na + 1}{n} = a + \frac{1}{n} < a + b - a = b$\\
$c > \frac{na - 1 + 1}{n} = a$\\
то есть $c \in (a, b)$.

\S3. Во всяком интервале бесконечно много рациональных чисел.

\D Пксть в некотором интервале $(a, b)$ количество рациональных чисел конечно. Обозначим $x_1$ наименьшее из них. Тогда в интервале $(a, x_1)$ нет ни одного рационпльного число, что противоречит теореме.
\skip
\subsection{Две теоремы о "бедности" счeтных множеств}
(38)

\T Всяякое бесконечное множество содержит счётное подмножество.

\D Пусть множество $A$ бесконечено. Тогда в нём есть элемент $a_1$. Множество $A\bsl \{a_1\}$ бесконечно, поэтому в нём есть элемент $a_3$. Ввиду бесконечности множества $A$ этот процесс не оборвётся ни на каком шаге; продолжая его и далее, получим множество $B = \{a_1, a_2,\cdots\}$, которое по построению будет счётным одмножеством $A$.

\T Всякое бесконечно подмножество счётного множества счётно: есди $A$ счётно, $B \subset A$ и $B$ бесконечно, то $B$ счётно.

\D Расположим элементы $A$ в виде последовательности: $A =\{a_1, a_2, a_3, \cdots\}$.\\
Будем нусеровать элементы $B$ в порядке их появления в этой последовательности. Тем самым каждый элемент $B$ будет занумерован ровно один раз и, так как $B$ бесконечено, для нумерации будет использован весь натуральный ряд.
\skip
\subsection{Теорема об объединении не более чеи счeтных множеств (с леммой)}
(39)

\L Пусть элементы множества $A$ предстаимы в виде бесконечной в обоих направлениях матрицы (занумерованы с помощью упорядоченной пары натуральных чисел по одному разу с использованием всех возможных пар). Тогда $A$ счётно.

\D Занумеруем элементы множества $A$ по диагоналям $A = \{a_{11}, a_{12}, a_{21}, a_{13}, a_{22}, \cdots\}$.

\Op Не более чем счётное множество~--- счётное, конечное или пустое множество

\T Не больее чем счётное объединение не более чем счётных множеств.

\D Пусть $B = \ds\bigcup\limits^{n}_{k=1} A_k$ или $B = \ds\bigcup\limits^{\infty}_{k=1} A_k$, множества $A_k$ не более чем счётны. Запишем элементы $A_1$ в первую строку матрицы, элементы $A_2\bsl A_1$~--- во вторую строку и так далее, то есть если задано множество $A_k$, то элементы $A_k \bsl \ds\bigcup\limits^{k - 1}_{j = 1} A_j$ запишем в $k$-тую строку матрицы. Таким образом все элементы множества $B$ окажутся записанными в клетки матрицы (но при этом некоторые клетки могут остаться пустыми). Значит $B$ Равномощно некоторому подмножеству счётного множества $\N \times \N$. А подмножество счётного либо счётно, либо конечно, либо пусто, то есть не более чем счётно.
\skip
\subsection{Счeтность множества рациональных чисел}
(40)\\

\T Множество рациональных чисел счётно.

\D Обозначим $\Q_+ \{x\in \Q: x > 0\}, \qquad \Q_- = \{x \in \Q: x < 0\}$\\
При всех $q \in \N$ множество $\Q_q = \{\frac{1}{q}, \frac{2}{q}, \frac{3}{q}, \cdots\}$ счётно. По теореме $Q_+ = \ds\bigcup^{\infty}_{q=1} \Q_q$ счётно. Аналогично $\Q_-$ счётно. По той же теореме\\
$\Q = \Q_+ \cup \Q_- \cup \{0\}$ счётно.

\S1. Если $a, b \in \R$, $a < b$, то $\Q \cap (a, b)$ счётно.
\skip
\subsection{Несчeтность отрезка}
(40)\\
\T Отрезок $[0, 1]$ несчётен.

\D Допустим противное пусть отрезок $[0, 1]$ счётен, то есть все числа отрезка $[0, 1]$ можно разположить в виде последовательности [0, 1] = $\{x_1, x_2, x_3, \cdots\}$.\\
Разобъём отрезок $[0, 1]$ на три равных отрезка $[0, \frac{1}{3}]$, $[\frac{1}{3}, \frac{2}{3}]$, $[\frac{2}{3}, 1]$ и обозначим через $[a_1, b_1]$ тот из них, который не содержит точки $x_1$. Далее разобъём отрезок $[a_1, b_1]$ на три отрезка и обозначим через $[a_2, b_2]$ тот из них, который не содержит точки $x_2$. Этот процесс продолжим неограниченно. В результате мы построим последовательность вложенных отрезков $\{[a_n, b_n\}^\infty_{n = 1}$, причём $x_n\notin [a_n, b_n]$ при любом $n$. По аксиоме о вложенных отрезках существует точка $x^*$, принадлежащая всем отрезкам $[a_n, b_n]$. Но тогда $x^* = x_m$ при некотором $m$. По построению $x^* \notin [a_m, b_m]$, что противоречит принадлежности $x^*$ всем отрезкам.

\S2. Множества вещественныз чисел $\R$ и иррациональных чисел $\R \bsl \Q$ несчётны.

\Op Если множество эквивалентно отрезку $[0, 1]$, то говорят, что оно имеет мощность континуума.

\Zam1. Любой невырожденный отрезок имеет мощность континнума. Также как и любой промежуток, вся прямая $\R$ и всё пространство $\R^m\ m \in \N$.

\Zam2. Множество всех функций $f: [0, 1] \to \R$, более богато элементами, чем отрезок $[0, 1]$, то есть оно не равномощно отрезку, но имеет часть, равномощную отрезку.

\Zam3. Мощность~--- класс эквивалентности. Два множества попадают в один класс, если они эквивалентны.

\Zam4. Если из бесконечного множества удалить конесное число элемнтов, множество окажется бесконечным, то получится множество равномощное исходному. Поэтому любое бесконечное множество имеет равномощное подмножество, не совпадающее с исходным множеством.
\skip
\subsection{Единственность предела последовательности. Ограниченность сходящейся последовательности}
(50)

\T \q Единственность предела последовательности. Последовательность в метрическом пространстве не может иметь более одного предела: если $x_n \to a$, $x_n \to b$, то $a = b$.

\D Предположим противное: пусть $a \neq b$. Тогда по аксиоме $\rho(a, b) > 0$. Возьмём $\varepsilon = \frac{\rho(a, b)}{2}$. По определению предела, $\exists N_1, N_2$, что $\forall n > N_1\ \rho(x_n, a) < \varepsilon$ и $\forall n > n_2\ \rho(x_n, b) < \varepsilon$. Тогда если $n > \max(N_1, N_2)$, то по аксилмам расстояния\\
$\rho(a, b) \le \rho(a, x_n) + \rho(x_n, b) < \varepsilon + \varepsilon = \rho(a, b)$\\
Что невозможно.

\T Сходящаяся последовательность ограничена.

\D Пусть $x_n \to a$. Взяв $\varepsilon = 1$ найдем $N$, что для всех номеров $n > N$ будет $\rho(x_n, a) < 1$. Пусть\\
$R = \max\{\rho(x1, a), \cdots, \rho(x_n, a), 1\}$,\\
тогда $\rho(x_n, a) \le R$ при всех $n \in \N$.
\skip
\subsection{Предельный переход в неравенстве. Теорема о сжатой подпоследовательности}
(51)

\T \q Предельныйпереход в неравенстве. Пусть ${x_n}$, ${y_n}$~--- вещественные последовательности $x_n \le y_n$ при всех натуральных $n$, $a, b \in \R$ $x_n \to a,\ y_n \to b$. Тогда $a \le b$.

\D Предположим противное: пусть $a > b$. Тогда $\varepsilon = \frac{a - b}{2} > 0$. По определению предела найдутся такие номера $N_1, N_2$, что $a = \varepsilon < x_n$ для всех $n > N_1$, а $y_n < b + \varepsilon$ для всех $n > N_2$. Значит если $n > \max\{N_1, N_2\}$, то\\
$y_n < b + \varepsilon = \frac{a + b}{2} = a - \varepsilon < x_n$,\\
что противоречит условию.

\Zam1. $x_n = -\frac{1}{n}$, $\frac{1}{n}$ показывает, что из $x_n < y_n$ не следует $\lim x_n < \lim y_n$.

\S1. 1. Если $x_n \le b$ при всех $n \in \N$ и существует $\lim x_N$, то $\lim x_n \le b$.\\
2. Если $x_n \ge a$ при всех $n \in \N$ и существует $\lim x_N$, то $\lim x_n \ge a$.\\
3. Если $x_n \in [a, b]$ при всех $n \in \N$ и существует $\lim x_N$, то $\lim x_n \in [a, b]$.

\Zam2. Свойство отрезка из утверждения 3 (неверное для других типов отрезков) называется замкнутостью.

\T \q О сжатой последовательности (О двух милиционерах). Пусть $\{x_n\},\ \{y_n\}, \{z_n\}$~--- вещественные последовательности, $x_n \le y_n \le z_n$, при всех $n \in \N,\ a \in \R$, $\lim x_n = \lim z_n = a$. Тогда предел $\{y_n\}$ существует и равен $a$.

\D Возьмём $\varepsilon > 0$. По определению, предела найдутся такие номера $N_1, N_2$, что $a - \varepsilon < x_n$ для всех $n > N_1$, а $z_n < a + \varepsilon$ для всех $n > N_2$. Пусть $N = \max\{N_1, N_2\}$. Тогда при $n > N$\\
$a - \varepsilon < x_n \le y_n \le z_n < a + \varepsilon$.\\
В силу произвольности $\varepsilon$ предел $\{y_n\}$ существует и равен $a$.

\Zam2. Отметим, что если $|y_n| \le z_n$ при всех $n \in \N$ и $z_n \to 0$, то $y_n\to 0$.

\Zam3. В обеих теорема достаточно  выполнения неравенств для всех номеров, начиная с некоторого.

\skip
\subsection{Бесконечно малые. Аррифметические действия над сходящимися последовательностями}
(53)

\Op Последовательность вещественных или комплексных чисел называется бесконечно малой, если она стремится к нулю.

\L Произведение бесконечно малой последовательности на ограниченную есть бесконечно малая: $\{x_n\}$~--- бесконечно малая числовая последовательность, $\{y_n\}$~--- ограниченная числовая последовательность, тогда $\{x_n y_n\}$~--- бесконечно малая.

\D В силу ограниченности $\{y_n\}$ Найдётся $K > 0$, что $|y_n| \le K$ при всех $n$. Возьмём $\eps > 0$. По определению предела последовательности $\{x_n\}$ существует такой номер $N$, что $|x_n| < \frac{\eps}{K}$ для всех $n > N$. Но тогда для всех $n > N$;\\
$|x_n y_n| < \frac{\eps}{K}	\cdot K = \eps$.\\
В силу произвольности $\eps$ это и означает $x_n y_n \to 0$.

(57)

\T \q Арифметические действия над сходящимися последовательностями в нормированном пространстве. Пусть $(X, ||\cdot||)$~--- нормиованное пространство, $\{x_n\},\ \{y_n\}$ в $X$, $\lambda_n$~--- числовая последовательность $x_0,\ y_0 \in X,\ \lambda_0 \in \R (\textit{или} \C)$, $x_n \to x_0,\ y_n \to y_0,\ \lambda_n \to \lambda_0$. Тогда\\
1. $x_n + y_n \to x_0 + y_0$\\
2. $\lambda_n x_n \to \lambda_0 x_0$\\
3. $x_n - y_n \to x_0 = y_0$
4. $||x_n|| \to ||x_0||$

\T \q Арифметические действия над сходящимися числовыми последовательностями. Gecnm ${x_n},\ {y_n}$~--- числовые последовательности, $x_0, y_0 \in \R$ (или$\C$), $x_n \to x_0,\ y_n\to y_0$. Тогда\\
1. $x_n + y_n \to x_0 + y_0$
2. $x_n y_n \to x_0 y_0$
3. $x_n - y_n \to x_0 - y_0$
4. $|x_n| \to |x_0|$
5. Если $y_n \neq 0$ при всех $n$ и $y_0 \neq 0$, то $\frac{x_n}{y_n} \to \frac{x_0}{y_0}$.

\D

1. Возьмём $\eps > 0$. По определению предела найдутся такие номера $N_1$ и $N_2$, что $||x_n - x_0|| < \frac{\eps}{2}$для всех $n > N_1$, а $||y_n - y_0|| < \frac{\eps}{2}$ для всех $n > \N_2$. Положим $N = \max\{N_1, N_2\}$. Тогда при всех $n > N$ юудет\\
$||(x_n + y_n) - (x_0 + y_0)|| \le ||x_n - x_0|| + ||y_n - y_0|| \le \frac{\eps}{2} + \frac{\eps}{2} = \eps$.

2. По неравенству треугольника\\
$||\lambda_n x_n - \lambda_0 x_0|| = ||(\lambda_n - \lambda_0) x_n + \lambda_0(x_n - x_0)|| \le |\lambda_n - \lambda_0||||x_n|| + ||\lambda_0||||x_n - x_0||$\\
$\{|\lambda_n - \lambda_0|\}$ и $\{||x_n - x_0||\}$~--- бесконечно малые. $\{||x_n||\}$~--- ограничена по теореме. А $\{|\lambda_0|\}$~--- постоянная. Тогда оба слагаемых бесконечно малые, по лемме, тогда их сумма тоже бесконечно малая.

3. $x_n - y_n = x_n + (-1)y_n \to x_0 + (-1)(y_0) = x_0 - y_0$.

4. Следует из $|||x_n|| - ||| \le ||x_n - x_0||$ и теоремы о двух милиционерах.

5. Достаточно доказать, что $\frac{1}{y_n} \to \frac{1}{y_0}$. Поскольку\\
$\frac{1}{y_n} - \frac{1}{y_0} = (y_0 - y_n)\cdot\frac{1}{y_0}\cdot\frac{1}{y_n}$\\
$\{y_0 - y_n\}$~-- бесконечно малая, $\{\frac{1}{y_0}\}$~--- ограниченная. Осталось доказать ограниченность $\{\frac{1}{y_n}\}$.\\
По определению предела для $\eps = \frac{|y_0|}{2} > 0$ существует номер $N$, что $||y_n - y_0|| < \eps$ для всех $n > N$. Тогда при всех $n > N$ по свойствам модуля\\
$|y_n| = |y_0 + y_n - y_0| \ge |y_0| - |y_n - y_0| > |y_0| - \eps = \frac{|y_0|}{2}$\\
Обозначим $k = \min\{|y_1|, \cdot, |y_n|, \frac{|y_0|}{2}\}$. $k > 0$ и $|y_n| \ge k$ при всех $n$. Следовательно $|\frac{1}{y_n}|\le \frac{1}{k}$ при всех $n$, что и зощначает ограниченность $\{\frac{1}{y_n}\}$
\skip
\subsection{Свойство скалярного произведения. Неравенство Коши-Буняковского-Шварца. Норма порождeнная скалярным произведением}
(59)

\Op Пусть $X$~--- векторное пространство над $\R$ или $\C$. Функция $\varphi: X \times X \to \R$ (или $\C$) называется скалярным произведением в $X$ (обозначаение $\varphi{x, y} = \langle a, b \rangle$), если она  удовлетворяет следующим свойствам.\\
1. Линейность по первому аргументу: для всех $x_1, x_2, y \in X$ и всех $\lambda, \mu \in \R(\C)$\\
$\langle\lambda x_1 + \mu x_2, y\rangle = \lambda \cdot\langle x_1, y\rangle + \mu\cdot\langle x_2, y\rangle$.\\
2. Эрмитовская симметричность\\
$\langle y, x\rangle = \ol{\langle x, y\rangle}$\\
3. Положительная определённость:\\
$\langle x, x\rangle \ge 0;\quad \langle a, b\rangle = 0 \lra x = \theta$\\
В вешественном случае черту можно опустить.\\
Некоторые свойства скалярного произведения:\\
1. $\langle x, y_1, + y_2\rangle = \langle x, y_1\rangle + \langle x, y_2\rangle$.\\
2. $\langle x, \lambda y\rangle = \ol{\lambda}\langle x, y\rangle$.\\
3. $\langle \theta, y\rangle = \langle x, \theta\rangle = 0$

\T \q Неравенство Коши-Буняковского-Шварца.(59)\\
$|\langle x, y \rangle|^2 \le \langle x, x\rangle\langle y, y\rangle$\\
Если рассматривать норму, порождённую скалярным произведением, то верно: $|\langle x, y \rangle| \le ||x||\cdot||y||$, так как $||x|| = \sqrt{\langle x, x\rangle}$.

\D Если $y = \theta$ неравенство выполнено. Иначе положим\\
$\lambda = -\frac{\langle x, y\rangle}{\langle y, y\rangle}$\\
Тогда в силу аксиом скаялярного произведения и равенства $\lambda\ol{\lambda} = |\lambda|^2$\\
$\langle x + \lambda y, x + \lambda y\rangle = \langle x, x\rangle + \ol{\lambda}\langle x, y\rangle + \lambda\langle y, x\rangle + |\lambda|^2 \langle y, y\rangle = \langle x, x\rangle + (-1 -1 + 1)\frac{|\langle x, y\rangle|^2}{\langle y, y\rangle}$\\
Таким образом\\
$\langle x, x\rangle\langle y, y\rangle - |\langle x, y\rangle|^2 = \langle y, y\rangle\langle x + \lambda y, x + \lambda y\rangle \ge $

\Zam1. Неравенство обращается в равенство тогда и толко тогда, когда вектора коллинеарны.

Вункция $p(x) = \sqrt{\ang xx}$~--- норма в $X$.\\
Положительная определённость следёет из аксиом.\\
$p(\lambda x) = \sqrt{\ang{\lambda x}{\lambda x}} = \sqrt{\lambda\ol{\lambda}\ang xx}= |\lambda| p(x)$\\
Докажем неравество треугольника\\
$p^2(x + y) = \ang{x + y}{x + y} = \ang xx + \ang xy + \ang yx + \ang yy = \ang xx + 2\Re\ang xy + \ang yy \le \ang xx + 2|\ang xy| + \ang yy \le p^2(x) + 2p(x)p(y) + p^2(y) = (p(x) + p(y))^2$\\

\Zam2. Неравенство обращается в равенство тогда и толко тогда, когда вектора $x, y$сонаправлены.

\D Еслм один из векторов нулевой, равенство очевидно. Пусть $x, y \neq \theta$.Обращение неравенства в равенство равносильно тому, что\\
$\Re\ang xy = |\ang xy| = ||x||\cdot||y||$\\
Из второго равенства вектора коллинеарны, то есть $x = \lambda y$. Подставляя, получим $\Re\lambda \ang yy = = |\lambda|\ang yy$. Отсюда $\lambda > 0$
\skip
\subsection{Неравенство Коши-Буняковского в $\R^m$ и $\C^m$. Сходимость и покоординатная сходимость}
(61)

\T \q Неравенство Коши-Буняковског и неравенство трейгольника в $\R^m$. Для любых вещественных чисел $x_1, \cdots, x_m$, $y_1, y_m$\\
$(\ds\sum\limits^{m}_{k = 1} x_k y_k)^2 \le (\sum\limits^m_{k - 1} x^2_k)(\sum\limits^m_{k = 1} y^2_k)$.\\
$\sqrt{\ds\sum\limits^m_{k = 1}(x_k + y_k)^2} \le \sqrt{\sum\limits^m_{k = 1}x^2_k} + \sqrt{\sum^m_{k = 1} y^2_k}$.

\D Первое~--- КБШ, второе правило~--- треугольника.

\Op Говорят, что послледовательность $\{x^{(n)}\}$ точек $\R^m$ сходится к пределу $x^{(0)} \in \R^m$ покоординатно, если $x^{(n)}_j \xra[n\to \infty]{} x^{(0)}_j$, для всех $j \in [1: m]$

\L В $\R^m$ покоординатная сходимость и сходимость по евклидовой норме равносильны

\D Утверждение следует из неравенств\\
$|x^{(n)} - x^{(0)}_j| \le |x^{(n)} - x^{(0)}| = \sqrt{\ds\sum\limits^m_{k = 1} (x^{(n)}_k - x^{(0)}_k)^2}\le \sqrt{m} \max\limits_{1\le k\le m} |x^{(n)}_l - x^{(0)}_k|$\\
и теоремы о предельном переходе в неравенствах.

\S. Сходимость последовательности комплексных чисел равносильна одновременной сходимости последовательности из их мнимых частей.
\skip
\subsection{Бесконечно большие и бесконечно малые. Арифметические действия над бесконечно большими}
Билет 15: Бесконечно большие и бесконечно малые. Арифметические действия над бесконечно большими
(65-67)

\Op Говорят, что вещественная последовательность $\{x_n\}$ стремится к\\
1) плюс бесконечености, и пишут\\
$\lim\limits_{n\to\infty} x_n = +\infty$ или $x_n \xra[n\to\infty]{} +\infty$, если\\
$\forall E > 0\ \exists N \in \N\ \forall n \in \N: n > N\ x_n > E$;\\
2) минус бесконечности, и пишут\\
$\lim\limits_{n\to \infty} x_n = -\infty$ или $x_n \xra[n\to \infty]{} -\infty$, если\\
$\forall E > 0\ \exists N\in\N\ \forall n\in\N: n > N\ x_n < -E$;\\
3) бесконечности, и пишут\\
$\lim\limits_{n\to \infty} x_n = \infty$ или $x_n\xra[]{n\to\infty} \infty$, если\\
$\forall E > 0 \exists N \in \N\ \forall b \in \N: n > N\ |x_n| > E$

\Zam1. Из определений 1 и 2 следует третье, обратно неверно $x_n = (-1)^nn$

\Op Последовательность называется бесконечно большой, если стремится к бесконечности.

\Zam2. Из определения следует, что если $x_n\to\infty$, то $x_n$ неограничена. Обратное неверно $x_n = (1 + (-1) ^n)n$ не ограничена и не стремится к бесконечности.

\Zam3. Понятно, что последовательность не может стремится одновремнно к бескончености и к конечному пределу, а также к бесконечности разных знаков. Тогда предел единственный в $\ol{\R}$.

\Zam4. Определение достаточно проверять лишь для достаточно больших $E$, можно также опустить требование $E > 0$.

\Zam5. Определение сходящейся последовательности не меняется:: последовательность называется сходящейся, если она имеет конечный предел. Бесконечно большие последовательности считаются расходящимися.

\L \q Связь между бесконечно большими и бесконечно малыми. Пусть $\{x_n\}$~--- числовая последовательность $x_n\neq 0$ ни прикаком $n$. Тогда последовательность $\{x_n\}$~--- бесконечно большая в том и только том случае когда $\{\frac{1}{x_n}\}$~--- бесконечно малая.

\D Пусть $x_n\to\infty$. Возьмём $\eps > 0$ и для числа $E = \frac{1}{\eps}$ подбереём $N$, что для всех $n > N$ $|x_n| > E$. Последнее равносильно $|\frac{1}{x_n} < \eps|$, что и означает стремление $\frac{1}{x_n}$ к нулю. в обратную сторону аналогично.

\T \q Арифметические действия над бесконечно большими. Пксть $\{x_n\}, \{y_n\}$~--- числовые последовательности.\\
1. Если $x_n \to +\infty, \{y_n\}$ ограничена снизу, то $x_n + y_n\to +\infty$\\
2. Если $x_n\to -\infty, \{y_n\}$ ограничена сверху, то $x_n + y_n \to -\infty$\\
3. Если $x_n \to \infty, \{y_n\}$ ограничена, то $x_n + y_n \to \infty$\\
4. Если $x_n \to \pm\infty, y_n \ge b > 0$ для всех $n$ (или $y_n \to b_1 > 0$), то $x_n \to \pm\infty$\\
5. Если $x_n \to \pm\infty, y_n \le b < 0$ для всех $n$ (или $y_n \to b_1 < 0$), то $x_n \to \mp\infty$\\
6. Если $x_n \to \infty, |y_n| \ge b > 0$ для всех $n$ (или $y_n \to b_1 \neq 0$), то $x_n \to \infty$\\
7. Если $x_n \to a\neq 0, y_n \to 0, y_n\neq 0$; при всех $n$, то $\frac{x_n}{y_n} \to \infty$\\
8. Если $x_n\to  a \in \C, y_n\to\infty$, то $\frac{x_n}{y_n} \to 0$\\
9. Если $x_N \to \infty, y_n \to b \in \C, y_n\neq 0$ при всех $n$, то $\frac{x_n}{y_n} \to \infty$.

\D Докажем отверждения 1, 6, 8\\
1. Возьмём $E > 0$. По определению ограниченности снизу ,найдётся такое число $m \in \R$, что $y_n \ge$ 0 при всех $n$. По определени. бесконечного предела существует такой номер $N$, что $x_n > E - m$ для всех $n > N$. Тогда для всех $n > N$ $x_n + y_n > E - m + m > E$.\\
6. Пусть $|y_n| \ge b > 0$ для всех $n$. Возьмём $E > 0$. По определению бесконечного предела существует такой номер $N$, что $|x_n| > \frac{E}{b}$ для всех $n > N$. Тогда для всех $n > N$ $|x _n y_n| > \frac{E}{b} \cdot b = E$\\
Gecnm $y_n \to b_1 \neq 0$. Положим $b = \frac{|b_1|}{2}$, если $b_1$~--- число и $b = 1$, если $b_1$~--- бесконечность. Тогда начиная с некоторого номера $|y_n| \ge b$ и применимо толко что доказанное.\\
8. По теореме о пределе произведения и лемме $\frac{x_n}{y_n} = x_n \cdot\frac{1}{y_n}\to a\cdot 0 = 0$

\Zam1. Часть утверждений теорем об арифметических свойствах можно объединить следующей формулировкой. Есле $\{x_n\}, \{y_n\}$`--- вещественные последоватедьности, $x_n \to x_0 \in \ol{\R}, y_n\ to y_0, \in \ol{\R}$, знак $*$ означает одно из четырёх арифметических действий и $x_0*y_0$ определено в $\ol{\R}$, то $x_n*y_n \to x_0*y_0$ \\
Теорема не позволет сделать заключение о значении в следующих четырёх случаях\\
1. $x_n \to +\infty, y_n \to -\infty\qquad x_n + y_n \to ?$\\
2. $x_n \to 0, y_n \to \infty\qquad x_n y_n \to ?$\\
3. $x_n \to 0, y_n \to 0 \qquad \frac{x_n}{y_n} \to ?$\\
4. $x_nto \infty, y_n \to \infty\qquad \frac{x_n}{y_n}\to ?$
\skip
\subsection{Свойства открытых множеств. Открытость шара. Внутренность}
(68-70)

 \Op Точка $a$ называется внутренней точкой множества $D$, если существует окрестность точки $a$, содержащаяся в $D$.

\Op Множество $D$ называется открытым, если все его точки внутренние.

\Op Пусть $(X, \rho)$~--- метрическое пространство, $a \in X, r > 0$. Множество\\
$B(a, r = \{x \in X: \rho()x, a < r\})$\\
называются открытым шаром радиуса $r$ с центром в точке $a$ или окрестностью ($r$-окрестностью) точки $a$ и обозначается $V_a(r)$ или $V_a$, если  значение $r$ несущественно.

Покажем, что открытый шар $B(a, r)$~--- открытое множество. Пусть $p \in B(a, r)$, то есть $\rho(p, a) < r$\\
Положим $h = r - \rho(p, a)$ $(h > 0)$ и проверим что $B(p, h) \subset B(a, r)$. Пусть $x \in B(p, h)$, то есть $\rho(x, p) < h$, тогда\\
$\rho(x, a) \le \rho(x, p) + \rho(p, a) < h + r - h = r$\\
То есть $x \in B(a, r)$.

\T \q Свойства открытых множеств.\\
1. Объежинение любого семейства открытых множеств открыто.\\
2. Перечечение конечного семейства открытых множеств открыто

\D 1. Пусть задано семейство открытых множеств $\{G_\alpha\}_{\alpha \in A},\ G = \ds\bigcup\limits_{\alpha \in A} G_\alpha,\ x \in G$. Докажем, что $x$~--- внутренняя точка $G$. Так как $x \in G$, найдётся $\alpha: x \in G_\alpha$, тогда существует шар $B(x, r) \subset G_\alpha$, тогда $B(x, r) \subset G$\\
2. Пусть задано конечное семейство открытых множество $\{G\}^n_{k = 1},\ G = \bigcup\limits^n_{k = 1} G_k,\ x \in G$. Тогда x принадлежит каждому из множеств и в силу открытости найдутся такие $r_1, r_2, \cdots, r_n$, что $B(x, r_k) \subset G_k$. При $r = \min\{r_1, \cdots, r_n\}$ $B(x, r) \subset G_k$ для любого $k \in [1 : n]$. Тогда, по определению, $B(x, r) \subset G$.

\Zam 1. Пересечени бесконечного семейства открытых множеств не обязано быть открытым. Например\\
$\ds\bigcap\limits^\infty_{n = 1} (-\frac{1}{n}, \frac{1}{n}) = \{0\}$\\
а одноточечное мнодество не является открытым в $\R$.

\Op Множество всех внутренних точне множества $D$ называется внутренностью $D$ и обозначается $\mathring{D}$ или Int $ D$

\Zam 2. Внутренность $D$ есть:\\
а) объединение всех открытых подмножеств $D$\\
б) максимальное по включению открытое подмножество $D$.

\D Пусть $G$~--- объединение всех открытых подмножеств $D$, тогда $G \subset D$, $G$ содержит любое открытое подмножество $G$ и открыто по теореме, то есть $G$~--- максимальное по включению открытое подмножество $D$. Если $x$~--- внутренняя точка $D$, то $D$ содержит окрестность $V_x$ точки $x$, а тогда $x \in V_x \subset G$. С другой стороны, если $x \in G$, то $x$ принадлежит некоторому открытому подмножеству $D$ и значит является его внутренней точкой и, тем более, внутренней точкой $D$.

\Zam3. Множество открыто тогда и только тогда, когда оно совпадает со своей внутренностью.
\skip
\subsection{Предельный точки. Связь Открытости и замкнутости. Свойство замкнутых множеств. Замыкание}
(70-75)

\Op Точка $a$ называется предельной точкой  или точкой сгущения множества $D$, если в любой окрестности точки $a$ найдётся точка множества $D$, отличная от $a$.\\
$\forall \dot{V_a}\ \exists x \in \dot{V_a} \cap D$

\Op Множество $D$ называется замкнутым (в $X$), если он содержит все свои предельные точки.

\T Множество открыто тогда и только тогда, когда его дополнение замкнуто.

\D Пусть $D^c$ замкнуто. Возьмём точку $x \in D$ и докажем, что $x$~--- внутренняя точка $D$; в силу произвольности $x$ это и будет обозначать открытость $D$. Поскольку $x \notin D^c$, а $D^c$ замкнуто $x$ не является прелельнойточкой $D^c$, то есть существует такая окрестность $\dot{V_x} \cap D^c = \varnothing$, тогда $V_x \cap D^c = \varnothing$, то есть $V_x \in D$, то есть $x$~--- внутрення точка.\\
Пусть $D$ открыто. Возьмём точку $x$, предельную для $D^c$, и докажем, что $x\in D^c$; в силу произвольности $x$ это и будет означать замкнутость $D^c$. Поскольку в любой окрестности точки $X$ найдётся точка $D^c$, $x$ не является внутренней точкой $D$, $x \notin D \Ra x \in D^c$

\T \q Свойства замкнутых множеств.\\
1. Пересечение любого семейства замкнутых множеств замкнуто.\\
2. Объединение конечного семейства замкнутых множеств замкнуто.

\D Следует из аналогичной теоремы для открытых множеств и формул.\\
$\ds\bigcap\limits_{\alpha \in A} F_\alpha = (\bigcup\limits_{\alpha \in A} F^c_\alpha)^c \qquad \bigcup\limits^n_{k = 1} F_k = (\bigcap^n_{k = 1}F^c_k)^c$

\Zam1. $\bigcup\limits_{q \in \Q} \{q\} = \Q$ не замкнуто в $\R$. Так как в любом интервале есть рациональное число, все точки числовой прямой являются предельными.

\Op Точка $a$ называется точкой прикосновения множества $D$, если в любой её окрестности есть точка множества $D$

\Op Множество всех точек соприкосновения называется замыканием $D$ и обозначается $\ol{D}$ или Cl$D$.

\Zam2.Точки соприкосновения~--- объединени множества предельных и изолированых точек $D$.

\Zam4. Точка $a$~--- точка прикосновения множества $D$ тогда и толко тогда, когда существует последовательность $\{x_n\}$ точек множества $D$ стремящаяся к $a$.

\D  Пусть $a$~--- точка прикосновения $D$/Если $a \in D$, то можно взять станционарную последовательность, все члены которой равны $a$. Иначе $a$~--- предельная точка $D$ и искомая последовательность сущесивует по замечанию к определению предельной точки.\\
Обратно, если существует последовательность $\{x_n\}$ с перечисленными свойствами, то по определению предела в любой окрестности точки $a$ найдётся член этой последовательности (туда даже попадают все члены, начиная с некоторого номера), то есть точка множества $D$.

\Zam4. Замыкание $D$ есть:\\
а) Пересечение всех замкнцтых множеств, содержащих $D$\\
б) Минимальное по включению замкнутое множество, содержащее $D$.

\D Пусть $F$~--- пересечение всех замкнутых множеств, содержащих $D$. Тогда $D \subset F$, $F$ содержится в любом замкнутом множестве, содержащем $D$, и $F$ замкнуто по теореме, то есть $F$~--- минимальное по включению замкнутое множество содержащее $D$. Если $x \in \ol{D}$ , то есть $x$ точка прикосновения $F$, а тогда $x \in F$ в силу замкнутости $F$. С другой стороны , если $x \notin \ol{D}$, то у точки $x$ существует окрестность $V_x$, содержащаяся в $D^c$. Тогда её дополнение $V^c_x$ замкнуто и содержит $D$, поэтому $F \subset V^c_x$, то есть $V_x \subset F^c$, тогда $x\notin F$;

\Zam5 Множество замкнуто тогда и толко тогда, когда оно совпадает со своим замыканием.

\Op Внутренность дополнения множества $D$ называется внешностью $D$ и обозначается Ext$D$

\Op Точка $a$ называется граничной точкой множества $D$, если\\
$\forall V_a \exists x_1\in D,\ x_2\in D^c: \{x_1, x_2\} \subset V_a$.

\Op множество всезх граничных точек называется границей и обозначается Fr$D$ или $\delta D$

\Op Множество всех предельных точек множества $D$ называется производным множеством множества $D$ и обозначается $D'$

\Zam6. 1. Ext$D = (\ol{D})^c$\\
2. $\delta D = \ol{D}\bsl \mathring{D}$\\
3. Граница замкнута\\
4. Множество $D'$ замкнуто
\skip
\subsection{Открытость и замкнутость относительно пространства и подпространчтва}
(75)

\T \q Открытость и замкнутость в прострвнстве и подпространстве. Пусть $(x, p)$~--- метрическое пространствоб $D \subset Y\subset X$.\\
1. $D$ открыто в $Y$ тогда и только тогда, когда существует множество $G$, открытое в $X$, что $D = G \cap Y$\\
2. $D$ замкнуто в $Y$ тогда и только тогда, когда существует такое множество $F$, замкнутое в $X$, что $D = F\cap Y$

\D 1. Пусть $D = Y \subset G$, где $G$ открыто в $X$. Возьмём точку $a \in D$. В силу открытости $G$ в $X$ существует окрестность $V^X_a$ точки $a$ в $X: V^X_a \subset G$. Тогда $V^Y_a = V^X_a \cap Y$~--- окрестность содержащаяся в $D$. В силу произвольности $a$ множество $D$ открыто в $Y$.\\
Обратно, пусть $D$ открыто в $Y$. Тогда для каждой точки $a \in D$ найдётся её окрестность в $Y$, содержащаяся в $D: V^Y_a = B^Y(a, r_a) \subset D$. Обозначим $G = \ds\bigcup_{a \in D} B^X(a, r_a)$. Тогда $G$ открыто в $X$ как объединение открытых множеств и $G \cap Y = \ds\bigcup\limits_{a \in D}(B^x(a, r_a)\cap Y) = \bigcup\limits_{a \in D}B^Y(a, r_a) = D$\\
2. По теореме замкнутость $D$ в $Y$ равносильна открытости $Y\bsl D$ в $Y$. По доказанному, последнее равносильно существованию такого открытого в $X$ множества $G$, что $Y\bsl D = G \cap Y$. Осталось обозначить $F = G^c$ и учесть, что соотношения $S = F \cap Y$ и $Y\bsl D = G\bsl Y$ равносильны.
\skip
\subsection{Компактность относительно пространства и подпространства}
(77)

\Op Семейство множество $\{G_\alpha\}_{\alpha \in A}$ называется покрытием множества $K$, если $K \subset \ds\bigcup_{\alpha \in A} G_\alpha$

\Op Пусть $(X, p)$~--- метрическое пространство, $K \subset X$. Покрытие $\{G_\alpha\}_{\alpha \in A}$ иножество $K$ нахывается открытым если при любом $\alpha \in A$ множество $G_a$ открыто в $X$.

\Op Подмножество $K$ метрического пространства $X$ называется компактным, если из люього открытого покрытия $K$ можно извлечь конечное подпокрытие:\\
$\forall\{G_\alpha\}_{\alpha \in A} \subset \ds\bigcup\limits_{\alpha\in A}G_\alpha \Ra \exists\alpha_1, \cdots, \alpha_N \in A\ K = \bigcup^N_{i = 1} G_{\alpha_i}$, $G_\alpha$ открыты в $X$.

\L Пусть $(X, p)$~--- метрическое пространство, $Y$~--- подпространство $X$, $K \subset Y$. Тогда свойства компактности $K$ в $X$ и $Y$ равносильны.

\D Пусть $K$ компактно в $X$. Возьмём покрытие $K$ множествами $V_\alpha$ открытыми в $Y$. По теореме $V_\alpha = G_\alpha \cap Y$, где множества $G_\alpha$ открыты в $X$. Множества $G_\alpha$ образуют покрытие $K$:\\
$K \subset \dsl \bigcup{\alpha \in N}{} V_\alpha \subset \dsl\bigcup{\alpha \in A}{} G_\alpha$.\\
Пользуясь компактностью $K$ в $X$, извлечём из покрытия $\iseq G\alpha A$ конечное подпокрытие $K \subset \dsl \bigcup{i = 1}N G_{\alpha_i}$, но поскольку $K\subset Y$,\\
$K \subset \dsl\bigcup{i=1}{N}(G_{\alpha_i}\cap Y) = \dsl \bigcup{i = 1}N V_{a_i}$\\
Итак, из произвольного покрытия $K$ множествами, открытыми в $Y$ можно извлечь конечное подпокрытие, что и означает компатность $K$ в $Y$.\\
Пусть теперь $K$ компактно $Y$. Возьмём покрытие $K$ множествами $G_\alpha$, открытыми в $X$. Положим $V_\alpha - G_\alpha \cap Y$; Тогда иножества $K$в $Y$ из него можно извлечь конечное подпокрытие $\{V_{\alpha_i}\}^N_i = 1$. Но тогда $\{V_{\alpha_i}\}^N_i = 1$ тоже покрытие $K$ и компактность $K$ в $X$ доказана.
\skip
\subsection{Компактность, замкнутость и ограниченность}
(78)

\T \q Простейшие свойства компактов. Пусть $(X, p)$~--- метрическое пространство, $K \subset X$\\
1. Если $K$ компатно, то $K$ замкнуто и ограничено.\\
2. Если $X$ компактно, а $K$ замкнуто, то $K$ компакт.

\D 1. Докажем, что $K^c$ открыто. Возьмём точку $a\in K^c$ и докажем, что $a$~--- внутренняя точка $K^c$; в силу произволности $a$ это и будет означать, что $K^c$ открыто. Для каждой точки $q \in K$ положим\\
$r_q = \frac{p(q, a)}{2},\quad V_q = B(a, r_q)\quad W_q = B(q, r_q)$\\
Тогда $V_q \cap W_q = \varnothing$. Семейство $\iseq WqK$~--- открытое покрытие компакта $K$. Извлечем из него конечное подпокрытие $\{W_{q_i}\}^N_{i = 1}$: $K \subset \dsl \bigcup{i=1}N W_{q_i} = W$. Тогда $V = \dsl \bigcap{i = 1}N V_{q_i}$~--- окрестность точки $a$, причём $V\cap W = \varnothing$. Тем более $V \cap K = \varnothing$, то есть $V \in K^c$.\\
Докажем, что $K$ ограничено. Зафиксируем точку $a \in X$ и рассмотрим покрытие множества $K$ открытыми шарами $\{B(a, n)\}^\infty_{n = 1}$. В силу компактности $K$ покрывается конечным набором шаров $\{B(a, n_i)\}^N_{i = 1}$ и, следовательно, содержится в шаре $B(a, \dsl\max{1 \le \le N}{} n_i)$.\\
2. Пусть $\iseq G\alpha A$~--- открытое покрытие $K$. Тогда, поскольку $K$ замкнуто, $\iseq G\alpha A \cap \{K^c\}$~--- открытое покрытие $X$. Пользуясь компактностью $X$ извлечём из него конечное подпокрытие $X$: $X = \dsl\bigcap{i=1}N G_{\alpha_i}\cup K^c$. Но тогда $\se {G_{\alpha_i}}{i = 1}N$~--- покрытие $K$.
\skip
\subsection{Две леммы о подпоследовательностях}
(81)

\Zam1. Для строго возрастающей последовательности $\{n_k\}$ при всех $k$ будет $n_k \ge k$. Действительно, $n_1 \ge 1$ и из неравенства $n_k \ge k$ следует, что $n_{k + 1} \ge n_k + 1 \ge k + 1$.

\L Всякая подпоследовательность сходящейся последовательности сходится, и притом к тому же пределу: если $\{x_n\}$~--- последовательность в метрическом пространстве $X$, $x_{n_k}$~--- её подпоследовательность, $a \in X, x_a$, То $x_{n_k} \to a$.

\D Возьмём $\eps > 0$. По определению предела существует такой номер $N$, что $p(x_n, a) < \eps$ для всех $n > N$. Но тогда, если $K > N$, то по замечанию и $n_k > N$, а значит $p(x_{n _k}, a) < \eps$.

\L Пусть $\{x_n\}$~--- последовательность в метричеком пространстве $X$, $\{x_{n_k}\}$ и $\{x_{m_t}\}$~--- её подпоследовательностьи, причём объединение множеств их индексов равно $\N, a \in X$. Тогда, если $x_{n_k} \to a,\ x_{m_l}\to a$, то и $x_n \to a$.

\D Возьмём $\eps > 0$. По определнию предела одной и другой подпоследовательности найдутся такие номера $k$ и $L$, что\\
$\rho(x_{n_k}, a) < \eps\quad$ для всех $k > K$\\
$\rho(x_{m_l}, a) < \eps\quad$ для всех $l > L$\\
Положим $N = \max\{n_k, m_l\}$. Если $n > N$, то или $n$ равно некоторому $n_k$, причём $k > K$, а тогда $\rho(x_n, a)< \eps$, или $n$ равно некоторому $m_l$, причём $l > L$, а тогда $\rho(x_n, a) < \eps$.

\Zam2. Леммы сохраняют силу для $a = \infty$ в нормированном пространстве и для $a = \pm \infty$ для вещественных последовательностей. В доказательстве следует заменить неравенство вида $\rho(x_n, a) < \eps$ на $|x_n| > E$ и т. п.
\skip
\subsection{Лемма о вложенных параллелепипедах. Компактность куба}
(79)

\L Пусть $\se{[a^{(n)}, b^{(n)}]}{n = 1}\infty$~--- последовательность вложенных параллелепипедов в $\R^m$, то есть\\
$a^{(n)}_k\le a^{(n)}_{k + 1}\le b^{(n)}_{k + 1} \le b^{(n)}_k$ при всех $n \in N$ и $j \in [1: m]$\\
Тогда $\dsl\bigcap{n = 1}\infty [a^{(n)}, b{(n)}] \neq \varnothing$.

\D При каждом $k \in [1: m]$ имеем последовательность вложенных отрезков $\se{[a^{(n)}_k, b^{(n)}_k]}{n = 1}\infty$. По аксиоме Кантора, найдётся точка $x^*_k$, принадлежащая всем отрезкам. Тогда $m$-мерная точка $x^* = (x^*_1, \cdots\, x^*_m)$ принадлежит всем параллелепипедам $[a^{(n)}, b^{(n)}]$.

\L Замкнутый куб в $\R^m$ компактен.

\D Пусть $I = [a, b]$~--- куб в $\R^m,\ \delta$~--- его диагональ. Допустим, тчо $I$ не компатен. Обозначим через $\{G_\alpha\}$ такое открытое покрытие $I$, из которого нельзя изввлечь конечное подпокрытие. Разделив каждый отрезок $[a_k, b_k]$ пополам, разобъём куб $I$ на $2^m$ кубов. Среди них найдётся тот, который не покрывается никаким конечным наборо множеств из семества $\{G_\alpha\}$ (так как иначе куб $I$ покрывается конечным набором множеств из этого семейства). Обозначим этот куб $I_1$. Продолжая процесс деления и далее получим последовательность вложенных замкнутых кубов $\se{I_n}{n = 1}\infty$ со следующими свойствами:\\
1) $I\supset I_1\supset I_2 \supset\cdots$.\\
2) $I_n$ не покрывается никаким конечным набором множеств из семейства $G_\alpha$\\
3) Если $x, y \in I_n$, то $|x - y| < \frac{\delta}{2^n}$\\
По лемме существует точка $x^*$, принадлежащая одновременно всем кубам. Следовательно, $x^* \subset I$. Тогда $x^*$ принадлежит некотороиу элементу покрытия $G_{\alpha^*}$. Поскольку $G_{\alpha^*}$ открыто, найдётся такое $r > 0$, что $B(x^*, r) \subset G_\alpha$. Так как $\frac{\delta}{2^n}\xra[n\to\infty]{} 0$, найдётся такое $n$, что $\frac{\delta}{2^n} < r$, то есть $I_n \subset B(x*, r)$. Значит куб $I_n$ покрывается одним множеством, что противоречит свойству 2.
\skip
\subsection{Характеристика компактов в $\R^m$. Принцип выбора}
(82)

\T \q Характеристика компактов в $\R^m$. пусть $K\subset \R^m$. Тогда следующи утверждения равносильны\\
1. $K$ замкнуто и ограничено\\
2. $K$ компатно\\
3. Из всякой плследовательности точек $K$, можно извлечь подпоследовательность, имеющую предел, принадлежащий $K$.

\D проведём по схемк $1 \Ra 2 \Ra 3 \Ra 1$

1 $\Ra$ 2. Поскольку $K$ ограничено, $K$ содержится в некотором замкнутом кубе I. Тогда $K$ замкнуто и по теореме, так как $K = K \cap I$ и $K$ замкнуто в $\R^m$. Куб $I$ компактен по лемме. По теореме заключаем, что $K$ компатно как замкнутое подмножество компакта.

2 $\Ra$ 3 Пусть $\se{x^{(n)}}{n = 1}\infty$~--- последовательность в $K$; обозначим через $D$ множество её значений. Если $D$ конечно, то из $(\{x^{(n)}\})$ можно выделить станционарную подпоследовательность, причём её предел будет совпадать со значением и, значит, приадлежать $K$.\\
Содержателен случай, когда $D$ бесконечно. Докажем от противного, что в этом случае в $K$ есть предельная точка $D$. Если в $K$ нет предельных точек $D$, то у каждой точки $q \in K$ найдётся окрестность $V_q$, содержащая не более одной точки множества $D$. Тогда получаем противоречие с компатностью $K: \iseq VqK$~--- открытое покрытие $K$, из которого нельзя извлечь конечное подпокрытия не только $K$, но даже $D$.\\
Пусть $a \in K$~--- предельная точка $D$. Тогда найдётся номер $n_1$, для которого $p(x^{(n_1)}, a) < 1$. Так как множество $B(a, \frac{1}{2}) \cap D$ бесконечно, найдётся номер $n_2 > n_1$, для которого $p(x^{(n_2)}, a) < \frac{1}{2}$. Этот процесс можно продолжать неограниченно: на шаге с номером $k$, поскольку множество $B(a, \frac{1}{k})\cap D$ бесконечно, найдётся номер $n_k > n_{k - 1}$, для которого $p(x^{n_k}, a) < \frac{1}{k}$. По построению $\{x^{(n_k)}\}$~--- подпоследовательность и $x^{(n_k)}\to a$.

3 $\Ra$ 1. Если $K$ не ограничено, то для каждого натурального $n$ найдётся такая точка $y^{(n)} \in K$, что $|y^{(n)}| > n$. Последовательность $\{y^{(n)}\}$ стремится к бескончности, а тогда из ней нельзя выделить сходящуюс подпоследовательность, так как по лемме любая её полпоследовательность стрмится к бесконечности.\\
Если же $K$ не замкнуто, то у $K$ есть предельная точка $b$, не принадлежащая $K$. Следовательно, существует последовательность точек  $K$, стремящаяся к $b$. Но тогда любая её подпоследовательность стремится к $b$ и, значит, не имеет предела, принадлежащего $K$.

\Zam2. Утверждение о том, что в $\R^m$ всякое замкнутое ограниченное множество компактно и его частный случай лемму называют \q Теоремой Гейне-Бореля.

\Zam2. Как уже отмечалось, $2 \Ra 1$ верна в любом метрическом пространстве, а $1 \Ra 2$~--- не в любом. На самом деле, утверждения 2 и 3 равносильны в любом метрическом пространстве. Утверждение $3\Ra 2$ оставим без доказательства. Свойство 3 называется секвенциальной компактоностью.

\Zam3. В ходе доказательства теоремы установлено, что всякое бесконечное подмножество компакта $K \subset \R^m$ имеет предельную точку, принадлежащую $K$.

\S1. \q Принцип выбора Больцано-Вейерштрасса. Из всякой ограниченной подпоследовательности в $R^m$ можно извлечь сходящуюся подпоследовательность.

\D В силу ограниченности все члены последвоательности принадлжат некоторому замкнутому кубу $I$. Поскольку $I$ компактен, из этой подпоследовательности можно извлечь подпоследовательность, имеющуь предел, принадлежащий $I$.

\Zam4. Если вещественная последовательность неограничена (сверху, снизу) то из неё можно извлечь подпоследовательность, стремящуюся к бесконечности(плюс-минус бесконечности). Для бесконечности без знака утверждение верно и в нормированном пространстве.

\D Для определённость докажем для $+\infty$. Так как последовательность $\{x_n\}$ не ограничена сверху, найдётся номер $n_1$, для которого $x_{n_1} > 1$. Далее найдётся номер $n_2 > n_1$, для которого $X_{n_2} > 2$ (иначе $\{x_n\}$ была бы ограниченв сверху числом $\max\{x_1, \cdots,x_{n_1}, 2\}$). Жтот процесс продолжаем неограниченно: на $k$-том шаге найдётся номер $n_k > n_{k - 1}$, ля которого $x_{n_k} > k$. Тогда $x_{n_k} \to +\infty$.
\skip
\subsection{Сходимость и сходимость в себе. Полнота $\R^m$}
(84)

\Op Пусть $\seq xn1\infty$~--- последовательность в метрическом пространстве $X$. Говорят, что последователбность $x_n$ сходится в себе, если\\
$\forall \eps > 0\ \exists N\ \forall n, l > N\ \rho(x_n, x_l) < \eps$.

\L 1. Сходящаяся в себе последователность ограничена\\
2. Если у сходящейся в себе последовательности есть сходящаяся подпоследовательность, то сама последовательность сходится.

\D 1. Пользуясь сходимостью $\{x_n\}$ в себе, подберём такой номер, что для всех $n, l > N$ будет $\rho(x_n, x_l) < 1$. В частночти, тогда $\rho(x_n, x_{N + 1}) < 1$ для всех $n > N$. Пусть $b \in X$. Следовательно, для всех $n > N$ по неравенству треугольника $\rho(x_n, b) < 1 + \rho(x_n + 1, b)$. Положим\\
$\R = \max\{\rho(x_1, b), \cdots, \rho(x_N, b), 1 + \rho(x_{N + 1}, b)\}$\\
Тогда $\rho(x_n, b) \le R$ для всех номеров $n$.\\
2. Пусть $\{x_n\}$ сходится в сеюе $x_{n_k} \to a$. Возьмём $\eps > 0$. По определению предела, найдётся такой нмер $K$, что $\rho(x_{n_k}, q) < \frac{\eps}{2}$ для всех $k > K$, а по определнию сходимости в себе, найдётся такой номер $N$, что $\rho(x_n, x_l) < \frac{\eps}{2}$ для всех $n, l > N$. Покажем, что найженное $N$~--- требуемое для $\eps$ из определения предела. Пусть $n > N$ Положим $M = \max\{N +1, K +1\}$; тогда $b_M \ge n_{N + 1} \ge N$ и аналогично $n_M \ge K$. Следовательно,\\
$\rho(x_n, a) \le \rho(x_n, x_{n_M}) + \rho(x_{n_M}, a) < \frac{\eps}{2} + \frac{\eps}{2} = \eps$\\
В силу произвольности $\eps$ это и означает, что $x_n \to a$.

\T 1. Во всяком метрическом пространстве любая сходящаяся помледовательность сходится в себе.\\
2. В $\R^m$ любая сходящаяся в себе последовательность сходится.

\D 1. Обозначим $\lim x_n = a$. Возьмём $\eps > 0$. По определению предела, найдётся такой номер $N$, что $\rho(x_n, a) < \frac{\eps}{2}$ для всех $n > N$. Тогда для всех $n, m > N$\\
$\rho(x_n, x_m) \le \rho(x_n, a) + \rho(a, x_m) < \frac{\eps}{2} + \frac{\eps}{2} =\eps$.\\
В силу произвольности $\eps$ это и значит, что $\{x_n\}$ сходится в себе.\\
2. Пусть $\pb xn$~--- сходящаяся в себе последовательность в $\R^m$. По пункту 1 леммы, она ограничена. По принципу выбора Болцано-Вейерштрасса из неё можно извлечь сходящуюся подпоследовательность, а тогда по пункту 2 леммы она сама сходится.

\Op Если в метрическом протсранстве $X$ любая сходящаяся в себе последователбность сходится, то это пространство называется полным.

\Zam1. Второе утверждение теоремы можно сформулировать так: пространство $\R^m$ полно.\\
Пример неполног пространства $\Q$, как подпротсранство $\R$. Если взять десятичные приближения к $\sqrt{2}$, то она будет сходиться в себе, но небудет иметь предела в $\Q$.

\Zam2. Утвержение о том, что в $\R^m$ сходимость и сходимость в себе равносильны, называют критерием Больцано - Коши сходимости последовательности\\
В пространстве $\R^m$ последовательность $\{\pb xn\}$ сходится тогда и толко тогда, когда\\
$\forall \eps > 0\ \exists N \in \N:\ \forall n, l > N|\pb xn - \pb xl| < \eps$\\
Критерий позволяет доказывать существавание предела, не используя само значения предела.
\skip
\subsection{Теорема о стягивающихся отрезках. Существование точной верхней границы}
(86)

\T \q О стягивающихся отрезках. Пусть $\se {[a_n, b_n]}{n = 1}\infty$~--- последовательность стягивающихся отрезков. Тогда пересечение всех этих отрезков состоит из одной точки, то есть
\F{$\exists c \in \R: \dsl \bigcap{n = 1}\infty [a_n, b_n] = \{c\}$}
при этом $a_n \to c,\ b_n \to c$

\D То, что пересечение не пусто, следует из аксиомы о вложенных отрезках. Пусть $c, d \in \dsl \bigcap{n = 1}\infty [a_n, b_n]$. Докажем, что $c = d$. Поскольку $a_n \le c \le b_n$ и $a_n \le d \le b_n$, имеем \F{$a_n - b_n \le c - d \le b_n - a_n$}. По теореме о предельном переходе в неравенстве, $0 \le c - d \le 0$, то есть $c = d$. Так как \F{$0 \le c - a_n \le b_n - a_n \quad 0 \le b_n - c \le b_n - a_n$} по теореме о сжатой последовательности $a_n \to c$ и $b_n \to c$.

\Op Пусть $E \subset \R,\ E \neq \varnothing,\ E$ ограничено сверху. Наименьшая из верхних границ множества $E$ называется точной верхней границей, или верхней гранью, или супремумом множества $E$ и обозначается $\sup E$.

\T \qСуществование верхней грани. Всякое непустое ограниченное сверху множетво имеет верхнюю грань.

\D По условию, существует точка $x_0 \in E$ и $M$~--- верхняя граница $E$, $x_0 \le M$. Обозначим $[a_1, b_1] = [x_0, M]$. Отрезок $[a_1, b_1]$ Удовлетворяет двум условиям:\\
1. $[a_1, b_1] \cap E \neq \varnothing$\\
2. $(b_!, +\infty) \cap E = \varnothing$\\
Рассмотрим середину отрезка $[a_1, b_1]$~--- точку $\frac{a_1 + b_1}2$. Положим $[a_2, b_2] = [a_1, \frac{a_1 + b_1}2]$, если $(\frac{a_1 + b_1}2, b_1] \cap E = \varnothing$, и $[a_2, b_2] = [\frac{a_1 + b_1}2, b_1]$, если $(\frac{a_1 + b_1}2, b_1]\cap E \neq \varnothing$. В обоих случаях условия сохраняются.\\
Будем продолжать этот процесс неограниченно, при этом оставляя выполняться эти условия. При этом, $b_n - a_n = \frac{b_1 - a_1}{2^{n - 1}} \to 0$. По теореме о стягивающихся отрезках существует единственная точка $c$, принадлежащая одновременно всем отрезкам, причём $a_n \to c,\ b_n\to c$\\
Проверим, что $c = \sup E$. Если $x \in E, n \in N$, то $x\le b$ по свойству 2. По теореме о предельном переход в неравенстве $x \le c$. То есть $c$~-- верхняя граница $E$. Возьмём $\eps > 0$ и дркажем, что $c - \eps$ не является верхней гранцей $E$. Так как $a_n \to c$, найдётся номмер $N$, для которого $a_N > c - \eps$ (по определению предела все элементы с некоторого номера удовлетворяют этому неравенству). По свойству 1, найдётся точка $x \in [a_N, b_N] \cap E$, а тогда $x > x - \eps$.

\Zam2. Если множество $E$ не ограничено сверху считают, что $\sup E = +\infty$, при этом определение супремума в $\ol{\R}$ существует у любого непустого множества. Ограниченночть $Y$ сверху равносильна неравенству $\sup E < +\infty$

\Zam3. Если $D \subset E\subset \R,\ D \neq \varnothing$, то $\sup D \le \sup E$

\D Если $sup E = +\infty$, то неравенство тривиально. Пусть $\sup E < +\infty$. Если $x \in D$, то $x \in E$ и, слеедовательно, $x \le \sup E$, то есть $\sup E$ какая-то верхняя граница $D$. Но $\sup D$~--- наименьшая верхняя граница $D$, поэтому $\sup D \le \sup E$.

\Zam4. Если $E, F \subset \R,\ E, F \neq \varnothing, t > 0$, то\\
$\sup(E + F) = \sup E +\sup F$\\
$\sup(tE) = t\sup E$\\
$\sup(-E) = -\inf E$
\skip
\subsection{Предел монотонной последовательности}
(91)

\T \q Предел монотонной последовательности.\\
1. Всякая возрастающая ограниченная сверху последовательность сходится.\\
2. Всякая убывающая ограниченная снизу последовательность сходится.\\
3. Всякая монотонная ограниченная последовательность сходится.

\D Докажем первое утверждение. Пусть последовательность $\{x_n\}$ ограничена сверху. По теореме существует $\sup x_n = c \in \R$. Докажем, что $c = \lim x_n$. Возьмём $\eps > 0$. По определению супремума (так как $c - \eps$ не является верхней границей последовательности), найдётся такой номер $N$, что $x_N > c - \eps$. В силу возрастания поледовательности, при любом $n > N$ будет $x_n \ge x_N$. Снова по определению супремума $x_n \le c$ при всех $n$. Итак для любого $n > N$\F{$c - \eps < x_N \le x_n \le c < c + \eps$} В силу произвольности $\eps$ это значит $c = \lim x_n$.\\
Второе утверждение доказываетя аналогично. Третье следует из первых двух.

\Zam1. Если последовательность возрастает и не ограничена сверху, то она стремится к плюс бесконечности.

\D Возьмём $E > 0$. Так как $\{x_i\}$ не ограничена сверху, найдется такой номер $N$, что $x_N > E$. Тогда для любого номера $n > N$, в силу возрастания последовательности, тем более $x_n > E$.

\Zam2. Доказано, что любая монотонная последовательность имеет предел в $\ol{\R}$, при этом для всех возрастающих последовательностей \F{$\lim x_n = \sup x_n$} а для убывающих \F{$\lim x_n = \inf x_n$}
\skip
\subsection{Неравенство Я. Бернулли, $\lim z^n$, число $e$, формула Герона}
(92)

\L \q Неравенство Я. \qБернулли. Если $n \in \Z_+,\ x > -1$, то \F{$(1 + x)^n \ge 1 + nx$}.

\D При $n = 0, n = 1$ (база индукции) неравенство очевидно, обращается в равенство. Сделаем переход: Пусть неравенство верно для номера $n$. Тогда \F{$(1 + x)^{n + 1} = (1 + x)^n(1 + x) \ge (1 + nx)(1 + x) = 1 + (n + 1)x + nx^2 \ge 1 + (n + 1)x$}

\Pr1. Пусть $x \in \C,\ |z| < 1$. Докажем, что $\dsl \lim{n\to\infty}{} z^n = 0$.

В самом деле, последователность $\{|z|^n\}$ убывает и ограничена снизу нулём. Следовательно, существует конечный $\dsl\lim{n\to\infty}{}|z|^n=a$. Перейдя в равенство $|z|^{n + 1} = |z||z|^n$ к пределу, получим $a = |z|a \lra (1 - |z|)a = 0$, отсюда $a = 0$, поскольку $|z| < 1$, таким образом $z^n \to 0$, что равносильно $x^n \to 0$.

\Pr2. \q Число $e$. Докажем, что последовательность $x_n = (1 + \frac 1n)^n$ сходится.

Положим $y_n = (1 + \frac 1n)^{n + 1}$. Ясно, что последовательность $\{y_n\}$ ограничена снизу единицей, кроме того она убывает \F{$\frac{y_{n - 1}}{y_n} = \frac{(1 + \frac{1}{n - 1})^n}{(1 + \frac 1n)^{n + 1}} = \frac{(\frac n{n - 1})^ n}{(\frac{n + 1}n)^{n + 1}} = (\frac{n^2}{n^2 - 1})^{n + 1}\frac{n - 1}n = (1 + \frac 1{n^2-1})^{n + 1}\frac{n - 1}n \ge (1 + \frac{n + 1}{n^2 - 1}) \frac{n - 1}n \ge 1$} Следовательно, $\{y_n\}$ сходится, а тогда по теореме о пределе частного и $x_n = \frac{y_n}{1 + 1/n}$. сходится к этому же пределу.

\Pr3. \q Формула Герона. Пусть $a > 0, x_0 > 0$, \F{$x_{n + 1} = \frac 12 (x_n + \frac a{x_n})$} Ясно, что $x_n > 0$ при всех $n$ и значит, $\{x_n\}$ ограничена снизу. Воспользоваашись очевидным неравенством \F{$t + \frac 1t \ge 2,\quad t > 0$} получим, что при всех $n \in \Z_+$ \F{$x_{n + 1} = \frac{\sqrt{a}}2(\frac{x_n}{\sqrt{a}} + \frac a{x_n}) \ge \frac{\sqrt{a}}{2}2= \sqrt{a}$} Поэтом у для всех $n \in \N$\F{$x_{n + 1} = \frac{x_n}2 (1 + \frac a{x^2_n}) \le x_n$} То есть последовательность $\{x_n\}$ убывает. Следовательно, она сходится; обозначим $\lim x_n =\beta$. Перейдя к пределу в равенстве получим \F{$\beta = \frac12(\beta + \frac a\beta)$} Откуда $\beta = \sqrt{a}$, так как $\beta \ge 0$\\
Равенство $\lim x_n = \sqrt{a}$ называется формулой герона и используется для приближенного вычисления корня.

\Zam3. Пусть $x_n > 0$ при всех $n$, $\lim \frac{x_{n + 1}}{x_n} < 1$ тогда $x_n \to 0$.\\
Отсюда следует, что
\F{$\dsl\lim{n\to\infty}{} \frac{n^k}{a^n} = 0 \quad a > 1,\ k \in \N$}
\F{$\dsl\lim{n\to\infty}{} \frac{a^n}{n!} = 0 \quad a \in \R$}
\F{$\dsl\lim{n\to\infty}{} \frac{n!}{n^n} = 0$}
\skip
\subsection{Верхний и нижний пределы последовательностей}
(95)

\Op Точка $a \in \ol{\R}$ называется частичным пределом последовательности $\{x_n\}$, если существует подпоследовательность $\{x_{n_k}\}$, стремящаяся к $a$.

\Op Пусть последовательность $\{x_n\}$ ограничена сверху. Величина
\F{$\ol{\dsl\lim{n\to\infty}{}}x_n = \dsl\lim{n\to\infty}{} \dsl\sup{k \ge n}{} x_k$} называется верхним пределом последовательности $\{x_n\}$

Пусть последовательность $\{x_n\}$ ограничена снизу. Величина
\F{$\ul{\dsl\lim{n\to\infty}{}}x_n = \dsl\lim{n\to\infty}{} \dsl\inf{k \ge n}{} x_k$} называется нижним пределом последовательности $\{x_n\}$

\Zam1. Последовательности $z_n - \dsl\sup{k \ge n}{} x_k$ и $y_n = \dsl\inf{k \ge n}{} x_k$ иногда называют верхней и нижней огибающими последовательности $\{x_n\}$.\\
Если $\{x_n\}$не ограничена сверху, то $z_n = +\infty$ при всех $n$ и поэтому полагают $\ol{\lim} x_n = +\infty$. Аналогично, если $\{x_n\}$ не ограничена снизу полагают $\ul{\lim} x_n = -\infty$.

\Zam2. Верхний и нижний пределы вещественной последовательности $\{x_n\}$ существуют в $\ol{\R}$, причем $\ul{\lim} x_n \le \ol{\lim}x_n$

\D Пусть $\{x_n\}$ ограничена и сверху и снизу. Так как при переходе к подмножеству супремум не увеличивается, а инфинум не уменьшается $\{y_n\}$ возрастает, а $\{z_n\}$ убывает. При всех $n$ $y_q \le y_n \le z_n \le z_1$. По теореме о преле монотонной последовательности $\{y_n\},\ \{z_n\}$ сходятся, то есть существуют конечные пределы $\lim b$. Если хотя бы одна последовательность неограничена, то очевидно.

\T \q О верхнем и нижнем пределе последовательности. Пусть $\{x_n\}$~--- вещественная последовательность, тогда справедливы следующий утверждения\\
1. Верхний предел~--- наибольший, а нижний~--- наименьший из частичных пределов $\{x_n\}$\\
2. Предел $\{x_n\}$ в $\ol{\R}$ существует тогда и только тогда, когда $\ul{\lim} x_n = \ol{\lim} x_n$, при этом $\lim x_n$ равен этому значению.

\D \RNumb{1}. Пусть $\{x_n\}$ ограничена и сверху и снизу\\
1. Обозначим $b = \ol{\lim}x_n$. Докажем, что $b$~--- частичный предел $\{x_n\}$, для чего построим полпоследовательность последовательности $\{x_n\}$, стремящуюся к $b$. При всех $n$ будет $z_n \ge b$. Поскольку \F{$z_1 = \sup\{x_1, x_2,\cdots\} > b -1$} найдётся номер $n_1$, для которого $x_{n_1} > b - 1$. Поскольку \F{$z_{n_1 + 1} = \sup\{x_{n_1 + 1}, x_{n_1 + 2}, \cdots\} > b - \frac 12$} найдется номер $n_2 > n_1$, для которого $x_{n_2} > b - \frac 12$. Этот процесс может продолжаться неограничено. Таким образом, построена подпоследовательность $\{x_n\}$, члены которой удовлетворяют неравенству \F{$b - \frac 1k < x_{n_k} \le z_{n_k}$} Подпоследоватеьность $\{z_{n_k}\}$ последовательности $\{x_n\}$, стремящейся к $b$ тоже стремится к $b$. По теореме о сжатой последоавтельности и $x_{n_k} \to b$.\\
Если $\{x_{m_l}\}$~--- подпоследовательность $\{x_n\}$, $\{x_{m_l}\} \to \beta$, то, сделав предельный переход в неравенстве $x_{m_l} \le z_{m_l}$, получим, что $\beta \le b$, то есть $b$~--- наибольший частичный предел $\{x_n\}$.\\
Аналогично для $\ul{\lim} x_n$~--- наименьшего частичного предела.
2. По определению $y_n$ и $z_n$, при всех $n$ будет \F{$y_n \le x_n \le z_n$} Если $\ul{\lim} x_n = \ol{\lim} x_n$, то по теореме о сжатой последовательности существует $\lim x_n$ и $\lim x_n = \ul{\lim} x_n = \ol{\lim} x_n$

\RNumb{2}. Пусть $\{x_n\}$ ограничена сверху, но не снизу. Тогда, по определению, $\ul{\lim} x_n = -\infty$. По замечанию 2 к принципу выбора, из $\{x_n\}$ можно выбрать подпоследовательность, стремящуюся к $-\infty$, то есть $-\infty$~--- частный предел $\{x_n\}$. Разумеется $-\infty$ меньше любого другого частичного предела, если они есть. То что $\ol{\lim} x_n$~--- наибольший частичный предел, доказано в пункте \RNumb{1}. Если $\ul{\lim} x_n = \ol{\lim} x_n$, то есть $z_n \to -\infty$, то и $x_n \to -\infty$, так как $x_n \le z_n$. Обратно, если $\lim x_n = -\infty$, то и $\ol{\lim} x_n = -\infty$, как частичный период.

\RNumb{3}. Если $x_n$ не ограничена ни сверху, ни снизу, то превое утверждение очевидно, а второе не реализуется
\skip
\subsection{Равносильность определений предела отображения по Коши и по Гейне}
(99)

\Op на $\eps$-языке (по Коши)\\
$\forall \eps > 0\ \exists \delta > 0\ \forall x \in D \bsl \{a\}: \rho_X(x, a) < \delta\ \rho_Y(f(x), A) < \eps$.

\Op на языке последовательностей (по Гейне)\\
$\forall \{x_n\}: x_n\in D\bsl\{a\}, x_n \to a f(x_n) \to A$.

(102)
\T Определения предела отображения по Коши и по Гейне эквивалентны.

\D Для определённости докажем теорему при $a \in X, A \in Y$.

1. Пусть $A$~--- предел отображения $f$ в точке $a$ по Коши; докажем, что тогда $A$~--- предел и по Гейне. Возьмём последовательность $\{x_n\}$ со свойствами из определения Гейне: $x_n \in D,\ x_n \neq a,\ x_n\to a$. Требуется доказать, что $f(x_n) \to A$. Возьмём $\eps > 0$. По определению Коши подберём такое $\delta> 0$, что для всех $x\in D$, для которых $x\neq a$ и $p_X(x, a) <\delta$  будет $p_Y(f(X), A) < \eps$. По определению предела последовательности $\{x_n\}$ для числа $\delta$ найдётся такой  номер $N$, что при всех $n > N$ верно неравенство $p_X(x_n, a) < \delta$. Но тогда $p_Y(d(x_n), A) < \eps$ Для всех $n > N$ В силу произвольности $\eps$ это значит, что $f(x_n) \to A$.

2. Пусть $A$~--- предел отображения $f$ в точке $a$ по Гейне; докажем, что тогда $A$~--- предел $f$ по Коши. Предположим противное: пусть $A$ не есть предел по Коши. Записывая отрицание определения Коши, получаем
\F{$\exists \eps^* > 0\ \forall \delta > 0 \exists x \in D \bsl\{a\}: p_X(x, a) < \delta,\ p_Y(f(X), A) \ge \eps^*$} Следовательно, для каждого $n \in \N$ по число $\delta = \frac 1n$ найдётся такая точка $x_n$, что \F{$x_n \in D \bsl\{a\}\quad p_X(x_n, a) < \frac 1n,\quad p_Y(f(x_n), A) \ge \eps^*$} По теореме о сжатой последовательности, построенная последоввательность $x_n$ стремится к $a$, так как \F{$0 < p_X(x_n, a) < \frac 1n$.} Тогда, по определению Геёне, $f(x_n) \to A$. По определению предела последовательности $\{f(x_n)\}$ для числа $\eps^*$ найдётся такой номер $N$, что для всех номеров $n > N$ будет $p_Y(f(x_n), A) < \eps^*$, что противоречит условию.
\skip
\subsection{Простейшие свойства отображений, имеющих предел (единственность предела, локальная ограниченность, арифметические действия)}
(103-106)

\T \q Единственность предела отображения. Отображение в данной точке не может иметь более одного предела: если $X, Y$~--- метрические пространства, $f: D \subset X \to Y, a$~--- предельная точка $D,\ A, B \in Y,\ f(x)\t A,\ f(x)\t B$, то $A = B$.

\D Возьмём последовательность $\{x_n\}: x_n \in D,\ x_n \neq a,\ x_n \to a$. По определению Гейне $f(x_n)\to A$ и $f(x_n) \to B$. По единственности предела последовательности, $A = B$.

\Zam1. Если $Y = \R$, то, как и для последовательностей, а теореме можно считать, что $A, B \in \ol{\R}$.

\T. \q Локальная ограниченность отображения имеющего предел. Пусть $X, Y$~--- метрические пространства, $f: D\subset X\to Y$, $a$~--- предельная точка $D,\ a\in Y,\ f(x)\t A$. Тогда существует такая окрестность $V_a$ точки $a$, что $f$ ограниченв в $V_a\cap D$ (то есть $f(V_a \cap D)$ содержится в некотором шаре в $Y$).

\D Возьмём окрестность $V_A = B(A, 1)$. По определению предела на языке окрестностей, найдётся такая окрестность $V_a$ точки $a$, что $f(V_a \cap D) \subset B(A, 1)$. Если $a \notin D$, то на этом доказательство заканчивается. Иначе
\F{$f(V_a \cap D) \subset B(A, R)$, где $R = \max\{1, \rho_Y(f(a), A)\}$}

\Zam2. Отображение, имеющее предел в точке не обязано быть ограниченным. Например функция $f(x) = x$. Поэтому в названии теоремы присутствуетс лово "локальная".

\Zam3. Если $X$~--- метрическое пространство, $Y$~--- нормированное пространство с нулём $\theta,\ D \subset X,\ a$~--- предельная точка $D,\ g: D\to Y,\ \dsl\lim{x\to a}{}g(x) = B,\ B \neq \theta$, то существует такая окрестность $V_a$, что $g(x) \neq \theta$ для всех $x \in \dot{V_a}\cap D$.

\D Пусть не так: тогда для любого $n \in \N$ существует точка $x_n \in V_a(\frac1n)\cap D$, для которогй $g(x_n) = \theta$. Построенная последовательность $\{x_n\}$ стремиться к $a$. По определению предела $g(x_n) \to B$, откуда $B = \theta$, что протмворечит условию.

\T \q Арифмметические действия над отображениями, имеющими предел. Пусть $X$~--- метрическое пространство, $Y$~--- нормированное пространство, $D \subset X,\ f, g: D \to Y,\ \lambda: D\to \R(\C),\ a$~--- предельная точка $D,\ A, B \in Y,\ \lambda_0 \in \R(\C),\ f(x) \t A,\ g(x)\t B,\ \lambda(x)\t \lambda_0$. Тогда \\
1. $f(x) + g(x) \t A + B$;\\
2. $\lambda(x)f(x) \t \lambda_0 A$;\\
3. $f(x) - g(X) \to A - B$;\\
4. $\|f(x)\| \t \|A\|$

\T \q Арифметические действия над функциями, имеющими предел. Пусть $X$~--- метрическое пространство, $f, g: D \subset X \to \R(\C),\ a$~--- предельная точка $D,\ a, b \in\R,\ f(x) \t A,\ g(x)\t B$. Тогда\\
1. $f(x) + g(x) \t A + B$;\\
2. $f(x)g(x) \t A B$;\\
3. $f(x) - g(x) \t A - B$;\\
4. $|f(x)| \t |A|$;\\
5. Если $B \neq 0$, то $\frac{f(x)}{g(x)} \t \frac AB$.

\D С помощью определения на языке последовательностей эти теоремы сводятся к аналогичным про последовательности. Докажем, например, первое утверждение. Возьмём последовательность $\{x_n\}$ со свойствами $x_n \in D,\ x_n \neq a, x_n \to a$. Тогда по определению Гейне $f(x_n) \to A, g(x_n) \to B$. По теореме о пределе суммы для последовательностей $f(x_n) + g(x_n) \to A + B$. В силу произвольности последовательности $\{x_n\}$ это и означает, что $f(X) + g(x) \t A + B$. При доказательстве утверждения о пределе частного следует учесть, что, по замечанию 3, существует такая окрестность $V_a$, что частное $\frac fg$ определено по крайней мере на множестве $V_a \cap D$.

\Zam 4. Теорема про функции верна и для бесконечных пределов, за исключением случаев неопределённости вида $\infty - \infty, 0\cdot\infty, \frac 00, \frac\infty\infty$.

\Zam5. Определение бесконечно малой и бесконечно большой переносятся на функции (и отображения со значениями в нормированно пространстве). Так, функция, стремящаяся к нулю в точке $a$ называется бесконечно малой в точке $a$. Утверждение о том, что  произведение бесконечно малой функции на ограниченнуб есть бесконечно малая, и о связи иежду бесконечно большими и бесконечно малыми сохраняют свою силу.
\skip
\subsection{Предельный переход в неравенстве для функций. Теорема о сжатой функции}
(106)

\T \q Предельный переход в неравенстве для функций. Пусть $X$~--- метрическое пространство, $f, g: D\subset X\to \R,\ a$~--- предельная точка $D,\ f(x)\le g(x)$ для всех $x\in D\bsl \{a\},\ A, B\in \ol{\R}, f(x)\t A,\ g(x)\t B$. Тогда $A \le B$

\D Возьмём последовательность $\{x_n\}$ со свойствами $x_n \in D,\ x_n \neq a,\ x_n\to a$. Тогда, по определению Гейне, $f(x_n) \to A,\ g(x_n)\to B$. По теореме о предельном переходе в неравенстве для последовательностей $A \le B$.

\T \q О сжатой функции. Пусть $X$~--- метрическое пространство, $f, g, h: D\subset X \to \R, a$~--- предельная точка $D,\ f(x)\le g(x)\le h(x)$ для всех $x \in D\bsl\{a\},\ A \in \R,\ f(x)\t A,\ h(x)\t A$. Тогда и $g(x)\t A$.

\D Возьмём последовательность $\{x_n\}$ со свойствами $x_n \in D,\ x_n \neq a,\ x_n\to a$. Тогда, по определению Гейне $f(x_n) \to A,\ h(x_n) \to A$. Кроме того, по условию для всех $n \in N$ \F{$f(x_n) \le g(x_n) \le h(x_n)$.} По теореме о сжатой последовательности $g(x_n) \to A$. В силу произвольности последовательности $\{x_n\}$ это и значит, что $g(x)\t A$.

\Zam1. Аналогично доказывается, что если $f(x) \le g(z)$ для всех $x \in D\{a\}$ и $f(x) \t +\infty (g(x)\t -\infty)$,то и $g(x) \t +infty (f(x) \t -\infty)$.

\Zam2. В теоремах и замечании достаточно выполнения неравенств на множестве $\dot{V_a}\cap D$, где $V_a$~--- какая-нибудь окрестность точки $a$.

Пусть $f: D \subset X\to Y,\ D_1 \subset S,\ a$~--- предельная точка $D_1$ (а следовательно и $D$). Тогда если предел $f$ в точке $a$ существует и равен $A$, то предел сужения $f$ на $D_1$ в точке $a$ также существует и равен $A$. В самом деле, если соотношение $f(X) \in V_A$ выполняется для всех $x \in \dot{V_a}\cap D$, то оно тем более выполняется для всех $x$ из $\dot{V_a}\cap D_1$. Однако возможна ситуация, когда предел сужения существует, а предел отображения нет.
\skip
\subsection{Предел монотонной функции}
(108)

\T \q О пределе монотонной функции. Пусть $f: D\subset \R\to\R,\ a \in (-\infty, +\infty),\ D_1 = D\cap(-\infty, a), a$~--- предельная точка $D_1$.\\
1. Если $f$ возрастает и ограничена сверху на $D_1$, то существует конечный предел $f(a-)$.\\
2. Если $f$ убывает и ограничена снизу на $D_1$, то существует конечный предел $f(a-)$.

\D Докажем первое утверждение, второе доказыается аналогично. Положим $A = \dsl\sup{x\in D_1}{} f(x)$; тогда $A \in \R$ в силу ограниченности функции свурху. Докажем, что $f(a-) = A$. Возьмём $\eps > 0$. По определению верхней грани существует такая точка $x_0 \in D_1$, что $f(x_0) > A - \eps$. Но тогда для всех таких $x\in D_1$, что $x > x_0$, в силу возрастания $f$ \F{$A - \eps < f(x_0) \le f(x) \le A < A + \eps$.} Теперь положим $\delta = a - x_0$ при $a \in \R$ или $\Delta = \max\{x_0, 1\}$ при $a = +\infty$; Тогда неравенство из определения предела выполнено для всех таких $x \in D$, что $0 < a - x < \delta$ (соответсвенно, $x > \Delta$).

\Zam2. Аналогично утверждениям теоремы доказываются\\
3. Если $f$ возрастает и не ограничена сверху на $D_1$, то предел $f(a-)$ существует и равен $+\infty$.\\
4. Если $f$ убывает и не ограничена снизу на $D_1$, то предел $f(a-)$ существует и равен $-\infty$.

\Zam2. Аналогично формулируется и доказывается теорема для правостороннего предела.\\
Пусть $f: D\subset \R\to\R,\ a \in (-\infty, +\infty),\ D_2 = D\cap(a, +\infty), a$~--- предельная точка $D_2$.
1. Если $f$ возрастает и ограничена снизу на $D_2$, то существует конечный предел $f(a+)$.\\
2. Если $f$ убывает и ограничена сверху на $D_2$, то существует конечный предел $f(a+)$.\\
3. Если $f$ возрастает и не ограничена снизу на $D_2$, то предел $f(a+)$ существует и равен $-\infty$.\\
4. Если $f$ убывает и не ограничена сверху на $D_2$, то предел $f(a+)$ существует и равен $+\infty$.
\skip
\subsection{Критерий Больцано-Коши для отображений}
(110)

\T \q Критерий Больцано-Коши для отображений. Пусть $X, Y$~--- метрические пространства, $Y$ полно, $f: D\subset X \to Y,\ a$~--- предельная точка $D$. Тогда существование в точке $a$ предела $f$, принадлежащего $Y$, равносильно следующему утверждению.
\F{$\forall \eps > 0\s\exists V_a\s\forall \ol{x}, \ool{x} \in \dot{V_a}\cap D\s\rho_Y(f(\ol{x}), f(\ool{x})) < \eps$.}

\D 1. Пусть $\dsl\lim{x\to a}{}f(x) = A \in Y$. Возьмём $\eps > 0$. По определению предела, найдётся такая окрестность $V_a$ точки $a$, что $p_Y(f(X), A) < \frac \eps2$ Для всех $x \in \dot{V_a} \cap D$. Тогда если $\ol{x}, \ool{x} \in \dot{V_a}\cap D$, то
\F{$\rho_Y(f(\ol{x}), f(\ool{x})) \le \rho_Y(f(\ol{x}), A) + \rho_Y(A, f(\ool{x})) < \eps$} В силу произвольности $\eps$ формула выполнена.

2. Пусть выполнена формула. Докажем существование предела $f$ в точке $a$ на языке последовательностей. \Geine, и докажем, что существует $\lim f(x_n) \in Y$. Возьмём $\eps > 0$ и подберём окрестность $V_a$ из формулы. По орпеделению предела $\{x_n\}$ найдётся такой номер $N$, что $x_n \in V_a$ для всех $n > N$; тогда $x_n \in \dot{V_a}\cap D$ для тех же $n$. По выбору $V_a$, для всех $n, l > N$ будет $\rho_y(f(x_n), f(x_1)) < \eps$. таким образом последовательность сходится в себе, а, значит, в силу полноты $Y$, сходится к некоторому пределу, принадлежащему $Y$. Тогда, в силу замечания к определению предела существует $\dsl\lim{x\to a}{} f(X) \in Y$.

\Zam1. Полнота $Y$ использовалась только во второй части доказательства.
\skip
\subsection{Двойной и повторный пределы, примеры}
(111)

\Op  Пусть $D_1, D_2 \subset\R,\s a$~--- предельная точка $D_1,\s b$~--- предельная точка $D_2,\s D \supset (D_1 \bsl\{a\}) \times (D_2 \bsl\{b\}),\s f: D \to \R$.

1. Если для каждого $x \in D_1 \bsl\{a\}$ существует конечный предел
\FF{\varphi(X) = \dsl\lim{y\to b}{} f(x, y),} то предел функции $\varphi$ в точке $a$ называется повторным пределом функции $f$ в точке $(a, b)$: \FF{\dsl\lim{x\to a}{} \varphi(x) = \dsl\lim{x\to a}{} \dsl\lim{y\to b}{} f(x, y)}

2. Аналогично для $y \in D_2 \bsl \{b\}$ \f($\dsl\lim{y\to b}{} \psi(x) = \dsl\lim{y\to b}{} \dsl\lim{x\to a}{} f(x, y)$)/

3. Точку $A$ называют двойным пределом функции $f$ в точке $(a, b)$ и пишут \FF{\dsl\lim{x\to a, y\to b}{} f(x, y) = A, \qquad f(x, y)\xra[x\to a, y\to b]{} A,} если для любой окрестности $V_a$ точки $A$ существуют такие окрестности $V_a$ и $V_b$ точек $a, b$, что $f(x, y) \in V_A$ для всех $x \in \dot{V_a}\cap D_1,\s y \in \dot{V_b}\cap D_2$.

\T \q О двойном и повторном пределе. Пусть $D_1, D_2 \subset \R,\s a$~--- предельная точка $D_q,\s b$~--- предельная точка $D_2,\s D \supset (D_1 \bsl\{a\}) \times (D_2 \bsl\{b\}),\s f: D \to \R$ и выполнены условия:\\
1. существует конечный или бесконечный двойной предел $\dsl\lim{x\to a,y\to b}{} f(x,y) = A$;\\
2. для каждого $x \in D_1 \bsl\{a\}$ существует конечный предел \FF{\varphi(x) = \dsl\lim{y\to b}{} f(x, y).} Тогда повторный предел $\dsl\lim{x\to a} \varphi(x)$ существует и равен $A$.

\D Для определённости пусть $A \in \R$. Возьмём $\eps > 0$. По определению, двойного предела, найдутся такие окрестности $V_a, V_b$, что для всех $x \in \dot{V_a}\cap D_1,\s y \in \dot{V_b}\cap D_2$ выполняется неравенство \FF{|f(x, y) - A| < \frac \eps2.} Устремляя в нём $y$ к $b$ и пользуюясь непрерывностью модуля, получаем \FF{|\varphi(x) - A| \le \frac \eps2 < \eps} для всех $x \in \dot{V_A}\cap D_1$, что и означает требуемое. В случае бесконечного предела следует изменить неравенство на соответсвующее.

\S1. При выполнении всех трёх условий оба повторных предела существуют и равны двойному.

\Pr1. Пусть $f(x, y) = \frac{x^2 - y^2}{x^2 + y^2}$, тогда в точке $(0, 0)$ повторные пределы различны и равны 1 и -1 соответсвенно, двойного предела не существует.

\Pr2. $f(x, y) = \frac{xy}{x^2 + y^2}$ в точке (0, 0) повторные пределы равны 0, но двойного предела не существует, так как по прямой $y = x$ предел равен $\frac 12$.

\Pr3. $f(x, y) = x\sin\frac 1y + y\frac 1x$ в (0, 0) повторных пределов нет, а двойной существует и равен 0, так как $|f(x, y)| \le |x| + |y|$
\skip
\subsection{Замена на эквивалентную при вычислении пределов. Асимптоты}
(163)

\def\ti{\tilde}

\T \q Замена на эквивалентную при вычислении пределов. Пусть $X$~--- метрическон пространство, $f, \ti{f}, g, \ti{g}: D\subset X\to \R(\C),\s x_0$~--- предельная точка $D$, \FF{f(x) \sim \ti{f}(x),\s g(x) \sim \ti{g}(x),\quad x \to x_0.} Тогда справедливы следующие утверждения.\\
1. $\dsl\lim{x\to x_0}{} f(x)g(x) = \dsl\lim{x\to x_0}{} \ti{f}(x)\ti{g}(x)$.\\
2. Если $x_0$~--- предельная точка области определения $\frac fg$, то $\dsl\lim{x\to x_0}{} \frac {f(x)}{g(x)} = \dsl\lim{x\to x_0}{} \frac{\ti{f}(x)}{\ti{g(x)}}$\\
В обоих утверждениях пределы одновременно существуют и равны или не существуют.

\Zam1. Если $g(x) \not\equiv 0$ в $\dot{V_a}\cap D$, то и $\ti{g}(x) \not\equiv 0$ в $\dot{\ti{V}}_{x_0}\cap D$ и обратно. Поэтому точка $x_0$ одновременно является или не является предельной для областей определения $\frac fg$ и $\frac{\ti{f}}{\ti{g}}$.

\D По определению эквивалентной функции, существуют окрестности $U_{x_0}, V_{x_0}$ и функции $\varphi, \psi$, стремящиеся к 1 при $x \to x_0$, такие, что \FF{f = \varphi \ti{f}$ на $\dot{U}_{x_0}\cap D, \quad g = \psi \ti{g} \textit{на} \dot{V}_{x_0}\cap D.} Тогда на множестве $\dot{W}_{x_0} \cap D$, где $W_{x_0} = U_{x_0} \cap V_{x_0}$, верны оба равенства. Значит, на $\dot{W}_{x_0}\cap D$ \F{$fg = (\varphi\psi)(\ti{f}\ti{g})$}. Следовательно если $\dsl\lim{x\to x_0}{} \ti{f}\ti{g}$ существует и равно $A$, то по теореме о пределе произведения $\dsl\lim{x\to x_0}{} f(x)g(x)$ существует и равен $A$. Верно и обратное. Аналогично доказыается для предела частного (может понадобиться сузить  окрестность, чтобы $\varphi, \psi$ не обращались в ней в нуль).

Пусть $f \sim g, f\sim h$. Если $f - h = o(f - g)$, то говорят, что асимптотически равенство $f\sim h$ точнее чем $f\sim g$.

(167)

\Op Пусть $x_0 \in \R$, функция $f$ задана по крайненй мере на $\ang a{x_0}$ или $\ang{x_0}b$ и действует в $\R$ прямая $x = x_0$ называется вертикальной асимптотой функции $f$, если $f(x_0+)$ или $f(x_0-)$ равны $+\infty$ или $-\infty$.

\Op Пусть $\langle a, +\infty) \subset D \subset \R,\s f: D\to \R,\s \alpha, \beta \in \R$. Прямая $y = \alpha x + \beta$ называется наклонной асимптотой функции $f$ при $x \to + \infty$, если \F{$f(x) = \alpha x + \beta + o(1),\quad x\to +\infty$}
Аналогично определяется наклонная асимптота при $x \to -\infty$ функции заданной по крайней мере на $(-\infty, b)$.
\skip
\subsection{Единственность асимптотического разложения}
(166)

\T \q Единственность асимптотического разложения. Пусть $X$~--- метрическое пространство, $D \subset X,\s x_0$~--- предельная точка $D,\s n \in \Z_+,\s f, g_k: D\to \R(\C)(k \in [0: n])$, при всех $k \in [0: n - 1]$ \F{$g_{k + 1}(x) = o(g_k(X))\quad x \to x_0$,} и для любой окрестности $V_{x_0}$ существует точка $t \in \dot{V}_{x_0} \cap D$, в которой $g_n(t) \neq 0$. Тогда, если асимптотическое разложение $f$ по системе $\{g_k\}$ существует, то оно едиственное: из равенств \F{$\dsl\sum{k = 0}n c_k g_k(x) + o(g_n(x)), \quad x\to x_0$,}
\F{$\dsl\sum{k = 0}n d_k g_k(x) + o(g_n(x)), \quad x\to x_0$,} следует, что $c_k = d_k$ при всех $k \in [0: n]$.

\D По индукции заключаем, что \F{$g_k(x) = o(g_l(x)), \quad x\to x_0,\s l < k$.} Обозначим \F{$E_k = \{x \in D: g_k(x) \neq 0\}, \quad k \in [0: n]$.} Если бы функция $g_k$ тождественно обращалась в ноль на множестве вида $\dot{U}_{x_0} \cap D$, то и функция $g_n = \varphi_k g_k$, где $\varphi_k$~--- функция из определения символа $o$ , обращалась бы в тождественный ноль а множестве $\dot{V}_{x_0} \cap D$, что противоречит условию. Следовательно, $x_0$~--- предельная точка каждого $E_k$.\\
Допустим противное: пусть $c_k = d_k$ не при всех $k \in [0: n]$. Положим \F{$m = \min\{k \in [0: n]: c_k \neq d_k\}$}.
Из разложений следует, что \F{$f(x) = \dsl\sum{k = 0}m c_k g_k(x) + o(g_n(x)), \quad x\to x_0$,}\F{$f(x) = \dsl\sum{k = 0}m d_k g_k(x) + o(g_n(x)), \quad x\to x_0$.} Вычтя получим \F{$0 = (c_m - d_m)g_m(x) + o(g_m(x)),\quad x\to x_0$.} Поделив на $g_m(x)$ при $x \in E_m$ и перейдя к пределу по множеству $Е_m$, получим $c_m = d_m$, что противоречит определению $m$.
\skip
\subsection{Непрерывность. Точки разрыва и их классификации, примеры}
(114)\\
Пусть $(X, \rho_X)$ и $(Y, \rho_Y)$~--- метрические пространства, $f: D\subset X \to Y$, $x_0 \in D$. Отображение $f$ назывется непрерывным в точке $x_0$ если выполняется одно из следующих утвержений.\\
\q1. Предел отображения $f$ в точке $x_0$ существует и равен $f(x_0)$. (Применимо, если $x_0$~--- предельная точка D).\\
\q2. \q На $\varepsilon$-языке или по Коши.\\
$\forall \varepsilon > 0\ \exists \delta > 0\ \forall x \in D:\ \rho_X(x, x_0) < \delta \Ra \rho_Y(f(x), f(x_0)) < \varepsilon$\\
\q3. \q На языке окрестностей.\\
$\forall V_{f(x_0)}\ \exists V_{x_0}\ f(V_{x_0} \cap D) \subset V_{f(x_0)}$\\
\q4. \q На языке последовательностей или по Гейне.\\
$\forall \{x_n\}:\ x_n \in D, x_n \to x_0\ f(x_n) \to f(x_0)$\\
\q5. Бесконечно малому приращению аргумента соответствует бесконечно малое приращение отображения. $\Delta y \xra[\Delta x \to \theta_X]{} \theta_Y$

Отображение называется непрерывным на множестве $D$, если оно непрерывно в каждой точке множества $D$.\\
Множество отображений $f: D\subset X \to Y$ непрерывных на множестве $D$, обозначают $C(D\subset X\to Y)$ или $C(D\to Y)$

\Zam 1. Равносильность определений, когда $x_0$ предельная точка $D$, следует из равносильности различных определений предела. Под номерами 2, 3, 4 записан тот факт, что точка $a = f(x_0)$ является пределом отображения $f$ в точке $x_0$ с одним отличием в каждом случае. 2: опущено условие $x \neq x_0$; 3: окрестность не проколота; 4: опущено условие $x\neq x_0$. Так как это ничего не портит. Определение 5 на любом из языков записывается также, как и определение 1.

\Op пусть $f: D \subset X \to Y, x_0 \to D$. Если отображение $f$ не является непрерывным в точке $x_0$, то говорят, что $f$ разрывно (терпит разрыв, испытывает разрыв)  вточке $x_0$, а точку $x_0$ называют точкой разрыва.

\Pr1. Функция сигнум
\F{sign$x = \begin{cases}1, & x> 0\\
0, & x = 0,\\
-1, & x < 0\end{cases}$} Тогда $f(0+) = 1,\s f(0-) = 1,\s$ 0~--- точка неустранимого разрыва превого рода.

\Pr2. $f(x) = |$sign$(x)|$ Тогда $f(0+) = f(0-) = 1$ и 0~--- точка рустранимого разрыва первого рода.

\Pr3. $f(x) = \frac 1x$ ТОгда $f(0+) = +\infty,\s f(0-) = -\infty$ и 0~--- точка разрыва второго рода.

\Pr4. $f(x) = \frac 1{x^2}$. Тогда $f(0+) = f(0-) = +\infty$ и 0~--- точка разрыва второго рода

\Pr5. $f(x) = \frac{x - 1}{x^2 - 1}$, тогда $f$ определена на $\R \bsl \{-1, 1\}$ и на области определения $f(x) = \frac 1{x + 1}$. Точка --1~--- точка разрыва второго рода, 1~--- точка устранимого разрыва. положим $f(1) = \frac 12$. получим непрерывную  точке 1 функцию.
\skip
\subsection{Арифметические действия над напрерывными отображениями. Стабилизация знака непрерывной функции.}
(122)

\T \q Арифметические действия над непрерывными отображениями. Пусть $X$~--- метрическое пространство, $Y$~--- нормированное пространство, $D\subset X,\s x_0 \in D$, отобрадения $f, g: D \to Y,\s \lambda: D \to \R(\C)$ нерперывны в  точке $x_0$ю Тогда отображения $f + g,\s f - g,\s \lambda f, |f|$, непрерывны в точке $x_0$.

\D Если $x_0$~--- изолированнач точка $D$, то утверждение тривиально. Если же $x_0$~--- предельная точка $D$, то теоремы о непрерывности следуют из теорем о пределах.

\Zam1. \q О стабилизации знака непрерывной функции. Если функция $g: D\to \R$ непрерывна в точке $x_0$, причём $g(x_0) \neq 0$, то существует такая окрестность $V_{x_a}$, что sign$g(x) = \textit{sign}g(x_0)$ для всех $x \in V_{x_0} \cap D$.

\D Для определённости рассмотрим случай, когда $g(x_0) > 0$. Допустим противное: пусть для любого $n \in \N$ существует точка $x_n \in V_{x_0}(\frac 1n) \cap D$, для которой $g(x_n) \le 0$. Построенная последовательность $\{x_n\}$ стремится к $x_0$. По определению непрерывности $g(x_n) \to g(x_0)$. по теореме о предельном переходе в неравенстве, $g(x_0) \le 0$, что противоречит условию.
\skip
\subsection{Непрерывность и предел композиции}
(124)

\T \q Непрерывность композиции. Пусть $X, Y, Z$~--- метрические пространства, $f : D \subset X \to Y,\s g: E \subset Y \to Z,\s f(D) \subset E, f$ непрерывно в точке $x_0 \in D,\s g$ непрерывно в точке $f(x_0)$. Тогда $g\circ f$ непрерывно в точке $x_0$.

\D Возьмём последовательность $\{x_n\}$, такую что $x_n \in D,\s x_n \to x_0$. Обозначим $y_n = f(x_n),\s y_0 = f(x_0)$; тогда $y_n, y_0 \in E$. По определению непрерывности $f$ в точке $x_0$ на языке последовательностей $y_n \to y_0$. По определению непрерывности $g$ в точке $y_0$ на языке последователбностей $g(y_n) \to g(y_0)$. то есть  $(g \circ f)(x_n) \to (g \circ f)(x_0)$. Последнее в силу проивольности $\{x_n\}$ и означает непрерывность $g \circ f$ в точке $x_0$.

\Zam2. Пусть $f(x) = x \sin\frac 1x,\s g (y) = |$sign$y|$. Тогда $f(x) \xra[x \to 0]{} 0,\s g(y) \xra[y \to 0]{} 1$, но композиция $g \circ f$ не имеет предела в нуле, так как $(g \circ f) (\frac 1{n\pi}) = 0 \to 0$, а $(g\circ f)(\frac 1{(n + 1/2)\pi}) = 1 \to 1$. Этот пример показывает, что утверждение "если $f(x) \t A, g(x) \xra[x \to A]{} B$, то $g \circ f(x) \t B$" может не выполняться. Если же запертить $f(x)$ принимать значение $A$, то утверждение становится верным.

Пусть $X, Y, Z$~--- метрические пространства, $f: D \subset X \to Y,\s g: E \subset Y \to Z,\s f(D) \subset E$ и выполнены условия:\\
1. $a$~--- предельная точка $D,\s f(x) \t A$;\\
2. $A$~--- предельная точка $E,\s g(x) \xra[x\to A]{} B$;\\
3. Существует такая окрестность $V_a$ точки $a$, что $f(X) \neq A$ для любого $x \in \dot{V}_a \cap D$.\\
Тогда $(g \circ f) (x) \t B$.
\skip
\subsection{Характеристика непрерывности отображения с помощью прообразов}
(125)

\T \q Характеристика непрерывности отображения с помощью прообразов. Пусть $X, Y$~--- метрические пространства, $f: X \to Y$. Тогда для непрерывности $f$ на $X$ необходимо и достаточно, чтобы при отображении $f$ прообраз любого открытого в $Y$ множества был открыт в $X$.

\D 1. Пусть $f$ непрерывно и множество $U$ открыто в $Y$. Докажем, что множество $f^{-1} (U)$ открыто в $X$. Для этого возьмём точку $a \in f^{-1}(U)$ и докажем, что $a$~--- внутренняя точка $f^{-1} (U)$. Так как $f(a)\in U$, а $U$ открыто, существует окрестность $V_{f(a)}$, содержащаяся в $U$. По определению непрерывности $f$ в точке $a$ найдётся окрестность $V_a$ такая, что $f(V_a) \subset V_{f(a)} \subset U$. Следовательно $V_a \subset f^{-1}(U)$, то есть $a$~--- внутренняя точка $f^{-1}(U)$.

2. Пусть прообраз любого открытого множества открыт, $a \in X$. Докажем, что $f$ непрерывно в точке $a$; в силу произвольности $a$ это и будет означать непрерывность $f$ на всём $X$. Возьмём окрестностть $V_{f(a)}\subset Y$. По условию её прообраз $G = f^{-1}(V_{f(a)})$ открыт в $X$, при этом $a \in G$. Значит, найдётся окрестность $V_a: V_a \subset G$. Осталось проверить, что $f(V_a) \subset V_{f(a)}$. Тогда определение непрерывности $f$ в точке $a$ на языке окрестностей будет выполнено. Действительно, если $y \in f(V_a)$, то, по определению образа, $y = f(x)$ для некоторого $x \in V_a$; тем более $x \in G$. По определению прообраза $f(x) \in V_{f(a)}$, то есть $y \in V_f(a)$

\Zam1. Пусть $f:[0, 2] \to \R,\s f(X) = x$. Тогда \F{$f^{-1}(1, +\infty) = (1, 2]$.} Противоречий с теоремой нет: полуинтервал $(1, 2]$ открыт в $X = [0, 2]$, хотя и не является открытым в $\R$.
\skip
\subsection{Теорема Вейерштрасса о непрерывных отображениях, следствия}
(126)

\T \q Вейерштрасса О непрерывных отображениях. Пусть $X, Y$~--- метрические пространчтва, $X$~--- компактно, $f \in C(X\to Y)$. Тогда $f(x)$ компактно. Другими словами: непрерывный образ компакта~--- компакт.

\D Пусть $\iseq G\alpha A$~--- открытое покрытие множества $f(X): f(X) \subset \dsl\bigcup{\alpha \in A}{} G_\alpha$. По теореме о характеристике непрерывности с помощью прообразов, при всех $\alpha \in A$ множества $f^{-1}(G_\alpha)$ открыты в $X$. Проверим, что \F{$X = \dsl\bigcup{\alpha \in A}{} f^{-1} (G_\alpha)$.} В самом деле, если $a \in X$, то $f(A) \in Y$ и, значит, $f(a) \in G_\alpha$ при некотором $\alpha$, то есть $a \in f^{-1}(G_\alpha)$ при некотором $\alpha$. Следовательно, $X$ содержится в объединении $f^{-1} (G_\alpha)$. Обратное включение тривиально.

Пользуясь компактностью $X$, выделим из его откытого покрытия $\se{f^{-1}(G_\alpha)}{\alpha \in A}{}$ конечное подпокрытие: найдётся такой конечный нобор индексов $\alpha_1, \cdots, \alpha_N \in A$, что
\F{$X = \dsl\bigcup{i = 1}N f^{-1}(G_{\alpha_i})$.} Осталось проверить, что \F{$f(X) \subset \dsl\bigcup{i = 1}N G_{\alpha_i}$;} это и будет означать, что из произвольного открытого покрытия удастся извлечь конечное подпокрытие. Действительно, если $y \in f(X)$, то $y = f(x)$ для некоторого $x \in X$. Тогда найдётся такой номер $i \in [1:N]$, что $x \in f^{-1}(G_{\alpha_i})$. Последнее означает, что $f(x) \in G_{\alpha_i}$, то есть $y \in G_{\alpha_i}$.

\S1. Непрерывный образ компакта замкнут и ораничен.

\S2. \q Первая теорема Вейерштрасса о непрерывных функциях. Функция непрерывная на отрезке ограничена.

\Zam1. Оба условия: и непрерывность, и то, что область определения отрезок~--- существенны. Так, функции $f(x) = x,\s g(x) \frac 1x$ непрерывны, но не ограничены, соответсвенно, на $\R$ и $(0, 1]$. Функция \F{$h(x) = \begin{cases} \frac 1x, & x \in (0,1],\\0, & x = 0.\end{cases}$} Задана на $[0, 1]$, разрывна в точке 0, но не ограничена.

\S3. Пусть $X$ компактно, $f \in C(X\to \R)$. Тогда существует $\dsl\max{x \in X}{} f(X), \dsl\min{x \in X}{}$. Другими словами: непрерывная на компакте функция принимает наибольшее и наименьшее значение.

\D Остаётся доказать, что компактно подмножество $E$ числовой прямой $(E = F(X))$ имеет наибольший и наименьший элемент. Существует  $\sup E = b \in \R$. Докажем, что $b \in E$: это и будет означать, что $b = \max E$. По определению супремум для любого $n \in N$ найдётся такая точка $x_n \in E$ , что $b - \frac 1n < x_n \le b$. Построенная последовательность стремится к $b$. Следовательно, $b \in E$ в силу замкнутости $E$. Доказательство для минимума аналогично.

\S4. \q Вторая теорема Вейерштрасса о непрерывных функциях. Функция, непрерывная на отрезке, принимает наибольшее и наименьшее значение.

\Zam2. И здесь оба условия существенны. Так как функции $f, g, h$ из замечания 1 не имеют наибольшего значения. Наибольшего значения не имеет и ограниченная непрерывная функция $f_1(x) = x$ на $[0, 1)$.
\skip
\subsection{Теорема Кантора}
(129)

\Op Пусть $X, Y$~--- метрические пространства $f: X\to Y$. Отображение $f$ называется равномерно непрерывным на $X$, если \F{$\forall \eps > 0\s\exists \delta > 0\s\forall\ol{x}, \ool{x} \in X: p_X(\ol{x}, \ool{x}) < \delta\s p_Y(f(\ol{x}), f(\ool{x})) < \eps$.} Ясно, что всякое раномерно непрерывное отображение непрерывно.

\T \q Кантор. Непрерывное на компакте отображение равномерно непрерывно.

\D Пусть $X$~--- компактно, $f \in C(X \to Y)$. Предположим, что $f$ не являетс яравномерно непрерывным. Тогда существует такое $\eps^* > 0$, что при каждом $n \in \N$ для числа $\delta = \frac 1n$ найдутся точки $\ol x_n, \ool x_n \in X$: \F{$\rho_X(\ol x_n, \ool x_n) < \frac 1n,\quad \rho_Y(\ol y_n, \ool y_n) \ge \eps^*$,} где $\ol y_n = f(\ol x_n),\s \ool y_n = f(\ool x_n)$.\\
Пользуясь секвенциальной компактностью $X$, выделим из последовательности $\{\ol x_n\}$ точек $X$ подпоследовательность $\{\ol x_{n_k}\}$, имеющую предел в $X$: $\ol x_{n_k} \to c \in X$. Тогда и $\ool x_{n_k} \to c$, так как \F{$\rho_X(\ool x_{n_k}, c) \le \rho_X(\ool x_{n_k}, \ol x_{n_k}) + \rho_X(\ol x_{n_k}, c) < \frac 1{n_k} + \rho_X(\ol x_{n_k}, c) \to 0$}. По непрерывности $f$ в точке $c$ \F{$\ol y_{n_k} \to f(c),\quad \ool y_{n_k} \to f(c)$.} Следовательно, $\rho_Y(\ol y_{n_k}, \ool y_{n, k}) \to 0$ и, начиная с некоторого номера $\rho_Y(\ol y_{n_k}, \ool y_{n, k}) < \eps^*$, что противоречит построению.
\skip
\subsection{Теорема Больцано-Коши о непрерывных функциях}
(130)

\T \q Больцано-Коши О промежуточном значении непрерывной функции. Пусть $f$ непрерывна на $[a, b]$. Тогда для любого числа $C$, лежащего между $f(a)$ и $f(b)$ найдётся такое $c \in (a, b)$, что $f(c) = C$.

\D 1. Пусть числа $f(X), f(b)$ разных знаков: $f(a)f(b) < 0$; докажем, что существует такая точка $c \in (a, b)$, что $f(c) = 0$. Не умаляя общности, будем считать, сто $f(a) < 0 < f(b)$; второй случай рассматривается аналогично. Рассмотрим середину отрезка $[a, b]$~--- точку $\frac{a + b}2$. Если $f(\frac{a + b}2) = 0$, то теорема доказана~--- можно положить $c = \frac{a +b}2$. Иначе
\F{$[a_1, b_1 = \begin{cases}[\frac{a + b}2, b], & f(\frac{a + b}2) < 0,\\ [a, \frac{a + b}2], & f(\frac{a + b}2) < 0. \end{cases}$} В обоих случаях $f(a_1) < 0 < f(b_1)$. Продолжим этот процесс построения помежутков. Если процес не завершится (не будет найдена точка $c$), то будкт построена последовательность вложенных отрезков, таких что $f(a_n) < 0 < f(b_n)$. При этом отрезки стягивающиеся, так как $b_n - a_n = \frac{b - a}{2^n} \to 0$. По теореме о стягивающихся отрезках существует единственная точка $c$ принадлежащая одновременно всем отрезкам, при этом $a_b \to c,\s b_n \to c$. По теореме о предельном переходе в неравенстве $f(c) \le 0 \le f(c)$, то есть $f(c) = 0$.

2. Докажем теорему в общем случае. Пусть $\varphi = f - C$. Тогда $\varphi \in C[a, b]$ как разность непрерывных функций, $\varphi(a)\varphi(b) < 0$. По доказанному существует такая точка $c \in (a, b)$, что $\varphi(c) = 0$, то есть $f(c) = C$.

\Zam1. Теорему можно переформулировать так: если непрерывная на промежутке функция принимает два какие-то два значения, то она принимает все значения, лежащие между ними.\\
Здесь существенно и то, что функция непрерывна, и то, что она задана на промежутке. Функция sign, заданная на $\R$, разрывна в 0. Она принимает значения --1 и 1, но из чисел между ними только 0. Сужение функции на $\R\bsl\{0\}$ непрерывно, но не принимает значений, лежвщих между -1 и 1.

\Zam4. Другой способ доказательства теоремы Больцано-Коши~--- поверить, что если $f \in C[a, b],\s f(a) < 0 < f(b)$, то точка \F{$c = \sup\{x \in [a, b]: f(X) < 0\}$} есть корень функции $f$.
\skip
\subsection{Сохранение промежутка (с леммой о характере промежутков). Сохранение отрезка}
Билет 44: Сохранение промежутка (с леммой о характере промежутков). Сохранение отрезка
(131)

\L \q Характеристика промежутков. Пусть $E \subset \R$. Тогда следующие утверждения равносильны.\\
1. $E$~--- промежуток (возможно вырожденный).\\
2. Для любых $x, y$, принадлежащих $E\s (x < y),\s [x, y] \subset E$

\D Второе утверждение следует из первого тривиально. Докажем обратный переход. Пусть $e \neq \varnothing$. Обозначим $m = \inf R,\s M = \sup E$. Ясно, что $E \subset [m, M]$. Докажем, что $(m, M) \subset E$. Пусть $m < z < M$. Тогда по определению граней существуют точки $x, y \in E: x < z < y$. По условию $z \in E$

\T \q О сохранении промежутка. Непрерывный образ промежутка~--- промежуток.

\D Пусть $f \in C\ang ab$, \F{$m = \dsl\inf{x \in \ang ab}{} f(x),\quad M = \dsl\sup{x \in \ang ab}{} f(x)\quad (m, M \in \R)$.} По теореме Больцано-Коши о промеуточных значениях непрерывной функции, множество $E = f(\ang ab)$ выпукло, а, по лемме, $E$~--- промежуток, то есть $f(\ang a,b) = \ang mM$.

\Zam2. Промежуток $\ang mM$ может быть другого типа, нежели $\ang ab$

\S1. \q О сохранении отрезка. Непрерывный образ отрезка~--- отрезок. 

\D Действительно, множество $f([a, b])$~--- промежуток, а, по теореме Вейерштрасса, имеет наибольший и наименьший элемент.

\Zam3. Наибольшее и наименьшее значения не обязательно достигаются на концах отрезка.
\skip
\subsection{Теорема Больцано-Коши о непрерывных отображениях}
(133)

\Op Пусть $Y$~--- метрическое пространство, $E \subset Y$. Непрерывное отображение отрезка в множество $E$: \F{$\gamma \in C([a, b]) \subset \R \to E$} называется путём в $E$. Точка $\gamma(a)$ называется началом, $\gamma(b)$~--- концом пути.

\Op Пусть $Y$~--- метрическое пространство, $E \subset Y$. Множество $E$ называется линейн связным, если любые две его точки соединены путём.
\F{$\forall A, b \in E\s \exists \gamma \in C([a, b] \subset \R \to E) :\s \gamma (a) = A,\s \gamma(b) = B$}

\T \q Больцано-Коши О непрерывных отображениях. Пусть $X, Y$~--- метрические пространства, $X$ линейно связно, $f \in C(X \to Y)$. Тогда $f(X)$ линейно связно. Другими словами: непрерывный образ линейно свзного множества линейно связен.

\D Пусть $A, B \in f(X)$. Тогда, по определению образа, существуют точки $\alpha, \beta \in X: a = f(\alpha),\s B = f(\beta)$. Так как $X$ линейно связно, точки $\alpha, \beta$ можно соединить путём в $X$, то есть существует путь $\gamma \in C([a, b] \to X): \gamma(a) = \alpha,\s \gamma(b) = \beta$. Но тогда, по теореме о непрерывности композиции $f \circ \gamma$~--- путь в $f(X)$; при этом $(f \circ \gamma)(a) = A,\s (f \circ \gamma)(b) = B$.

\Zam4. Согласно лемме, на прямой линейно связными могут быть толко промежутки.

\Zam4. Теорема о сохранении промежутка, вообще говоря, не допускает обращения. Так, множество значений функции \F{$f(x) = \begin{cases}x, &x \in [0, 1],\\ 0, & x \in (1, 2].\end{cases}$} есть отрезок $[0, 1]$. Однако для монотонной функции обратное утверждение верно.
\skip
\subsection{Разрывы и непрерывность монотонной функции}
(134)

\T \q О разрывах и непрерывностях монотонной функции. Пусть $f: \ang ab \to \R,\s f$ монотонна. Тогда справедливы следующие утверждения.\\
1. $f$ не может иметь разрывов второго рода.\\
2. Непрерывность $f$ равносильна тому, что её множество значений~--- промежуток.

\D Пусть для определения $f$ возрастает.

1. Пусть $x_0 \in (a, b),\s x_1 \in \langle a, x_0).$ Тогда $f(x_1) \le f(x) \le f(x_0)$ для всех $x \in (x_1, x_0)$, пожтому $f$ возрастает и ограничена сверху на $\langle a, x_0)$. По теореме о пределе монотонной функции, существует конечный предел $f(x_0-)$, причем, по теореме о предельно переходе в неравенстве, $f(x_1) \le f(x_0-) \le f(x_0)$. Аналогично доказывается, что для любой точки $x_0 \in \langle a, b)$ существует конечный предел $f(x_0+)$, причём $f(x_0) \le f(x_0+) \le f(x_2)$ для всех $x_2 \in (x_0, b\rangle$.

2. Ввиду следствия о сохранении промежутка остается доказать достаточность. Пусть $f(\ang ab)$~--- промежуток. Докажем непрерывность $f$ слева в любой точке $x_0 \in \ang ab$ от противного. Пусть $f(x_0-) < f(x_0)$ (существование конечного левостороннего предела уже доказано). Возьмём $y \in (f(x_0-), f(x_0))$. Тогда если $a < x_1 < x_0$, то $y \in [f(x_1), f(x_0)]$. Следовательно, $y \in f(\ang ab)$, то есть $y$~--- значение функции. С другой стороны, для всех $x \in \langle a, x_0)$ будет $f(X) \le f(x_0-)< y$, а для всех $x \in [x_0, b\rangle$ будет $f(x) \ge f(x_0) > y$, то есть функция не принимает значение $y$. Полкченное противоречме доказывает, что $f(x_0-) = f(x_0)$. Аналогично $f$ непрерывна справа в любой точке $x_0\ in \langle a, b)$.
\skip
\subsection{Существование и непрерывность обратной функции}
(134)

\T \q О существовании и непрерывности обратной функции. Gecnm $f: C(\ang ab \to \R),\s f$ строго монотонна, \F{$m = \dsl\inf{x \in \ang ab}{} f(x), \quad M = \dsl\sup{x\in\ang ab}{} f(x)$.} Тогда справедливы следующие утверждения.\\
1. $f$ обратима, $f^{-1}: \ang mM \to \ang ab$~--- биекция.\\
2. $f^{-1}$ строго монотонна одноимённо с $f$.\\
3. $f^{-1}$ непрерывна.

\D Пусть для определения $f$ строго возрастает.\\
Если $x_1, x_2 \in \ang ab,\s x_1 < x_2$, то $f(x_1) < f(x_2)$; следовательно $f$ обратима. По теореме о сохранении промежутка $f(\ang ab) = \ang mM$. По общим свойтсвам обратимого отображения $f^{-1}$~--- биекция $\ang mM$ и $\ang ab$.\\
Докажем, что $f^-1$ строго возрастает. Если $y_1, y_2 \in \ang m, M,\s y_1 < y_2$, то $y_ 1 = f(x_1),\s y_2 = f(x_2)$, где $x_1, x_2 \in \ang ab,\s x_1 = f^{-1}(y_1),\s x_2 = f^{-1}(y_2)$. При этом $x_1 < x_2$, так как возможность $x_1 \ge x_2$ исключена в силу строгого возрастания $f$.\\
Возрастающая функция $f^{-1}$ задана на промежутке $\ang mM$, а её множество значений~--- промежуток $\ang ab$. По теореме о разрывах и непрерывности монотонной функции, она непрерывна.

\Zam1. Для обратимости строго омнотонной функции и строгой монотонности обратимой функции непрерывность не нужна.

\Zam2. 1. Множество точек разрыва монотонной функции не более чем счётно.\\
2. Если функция задана на промежутке, непрерывна и обратима, то она строго монотонно и, следовательно, обратная функция непрерывна.\\
3. Отображение, обратное к непрерывному, может окад=заться разрывным. Сопоставим каждой точке $x$ подуинтервала $[0, 2\pi)$ точку $f(x)$ единичной окружности $S$, такую что длина дуги, отсчитываемой от точки $f(0) = (1, 0)$ до точки $f(x)$, равна $x$ (или, что тоже самое $\arg f(x) = x$). Отображение $f$ биективно и непрерывно, но $f^{-1}$ терпит разрыв в точке $(1, 0)$. Близким к ней точкам окружности с отрицательной ординатой соответствуют точки полуинтервала, близкие к $2\pi$, а не к 0.\\
Но если отображение задано на компакте, непрерывно и обратимо, то обратное отображение непрерывно.\\
4. Существует обратимая функция $f: \R \to \R$, непрерывная в точке 0, но такая, что $f^{-1}$ разрывна в точке $f(0)$.
\skip
\subsection{Степень с произвольным показателем}
(136)

Степенную функцию с показателем $\alpha$, которая $x$ сопоставляет $x^\alpha$, будем обозначать $\epsilon_\alpha: \epsilon(x) = x^\alpha$. Заранее отметим, что области определнения степенных функций могут быть различны при разных показателях.

При $\alpha = 1$ функция $\epsilon_1 =$ id$_\R$, как уже отмечалось, непрерывна на $\R$.

При $\alpha = n\in\N$ по определению, $x^n = x\cdot x\cdot ... \cdot x$ $n$ раз $x \in \R$. Следовательно $\epsilon_n$непрерывна на $\R$, как произведение непрерывных.

При $\alpha = -n$, где $n \in \N$, полагаем \F{$x^{-n} = \frac 1{x^n}, \quad x \in \R \bsl \{0\}$}. Следовательно, функция $\epsilon_{-n}$ непрерывна на $\R\bsl \{0\}$ как частное непрерывных.

При $\alpha = 0$ по определению полагаем $x^0 = 0$ при всех $x \neq 0$, тогда можно в соответствии с общим соглашением доопределить  по непрерывности $x^0 = 1$ и при $x = 0$.

Если $n \in \N,\s n$ нечётно, то функция $\epsilon_n$ строго возрастает на $\R$, $\dsl\sup{x \in \R}{} \epsilon_n(x) = +\infty,\s \dsl\inf{x \in \R}{} \epsilon_n(x) = -\infty$; по теореме о сохранении промежутка $\epsilon_n(\R) = \R$. Если $n \in \N,\s n$ чётно, то функция $\epsilon_n$ строго возрастает на $\R_+, \dsl\sup{x\in \R_+}{} \epsilon_n(x) = +\infty,\s \dsl\min{x \in \R_+}{} \epsilon_n(x) = 0$; по теореме о сохранении промежутка $\epsilon_n(\R_+) = \R_+$. По теореме о существовании и непрерывности обратной функции существует и непрерывна функция \F{$\epsilon_{\frac 1n} = \begin{cases}\epsilon^{-1}_n, & n \text{ нечетно}, \\ (\epsilon_n|_{\R_+})^ {-1}, & n \text{ чётно},\end{cases}$} которая называется корнем $n$-ной степени и обозначается ещё $\sqrt[n]{(\cdot)}: \epsilon_{1/n} = x^{1/n} = \sqrt[n]{x}$. Итак, \F{$\epsilon_{1/n}: \R \xra[]{\textit{на}} \R,\qquad n$ чётно,} \F{$\epsilon_{1/n}: \R_+ \xra[]{\textit{на}} \R,\quad n$ чётно;} $\epsilon_{1/n}$ строго возрастает и непрерывна.

При $\alpha \in \Q$ $\alpha = r = \frac pq$~--- несократимая дробь $p \in \Z,\s q \in \N$. Полагаем \F{$x^r = (x^p)^{1/q}$,} для всех $x$, для которых правая часть имеет смысл. Другими словами $\epsilon_r = \epsilon_(1/q) \circ \epsilon_p$. Тогда $x^r$ определено в следующих случаях \F{$x > 0,\quad r$ любое,} \F{$x = 0,\quad r \ge 0$} \F{$x < 0,\quad q$ нечётно.} Функция $\epsilon_r$ непрерывна на своей области определения; она строго возрастает на $[0, +\infty)$ при $r > 0$, строго убывает на $(0, +\infty)$ при $r < 0$.

(145)

При всех $x > o,\s a \in \R$ по свойству $a^{xy} = (a^x)^y$  верна формула $x^ \alpha = e^{\alpha \ln x.}$ Поэтому степенная функция $\epsilon_\alpha$ непрерывна на $(0, +\infty)$ при всех $\alpha \in \R$. Если $\alpha$ иррационально, то \F{$\epsilon_\alpha:[0, +\infty) \xra[]{\textit{на}}[0, +\infty),\quad \alpha > 0$,} \F{$\epsilon_\alpha:(a, +\infty) \xra[]{\textit{на}}(0, +\infty),\quad \alpha < 0$.}  Непрерывность $\epsilon_\alpha$ в нуле при $\alpha > 0$ также имеет место: если $x_n > 0, x_n \to 0$, то $y_n = \ln x_n \to -\infty$ и $\epsilon_\alpha(x_n) = e^{\alpha y_n} \to 0 = e_\alpha(0)$
\skip
\subsection{Свойства показательной функции и логарифма}
(140)

\q1. Функция $\exp_a$ строго возрастает на $\R$ при $a > 1$ и строго убывает на $\R$ при $0 < a < 1$.

\D Пусть $a > 1,\s x < y$. Докажем, что $a^x < a^y$. Возьмём рациональные числа $\ol r, \ool r$, такие что \F{$x < \ol r < \ool r < y$} и две последовательности рациональных чисел $\{\ol r_n\}$ и $\{\ool r_n\}$, такие что \F{$\ol r_n < x < y < \ool r_n\quad \ol r_n \to x,\s \ool r_n \to y$.} Тогда в силу строгой монотонности показательной функции рационального аргумента \F{$a^{\ol r_n} < a^{\ol r} < a^{\ool r} < a^{\ool r_n}$.} По теореме о предельном переходе в неравенстве,
\F{$a^x \le a^{\ol r} < a^{\ool r} \le a^ y$.} Случай $0 < a < 1$ разбирается аналогично.

\q2. $a^{x + y} = a^x a^y$. В частности $a ^{-x} = \frac 1{a^x}$

\D Возьмём две последовательность рациональных чисел $\{\ol r_n\}$ и $\{\ool r_n\}$, стремящиеся к $x, y$ и перейдём к пределу в равенстве
\F{$a^{\ol r_n + \ool r_n} = a^{\ol r_n}a^{\ool r_n}$,} которое верно для рациональных чисел.

\q3. Показательная функция непрерывна на $\R$.

\D Непрерывность показательной функции в нуле доказывается на языке последовательностей. $\{x_n\}$~--- последовательность вещественных чисед, $x_n \to 0$. Возьмём $\eps > 0$ и зафиксируем номер $N_0$ для которого выполняется неравенство $1 - \eps < a^{-1/N_0} < a^{1/N_0} < 1 + \eps$. Тогда найдётся такой номер $N$,что для всех $n > N$ будет $-\frac 1{N_0} < x_n < \frac 1{N_0}$. В силу строгой монотонности показательной функции \F{$1 - \eps < a^{-1/N_0} < a^{x_n} < a^{a/N_0} < 1 + \eps$} для таких $n$. Это и означает, что $a^{x^n} \to 1$. Случай $0 < a < 1$ разбирается аналогично.\\
Непрерывность в произвольной точке $x_0$ следует из доказанной непрерывности в нуле \F{$a^{r_0 + \Delta x} - a^{x_0} = a^{x_0} (a^{\Delta x} - 1) \to 0$}.

\q4. $(a^x)^y = a^{xy}$.

\D Возьмём две последовательности рациональных чисел $\{x_n\},\s \{y_m\}: x_n \xra[n \to \infty]{} x,\s y_m \xra[m \to \infty]{} y$. По известному свойству степени с рациональным показателем $(a^{x_n})^{y_m} = a^{x_n y_m}$. Зафиксируем $m$ и устремими $n$ к $\infty$. Тогда, по определению показательной функции $a^{x_n y_m} \xra[n \to \infty] a^{xy_m}$ и $a^{x_n} \xra[n \to\infty] a^x$, а по непрерывности степенной функции с рациональным показателем $(a^{x_m})^{y_m} \xra[n\to\infty]{} (a^x)^{y_m}$. Поэтому $(a^x)^{y_m} = a^{xy_m}$. Осталось устремить $m$к $\infty$и воспользоваться непрерывностью показательной функции.

\q5. $(ab)^x = a^x b^x$

\D Сделаем предельный переход в равенстве для степеней с рациональным показателем.

\q6. $\exp_a: \R\xra[]{\textit{на}}(0, +\infty)$.

\D Пусть $a > 1$. Функция $exp_n$ строго возрастает, поэтому существуб пределы $\dsl\lim{x \to \pm \infty}{} a^x/$. по неравенству Бернулли
\F{$a^n = (1 + \alpha)^n \ge 1 + na \to +\infty, \quad a^{-n} = \frac 1{a^n} \to 0$.} Значит по свойствам промежутака $\exp_a(\R) = (0, + \infty)$. Кроме того, значение 0 не принимается в силу строгой монотонности: если $a^{x_0} = 0$, то $a^x < 0$ при $x < x_0$, чего быть не может. Доказательство при $0 < a < 1$ аналогично.

\q1. $\log_a(xy) = \log_a x + \log_a y$ при всех $x, y > 0$.

\D По свойству 2, \F{$a^{\log_a x + \log_a y} = a^{\log_a x}a^{\log_a y} = xy$.}

\q2. $\log_a x^\alpha = \alpha \log_a x$ при всех $x > 0,\s \alpha \in \R$ В частности $\log_a \frac 1x = -\log_a x$.

\D По свойству 4, \F{$a^{\alpha \log_a x} = (a^{\log_a x})^\alpha = x^\alpha$.}

\q3. $\log_a x = \frac{\log_b x}{\log_b {a}}$ при всех $x > 0$. В частности, $\log_a b = \frac 1{\log_b a}$.

\D По свойству 4 \F{$b^{\log_b a\log_a x} = (b^{\log_b a})^{\log_a x} = a^{\log_a x} = x$}.
\skip
\subsection{Непрерывность тригонометрических и обратных тригонометрических функций}
(146-154)

\L Если $0 < x < \frac \pi2$, то \F{$\sin x < x < \tg x$.}

\D Изобразим единичную окружность и угол в $x$ радиан.\\
\ig{sintan}\\
На рисунке \F{$\triangle OAB \subset \textit{сект.} OAB \subset \triangle OAD$.} Поэтому фигуры связаны неравенством \F{$S_{\triangle OAB} < S_{\textit{сект.} OAB} < S_{\triangle OAD}$} Учитывая, что \F{$S_{\triangle OAB} = \frac12 |OA||BC|$,} \F{$S_{\textit{сект.} OAB} = \frac 12|OA|^2 x,\quad S_{\triangle OAD} = \frac 12 |OA||AD|$}, \F{$|OA = 1|,\s |BC| = \sin x,\s |AD| = \tg x$.}

\S1. При всеx $x \in \R$ $|\sin x| \le |x|$

\D При $|x| \in (0, \frac \pi2)$ доказано по лемме, иначе $|sin x| < 1 < \frac \pi2 \le x$

\S2. Функции синус и косинус непрерывны на $\R$.

\D Для любой точки $x_0 \in \R$ имеем: \F{$|\sin x - \sin x_0| = |2\sin \frac{x - x_0}2 \cos\frac{x + x_0}2 \le 2\cdot\frac{|x - x_0|}2\cdot 1 = |x - x_0| \xra[x \to x_0]{} 0$.} Непрерывность косинуса доказывается с помощью формулы приведения $\cos x = \sin (\frac \pi2 - x)$ и теоремы о непрерывности композиций.

Тангенс и котангенс непрерывны на своих областях определения, по теореме о непрерывности частного.

$\arcsin = (\sin|_{[-\frac \pi2, \frac \pi2]})^{-1}$. По теореме о существовании и непрерывности обратной функции, функция арксинус строго возрастает и непрерывна

$\arccos = (\cos|_{[0, \pi]})^{-1}$. Аналогично функция арккосинус строго убывает и непрерывна.

$\arctg = (\tg|_{(-\frac \pi2, \frac \pi2)})^-1$. Аналогично функция арктангенс строго возрастает  и непрерывна.

$\arcctg = (\ctg|_{(0, \pi)})^{-1}$. Аналогично функция аркоктангенс строго убывает и непрерывна.
\skip
\subsection{Замечательные пределы}
Билет 51: Замечательные пределы
(154-158)

\q1.\F{$\dsl\lim{x\to 0}{} \frac{\sin x}x = 1$}
По лемме $\sin x < x < \tg x \Ra \cos x < \frac {\sin x}x < 1$, применяя предельный переход в неравенстве при $x \to 0$ $\frac{\sin x}x = 1$.

\S1. \F{$\dsl\lim{x\to 0}{}\frac{1 - \cos x}{x^2} = \frac 12,\qquad \dsl\lim{x\to 0}{}\frac{\tg x}{x} = 1$,}
\F{$\dsl\lim{x\to 0}{}\frac{\arcsin x}{x} = 1,\qquad \dsl\lim{x\to 0}{}\frac{\arctg x}{x} = 1$.}

\q2. \F{$\dsl\lim{x\to\infty}{} (1 + \frac 1x)^x = e$}

\D Напомним, что $e = \dsl\lim{n\to\infty}{} (1 + \frac 1n)^n$. Разница между этим и доказываемым в том, что теперь речь идёт о пределе не последовательности, а функции, заданной на $\R \bsl [-1, 0]$: аргумент $x$ необязан принимать натуральные и даже положительные значения.\\
Для доказательства воспользуемся языком последовательностей. Возьмём последовательность $\{x_n\}: x_n \to \infty$ и докажем, что $f(x_n) \to e$.\\
1. пусть сначала $x_n \in \N$ для всех $n$. Возьмём $\eps > 0$ и по определению числа $e$ подберём такой номер $K$, что для всех номеров (то есть натуральных чисел) $k > K$ будет $|f(k) - e| < \eps$. Но начиная с некоторого номера $x_n > k$, а тогда $|f(x_n) - c| < \eps$, что и означает выполнение требования.\\
2. Пусть $x_n \to +\infty$. Тогда, начиная с некоторого номера $x_n > 1$, поэтому, не уменьшая общности, можно считать, что все $x_n > 1$. Уменьшая или увеличивая основание и показатель степени получаем равенство \F{$(1 + \frac 1{[x_n] + 1})^{[x_n]} \le (1 + \frac 1{x_n})^x_n \le (1 + \frac 1{x_n}) ^ {[x_n] + 1}$,} которые перепишем в виде
\F{$\frac{f([x_n] + 1)}{1 + \frac 1{[x_n] + 1}} \le f(x_n) \le (1 + \frac 1{[x_n]})f([x_n])$.} Так как $\{[x_n]\}$ и $\{[x_n] + 1\}$~--- поседлвательности натуральных чисел, стремящихся к $+\infty$, то, по доказанному, $f([x_n]) \to e,\s f([x_n] + 1) \to e$. Следовательно, по теореме о сжатой последовательности, $f(x_n)$ стремится к $e$.\\
3. Пусть $x_n \to -\infty$, тогда $y_n = -x_n \to +\infty$. По доказанному, \F{$f(x_n) = (1 + \frac 1{-y_n})^{-y_n} = (\frac{y_n}{y_n - 1})^{y_n} = (1 + \frac 1{y_n - 1})f(y_n - 1) \to e$.}
4. Пусть $x_n \notin [-1, 0],\s x_n \to \infty$, а в остальном $\{
x_n\}$ произвольная. Если чисел отрицательных (положительных) конечно,т то $x_n \to +\infty (-\infty)$ и требуемое соотношение уже доказано, иначе разобъём на две полпоследовательность положительных и отрицательных чисел. Они обе стремятся к $e$, тогда и вся поседовательность сходится к $e$ по лемме.

\Zam1. Заменимв $x$ на $\frac 1x$ модно получить \F{$\dsl\lim{x\to 0}{} (1 + x)^{1/x} = e$.}

\q3. \F{$\dsl\lim{x\to 0}{}\frac{\log_a(1 +x)}x = \frac 1{\ln a},\qquad a > 0,\s a \neq 1$.} В частности, \F{$\dsl\lim{x\to 0}{} \frac{\ln(1 + x)}x = 1$.}

\D Так как $\log_a(1 + x) = \frac{\ln(1 + x)}{\ln a}$, достаточно доказать равенство для натурального логарирфма. Имеем \F{$\li{x}{0}\frac{ln(1 + x)}x = \li{x}{0} \ln(1 + x)^{1/x} = \ln \li{x}{0}(1 + x)^{1 / x} = \ln e = 1$.} Во втором равенстве воспользовались непрерывностью логарифма в точке $e$ и теоремой о непрерывности композиции (для её применения мы доопределим $(1 + x)^{1/x} = e$ при $x = 0$).

\q4. \F{$\li x0 \frac{(1 + x)^\alpha - 1}x = \alpha,\qquad \alpha\in\R$.}

\D При $\alpha = 0$ тривиально. Пусть $\alpha \neq 0$. Возьмйм последовательность $\{x_n\}: x_n \to 0,\s x_n \neq 0$; не уменьшая общности, можно считать, что $|x_n| < 1$. Тогда в силу непрерывности и строгой монотонности степенной функции $y_n = (1 + x_n)^\alpha - 1 \to 0,\s y_n \neq 0$. При этом \F{$\alpha\ln(1 + x_n) = \ln(1 + y_n)$.} Пользуясь замечательным пределом для логарифма находим
\F{$\frac{(1 + x_n)^\alpha - 1}{x_n} = \frac {x_n}{y_n} = \frac{y_n}{\ln(1 + y_n)}\alpha\frac{\ln(1 + x_n)}{x_n} \to \alpha$.}

\q5. \F{$\li{x}{0}\frac{a^x - 1}x = \ln a,\qquad a > 0$.} В частности \F{$\li{x}{0} = \frac{e^x - 1}x = 1$}

\D При $a = 1$ Доказывается равенство тривиально; пусть $a \neq 1$. Возьмём последователбность $\{x_n\}:x_n\to 0,\s x_n \neq 0$. Тогда в сид непрерывности и строгой монотонности показательной функции $y_n = a^{x_n} - 1 \to 0,\s y_n \neq 0$. При этом \F{$x_n\ln a = \ln(1 + y_n)$.} Пользуясь замечательным пределом для логарифма, находим \F{$\frac{a^{x_n} - a}{x_n} = \frac{y_n}{x_n} = \frac{y_n}{\ln(1 + y_n)} \ln a \to \ln a$.}
\skip
\subsection{Дифференцируемость и производная. Равносильность определений примеры}
(169)

\Op Пусть $f: \ang ab \to\R,\s x_0 \in \ang ab$. Существует такое число $A \in \R$, что \F{$f(x) = f(x_0) + A(x - x_0) + o(x - x_0),\quad x \to x_0$,} то функция называется диффернцируемой в точке $x_0$. При этом чисдо $A$ называется производной фугкции $f$ точке $x_0$.

\Op Пусть $f: \ang ab \to \R,\s x_0\in \ang ab$. Если существует предел \F{$\li{x}{x_0} \frac{f(x) - f(x_0)}{x - x_0}$,} равный числу $A \in \R$, то функция $f$ называется дифференцируемой в точке $x_0$, а число $A$~--- её производной в точке $x_0$.

\T Определения дифференцирцуемости и производной равносильны.

\D 1. Пусть $f$ дифференцируема, а $A$~--- ее производная в точке $x_0$, в смысле определения 1, которое говрит, что \F{$f(x) = f(x_0) + A(x - x_0) + \varphi(x)(x - x_0), \quad \varphi(x) \xra[x\to x_0]{} 0$.} Перенося в $f(x_0)$ в левую часть и деля на $x - x_0$ находим, что \F{$\frac{f(x) - f(x_0)}{x - x_0} = A + \varphi(x)\xra[x \to x_0]{} A$.} то есть $f$ дифференцирцема, а $A$~--- ее производная.\\
1. Обратно, пусть функция $f$ дифференцируем, а $A$~--- ее производная в смысле определения 2. Обозначим \F{$\varphi(x) = \frac{f(x) - f(x_0)}{x - x_0} - A$.} Тогда $\varphi(x) \xra[x\to x_0]{} 0$ и выполнено равенство из превой части доказательства, то есть $f$ дифференцируема, а $A$~--- призводная в смысле определения 1.

\Pr1. $f(x) = |x|$ $f'_\pm(0) = \li{x}{0\pm}\frac{|x| - 0}{x - 0} = \pm 1$. Поэтому она не дифференцируема в нуле.

\Pr2. $f(x) = x \sin \frac 1x$ при $x \neq 0,\s f(0) = 0$ $f'(0) = \frac{f(x) - f(0)}{x - 0} = \sin \frac 1x$ не имеет предела.

\Pr3. $f(x) = \sqrt[3]{x}$ $f'(0) = \frac{\sqrt[3]{x} - 0}{x - 0} \to x^{-2/3} \to +\infty$.

\Pr4. $f(x) =$ sign$x$ $f' = \frac{\textit{sign}x - \textit{sign}0}{x - 0} = \frac 1{|x|} \to +\infty$.
\skip
\subsection{Геометрический и физический смысл производной}
(174)

\q Геометрический (задача Лейбница о касательной).\\
Пусть $f: \ang ab \to \R,\s x_0 \in \ang ab,\s y_0 = f(x_0),\s f$ непрерывнав точке $x_0$. Точка $M_0 = (x_0, y_0)$. Возьмём на графике функции $f$ ещё одну точку $M_1 = (x_1б y_1):\s x_1 \in \ang ab,\s x_1 \neq x_0,\s y_1 = f(x_1)$. Проведём прямую $M_0 M_1$, которуб будем называть секущей. Уравнение секущей $M_0 M_1$ имеет вид\F{$u = y_0 + \frac{y_1 - y_0}{x_1 - x_0}(x_1 - x_0)$.} Касаткльной называетя предельное положение секущей при $M_1 \to M_0 (x_1 \to x_0)$ $k_{\textit{кас.}} = \li{x_1}{x_0} \frac{y_1 - y_0}{x_1 - x_0} = f'(x_0)$. Иными словами, производная в точке~--- угловой коэффициент касательной в этой точке. \F{$y = f(x_0) + f'(x_0)(x - x_0)$.}

\Zam1. Если $l(x) = f(x_0) + f'(x_0)(x - x_0)$,то \F{$f(x) - l(x) = o(x - x_0)\quad x \to x_0$}. Вто же время ни одна другая прямая не обладает этим свойством, поэтому егопринимают за определение касательной (не вертикальной).

\q физический (задача Ньютона о скорости).\\
Пусть материальная точка движется по прямой. Обозначим $s(t)$~--- путсьпройденный точкой за время от начального момента $t_0$ до $t$. Тогда путь.пройденный от момента $t_1$ до момента $t_1 + \Delta t$, равен $\Delta s = s(t_1 + \Delta t) - s(y_1)$. Средняя скорость между этими моментами времени вычисляется формулой $v_{\textit{ср}} = \frac{\Delta x}{\Delta t}$. $v_{\textit{мгн.}}(t_1) = \li{\Delta t}{0} v_{\textit{ср}}$~--- мгновенная скорость. По определению производной она равнв $s'(t_1)$. Подобным образом производная встречается и в ситуациях, когда речь идйт о скорости изменения одной величины относительно другой.
\skip
\subsection{Арифметические действия и производная}
(178)

Если $f, g: \ang ab \to \R$ дифференцируемы в точке $x \in \ang ab$

\q1. \q Производная суммы и разностию. то функция $f + g$ И $f - g$ дифференцируемы в этой точке и \F{$(f \pm g)'(x) = f'(x) \pm g'(x)$.}

\D По определению производной суммы. \F{$\frac{(f + g)(x + h) - (f + g)(x)}h = \frac{f(x + h) - f(x)}h + \frac{g(h + x) - g(x)}h \xra[h \to 0]{} f'(x) + g'(x).$} Это и означает, что сумма дифференцируема в точке и для производной суммы верно равенство.\\
Для разности доказывается аналогичнл.

\q2. \q Производная произведения. то функция $fg$ дифференцируема в точке $x$ и \F{$(fg)'(x) = f'(x)g(x) + f(x)g'(x)$.}

\D \F{$\frac{(fg)(h + x) - (fg)(x)}h = \frac{f(x + h) - f(x)}h g(x + h) + f(x) \frac{g(x + h) - g(x)}h \xra[h \to 0]{} f'(x)g(x) + f(x)g'(x)$.}

\S1. Если $\alpha \in \R$, то функция $\alpha f$ диффиренцируема в точке $x$ и \F{$(\alpha f)'(x) = \alpha f'(x)$.}

\S2. \q Линейность дифференцирования. Если $\alpha, \beta \in \R$, то функция $\alpha f + \beta g$ дифференцируема в точке $x$ и \F{$(\alpha g + \beta g)'(x) = \alpha f'(x) + \beta g'(x)$.}

\q3. \q Производная частоного. и $g(x) \neq 0$, то функция $\frac fg$ дифференцируема в точке $x$ и \F{$(\frac fg)'(x) = \frac{f'(x)g(x) - f(x) g'(x)}{g^2(x)}$.}

\D Прежде всего заметим, что в силу условия $g(x) \neq 0$ и непрерывности функции $g$ в точке $x$ существует такое $\delta > 0$, что $g$ не обращается в ноль на промежутке $(x - \delta, x + \delta)\cap\ang ab$. Поэтому частно $\frac fg$ определено на током промежутке, и модно ставить вопрос о дифференцируемости частного в очке $x$.
\F{$\frac{\frac fg(h + x) - \frac fg (x)}h = \frac 1{g(x + h)g(x)}(\frac{f(x + h) - f(x)}h g(x) - f(x)\frac{g(x + h) - g(x)}h) \xra[h \to 0]{} \frac{f'(x)g(x) - f(x)g'(x)}{g^2(x)}$}
\skip
\subsection{Производная композиции}
(180)

\T \q Производная композиции. Если функция $f: \ang ab \to \ang cd$ дифференцируема в точке $x \in \ang ab$, а функция $g: \ang cd \to \R$ дифференцируема в точке $f(x)$, то функция $g \circ f$ дифференцируема в точке $x$ и \F{$(g \circ f)'(x) = g'(f(x))\cdot f'(x)$.}

\D Обозначим $y = f(x)$. Воспользуемся определением 1 дифференцируемости и запишем \F{$f(x + h) = f(x) + f'(x)h + \alpha(h)h$,} \F{$g(y + k) = g(y) + g'(y)k + \beta(k)k$,} где функции $\alpha, \beta$ в нуле непрерывны и равны нулю. Подставляя во второе равенство $k = f'(x)h + \alpha(h)h = \varkappa(h)$, получаем
\F{$g(f(h + x)) = g(f(x)) + g'(f(x))(f'(x)h + \alpha(h)h) + \beta(\varkappa(h))\varkappa(h) = g(f(x)) + g'(f(x))f'(x)h + \gamma(h)h$,} где \F{$\gamma(h) = g'(y)\alpha(h) + \beta(\varkappa(h))(f'(x) + \alpha(h))$.} Ясно, что $\gamma(0) = 0$ b $\gamma$ непрерывна в нуле по теореме о непрерывности композиыции и результатов арифметических операций. Пожтому выполнено определение дифференцируемости композиции $g\circ f$ в точке $x$ и верно равенство.
\skip
\subsection{Производная обратной функции и функции, заданой параметрически}
(182)

\T \q Производная обратной функции. Пусть $f \in C\ang ab,\s f$ строго монотонна, дифференцируема в точке $x \in \ang ab,\s f'(x) \neq 0$. Тогда обратная функция $f^{-1}$ дифференцируема в точке $f(x)$ и \F{$(f^{-1})'(f(x)) = \frac 1{f'(x)}$.}

\D $f^{-1}$ существует, определена на промежутке $P$ (множество значений $f$), строго монотонна и непрерывна по теореме. Обозначим $y = f(x),\s h = f^{-1} (y + k) - f^{-1}(y) = \tau(k)$. ТОгда $h \neq 0,\s x = f^{-1}(y),\s x +h = f^{-1}(y _ k)$ и $f(x + h) - f(x) = k$.
\F{$\frac{f^{-1}(y + k) - f^{-1}(y)}k = \frac{\tau(k)}{f(x + \tau(k)) - f(x)}$} и найдём его предел при $k \to 0$. По условию, \F{$\frac h{f(x + h) - f(x)} \xra[h \to 0]{} \frac 1{f'(x)}$.} Но $\tau(k) \xra[k \to 0]{} 0$ по непрерывности $f^{-1}$ в точке y. Следовательно, \F{$\frac{f^{-1}(y + k) - f^{-1}(y)}k \xra[k \to 0]{} \frac 1{f'(x)}$} по теореме о непрерывности композиции.

\Zam1. Равенство можно переписать так: \F{$(f^{-1})'(x) = \frac 1{f'(f^{-1}(x))}$}

\Zam2. Дифференциал обратной функции в точке $x$~---  функция обратная диференциалу исходной в точке $x$.

\Zam3. Так как графики $f(x)$ и $f^{-1}(x)$ симметричны относительно $y = x$, касательные в симметричных точках тоже симметричны ($\tg \alpha = \ctg \beta$), то есть $f'(x) = \frac 1{(f^{-1})'(y)}$

\q Производная функции заданной параметрически.
Пусть $T$~--- множество, $\varphi, \psi: T \to \R$. Рассмотрим отображение $\gamma(\varphi, \psi): T \to \R^2$. Cистема \F{$\begin{cases}x = \varphi(t)\\ y = \psi(t)\end{cases}$} не всегда определяет функцию $y(x)$. Но если жто так ($\varphi$ обратима), то находим $t = \varphi^{-1}(x)$ из первого уравнения и подставляем во второе. Тогда \F{$y = \psi(\varphi^{-1}(x)),\quad x\in \varphi(T)$,} то есть $f = \psi \circ \varphi^{-1}$.\\
Обычно для встречающихся на практике систем множество $T$ можно разбить на несколько частей, на каждой из которых функция $\varphi$ обратима.\\
Пусть теперь $T = \ang ab,\s t \in \ang ab,\s \varphi \in C\ang ab,\s \varphi$ строго монотонна, $\varphi, \psi$ дифференцируемы в точке $t,\s \varphi'(t) \neq 0,\s f = \psi\circ\varphi^{-1}$~--- параметрически заданная функция. Тогда $f$ дифференцируема в точке $x = \varphi(t)$ и $f'(x) = \frac{\psi'(t)}{\varphi'(t)}.$ Следует из прафил дифференцирования композиции и обратной функции. Часто равенство записывают в виде $y'_x = \frac{y'_t}{x'_t}$.
\skip
\subsection{Производные элементарных функций}
(185)

\q1. $c' = 0$.

\q2. $(x^\alpha)' = \alpha x^{\alpha - 1}, \alpha \in \R$.

\D Пусть $x \neq 0$ считая, что $0 < |h| < |x|$. Пользуясь замечательным пределом для степенной функции получаем \F{$\frac{(x + h)^\alpha = x^\alpha}h = \frac{(1 + h/x)^\alpha - 1}{h/x}x^{\alpha - 1} \xra[h \to 0]{} \alpha x^{\alpha - 1}$.}

\q3. $(a^x)' = a^x \ln a,\quad a > 0,\s a \neq 1$.

\D По замечательному пределу для показательной функции \F{$\frac{a^{x + h} - a^x}h = a^x\frac{a^h - 1}h \xra[h \to 0]{} a^x \ln a$.}

\q4. $(\log_a x)' = \frac 1{x\ln a},\quad x > 0,\s a > 0,\ a \neq 1$.

\D По замечательному пределу для логарифма. \F{$\frac{\log_a(x + h) - \log_a x}h = \frac 1x \frac{\log_a(1 + h/x)}{h/x}\xra[h\to 0]{}\frac 1{x\ln a}$.}

\q5. $(\sin x)' = \cos x$.

\D По замечательному пределу для синуса и непрерывности косинуса \F{$\frac{\sin(x + h) - \sin x}h = \frac{\sin\frac h2 \cos(h + \frac h2)}h \xra[h\to 0]{} \cos x$.}

\q6. $(\cos x)' = -\sin x$.

\D По формуле для дифференцирования произведения и правиду дифференцирования композиции \F{$(\cos x)' = \sin (\frac \pi2 - x)' = \cos(\frac \pi2 - x)(-1) = -\sin x$.}

\q7. $(\tg x)' = \frac 1{\cos^2}$.

\D По формулам для производных синуса и коминуса и правилу дифференцирования частного \F{$(\tg x)' = (\frac{\sin x}{\cos x})' = \frac{(\sin x)'\cos x - (\cos x)'\sin x}{\cos^ x} = \frac{\cos^2 + \sin^2}{\cos^2 x} = \frac{1}{\cos^2 x}$.}

\q8. $(\ctg x)' = -\frac 1{\sin^2 x}$.

\D По формуле для производной тангенса и правилам дифференцирования композиции \F{$(\ctg x)' = \tg(\frac\pi2 - x)' = -\frac 1{\cos^2(\frac \pi2 - x)} = -\frac 1{\sin^2 x}$.}

\q9. $(\arcsin x)' = \frac 1{\sqrt{1 - x^2}},\quad x \in (-1, 1)$.

\D По правилу дифференцирования обратной функции \F{$(\arcsin x)' = \frac 1{(\sin y)'} = \frac 1{\cos y} = \frac 1{\sqrt{1 - \sin^2 y}} = \frac 1{\sqrt{1 - x^2}}$.}

\q10. $(\arccos x)' = -\frac 1{\sqrt{1 - x^2}},\quad x \in (-1, 1)$.

\D Так как $\arccos x = \frac \pi2 -\arcsin x$, \F{$(\arccos x)' = -(\arcsin x)' = -\frac 1{1 - x^2}$.}

\q11. $(\arctg x)' = \frac 1{1 + x^2}$.

\D По правилу дифференцирования обратной функции \F{$(\arctg x)' = \frac 1{(\tg y)'}= \cos^2 y = \frac 1{1 + \tg^2 y} = \frac 1{1 + x^2}$.}

\q12. $(\arcctg x)' = -\frac 1{1 + x^2}$.

\D Так как $\arcctg x = \frac\pi2 - \arctg x$ \F{$(\arcctg x)' = -(\arctg x)' = -\frac 1{1 + x^2}$.}
\skip
\subsection{Теорема Ферма}
(188)

\T \q Ферма. Пусть $f: \ang ab \to \R,\s x_0 \in (a, b),\s f(x_0) = \dsl\max{x \in (a, b)}{} f(x)$ или $f(x_0) = \dsl\max{x \in (a, b)}{} f(x),\s f$ дифференцируема в точке $x_0$. Тогда $f'(x_0) = 0$.

\D Пусть для определённости значение в точке $x_0$ наибольшее, то есть $f(x) \le f(x_0)$ при всех $x \in \ang ab$. Тогда $\frac{f(x) - f(x_0)}{x - x_0} \le 0$ при всех $x \in (x_0, b\rangle$. По теореме о предельном переходе в неравенстве \F{$f'(x_0) = f'_+(x_0) = \li{x}{x_0+} \frac{f(x) - f(x_0)}{x - x_0} \le 0$.}  Аналогично $\frac{f(x) - f(x_0)}{x - x_0} \ge 0$ при всех $x \in \langle a, x_0)$, и поэтому \F{$f'(x_0) = f'_-(x_0) = \li{x}{x_0-}\frac{f(x) - f(x_0)}{x - x_0} \ge 0$.} Следовательно $f'(x) = 0$.

\Zam1. Геометрический смысл теоремы Ферма: если во внутренней точке максимума (минимума) существует касательная, то эта касательная горизонтальна.

\Zam2. $f(x) = |x|$~--- пример функции не имеющей касательной в тоске минимума.

\Zam3. Условие, что $x_0$~--- внутренняя точка существенно: $f(x) = x^2$ на отрезке $[0, 1]$ принимает наибольшее значение в точке 1, при этом $f'(1) = 2$.
\skip
\subsection{Теорема Ролля}
Билет 59: Теорема Ролля
(189)

\T \q Ролля. Пусть функция $f$непрерывна на $[a, b]$, дифференцируема на $(a, b)$ и $f(a) = f(b)$. Тонда найдётся такая точка $x \in (a, b)$, что $f'(c) = 0$.

\D По теореме Вейерштрасса, существуют точки $x_1, x_2 \in [a, b]$, что $f(x_1) = \dsl\max{x \in [a, b]}{} f(x),\s f(x_2) = \dsl\min{x \in [a, b]}{} f(x)$. Если $x_1, x_2$~--- концевые точки $[a, b]$, то по условию $f(x_1) = f(x_2)$,то есть наибольшее и наименьшее значения $f$ совпадают, поэтому $f$ постоянна на $[a, n]$ и в качестве $c$ можно взять любую точку $(a, b)$. Если же $x_1$ или $x_2$ лежит в $(a, b)$, то, по теореме Ферма, $f'(x_1) = 0$ или $f'(x_2) = 0$; поэтому можно положить $c = x_1$ или $c = x_2$.

\Zam1. Геометрический смысл теормы Ролля: в условиях теоремы найдётся точка с горизонтальной касательной.

\Zam2. Все условия теоремы Ролля существенны. $f(X) = x (x \in [0, 1)) f(1) = 0$ разрывна в точке 1. Функция $f(x) =\sqrt{|x|} (x \in [-1, 1])$ не имеет производной в точке 0. $f(x) = x$ принимает разные значения на концах, но остальным условиям удовлетворяет.

\Zam3. Из дифференцируемости $f$ следует ее непрерывность, поэтому заключение теоремы выролняется для дифференцируемых на $[a, b]$ функций. В теореме Ролля функции разрешается не иметь производной на концах. Так отрезок $f(x) = \sqrt{1 - x^2}$ не дифференцируема на концах отрезка $[-1, 1 ]$, но условиям теоремы удовлетворяет.

\Zam 4. Из теоремы Ролля следует, что между любыми двумя нулями дифференцируемой функции всегда лежит нольеё производной.
\skip
\subsection{Формулы Лагранжа и Коши, следствия}
(190)

\T \q Лагранжа. Пусть функция непрерывна на $[a, b]$ и дифференцируема на $(a, b)$. Тогда найдётся такая точка $c \in (a, b)$, что \F{$\frac{f(b) - f(a)}{b - a} = f'(c)$}.

\T \q Коши. Пусть функции $f, g$ непрерывны на $[a, b]$ и дифференцируемы на $(a, b)$. Тогда найдётся такая точка $c \in (a, b)$, что \F{$\frac{f(b) - f(a)}{g(b) - g(a)} = \frac{f'(c)}{g'(c)}$.}

\Zam1. Теорема Лагранжа~--- частный случай теоремы Коши, поэтому ее доказывать небудем, но она применяется часто, поэтому её выделили в отдельную теорему.

\D Заметим, что $g(a) \neq g(b)$, так как иначе по теореме Ролля нашлась бы точка $t \in (a, b)$, в которой $g'(t) = 0$. Положим $\varphi = f - Kg$, где $K = \frac{f(b) - f(a)}{g(b) - g(a)}$, чтобы $\varphi(a) = \varphi(b)$. Тогда $\varphi$ удовлетворяет условиям теоремы Ролля. Поэтому найдётся такая точка $c \in (a, b)$, что $\varphi'(c) = 0$, то есть $f'(c) = Kg'(c)$, что равносильно требуемому.

\Zam7. Пусть функция $f$непрерывна на $\ang ab$ и дифференцируема на $(a, b)$. Тогда для любых различных точек $x, x+ \Delta x$ из $\ang a, b$ найдётся такое $\theta \in (0, 1)$, что \F{$f(x +\Delta x) - f(x) = f'(x + \theta \Delta x)\Delta x$} Доказательство по теореме Лагранжа с концами $x, x + \Delta x$. Надо учесть, что $c$ между $x$ и $x + \Delta x$, то есть $\theta = \frac{c - x}{\Delta x} \in (0, 1)$.

\S1. \q оценка приращения функции. Пусть функция $f$непрерывна на $\ang ab$, дифференцируема на $(a, b)$, а число $M > 0$ такого, что $|f'(t)| \le M$ для всех $t \in (a, b)$. ТОгда для любых точек $x$ и $x + \Delta x$ из $\ang ab$ \F{$|f(x + \Delta x) - f(x)| \le M|\Delta x|$.} Другими словами, если производная функции ограничена, то приращение функции не более чем в $M$ раз превзойдет приращение аргумента.\\
Очевидно вытекает из замечания.

\S2. Функция,имеющая на $\ang ab$ ограниченную производную, равномерно непрерывна на $\ang ab$.

\D Пусть $M > 0$ таково, что $|f'(t)| \le M$ для всех $t \in \ang ab$. Возьмем $\eps > 0$ и положим $\delta = \frac \eps M$. Тогда, если $x, y \in \ang ab,\s |x-y| < \delta$,то последствию 1 \F{$|f(x) - f(y)| \le M|x-y| \le M\delta = \eps$,} что и доказывает равномерную непрерывность $f$.
\skip
\subsection{Правило Лопиталя раскрытия неопределeнностей вида $\frac{0}{0}$, примеры}
(194)

\T \q Правило Лопиталя для раскрытия неопределенностей вида $\frac 00$. Пусть $-\infty \le a < b \le +\infty$, функции $f, g$ дифференцируемы на $(a, b),\s g'(t) \neq 0$ для любого $t \in (a, b),\s \li{x}{a+} f(x) = \li{x}{a+} g(x) = 0$ и существует предел \F{$\li{x}{a+}\frac{f'(x)}{g'(x)} = A \in \ol{\R}$.} Тогда предел $\li{x}{a+} \frac{f(x)}{g(x)}$ также существует и равен $A$.

\D 1. Пусть $a \in \R$. Доопределим функции в точке $a$ нулем: $f(a) = g(a) = 0$. Тогда доопределенные функции $f, g$ будут непрерывны на $[a, b)$. Возбмем последовательность $\{x_n\}: x_n \in (a, b),\s x_n \to a$, и докажем, что $\frac{f(x)}{g(x)} \to A$. Функции $f$ и $g$ удовлетворяют условиям теоремы Коши на каждом отрезке $[a, x_n]$. Поэтому для любого $n \in \N$ найдётся такая точка $c_n \in (a, x_n)$, что \F{$\frac{f(x_n)}{g(x_n)} = \frac{f(x_n) - f(a)}{g(x_n) - g(a)} = \frac{f'(c_n)}{g'(c_n)}$.} По теореме о сжатой последовательности, $c_n \to a$. По определению правостороннего предела на языке последовательностей, $\frac{f'(c_n)}{g'(c_n)} \to A$, а тогда  в силу произволности $\{x_n\}$ и $\frac{f(x)}{g(x)} \xra[x \to a+]{} A$.

2. Пусть $a = \infty$. В силк локадьности предела можно считать, что $b < 0$. Положим $\varphi(t) - f(-\frac 1t),\s \psi(t) = g(-\frac 1t) (t \in (0, -\frac 1b))$. Тогда \F{$\varphi'(t) = \frac 1{t^2}f'(-\frac 1t),\quad \psi(t)=\frac 1{t^2}g'(-\frac 1t) \neq 0$,}
\F{$\li{t}{0+} \varphi(t) = \li x{-\infty},\quad \li t{0+} \psi(t) = \li x{-\infty} g(x) = 0$,} \F{$\li x{0+} \frac{\varphi'(t)}{\psi'(t)} = \li x{-\infty} \frac {f'(x)}{g'(x)} = A$.} По доказанному, \F{$\li x{-\infty} \frac{f(x)}{g(x)} = \li t{0+}\frac{\varphi(t)}{\psi(t)} = A$.}

\Zam1. Утверждение, аналогичное теореме справедливы и для левостороннего предела, а следовательно, и для двухстороннего предела

\Pr1. $\li x0\frac{\sin x}{x} = \frac{\cos x}1 = 1$
\skip
\subsection{Правило Лопиталя раскрытие неопределeнностей вида $\frac{\infty}{\infty}$}
(195)

\T \q Правило Лопиталя для неопределённостей вида $\frac \infty\infty$. Пусть $\infty \le a < b \le +\infty$, функции $f$ и $g$ диференцируемы на $(a, b),\s g'(t) \neq 0$ для любого $t \in (a, b,\s \li x{a+}g(x) = \infty)$ и существует предел \F{$\li x{a+}\frac{f'(x)}{g'(x)} = A \in \ol{\R}$.} Тогда предел $\li x{a+} \frac{f(x)}{g(x)}$ тоже существует и равен $A$.

\D 1. Пусть $A = 0$. Возьмём последовательность $\{x_n\}$ со свойствами: $x_n \in (a, b),\s x_n \to a$, и докажем, что $\frac{f(x_n)}{g(x_n)}\to 0$. Зафиксируем число $\sigma > 0$. по условию, найдется такое $y \in (a, b)$, что для любого $c \in (a, y)$ будет $g(c)\neq 0$ и $|\frac{f'(с)}{g'(c)}| < \sigma$. Начиная с некоторого номера $x_n \in (a, y)$, поэтому можно считать, что $x_n \in (a, y)$ для всех $n$. По теореме Коши, для любого $n$ найдётся такое $c_n \in (x_n, y)$, что
\F{$\frac{f(x_n)}{g(x_n)} = \frac{f(x_n) - f(y)}{g(x_n) - g(y)}\frac{g(x_n) - g(y)}{g(x_n)} + \frac{f(y)}{g(x_n)} = \frac{f'(c_n)}{g'(c_n)}(1 - \frac{g(y)}{g(x_n)}) + \frac{f(y)}{g(x_n)}$.} Учитывая, что $g(x_n) \to \infty$, находим \F{$|\frac{f(x_n)}{g(x_n)}| \le \sigma(1 + |\frac{g(y)}{g(x_n)}|) + |\frac{f(y)}{g(x_n)}| \xra[n \to \infty]{} \sigma$.} Поэтому $\ol{\lim} |\frac{f(x_n)}{g(x_n)}| \le \sigma$. Но так как $\sigma$ произвольно $\lim\frac{f(x_n)}{g(x_n)} = 0$.

2. Пусть $A \in \R$ произвольно. Положим $h = f - Ag$. Тогда \F{$\li x{a+} \frac{h'(x)}{g'(x)} = \li x{a+} (\frac{f'(x)}{g'(x)} - A) = 0$.} По доказанному, $\frac{h(x)}{g(x)} \xra[x\to a+]{} 0$, то есть $\frac{f(x)}{g(x)} \xra[x\to a+]{} A$.

3. Пусть $A = +\infty$ расматривается аналогично случаю $A = 0$. При этом всместо $|frac{f'(c)}{g'(c)}| < \sigma$ используется неравенство $\frac{f'(c)}{g'(c)} > M$ и доказывается, что $\ul{\lim} \frac{g(x_n)}{g(x_n)} \ge M$. Случай $A = -\infty$ разбирается переходом к функции $-f$.

\Zam1. Утверждение, аналогичное теореме справедливы и для левостороннего предела, а следовательно, и для двухстороннего предела

\Zam2. В теореме функци $f$ не предполагается бесконечно большой, хотя на практике правило Лопиталя обычно применяют при наличии неопределеностей.

\Zam3. В цчловиях правила Лопиталя существование предела отношения функций выводится из существования предела отношений их производных. Обратное неверно. Если $g(x) = x,\s f(x) = x + \sin x$, то предел на бесконечности 1, а отношение производных предела не имеет.

\Pr1. $\li x{+\infty} \frac{\ln x}{x^\alpha} = 0,\quad \alpha > 0$.\\
$\li x{+\infty} \frac{\ln x}{x^\alpha} = \li x{+\infty} \frac{1/x}{\alpha x^{\alpha - 1}} = \li x{+\infty} \frac 1{\alpha x^\alpha} = 0$.

\Pr2. При $a > 1$. $\li x{+\infty} \frac x{a^x} = \li x{+\infty} \frac 1{a^x \ln a} = 0$\\
Тогда $\li x{+\infty} \frac{x^k}{a^x} = \li x{+\infty} (\frac{x}{(a^{1/k})^x})^k = 0$.
\skip
\subsection{Теорема Дарбу, следствия}
Билет 63: Теорема Дарбу, следствия
(198)

\T \q Дарбу. Если функция $f$ дифференцируема на $[a, b]$, то для любого числа $C$, лежащего между $f'(a)$ и $f'(b)$, найдется такое $c \in (a, b)$, что $f'(x) = C$.

\D 1. Пусть сначала $f'(a)$ и $f'(b)$ разных знаков; докажем, что существует такое $c \in (a, b)$, что $f(c) = 0$. Для определенности будем считать, что $f'(a) < 0 < f'(b)$. Поскольку $f$ непрерывна на $[a, b]$, по теореме Вейерштрасса найдётся точка $c \in [a, b]$, для которой $f(c) =\dsl\min{x\in[a, b]} f(x)$. Если $c\in (a, b)$, то, по теореме Ферма, $f'(c) = 0$. Поэтому достаточно доказать, что $c \neq a$ и $c \neq b$. Если $c = a$, то есть функция принимает наименьшее значение на левом конце отрезка, то $\frac{f(x) - f(a)}{x - a} \ge 0$ при всех $x \in (a, b]$, а поэтому и $f'(a) \ge 0$, что противоречит условию. Аналогично доказывается, что $c\neq b$.

2. Рассмотрим теперь общий случай. Пусть для определености $f'(a) < C < f'(b)$. Положим $\varphi(x) = f(x) - Cx$. Тогда \F{$\varphi'(a) = f'(a) - C < 0  < f'(b) - c = \varphi'(b)$.} По доказанному, найдется такое $c \in (a, b)$, что $\varphi'(c) = 0$, то есть $f'(c) = C$.

\S1. Если функция $f$ дифференцируема на $\ang ab$, то $f'(\ang ab)$~--- промежуток.\\
При доказательнстве сослаться на лемму 1 параграфа 2 главы 3 о характерисике промежутков.

\S2. Производная дифференцируемой на промежутке функции не может иметь на нем разрывов первого рода.
\skip
\subsection{Вычисления старших производных: линейность, правило Лейбница, примеры}
(199)

Пусть $f: D \subset \R \to \R,\s D_1$~--- множество дифференцируемости $f$, $f': D_1 \to \R$. Каждая точка $x_0 \in D_1$ удовлетворяет следующему условию: существует такое $\delta > 0$, что $(x_0 - \delta, x_0 + \delta) \cap D$~--- невырожденный промежуток.\\
Далее следует назвать $f''$~--- второй производной и т.д.\\
Производная порядка $n$ фунуции $f$ обозначается $\pb fn$. $\pb f1 = f'$, производные высших порядков определяются по индукции.

\Op Пусть $n - 1 \in \N$, множество $D_{n - 1}$ и функция $\pb f{n-1}: D_{n - 1} \to \R$ уже определены. Обозначим, через $D_n$ множество всех точек $x_0 \in D_{n - 1}$, для которых существует такое $\delta > 0$, что \F{$(x_0 - \delta, x_0 + \delta) \cap D_{n - 1} = (x_0 - \delta, x_0 + \delta) \cap D$,} и $\pb f{n - 1}$ дифференцируема в точке $x_0$.  Если $x_0 \in D_n$, то $f$ называется дифференцируемой $n$ раз в точке $x_0$. Функция \F{$\pb fn = (\pb f{n - 1})'|_{D_n}: D_n \to \R$} называетя производной порядка $n$, или короче, $n$-ной производной функции $f$. Другими словами, \F{$\pb fn (x_0) = \li x{x_0} \frac{\pb f{n - 1}(x) - \pb f{n - 1}(x_0)}{x - x_0},\quad x_0 \in D_n$}. Под нулевой производной подразумевается сама функция $\pb f0 = f$. Односторонние производные высших порядков определяются равенствами \F{$\pb fn_+(x_0) = \pb {(f|_{D \cap [x_0, +\infty)})}n (x_0), \quad \pb fn_-(x_0) = \pb{(f|_{D \cap (-\infty, x_0]})}n (x_0)$.} Другими словами \F{$\pb fn_\pm(x_0) = \li x{x_0\pm} \frac{\pb f{n - 1}(x) - \pb f{n - 1}_\pm (x_0)}{x - x_0}$.}

\T \q Арифметические действия над старшими производными. Пусть $n \in \N$, функции $f, g: \ang ab \to\R$ дифференцируемы $n$ раз в точке $x \in \ang ab$. Тогда\\
1) при любых $\alpha, \beta \in \R$  функция $\alpha f + \beta g$ дифференцируема $n$ раз в точке $x$ и \F{$\pb{(\alpha f + \beta g)}n(x) = \alpha \pb fn(x) + \beta \pb gn (x)$;}
2) функция $fg$ дифференцируема $n$ раз в точке $x$ и \F{$\pb{(fg)}n(x) = \dsl\sum{k = 0}n C^k_n\pb fk(x) \pb g{n - k}(x)$.}

\D Первое утверждение очевидно по индукции. Докажем второе (правило Лейбница) по индукции. При $n = 1$ равенство известно. Пусть утверждение верно для всех номеров не больших $n$, докажем для $n + 1$. Опуская обозначение аргумента $x$, имеем.
\F{$\pb{(fg)}{n + 1} = (\dsl\sum{k = 0}n C^k_n\pb fk \pb g{n-k})' = \dsl\sum{k = 0}n C^k_n \pb f{k + 1}\pb g{n - k} + \dsl\sum{k = 0}n C^k_n\pb fk\pb g{n + 1 - k} = \pb f{n + 1}\pb g0 + \dsl\sum{k = 1}{n} (C^{k - 1}_n + C^k_n) \pb fk \pb g{n+ 1 - k} + \pb f0 \pb g{n + 1} = \dsl\sum{k = 0}{n + 1} C^k_{n + 1} \pb fk \pb g{n + 1 - k}$}

\Pr1. $\pb {(x^\alpha)}n = \alpha(\alpha - 1)\cdots(\alpha - n + 1)x^{\alpha - n}.$\\
При $n = 1$ равенство известно. Индукционный переход \F{$x^{\alpha - n} = (\alpha - n)x^{\alpha - n - a}$.}
При $\alpha = -1$ \F{$\pb {(\frac 1x)}n = \frac{(-1)^n n!}{x^{n + 1}}$}

\Pr2. $\pb{(\ln x)}n = \frac{(-1)^{n - 1} (n - 1)!}{x^n}$.\\
Так как $\pb{(\ln x)}n = \frac 1x$, этот пример вытекает из предыдущего.

\Pr3. $\pb{(a^x)}n = a^x\ln^n a,\quad a > 0$. В частности $\pb{(e^x)}n = e^x$.

\Pr4. $(\sin x)' = \cos x,\s (\sin x)'' = -\sin x,\s (\sin x)''' = -\cos x,\s (\sin x)'''' = \sin x$, далее последовательность повторяется с периодом четыре. По формулам приведения можно записать \F{$\pb{(\sin x)}n = \sin(x + \frac {n\pi}2)$.}

\Pr 4. Аналогчно предыдущему примеру \F{$\pb{(\cos)}n = \cos(x + \frac{n\pi}2).$}
\skip
\subsection{Формула Тейлора с остаточным членом в форме Пеано}
(206)

\Op Пусть $n \in \N$ функция $f$ дифференцируема $n$ раз в точке $x_0$, или $n = 0$, а функция непрерывна точке $x_0$. Многочлен \F{$T_{n, x_0}f(x) = \dsl\sum{k = 0}n \frac{\pb fk (x_0)}{k!} (x - x_0)^k$} называется многочленом Тейлора порядка $n$ функции $f$ с центром в точке $x_0$. Разность \F{$R_{n, x_0}f(x) = f(X) - T_{n, x_0}f(x)$} называют остаточным членом или остатком формулы Тейлора, а равенство \F{$f(x) = \T_{n, x_0}f(x) + R_{n, x_0}f(x)$} --- формулой Тейлора.


\T \q Формула Тейлора -- Пеано. Пусть $n \in \N$, функция $f: \ang ab \to \R$ дифференцируема $n$ раз в точке $x_0 \in \ang ab$. Тогда \F{$f(x) = \dsl\sum{k = 0}n \frac{\pb fk (x_0)}{k!} (x - x_0)^k + o((x - x_0)^n),\quad x\to x_0$.}

\D Для краткости будем писать $T = T_{n, x_0} f,\s R = R_{n, x_0}f$. Тркбуется доказать, что $R(x) = o((x - x_0)^n)$. Поскольку $R = f - T$, а $\pb Tm (x_0) = \pb fm(x_0)$ при всех $m \in [0: n]$, имееи $\pb Rm(x_0) = 0$ при всех $m \in [0: n]$.\\
Поэтому достаточно доказать, что если $n \in \N$, функция $R$ дифференцируема $n$ раз в точке $x_0$ и $\pb Rm (x_0 = 0)$ при всех $m \in [0: n]$, то $R(x) = o((x - x_0)^n)$ при всех $x \to x_0$. Докажем по индукции по $n$.\\
База индукции $n = 1$. Так как $R(x_0) = R'(x_0) = 0$, по определению дифференцируемости получаем \F{$R(x) = R(x_0) + R'(x_0)(x- x_0) + o(x - x_0) = o(x - x_0) = o(x - x_0).\quad x\to x_0$.}
Индукционный переход: предположим, что для номера $n$ утверждение верно; докажем для номера $n + 1$. Пусть $\pb Rm(x_0) = 0$ при всех $m \in [0: n + 1]$ докажем, что \F{$\frac{R(x)}{(x - x_0)^{n + 1}} \xra[x \to x_0]{} 0.$} Доказательство будем вести на язвке последовательностей. Возбмём последовательность $\{x_\nu\}$ со свойствами $x_0 \in \ang ab,\s x_\nu \neq x_0,\s x_\nu \to x_0$. Тогда для каждого $\nu$ по формуле Лагранжа найдётся такая точка $c_\nu$, лежащая между $x_\nu$ и $x_0$, что \F{$\frac{R(x_\nu)}{(x_\nu - x_0) ^{n + 1}} = \frac{R(x_\nu) - R(x_0)}{(x_\nu - x_0)^{n + 1}} = \frac{R'(c_\nu)}{(x_\nu - x_0)^n}$.} Из неравенства $|x_\nu - x_0| < |x_\nu - x_0|$ следует,что $c_\nu \to x_0$. По индукционному предположению, применненному к функции $R'$ у которой все производные до $n$-ной включительно в точке $x_0$ равны 0, \F{$|\frac{R(x_\nu)}{(x_\nu - x_0)^{n + 1}}| \le |\frac{R'(x_\nu)}{(c_\nu - x_0)^n}| \to 0$,} что и требовалось доказать.
\skip
\subsection{Формула Тейлора с остаточным членом в форме Лагранжа}
(208)

\T \q Формула Тейлора -- Лагранжа. Пусть $n \in \Z_+,\s f \in \pb Cn \ang(ab),\s f$ дифференцируема $n + 1$ раз на $(a, b),\s x_0, x \in \ang ab,\s x \neq x_0$. Тогда существует такая точка $c$, лежащая между $x$ и $x_0$, что \F{$f(X) \dsl\sum{k = 1}n \frac{\pb fk (x_0)}{k!} (x - x_0)^k + \frac{\pb f{n + 1}(c)}{(n + 1)!}(x - x_0)^{n + 1}$.}

\D Обозначим через $\Delta$ интервал с конуами $x, x_0$, тогда $\ol{}\Delta$ обозначает отрезок с этими же концами. Положим $\psi(t) = (x - t)^{n + 1}$, \F{$\varphi(t) = f(x) - f(t) - \dsl\sum{k = 1}n \frac{\pb fk (t)}{k!}(x - t)^k, \quad t \in \ol{\Delta}$} Функции $\varphi$ и $\psi$ непрерывны на $\ol{\Delta}$ и дифференцируемы на $\Delta$, причем \F{$\psi'(t) = -(n + 1)(x - 1)^n \neq 0$} для любого $t \in \Delta$ найдем производную $\varphi$:
\F{$\varphi'(t) = -f'(t) - \dsl\sum{k = 1}n(\frac{\pb f{k + 1}(t)}{k!}(x-t)^k - \frac{\pb fk(t)}{k!}k(x - t)^{k - 1})=$}
\F{$= -f'(t) - \dsl\sum{k = 1}n\frac{\pb f{k + 1}(t)}{k!}(x - t)^k + \dsl\sum{k = 1}n \frac{\pb fk(t)}{(k - 1)!}(x - t)^{k - 1}=$}
\F{$-\dsl\sum{k = 0}n \frac{\pb f{k + 1}(t)}{k!}(x - t)^k + \dsl\sum{j = 0}{n - 1} \frac{\pb f{j + 1}(t)}{j!}(x - t)^j = -\frac{\pb f{n + 1}(t)}{n!}(x - t)^n$} Кроме того, $\varphi(x) = 0,\s \varphi(x_0) = R_{n, x_0}f(x),\s \psi(x) = 0,\s \psi(x_0) = (x - x_0)^{n + 1}$.\\
По теореме Коши о среднем, найдется такая точка $c \in \Delta$, что \F{$\frac{\varphi(x) -\varphi(x_0)}{\psi(x) - \psi(x_0)} = \frac{\varphi'(c)}{\psi'(c)}$.} Подставляя значения функций и производных, получаем
\F{$\frac{0 - R_{n, x_0}f(x)}{0 - (x - x_0)^{n + 1}} = \frac{-\pb f{n + 1}(x)(x - c)^n}{n!(n + 1)(x - c)^n}$,} что равносильно \F{$R_{n, x_0}f(x) = \frac{\pb f{n + 1}(c)}{(n + 1)!}(x - x_0)^{n + 1}$.}
\skip
\subsection{Тейлоровское разложение функций $e^x$, $\sin x$, $\cos x$, $\ln(1 + x)$, $(1 + x)^\alpha$}
(212-215)

$e^x = \ds\sum\limits^n_{k = 0} \frac{x^k}{k!} + o(x^n)$

$e^x = \ds\sum\limits^n_{k = 0} \frac{x^k}{k!} + \frac{e^{\theta x}}{(n + 1)!}x^{n + 1}$

$\sin x = \ds\sum\limits^n_{k = 0} \frac{(-1)^k}{(2k + 1)!}x^{2k + 1} + o(x^{2n + 2})$

$\sin x = \ds\sum\limits^n_{k = 0} \frac{(-1)^k}{(2k + 1)!}x^{2k + 1} + \frac{\sin(\theta x + \frac{(2n + 3)\pi}{2})}{(2n + 3)!}x^{(2n + 3)}$

$\cos x = \ds\sum\limits^n_{k = 0} \frac{(-1)^k}{(2k)!}x^{2k} + o(x^{2n + 1})$

$\cos x = \ds\sum\limits^n_{k = 0} \frac{(-1)^k}{(2k)!}x^{2k} + \frac{\cos(\theta x + \frac{(2n + 2)\pi}{2})}{(2n + 2)!}x^{(2n + 2)}$

$\ln(1 + x) = \ds\sum\limits^n_{k = 1}\frac{(-1) ^{k -1}}{k}x^k + o(x^n)$

$(1 + x)^{\alpha} = \ds\sum\limits^n_{k = 0} C^k_\alpha x^k + o(x^n)$

\D 1 Так как $\pb{(e^x)}k = e^x,\s \pb{(e^x)}k|_{x = 0} = 1$.

\D2 Из формулы \FF{\pb{\sin x}m = \sin(x + \frac \pi2) } при $k \in \Z_+$ находим \FF{\pb{\sin x}{2k}|_{x = 0} = 0,\quad \pb{(\sin(x))}{2k + 1}|_{x = 0} = (-1)^k. }

\D4 Поскольку при всех $k \in \N$ \FF{\pb{\ln(1 + x)}k = \frac{(-1)^{k - 1}(k - 1)!}{(1 + x)^k},\quad \pb{(\ln(1 + x))}k|_{x = 0} = (-1)^{k - 1}(k - 1)! ,} а $\ln 1 = 0$, получаем формулу.

\D5 Пусть $\alpha \in \R$. Положим \FF{C^k_n = \frac{\alpha(\alpha - 1)\cdot(\alpha - k + 1)}{k!},\quad k \in Z_+ .} Тогда поучаем формулу.
\skip
\subsection{Иррацианальность числа $e$}
(213)

\T \q Число $e$ иррационально.

\D Допустим противное $e = \frac mn,\s m, n \in \N$. Было доказано, что $2 < e < 3$. Поэтому $n \ge 2$, так как $e \notin \Z$. Умножим равенство на $n!$: \FF{(n - 1)!\cdot m = \dsl\sum{k = 0}n \frac{n!}{k!} + \frac{e^\theta}{n + 1},\quad \theta \in (0, 1).} Отсюда $\frac{e^\theta}{n + 1} \in \Z$, что абсурдно, так как $n + 1 \ge 3$, а $e^\theta < e < 3$.
\skip
\subsection{Применение формулы Тейлора к раскрытию неопределeнностей}
(216)

\T \q Применения формулы Тейлора для раскрытия неопределённостей. Пусть $f, g :\ang ab \to\R,\s x_0 \in \ang ab,\s n\ \in\N$,функции $f, g$ дифференцируемы $n$ раз в точке $x_0$: \FF{f(x_0) = f'(x_0) = \cdot = \pb f{n - 1}(x_0) = 0,} \FF{g(x_0) = g'(x_0) = \cdots = \pb g{n - 1}(x _0) = 0.} $\pb gn(x_0) \neq 0$. Тогда \FF{\li x{x_0} \frac{f(x)}{g(x)} = \frac{\pb fb(x_0)}{\pb gn (x_0)}.}

\D По формуле Тейлора -- Пеано при $x\to x_0$ \FF{f(x) = \frac{\pb fn (x_0)}{n!}(x - x_0)^n + o((x- x_0)^n),} \FF{g(x) = \frac{\pb gn(x_0)}{n!}(x -x_0)^n + o((x - x_0)^n) .} Поскольку $\pb gn(x_0) \neq 0$, существует такая окрестность $V_{x_0}$, что $g(x) \neq 0$ для любого $x \in \dot{V}_{x_0} \cap \ang ab$. Значит частное $\frac{f(x)}{g(x)}$ определено при всех таких $x$. Сокращая дробь на $\frac{(x - x_0)^n}{n!}$ получаем \FF{\frac{f(x)}{g(x)} = \frac{\pb fn(x_0) + o(1)}{\pb gn(x_0) + o(1)} \xra[x \to x_0]{} \frac{\pb fn(x_0)}{\pb gn(x_0)} }
\skip
\subsection{Критериий монотонности функции}
(217)

\T \q Критерий монотонности функции. Пусть функция $f$ непрерывна на $\ang ab$ и дифференцируема на $(a, b)$. Тогда $f$ возрастает(убывает) на $\ang ab$ в том и только том случае, когда $f'(x) \ge 0 (f'(x) \le 0)$ для всех $x \in \ang ab$.

\D 1. Необходимость. Пусть $f$ возрастает. Возьмем $x\in (a, b)$, тогда $f(y)\ge f(x)$ для всех $y \in (a, b)$, поэтому \FF{f'(x) = f'_+(x) = \li y{x+} \frac{f(y) - f(x)}{y - x} \ge 0 .}

2. Достаточность. Пусть $f'(X) \ge 0$ для всех $x \in \ang ab$ Возьмем $x_1, x_2 \in \ang ab: x_1 < x_2$ и докажем, что $f(x_1) \le f(x_2)$. По теореме Лагранжа существует такое $c \in (x_1, x_2)$, что \FF{f(x_2) - f(x_1) = f'(c)(x_2 - x_1) \ge 0}.
Случай убывающей функции рассматривается переходом к $-f$.

\S1. \q Критерий постоянства функции. Пусть $f: \ang ab\to \R$. Тогда $f$ постоянна тогда и только тогда, когда $f'(x) = 0$ при всех $x \in \ang ab$.

\D Известно, что производная постоянной функции равна 0. Обратно если $f \in C\ang ab$ и $f'(x) = 0$ для всех $x \in (a, b)$, то по теореме функция $f$ одновременно и возрастает и убывает, то есть постоянна.

\S2. \q Критерия строгой монотонности функции. Пусть $f$ непрерывна на $\ang ab$. Тогда $f$ строго возрастает на $\ang ab$ в том и только том слкучае, когда:\\
1) $f'(x) \ge 0$ для всех $x\in (a, b)$;\\
2) $f'$ не обращается в ноль тождественно ни на каком интервале.

\D По следствию 1, $f$ не постоянна ни на каком интервале. Поэтому из строго строгого возрастания $f$ вытекает утверждение 2. а утверждение 1 верно по теореме 1.\\
Пусть теперь выполнены утверждения 1 и 2. Из нетрицательности производной следует возрастание $f$. Если возрастание нестроое, то найдутся точки $x_1, x_2 \in \ang ab$, что $x_ 1 < x_2,\s f(x_1) = f(x_2)$. Тогда $f$ постоянна на $[x_1, x_2]$, что противоречит условию 2.

\Zam. Теорема и оба следствия обобщаются наситуацию, когда $f$ непрерывна на $\ang ab$, а дифференцируема на $\ang ab$ за исклбчением конечного множества точек.

\D Пусть $a_1,\cdots, a_n$~--- все те точки интервала $\ang ab$, в которых $f$ не дифференцируемма; $a_1 < \cdots < a_n$. Если $f$ возрастает на $\ang a, b$, то $f$ возрастает на каждом промежутке $\langle a, a_1],\s [a_1, a_2],\cdots,[a_n, b\rangle$. Тогда $f'\ge 0$ на каждом рпромежутке по теореме.\\
Обратно, если $f \in C\ang ab$ и $\s f' \ge 0$ на каждом промежутке, то $f$ возрастает на кждом из них и, следовательно, на $\ang ab$.
\skip
\subsection{Доказательство неравенства с помощью производной, примеры}
(219)

\T \q Доказательство неравенств с помощью производной. Пусть функции $a, g$ непрерывны на $[a, b\rangle$ и дифференцируема на $(a, b),\s f(a) \le g(x)$ и $f'(x) \le g'(x)$ для всех $x \in (a, b)$. Тогда $f(x) < g(x)$ для всех $x \in [a, b\rangle$.

\D Положим $h = g - f$. Тогда $h' = g' - f' \ge 0$ на $(a, b)$. По теореме функция $h$ возрастает на $[a, b\rangle$. следовательно для всех $x \in [a, b\rangle$ \FF{h(x) \ge h(a) = g(a) - f(a) \ge 0,} то есть $g(x) \ge f(x)$.

\Zam1. Аналогичное утвердение справедливо вместе сдоказательством и в случае, когда исходно значение функций сравниваются на правом конце.\\
Пусть функции $f, g$ непрерывны на $\langle a, b]$  и дифференцируема на $(a, b),\s f(b) \ge g(b)$ и $f'(x) \ge g'(x)$ для всех $x \in  (a, b)$. Тогда $f(x) \ge g(x)$ для всех $x \in \langle a, b]$.

\Zam2. Если в условиях теоремы будет $f'(x) < g'(x)$ для всех $x \in (a, b)$, то $f(x) < g(x)$ для всех $x \in (a, b)$.

\Pr1. Докажем, что $\cos x > 1 - \frac{x^2}2$ при всех $x\neq 0$.\\
В силу чётности обеих сторон, достаточно локазать при $x > 0$. Положим $f(x) = 1 - \frac{x^2}2,\s g(x) = \cos x$, тогда $f(0) = g(0) = 1$ и \FF{g'(x) = -\sin(x) > -x = f'(x) \textit{ при всех } x > 0.}

\Pr2. Докажем, что $\sin x > x = \frac{x^3}6$ при всех $x > 0$.\\
Положим $f(x) = x - \frac{x^3}6,\s g(x) = \sin x$. Тогда $f(0) = g(0) = 0$. По предыдущему примеру, \FF{g'(x) = \cos x > 1- \frac{x^2}2 = f'(x)\textit{ при всех }x > 0.}

\Pr 3. Докажем, что $\sin x > \frac 2\pi x$ при всех $x \in (0, \frac\pi2)$.\\
Положим $f(x) = \frac{\sin x}{x}$ при $x \neq 0,\s f(0) = 1$ \FF{f'(x) = \frac{x\cos x - \sin x}{x^2} = \frac{\cos x(x - \tg x)}{x^2} < 0.} Так как $x < \tg x$, а $\cos x$ и $x^2$ неотрицательны на заданном промежутке. Тогда $f$ строго убывает на промежутке, то есть $f(x) > f(\frac\pi2)$.
\skip
\subsection{Необходимое условие экстремума. Первое правило исследования критических точек}
(222)

\T \q Необходимое суловие экстремума. Пусть $f: \ang ab \to \R,\s x_0 \in (a, b)$~--- точка экстремума$f,\s f$ дифференцируема в точке $x_0$ Тогда $f'(x_0) = 0$.

\D По определению точки экстремума существует такое $\delta > 0$, что \FF{f(x_0) = \dsl\max{x \in (x_0 - \delta, x_0 + \delta)}{} f(x) \textit{ или } f(x_0) = \dsl\min{x \in (x_0 - \delta, x_0 + \delta)}f(x) .} Остается применить теорему Ферма к функции $f|_{(x_0 - \delta, x_0 + \delta)}.$

\Zam1. Как и в теореме Ферма, существенно, что $x_0$~--- внутренняя точка промежутка.

\Zam2. Условие $f'(x_0)$ неявляется достаточным $f(x) = x^3$.

\Zam3. Функция может быть не дифференцируемой в точке экстремума $f(x) = |x|$.

\T \q Первое правило исследования критических точек. Пусть $f: \ang ab \to \R,\s x_0 \in (a, b)$, функция $f$ непрерывна в точке $x_0$ и дифференцируема на $(a, b) \bsl \{x_0\}$ и существует такое $\delta > 0$, что $f'$ сохраняет знак на $(x_0 - \delta, x_0)$ и $(x_0, x_0 + \delta)$. Обозначим производные на этих промежутках $f'_1$ и $f'_2$ соответсвенно.\\
1. Если $f'_1 < 0$ и $f'_2 > 0$, то $x_0$~--- точка строгого минимума $f$.\\
2. Если $f'_1 > 0$ и $f'_2 < 0$, то $x_0$~--- точка строгого минимума $f$.\\
3. Если $f'_1 > 0$ и $f'_2 > 0$, то $x_0$~--- точка строгого возрастания $f$.\\
4. Если $f'_1 < 0$ и $f'_2 < 0$, то $x_0$~--- точка строгого убывания $f$.

\D Для определенности докажем 1 и 3.

1. Функция строго убывает на $(x_0 - \delta, x_0]$ и на $[x_0, x_0 + \delta)$. Поэтому $f(x) < f(x_0)$ как при всех $x\in (x_0 - \delta, x_0)$, так и при $x \in (x_0, x_0 + \delta)$. Значит $x_0$~--- Точка строгого минимума $f$.

2. Функция строго возрастает на $(x_0 - \delta, x_0]$ и на $[x_0, x_0 + \delta)$. Поэтому $f(x) < f(x_0)$ для всех $x \in (x_0 \ delta, x_0)$ и $f(x) > f(x_0)$ для всех $x \in (x_0, x_0 + \delta)$. То есть $x_0$~--- точка чтрогого возрастания.
\skip
\subsection{Второе правило исследования критических точек. Производная функции $e^{-1/x^2}$}
(224)

\T \q Второе проавило исследования критических точек. Пусть $f: \ang ab \to \R,\s x_0\in(a, b),\s n \in \N$, функция $f$ дифференцируема $n$ раз в точке $x_0$. \FF{f'(x_0) = \cdots = \pb f{n - 1}(x_0) = 0,\quad \pb fn(x_0) \neq 0}
1. Если $n$ четно и $\pb fn(x_0) > 0$, то $x_0$~--- точка строгого минимума $f$.\\
2. Если $n$ четно и $\pb fn(x_0) < 0$, то $x_0$~--- точка строгого максимума $f$.\\
3. Если $n$ нечетно и $\pb fn(x_0) > 0$, то $x_0$~--- точка строгого возрастания $f$.\\
4. Если $n$ нечетно и $\pb fn(x_0) < 0$, то $x_0$~--- точка строгого убывания $f$.

\D Запишем формулу Тейлора -- Пеано: \FF{f(x) = \dsl\sum{k = 0}n \frac{\pb fk(x_0)}{k!}(x - x_0)^k + o((x - x_0)^n).} Учитывая определение символа $o$ и обнуление производных $f$, перепишем это равенство в виде \FF{f(x) - f(x_0) = (x - x_0)^n(\frac{\pb fn(x_0)}{k!} + \varphi(x)),} где $\varphi(x) \xra[x \to x_0]{} 0$. Доопределим $\varphi(x_0) = 0$. Существует такое $\delta > 0$, что для всех $x \in (x_0 - \delta, x_0 + \delta)$ \FF{\sig(f(x) - f(x_0)) = \sig((x - x_0)^n\pb fn(x_0)).} Осталось сравнить знаки сомножителей.

\Zam. Может быть, что функция не постоянная, но $\forall n \in \N,\s \pb fn(x_0) = 0$. \FF{f(x) = \begin{cases}e^{-\frac 1{x^2}}, & x \neq 0,\\0, & x = 0.\end{cases}} \FF{f \in \pb C\infty (\R),\s \forall n \in \N \pb fn(0) = 0.}

\D Очевидно, что $f \in C(\R \bsl\{0\})$\\
1.Докажем, что $\pb fn = P_n(\frac 1x)e^{-1/x^2}$, где $P_n$~--- многочлен какой-то степени. База $n = 0,\s P_0 = 1$.\\
Переход: \FF{\pb f{n + 1}(x) = P'_n\lr{(\frac 1x})e^{-1/x^2}\lr{(-\frac 1{x^2}}) + P_n\lr{(\frac 1x})e^{-1/x^2}\lr{(\frac 2{x^3}})= P_{n + 1}\lr{(\frac 1x})e^{-1/x^2} } 

2. $\forall n \in \Z_+ \exists \pb fn(0) = 0?$\
База: $n = 0$, по заданию функции.\\
Переход: \FF{\pb f{n + 1}(0) = \li{x}{0} \frac{\pb fn (x) - \pb fn(0)}{x} = \li x0 \frac 1x P_n\lr{(\frac 1x})e^{-1/x^2} = 0 }
\skip
\subsection{Лемма о трeх хордах и односторонняя дифференцируемость выпуклой функции}
(228)

\L \q О трех хордах. Пусть функция $f$ выпукла вниз на $\ang ab,\s x_1, x_2, x_3 \in \ang ab,\s x_1 < x_2 < x_3$. Тогда \FF{\frac{f(x_2) - f(x_1)}{x_2 - x_1} \le \frac{f(x_3) - f(x_1)}{x_3 - x_1} \le \frac{f(x_3) - f(x_2)}{x_3 - x_2} }
\ig{threechords}

\D По определению выпуклости $f(x_2) \le tf(x_1) + (1 - t) f(x_3)$, где $t = \frac{x_3 - x_2}{x _3 - x_1},\s 1-t = \frac{x_2 - x_1}{x_3 - x_1}$. Преобразуем неравенство двумя способами. С одной стороны, \FF{f(x_2) \le f(x_1) + (1 - t)(f(x_3) - f(x_1)) = f(x_1) _ (x_2 - x_1)\frac{f(x_3) - f(x_2)}{x_3 - x_1} ,} что равносильно левому неравенству. С другой стороны, \FF{f(x_2) \le f(x_3) - t(f(x_3) - f(x_1)) = f(x_3) - (x_3 - x_2)\frac{f(x_3) - f(x_1)}{x_3 - x_1},} что равносильно правому неравенству.

\T \q Односторонняя дифференцируемость выпуклой функции. Пусть $f$ выпукла вниз на $\ang ab$. Тогда для любой точки $x \in \ang ab$ существуют конечные $f'_-(x), f'_+(x)$, причем $f'_-(x) \le f'_+(x)$.

\D Возьмем $x \in (a, b)$ и положим \FF{g(\xi) = \frac{f(\xi) - f(x)}{\xi - x},\quad \xi \in \ang ab \bsl\{x\}.} По лемме о трех хордах $g$ возрастает на $\ang ab\bsl\{x\}$. Поэтому, если $a < \xi < x < \eta < b$, то $g(\xi) \le g(\eta)$, то есть \FF{\frac{f(\xi) - f(x)}{\xi - x} \le \frac{f(\eta) - f(x)}{\eta - x}. } Следовательно, $g$ ограничена на $\langle a, x)$ сверху, а на $(x, b\rangle$ снизу. По теореме о пределе монотонной функции, существуют конечные пределы $g(x-)$ и $g(x+)$, которые, по определения, являются односторонними производными $f'_-(x)$ и $f'_+(x)$. Устремляя $\xi$ к $x$ слева, а $\eta$~--- справа, получаем, что $f'_-(x) \le f'_+(x)$.
\skip
\subsection{Выпуклость и касательные. Опорная прямая}
(231)

\T \q Выпуклость и касательные. Пусть функция $f$ дифференцируема на $\ang ab$. Тогда $f$ выпукла вниз на $\ang ab$ в том и только в том случае, когда график $f$ лежит не ниже любой своей касательной, то есть для любых $x, x_0 \in \ang ab$ \FF{f(x) \ge f(x_0) + f'(x_0)(x - x_0).}

\D 1. Необходимость. Пусть $f$ выпукла вниз, $x, x_0 \in \ang ab$. Если $x > x_0$, то, по лемме о трех хордах, для любого $\eta \in (x_0, x)$ \FF{\frac{f(\eta) - f(x_0)}{\eta - x_0} \le \frac{f(x) -f(x_0)}{x - x_0} .} Устремляя $\eta$ к $x_0$ справа, получаем неравенство \FF{f'(x_0) \le \frac{f(x) - f(x_0)}{x - x_0} ,} равносильное требуемому.\\
Если $x < x_0$, то, по лемме о трех хордах, для любого $\xi \in (x, x_0)$ \FF{\frac{f(\xi) - f(x_0)}{\xi - x_0} \ge \frac{f(x) - f(x_0)}{x - x_0}. } Устремляя $\xi$ к $x_0$ слева получаем неравенство \FF{f'(x_0) \ge \frac{f(x) - f(x_0)}{x - x_0},} равносильное требуемому (при домножении на $x - x_0 < 0$ меняется знак неравенства).

2. Достаточость. Пусть для любых $x, x_0 \in \ang ab$ верно неравенство. Возьмем $x_1, x_2 \in \ang ab: x_1 < x_2$ и $x \in (x_1, x_2)$. Применяя неравенство дважды: сначала к точкам $x_1, x$, а затем~--- к $x_2, x$, получаем \FF{f(x_1) \ge f(x) + f'(x)(x_1) - x,\qquad f(x_2) \ge f(x) + f'(x)(x_2 - x),} что равносильно \FF{\frac{f(x) - f(x_0)}{x -x_0} \le f'(x) \le \frac{f(x_2) -f(x)}{x_2 - x} .} Крайние части составляют неравенство из определения выпуклости.

\Op Пусть $f: \ang ab\to \R,\s x_0 \in \ang ab$. Прямая, задаваемая уравнением $y = l(x)$, называется опорной прямой для функции $f$ в точке $x_0$, если \FF{f(x_0) = l(x_0)\quad \textit{ и } \quad f(x) \ge l(x) \forall x \in \ang ab .} Если же \FF{f(x_0) = l(x_0) \quad \textit{ и } \quad f(x) > l(x) \forall x \in \ang ab \bsl\{x_0\},} то прямая называется строго опорной для функции $f$ в точке $x_0$.

\S2. пусть функция $f$ (строго) выпукла вниз на $\ang ab$. Тогда для любой точки $x_0 \in (a, b)$ существует (строго) опорная прямая функции $f$ в точке $x_0$.

\D По теореме, в каждой точке $x_0 \in (a, b)$ функция имеет односторонние касательные, а они в свою очередь являются (строго) опорнымы прямыми.
\skip
\subsection{Критерии выпуклости функции}
(234)

\T \q Дифференциальные критерии выпуклости. 1. Пусть функция $f$ непрерывна на $\ang ab$ и дифференцируема на $(a, b)$. Тогда $f$ (строго) выпукла вниз на $\ang ab$ в том и только том случае, когда $f'$ (строго) возрастает на $(a, b)$.\\
2. Пусть функция $f$ непрерывна на $\ang ab$ и дважды дифференцируемана $\ang ab$. Тогда $f$ выпукла вниз на $\ang ab$ в том и только том случае, когда $f''(x) \ge 0$ для всех $x \in (a, b)$.

\D 1. Необходимость. Возьмем $x_1, x_2 \in (a, b): x_1 < x_2$. По теореме о выпуклости и касательных \FF{f'(x_1) \le \frac{f(x_2) - f(x_1)}{x_2 - x_1} \le f'(x_2), } что и означает возрастание $f$.

Возьмем $x_1, x_2 \in \ang ab: x_1 < x_2$ и $x \in (x_1, x_2)$. По теореме Лагранжа, существуют такие $c_1 \in (x_1, x)$ и $c_2 \in (x, x_2)$, что \FF{\frac{f(x) - f(x_ 1)}{x -x_1} = f'(c_1),\quad \frac{f(x_2) - f(x)}{x_2 - x} = f'(c_2) .} Тогда $x_1 < c_1 < x < c_2 < x_2$, а $f$, по условию, возрастает, поэтому $f'(c_1) < f'(c_2)$, то есть \FF{\frac{f(x) - f(x_1)}{x - x_1} \le \frac{f(x_ 2) - f(x)}{x_2 - x} ,} что равносильно неравенству из определения выпуклости.

2. По пункту 1, выпуклость $f$ равносильна возрастанию $f'$, которое, по критерию монотонности, равносильно неотрицательности $f''$.
\skip
\subsection{Неравенство Йенсена}
(238)

\T \q Неравенство Йенсена. Пусть функция $f$ выпукла вниз на $\ang ab,\s n \in \N$. Тогда для любых $x_1,\cdots, x_n \in \ang ab$ и $p_1, \cdots, p_n > 0$ \FF{f\lr{(\frac{\dsl\sum{k = 1}n p_kx_k}{\dsl\sum{k = 1}n p_k}}) \le \frac{\dsl\sum{k = 1}n p_k f(x_k)}{\dsl\sum{k = 1}n p_k}. }

\Zam1. Числа $p_k$ называются весами, а отношение $\frac{\dsl\sum{k = 1}n p_k x_k}{\dsl\sum{k = 1}n p_k} $~--- взвешенным средним (арифметическим) чисел $x_1, \cdots, x_n$. Неравенство Йенсена можно сформулировать так: значение выпуклой вниз функции от взвешенного среднего не превосходит взвешенного среднего значений функций.

\Zam2. Не уменьшая общности можно считать что $\dsl\sum{k = 1}n p_k =1$. При этом условие неравенства Йенсена принимают вид \FF{f\lr{(\dsl\sum{k = 1}n p_k x_k}) \le \dsl\sum{k = 1}n p_k f(x_k) .}

\D Пусть сумма $p_k$ равна 1. Положим \FF{x^* = \dsl\sum{k = 1}n p_k x_k.} Сразу отметим, что если $x_1 = \cdots = x_n$, то $x^*$ с ними совпадает, а неравенство Йенсена обращается в равенство.\\
Пусть среди $x_1,\cdots, x_n$ есть различные. Проверим, что $x^* \in (a, b)$. Действительно, хоть одно из чисел $x_k$ меньше $b$, поэтому \FF{x^* < \dsl\sum{k = 1}n p_k b = b.} Аналогично, $a < x^*$.\\

В точке $x^*$ у функции $f$ существует опорная прямая; пусть она задается уравнением $l(x) = \alpha(x) + \beta$. По определению опорной прямой $l(x^*) = f(x^*)$ и $l(x_k) \le f(x_k)$ при всех $k$. Поэтому, \FF{f(x^*) = l(x^*) = \alpha\dsl\sum{k = 1}n p_k x_k + \beta = \dsl\sum{k = 1}n p_k(\alpha x_k + \beta) = \dsl\sum{k = 1}n p_k l(x_k) \le \dsl\sum{k = 1}n p_k f(x_k) }

\Zam3. Если $f$ строго выпукла, а среди $x_k$ есть различные, то неравенство Йенсена строгое.

\Zam. При $n = 2$ неравенство Йенсена совпадает с неравенством из определения выпуклости.
\skip
\subsection{Неравенства Юнга и Гeльдера}
(240)

\Op Числа $p, q \in (1, +\infty)$, связанные соотношением $\frac 1p + \frac 1q = 1$, называются сопряженными показателями.\\
Ясно, что $q = \frac{p}{p - 1},\s p = \frac q{q - 1}$.

\L \q Неравенство Юнга. Пусть $p, q$~--- сопряженные показатели, $a, b \in [0, +\infty)$, тогда $ab \le \frac{a^p}p + \frac{b^q}{q}$.

\D Если $a = 0$ или $b = 0$ очевидно. По неравенству Йенсена для $f(x) = \ln x$ (функция выпукла вверх) \FF{\ln\lr{(\frac {a^p}p + \frac{b^q}q }) \ge \frac 1p \ln(a^p) + \frac 1q \ln(a^q) = \ln a + \ln b .} При возведеениии $e$ в степень обеих честей неравенства, получается требуемое

\T \q Неравенство Гёльдера. Пусть $a, b \in\R^n$ или $\C^n,\s p > 1,\s \frac 1p + \frac 1q = 1$. Тогда \FF{\lr{|\dsl\sum{k = 1}n a_k b_k}| \le \lr{(\dsl\sum{k = 1}n |a_k|^p})^{1/p} \lr{(\dsl\sum{k = 1}n |b_k|^q})^{1/q} .}

\D Так как \FF{\lr{|\dsl\sum{k = 1}n a_k b_k}| \le \dsl\sum{k = 1}n |a_k b_k|,} достаточно доказать неравенство Гёльдера для чисел $|a_k|, |b_k|$. Поэтому, не уменьшая общности, иожно считать, что $a_k, b_k \in \R_+$. Более того, можно считать, что все $b_k > 0$. Действительно, если неравенство доказано для положительных чисел $b_k$, то доказано и неравенство, так как сумма $a_k b_k$ не изменится, сумма $a^q_k$ уселичиться, сумма $b^q_k$ не изменится (при добавелении пар ($a_k, b_k$), где $b_k = 0$).
Функция $f(x) = x^p$ строго выпукла вниз на $[0, +\infty)$. Положим $p_k = b^q_k,\s x_k = a_k b^{1-q}_k$ и применим неравенство Йенсена: \FF{\lr{( \frac{\dsl\sum{k = 1}n p_k x_k}{\dsl\sum{k = 1}n p_k}})^p \le \frac{\dsl\sum{k = 1}n p_k x^p_k}{\dsl\sum{k = 1}n p_k}. } Учитывая, что \FF{p_k x_k = a_k b_k,\quad p_k x^p_k = b^q_k a^q_k b^{p(1-q)}_k = a^p_k,} Получаем \FF{\lr{(\dsl\sum{k = 1}n a_k b_k})^p \le \lr{(\dsl\sum{k = 1}n a^p_k})\lr{(\dsl\sum{k = 1}n b^q_k})^{p - 1}.} Остается возвести обе части неравенства в степень $\frac 1p$ и воспользоваться тем, что $1 - \frac 1p = \frac 1q$.
\skip
\subsection{Неравенство Минковского и неравенство Коши между средними}
(243)

\T \q Неравенство Минковского. Пусть $a, b \in \R^n(\C^n),\s p \ge 1$. Тогда \FF{\lr{(\dsl\sum{k = 1}n |a_k + b_k| })^{1/p} \le \lr{(\dsl\sum{k = 1}n |a_k|^p })^{1/p} + \lr{(\dsl\sum{k = 1}n |b_k|^p })^{1/p} .}

\D При $p = 1$ неравенство Минковского сводится к неравенству треугольника для модуля. Пусть $p > 1,\s q = \frac p{p - 1}$. Обозначим $C = \sum\limits^n_{k = 1}|a_k + b_k|^p$. Применим неравенство треугольника, а затем неравенство Гёльдера:
\FF{C = \dsl\sum{k = 1}n |a_k + b_k||a_k + b_k|^{p - 1} \le \dsl\sum{k = 1}n |a_k||a_k + b_k|^ {p - 1} + \dsl\sum{k = 1}n |b_k||a_k + b_k|^{p - 1}|\le} \FF{\le\lr{(\dsl\sum{k = 1}n |a_k|^p })^{1/p} \lr{(\dsl\sum{k = 1} n |a_k + b_k|^{(p-1)q} })^{1/q} +}\FF{+ \lr{(\dsl\sum{k = 1}n |b_k|^p })^{1/p} \lr{(\dsl\sum{k = 1} n |a_k + b_k|^{(p-1)q} })^{1/q}=}
\FF{= \lr{\{\lr{( \dsl\sum{k = 1}n |a_k|^p})^{1/p} + \lr{(\dsl\sum{k = 1}n |b_k|^{1/p} }) }\} C^{1/q} .} Если $C = 0$ неравенство очевидно, иначе сократим на $C^{1/q}$.

\T \q Неравенство Коши о средних. Пусть $a_1, \cdots, a_n \ge 0$, тогда \FF{\sqrt[n]{a_1 a_2\cdots a_n} \le \frac{a_1 + a_2 + \cdots + a_n} n}.

\D Если какой-то $a_k = 0$, то неравенство очевидно.\\
Иначе по неравенству Йенсена для $f(x) = \ln x$ (выпуклая вверх) \FF{\ln \lr{(\frac{a_1 + \cdots + a_n}n }) \ge \frac{\ln a_1 + \ln a_2 +\cdots + \ln(a_n)}n.} При возведение числа $e$ в степень обоих частей неравенства, получаем требуемое неравенство.
\skip
\subsection{Метод касательных}
Билет 80: Метод касательных
(-)

\T \q Приближенное решение уравенений методом касательных (Ньютона). $f(x) = 0,\s f \in \pb C2 [a, b],\s f', f''$ строго сохраняют знак, $f(a)f(b) < 0$. $\iseq xn1$~--- последовательность точек таких, что \FF{l_n(x): y = \frac{f(x_{n - 1})}{x_{n - 1} - x_n}(x - x_{n - 1}) + f(x_{n - 1})} это касательная к функции $f$ в точке $x_n$. Причем, если \FF{\sig(f') = \sig(f''),} то $x_0 = b$, иначе $x_0 = a$. Тогда $x_n \to \xi: f(\xi) = 0$, причем $|\xi - x_{n + 1}| < $ ДОПИСАТЬ.

\D Разберем случай $f' > 0,\s f'' >0$\\
Так как $l(x)$~--- касательная в точке $x_n$, $x_{n + 1} \in (\xi, x_n)$, тогда $\{x_n\}$ убывает и ограничена снизу, значит имеет предел $\beta$.  $x_{n + 1} = x_n - \frac{f(x_n)}{f'(x_n)}$, тогда $\beta = \beta - \frac{f(\beta)}{f'(\beta)}$, тогда $f(\beta) = 0$, то есть $\beta = \xi$.

$x_{n + 1} - \xi = x_n - \xi - \frac{f(x_n)}{f'(x_n)}$. \FF{0 = f(\xi) = f(x_n) + f'(x_n)(\xi - x_n) + \frac{f''(c)}2(\xi - x_n)^2,\s \xi < c < x_n .} Получаем \FF{-\frac{f(x_n)}{f'(x_n)} + x_n - \xi = \frac{f''(c)}{2f'(x_n)}(x_n - \xi)^2 \lra} \FF{\lra x_{n + 1} - \xi = \frac{f''(c)}{2f'(x_n)}(x_n - \xi)^2 \lra |x_{n + 1} - \xi| = \lr{|\frac{f''(c)}{2f'(x_n)}}|(x_n - \xi)^2 \le} \FF{\le \lr{|\frac{f'(c)}{f''(c)}(x_n - \xi)^2 }| \le \frac 12 \dsl\max{(a, b)}{}\lr{(\frac{f'}{f''} }) (x - \xi)^2}
\skip
\newpage
\section{Требуемые определения}
\subsection{Инъекция}
(31)\\
Если отображение $f: X \to Y \lra \forall x_1, x_2 \in X\ x_1\neq x_2 \Ra f(x_1)\neq f(x2)$, то оно называется инъекцией, инъективным, обратимым, то есть при любом $y \in Y\ f(x)=y$ имеет не более одного решения
\skip
\subsection{Сюръекция}
(31)\\
Если у отображение $f: X \to Y$ $f(X) = Y$, то отображение называется сюръективным, или сюръекцией, или отображением "на". То есть $\forall y \in Y\ f(x) = y$ имеет хотя бы одно решение.
\skip
\subsection{Биекция}
(32)\\
Если отображение $f: X \to Y$ одновременно и сюръективно и инъективно, то его называют биекцией или взаимно однозначным отоборажением (соответствием). То есть $\forall y \in Y\ \exists!\ x\in X:\ f(x)=y$.
\skip
\subsection{Образ}
(30)\\
Пусть $f: X \to Y$, $A \subset X$. Множество $f(a) = \{y \in Y: \exists x \in A\ f(x) = y\}$ называется образом множества $A$ при отображении $f$. $f(x)$~--- образ множества, $X$~--- множество значений отображения $f$.
\skip
\subsection{Прообраз}
(31)\\
Пусть $f: X \to Y$, $B \subset Y$. Множество $f^{-1}(B) = \{x\in X:\ f(x) \in B\}$ называется прообразом множества $B$ при отображении $f$.
\skip
\subsection{Обратное отображение}
(33)\\
Пусть $f: X \to Y$, $f$ обратимо. Отображение , которое каждому $y$ из множества $f(X)$ сопоставляет то(единственное) значение $x$ из $X$, для которого $f(x) = y$ называется обратным к $f$ и обозначается $f^{-1}$. $f^{-1}: f(X) \to X$. Очевидно, что $f^{-1}$~--- биекция между $f(X)$ и $X$. График обратной функции симметричен относительно $y=x$ графику обычной функции.
\skip
\subsection{Предел последовательности}
(43)\\
Пусть $\{x_n\}^{\infty}_{n = 1}$~--- последовательность вещественных чисел. Число $a \in \R$ называют предел последовательности $\{x_n\}$ и пишут $\lim\limits_{n\to \infty} x_n = a$ или $x_n \xra[n \to \infty]{} a$. Если $\forall \varepsilon > 0 \, \exists N \in \N\,: \forall n \in \N,\, n > N\ |x_n - a| < \varepsilon$. Если у последовательности есть предел ~--- её называют сходящейся, иначе~--- расходящейся.
\skip
\subsection{Предел функции}
(100)\\
Пусть $f: D \subset \R \to \R$, $a \in \R$~--- предельная точка $D$, $A \in \R$. Число $A$ называют пределом функции $f$ в точке $a$ и пишут $\lim\limits_{x\to a} f(x) = A$ или $f(x) \xra[x\to a]{} A$, если $\forall\varepsilon > 0\ \exists\delta > 0\ \forall x \in D: x \neq a, [x - a] < \delta \Ra |f(x) - A| < \varepsilon$.
\skip
\subsection{Предел отображения}
(99)\\
Пусть $(X, \rho_X)$ и $(Y, \rho_y)$~--- метрические пространства, $f: D \subset X \to Y$, $a \in X$~--- предельная точка $D$, $A \in Y$. Точку $A$ называют пределомотображения $f$ в точке $a$ и пишут $\lim\limits_{x\to a}f(x) = A$ или $f(x)\xra[x\to a]{}, A$, если выполняется одно из следующих условий:\\
\q1. \qОпределение на $\varepsilon$-языке (по Коши).\\
$\forall \varepsilon > 0\ \exists\delta > 0\ \forall x \in D \bsl \{a\}:\ \rho_X(x, a) < \delta \Ra \rho_Y(f(x), A) < \varepsilon$.\\
\q2. \qОпределение на языке окрестностей.\\
$\forall V_A\ \exists V_a\ f(\dot{V_a} \cap D) \subset V_A$\\
Для любой окрестности $V_A$ точки $A$ существует такая окрестность $V_a$ точки $a$, что образ пересечения проколотой окрестности $\dot{V_a}$ с множеством $D$ при отображении $f$ содержится в окрестности $V_A$.\\
$\forall V_a\ \exists V_a \forall x \in \dot{V_a} \cap D \Ra f(x) \in V_A$. Очевидно, что это~--- переформулировка исходного утверждения.\\
\q3. \qОпределение нв языке последовательностей (по Гейне).\\
$\forall \{x_n\}: x_n \in D \bsl \{a\},\ x_n \to a\Ra(x_n) \to A$
\skip
\subsection{Метрическое пространство}
(46)\\
Функция $\rho: X \times X \to \R_+$ называется метрикой или расстоянием в множестве $X$, если она удовлетворяет следующим условиям:\\
\q1. $\rho(x, y) = 0 \lra x = y \qquad x, y \in X$\\
\q2. $\rho(x, y) = \rho(y, x) \qquad x, y \in X$\\
\q3. $\rho(x, z) \le \rho(x, y) + \rho(y, z) \quad x, y, z \in X$\\
Пара $(X, \rho)$~--- множество с метрикой в нём~--- называется  метрическим пространством.
\skip
\subsection{Векторное пространство}
(53)\\
Пусть $K$~--- поле, $X$~--- множество, и над элементами $X, K$ определены две операции: сложение $X \times X \xra[]{+} X$ и умножение $K \times X \xra[]{} X$, удовлетворяющие следующим условиям:
\\ $x, y, z \in X$, $\lambda, \mu \in K$\\
\q1. $(x + y) + z = x + (y + z)$\\
\q2. $x + y = y + x$\\
\q3. $\exists \theta \in X:\ 0\cdot x = \theta$\\
\q4. $(\lambda + \mu)x = \lambda x + \mu x$\\
\q5. $\lambda(x + y) = \lambda x + \lambda y$\\
\q6. $(\lambda\mu)\cdot x= \lambda \cdot (\mu x)$\\
\q7. $\exists1 \in K:\, 1 \cdot x = x$\\
Тогда $X$ называюттлинейным пространством или линейным множеством над полем $K$. Элементы $X$ называют векторами, элементы $K$~--- скалярами.
\skip
\subsection{Нормированое пространство}
(54)\\
Пусть $X$~--- векторное пространство над $\R$ или $\C$. Нормой в $X$ называют функцию $p: X\to\R_+$, удовлетворяющаяя следующим условиям.\\
\q1.\q Положительная определённость.\\
$p(x) = 0 \lra x = \theta$\\
\q2.\q Положительная однородность.\\
$p(\lambda x) = |\lambda|p(x)$\\
\q3.\q неравенство треугольника (полуаддитивность).\\
$p(x + y) \le p(x) + p(y)$\\
Принято обозначать норму $p(x) = ||x||$.\\
Пара $(X, ||\cdot||)$ называется нормированным протсранством.
\skip
\subsection{Неравенство Коши-Буняковского}
(59)\\
$|\langle x, y \rangle|^2 \le \langle x, x\rangle\langle y, y\rangle$\\
Если рассматривать норму, порождённую скалярным произведением, то верно: $|\langle x, y \rangle| \le ||x||\cdot||y||$, так как $||x|| = \sqrt{\langle x, x\rangle}$.
\skip
\subsection{Внутренние точки}
(68)\\
Точка $a$ называется внутренней точкой множества $D$, если существует окрестность точки $a$, содержащаяся в $D$\\
Множество называется открытым, если все его точки внутренние.
\skip
\subsection{Предельные точки}
(70)\\
Точка $a$ называется предельной точкой или точкой сгущения множества $D$, если в любой проколотой окрестности точки $a$ найдётся точка множества $D$.\\
$a$~--- предельная точка $D\, \lra \,\forall \dot{V_a}:\, \dot{V_a} \cap D \neq \varnothing$
\skip
\subsection{Открытые множества}
(68)\\
Множество называется открытым, если все его точки внутренние (содержатся в множестве с некоторой своей окрестностью).
\skip
\subsection{Замкнутые множества}
(71)\\
Множество $D$ называется замкнутым (в $X$), если содержит все свои предельные точки.
\skip
\subsection{Компактные множества}
(77)\\
Подмножество $K$ метрического пространства $X$ называется компактным, если из любого открытого покрытия $K$ можно извлечь конечно подпокрытие.\\
$\forall \{G_n\}_{\alpha \in A}:\ K \subset \bigcup\limits_{\alpha \in A} G_{\alpha} \Ra \exists \alpha_1, \cdots, \alpha_N \in A\ K \subset \bigcup\limits^N_{i = 1} G_{\alpha_i}$ ($G_{\alpha}$~--- открытые множества)
\skip
\subsection{Компактность в евклидовом пространстве}
(82)\\
Пусть $K \subset \R^m$, тогда следующие утверждения равносильны:\\
\q1. $K$ замкнуто и ограничено\\
\q2. $K$ компактно\\
\q3. Из всякой последовательности точек $K$ можно извлечь бесконечную подпоследовательность, имеющую предел, принадлежащий $K$.
\skip
\subsection{Сходимость в себе}
(84)\\
Пусть $\{x_n\}^\infty_{n = 1}$~--- последовательность в метрическом пространстве $X$. Говорят, что последовательность сходится в себе, если для любого положительного чисда $\varepsilon$ существует такой номер $N$, что для всех номеров $n$ и $l$, больших $N$ выполняется равенство $\rho(x_n, x_l) < \varepsilon$.\\
$\forall \varepsilon > 0\ \exists N\ \forall n, l > N\ \rho(x_n, x_l) < \varepsilon$
\skip
\subsection{Полнота метрического пространства}
(85)\\
Если в метрическом пространстве $X$ любая сходящаяся в себе последовательность сходится, то пространство $X$ называют полным.
\skip
\subsection{Ограниченность множества}
(50)\\
Подмножество $D$ метрического пространства $X$ называется ограниченным, если оно содкржится в некотором шаре:\\
$\exists a \in X, R > 0\ D \in \overline{B}(a, R)$\\
Последовательность $\{x_n\}$ в метричскои пространстве $X$ называется ограниченной, если множество её значений ограничено:\\
$\exists a \in X, R > 0\ \forall n \in \N\ \rho(x_n, a) \le R$\\
Открытый шар или закрытый (увеличим в два раза радиус) и в какой точке его центр (увеличим радиус на расстояние между старым и новым центрами) не имеет значения.
\skip
\subsection{Точные границы}
(87)\\
Пусть $E \subset \R$, $E \neq \varnothing$ ограничено снизу. Наибольшая из нижних границ множества $E$ называется точной нижней границей или нижней гранью и обозначается $\inf E$ (аналогично с точной верхей границей~--- $\sup E$):\\
$b = \sup E \lra 
\begin{cases}
\forall x \in E & x \le b\\
\forall\varepsilon > 0\ \exists x \in E: & x > b - \varepsilon;	
\end{cases}$\\
$a = \inf E \lra
\begin{cases}
\forall x \in E & x \ge a\\
\forall\varepsilon > 0\ \exists x \in E: & x < a + \varepsilon;	
\end{cases}$

\skip
\subsection{$\O$ символика}
(159)\\
Пусть $X$~--- метрической пространство, $D \subset X$, $f, g: D \to \R$ или $\C$, $x_0$~--- предельная точка $D$. Если существуют функция $\varphi: D \to \R$ или $\C$ и окрестность $V_{x_0}$ точки $x_0$, такие что $f(x) = \varphi(x)g(x)$ для всех $x \in V_{x_0} \cap D$ и\\
\q1. $\varphi$ ограничена на $V_{x_0} \cap D$, то говрят, что функции $f$ ограничена по сравнению с $g$ при $x \to x_0$, и пишут $f(x) = \O(g(x)) \qquad x \to x_0$.\\
\q2. $\varphi \xra[x\to x_0]{} 0$, то говорят, что функция $f$~--- бесконечно малая по сравнению с $g$ при $x \to x_0$, и пишут $f(x) = o(g(x)) \qquad x \to x_0$.\\
\q3. $\varphi(x) \xra[x\to x_0]{} 1$, то говорят, что функции $f$ и $g$ эквивалентны или асимптотически равны при $x \to x_0$, и пишут $f(x) \sim g(x) \qquad x \to x_0$.
\skip
\subsection{Непрерывность}
(114)\\
Пусть $(X, \rho_X)$ и $(Y, \rho_Y)$~--- метрические пространства, $f: D\subset X \to Y$, $x_0 \in D$. Отображение $f$ назывется непрерывным в точке $x_0$ если выполняется одно из следующих утвержений.\\
\q1. Предел отображения $f$ в точке $x_0$ существует и равен $f(x_0)$. (Применимо, если $x_0$~--- предельная точка D).\\
\q2. \q На $\varepsilon$-языке или по Коши.\\
$\forall \varepsilon > 0\ \exists \delta > 0\ \forall x \in D:\ \rho_X(x, x_0) < \delta \Ra \rho_Y(f(x), f(x_0)) < \varepsilon$\\
\q3. \q На языке окрестностей.\\
$\forall V_{f(x_0)}\ \exists V_{x_0}\ f(V_{x_0} \cap D) \subset V_{f(x_0)}$\\
\q4. \q На языке последовательностей или по Гейне.\\
$\forall \{x_n\}:\ x_n \in D, x_n \to x_0\ f(x_n) \to f(x_0)$\\
\q5. Бесконечно малому приращению аргумента соответствует бесконечно малое приращение отображения. $\Delta y \xra[\Delta x \to \theta_X]{} \theta_Y$

Отображение называется непрерывным на множестве $D$, если оно непрерывно в каждой точке множества $D$.\\
Множество отображений $f: D\subset X \to Y$ непрерывных на множестве $D$, обозначают $C(D\subset X\to Y)$ или $C(D\to Y)$
\skip
\subsection{Теоремы Больцано-Коши о непрерывных функциях}
(130, 133)\\
\qТеорема Больцано -- Коши о промежуточном значении.
Пусть функция $f$ непрерывна на $[a, b]$. Тогда для любого числа $C$, лежащего между $f(a)$ bи$f(b)$, найдётся такое $c \in (a, b)$, что $f(c) = C$\\
\qТеорема Больцано -- Коши о непрерывных отображениях.\\
Пусть $X, Y$~--- метрические пространства, $X$ линейно связно, $f \in C(X\to Y)$. Тогда $f(x)$ линейно связно.\\
Другими словами непрерывный образ линейно связного образа линейно связен.

\q Линейная связность.\\
$Y$~--- метрическое пространство, $E \subset Y$. Множество $E$ называется линейно связным, если любые две его точки можно соединить путём в $E$: $\forall A, B \in E\ \exists \gamma \in C([a, b] \subset \R \to E):\ \gamma(a) = A, \gamma = A, \gamma(b) = b$

\skip
\subsection{Теоремa Вейерштрасса о непрерываных отображениях}
(126)\\
Пусть $X$, $Y$~--- метрические пространства, $X$ компактно, $f \in C(X \to Y)$. Тогда $f(X)$ компактно. Другими словами: непрерывный образ компакта~--- компакт.\\
\q Первая теорема Вейерштрасса о непрерывных фунциях.\\
Функция, непрерывная на отрезке, ограничена.\\
\q Вторая теорема Вейерштрасса о непрерывных функциях.\\
Функция, непрерывная на отрезке, принимает свои наибольшее и наименьшее значения.

\skip
\subsection{Равномерная непрерывность}
(128)\\
Пусть $X, Y$~--- метрические пространства, $f: X \to Y$. Отображение называется равномерно непрерывным на $X$, если\\
$\forall \varepsilon > 0\ \exists \delta > 0\ \forall \ol{x}, \ol{\ol{x}} \in X:\ p_X(\ol{x}, \ol{\ol{x}}) < \delta\ p_Y{f(\ol{x}), f(\ol{\ol{x}})}$
\skip
\subsection{Теорема Кантора}
(129)\\
Непрерывное отображение на компакте равномерно непрерывно.\\
Для функций: непрерывная на отрезке функция равномерно непрерывна.
\skip
\subsection{Замечательные пределы}
(154)\\
\q1. $\lim\limits_{x\to 0} \frac{\sin x}{x} = 1$\\
\q2. $\lim\limits_{x\to\infty} (1 + \frac{1}{x})^x = e$\\
\q3. $\lim\limits_{x\to 0} \frac{\log_a(1 + x)}{x} = \frac{1}{\ln a}$\\
\q4. $\lim\limits_{x\to 0} \frac{(1 + x)^{\alpha} - 1}{x} = \alpha$\\
\q5. $\lim\limits_{x\to 0} \frac{a^x - 1}{x} = \ln a$\\
\skip
\subsection{Дифференцируемость и производная}
(169)

\q первое определение.\\
Пусть $f: \langle a, b\rangle \to \R$, $x_o \in \langle a, b\rangle$. Если существует такое числоа $A \in \R$, что\\
$f(x) = f(x_0) + A(x - x_0) + o(x - x_0) \quad x \to x_0$,\\
то функция называется дифференцируемой в точке $x_0$. При этом число $A$ называется производной функции в точке $x_0$.

\q второе определение.\\
Пусть $f: \langle a, b\rangle \to \R$, $x_o \in \langle a, b\rangle$.
Если существует предел\\
$\lim\limits_{x\to x_0} \frac{f(x) - f(x - x_0)}{x - x_0}$,\\
равный числу $A \in \R$, то функция $f$ называется дифференцируемой в точке $x_0$, а число $A$~--- её производной в точке $x_0$.
\skip
\subsection{Формулы дифференцирования}
(185)\\
\q1. $c' = 0$\\
\q2. $(x^{\alpha})' = \alpha x^{\alpha - 1}$\\
\q3. $(a^x)' = a^x \ln a$\\
\q4. $(\log_a x)' = \frac{1}{x\ln a}$\\
\q5. $(\sin x)' = \cos x$\\
\q6. $(\cos x)' = -\sin x$\\
\q7. $(\tg x)' = \frac{1}{\cos^2 x}$\\
\q8. $(\ctg x)' = -\frac{1}{\sin^2 x}$\\
\q9. $(\arcsin x)' = \frac{1}{\sqrt{1 - x^2}}$\\
\q10. $(\arccos x)' = -\frac{1}{\sqrt{1 - x^2}}$\\
\q11. $(\arctg x)' = \frac{1}{1 + x^2}$\\
\q12. $(\arcctg x)' = -\frac{1}{1 + x^2}$\\
\skip
\subsection{Правила дифференцирования}
(178)\\
$(f \pm g)'(x) = f'(x) \pm g'(x)$\\
$(fg)'(x) = f'(x)g(x) + g'(x)f(x)$\\
$(\alpha f)'(x) = \alpha f'(x)$\\
$(\frac{f}{g})'(x) = \frac{f'(x)g(x) - g'(x)f(x)}{g^2(x)}$\\
$(g \circ f)'(x) - g'(f(x))\cdot f'(x)$\\
$(f^{-1})'(f(x)) = \frac{1}{f'(x)}$\\
$y'_x = \frac{y'_t}{x'_t}$
\skip
\subsection{Формула Лагранжа}
(190)\\
Пусть функция $f$ непрерывна на $[a, b]$ и дифференцируема на $(a, b)$. Тогда найдётся такая точка $c \in (a, b)$, что\\
$\frac{f(b) - f(a)}{b - a} = f'(c)$
\skip
\subsection{Формула Тейлора с остатком в виде Пеано}
(206)\\
Пусть $n \in \N$, функция $f: \langle a, b\rangle \to \R$ дифференцируема $n$ раз в точке $x_0 \in \langle a, b\rangle$. Тогда\\
$f(x) = \ds\sum\limits^{n}_{k = 0} \frac{f^{(k)}(x_0)}{k!}(x - x_0)^k + o((x - x_0)^n) \qquad x\to x_0$
\skip
\subsection{Формула Тейлора с остатком в виде Лагранжа}
(208)\\
Пусть $n \in \Z_+, f \in C^{(n)}\langle a, b\rangle$, $f$ дифференцируема $n + 1$ раз на $(a, b)$, $x_0, x\in \langle a, b\rangle$, $x \neq x_0$. Тогда существует точка $c$, лежащая между $x$ и $x_0$, что\\
$f(x) = \ds\sum\limits^{n}_{k = 0} \frac{f^{(k)}(x_0)}{k!}(x - x_0)^k + \frac{f^{(n + 1)}(c)}{(n + 1)!}(x - x_0)^{n + 1}$
\skip
\subsection{Основные Тейлоровские разложения}
(212-215)

$e^x = \ds\sum\limits^n_{k = 0} \frac{x^k}{k!} + o(x^n)$

$e^x = \ds\sum\limits^n_{k = 0} \frac{x^k}{k!} + \frac{e^{\theta x}}{(n + 1)!}x^{n + 1}$

$\sin x = \ds\sum\limits^n_{k = 0} \frac{(-1)^k}{(2k + 1)!}x^{2k + 1} + o(x^{2n + 2})$

$\sin x = \ds\sum\limits^n_{k = 0} \frac{(-1)^k}{(2k + 1)!}x^{2k + 1} + \frac{\sin(\theta x + \frac{(2n + 3)\pi}{2})}{(2n + 3)!}x^{(2n + 3)}$

$\cos x = \ds\sum\limits^n_{k = 0} \frac{(-1)^k}{(2k)!}x^{2k} + o(x^{2n + 1})$

$\cos x = \ds\sum\limits^n_{k = 0} \frac{(-1)^k}{(2k)!}x^{2k} + \frac{\cos(\theta x + \frac{(2n + 2)\pi}{2})}{(2n + 2)!}x^{(2n + 2)}$

$\ln(1 + x) = \ds\sum\limits^n_{k = 1}\frac{(-1) ^{k -1}}{k}x^k + o(x^n)$

$(1 + x)^{\alpha} = \ds\sum\limits^n_{k = 0} C^k_\alpha x^k + o(x^n)$
\skip
\subsection{Сравнение логарифмической, степенной и показательной функций}
(196)\\
$\lim\limits_{x\to +\infty} \frac{\ln x}{x^\alpha} = 0 \qquad \forall\alpha > 0$\\
$\lim\limits_{x\to +\infty} \frac{x^k}{a^x} = 0 \qquad \forall a > 1, k \in \R$

\skip
\subsection{Точки экстремума и их отыскание}
(220)\\
Пусть $f: D \subset \R \to \R_+$, $x_0 \in D$. Если существует такое $\delta > 0$, что:
\begin{itemize}
	\item для любого $x \in (x_0 - \delta, x_0 + \delta) \cap D$ выполняется неравенство $f(x) \le f(x_0)$, то $x_0$ называется точкой максимума функции $f$;
	\item для любого $x \in (x_0 - \delta, x_0 + \delta) \cap (D\bsl \{x_0\})$ выполняется неравенство $f(x) < f(x_0)$, то $x_0$ называется точкой строгого максимума функции $f$.
\end{itemize} 

Если противоположные неравенства, то $x_0$ соответсвенно точка минимума и точка строгого минимума.

Необходимое условие экстремума\\
Пусть $f: \langle a, b\rangle \to \R$, $x_0\in (a, b)$~--- точка экстремума $f$. $f$ дифференцируема в точке $x_0$. Тогда $f'(x_0) = 0$ (Лемма Ферма).
\skip
\subsection{Определение выпуклости}
(226)\\
Функция $f: \langle a, b\rangle \to \R$ называется выпуклой вниз на $\langle a, b\rangle$, если для любых $x_1, x_2 \in \langle a, b\rangle $ и $t \in (0, 1)$ выполняется неравенство\\
$f(tx_1 + (1 - t)x_2) \le tf(x_1) + (1 - t)f(x_2)$\\
Если при $x_1 \neq x_2$ неравенство становится строгим, то функция называется строго выпуклой вниз\\
Если выполняются потивоположные неравенства, то функция выпукла вверх и строго выпуклая вверх соответсвенно.

\skip
\subsection{Критерий выпуклости}
(234)\\
\q1. Пусть функция f непрерывна на $\langle a, b\rangle$ и дифференцируема на $(a, b)$. Тогда $f$ (строго) выпукла вниз на $\langle a, b\rangle$ в том и только том случае, когда $f'$ (строго) возрастает на $(a, b)$\\
\q2. Пусть функция f непрерывна на $\langle a, b\rangle$ и дважды дифференцируема на $(a, b)$. Тогда $f$ выпукла вниз на $\langle a, b\rangle$ в том и только том случае, когда $f''(x) \ge 0\ \forall x \in (a, b)$.
\skip


\end{document}



















